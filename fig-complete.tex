% !TEX root = hazelnut-dynamics.tex
\begin{figure}[h]
\judgbox{\isComplete{\htau}}{$\htau$ is complete}
\begin{mathpar}
\inferrule[BComplete]{ }{
  \isComplete{b}
}

\inferrule[ArrComplete]{
  \isComplete{\htau_1}\\
  \isComplete{\htau_2}
}{
  \isComplete{\tarr{\htau_1}{\htau_2}}
}
\end{mathpar}

\vsepRule

\judgbox{\isComplete{\hexp}}{$\hexp$ is complete}
\begin{mathpar}
\inferrule[EVarComplete]{ }{
  \isComplete{x}
}

\inferrule[EConstComplete]{ }{
  \isComplete{c}
}

\inferrule[ESynLamComplete]{
  \isComplete{\htau}\\
  \isComplete{\hexp}
}{
  \isComplete{\halam{x}{\htau}{\hexp}}
}

\inferrule[EAnaLamComplete]{
  \isComplete{\hexp}
}{
  \isComplete{\hlam{x}{\hexp}}
}

\inferrule[EApComplete]{
  \isComplete{\hexp_1}\\
  \isComplete{\hexp_2}
}{
  \isComplete{\hap{\hexp_1}{\hexp_2}}
}

\inferrule[EAscComplete]{
  \isComplete{\hexp}\\
  \isComplete{\htau}
}{
  \isComplete{\hexp : \htau}
}
\end{mathpar}

\vsepRule


\judgbox{\isComplete{\dexp}}{$\dexp$ is complete}
\begin{mathpar}
\inferrule[DVarComplete]{ }{
  \isComplete{x}
}

\inferrule[DConstComplete]{ }{
  \isComplete{c}
}

\inferrule[DLamComplete]{
  \isComplete{\htau}\\
  \isComplete{\dexp}
}{
  \isComplete{\dlam{x}{\htau}{\dexp}}
}

\inferrule[DApComplete]{
  \isComplete{\dexp_1}\\
  \isComplete{\dexp_2}
}{
  \isComplete{\dap{\dexp_1}{\dexp_2}}
}

\inferrule[DCastComplete]{
  \isComplete{\dexp}\\
  \isComplete{\htau_1}\\
  \isComplete{\htau_2}
}{
  \isComplete{\dcasttwo{\dexp}{\htau_1}{\htau_2}}
}
\end{mathpar}

\caption{Complete H-types, external H-expressions, and internal H-expressions}
\label{fig:complete}
\end{figure}
