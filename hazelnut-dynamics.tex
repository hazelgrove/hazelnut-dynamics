\def\OPTIONConf{1}%
%\newif\iftr\trfalse
\newif\iftr\trtrue
%\documentclass[12pt]{article}
\RequirePackage{etex}
\documentclass[preprint,9pt]{sigplanconf}
% The following \documentclass options may be useful:
%
% 10pt          To set in 10-point type instead of 9-point.
% 11pt          To set in 11-point type instead of 9-point.
% authoryear    To obtain author/year citation style instead of numeric.
\usepackage[T1]{fontenc}
\usepackage{amsmath}
\usepackage{amssymb}
\usepackage{amsthm}
\usepackage{ stmaryrd }
\usepackage{mathpartir}
\usepackage{stmaryrd}
\usepackage{wasysym}
\usepackage{multicol}
\usepackage{extarrows}
\usepackage[usenames,dvipsnames,svgnames,table]{xcolor}
\definecolor{light-gray}{gray}{0.9}
\usepackage{soul}
\setulcolor{red}
\usepackage{mathpazo}
\usepackage{colortab}
\usepackage{url}
\usepackage{todonotes}
\usepackage{listings}
\lstset{tabsize=2,
basicstyle=\ttfamily\fontsize{8pt}{1em}\selectfont}
\usepackage{microtype}
\sloppy
\def \TirNameStyle #1{\small\rulename{#1}}
\renewcommand{\MathparLineskip}{\lineskiplimit=.3\baselineskip\lineskip=.3\baselineskip plus .2\baselineskip}

\usepackage{joshuadunfield}
\usepackage{llproof}
\usepackage{rulelinks}

%\usepackage{balance} %% Balance last page's two columns

\newtheorem{theorem}{Theorem}
\newtheorem{lemma}[theorem]{Lemma}
\newtheorem{corollary}{Corollary}
\newtheorem{definition}{Definition}
\newenvironment{proof-sketch}{\noindent{\emph{Proof Sketch.}}}{\qed}
\makeatletter

\renewcommand\topfraction{0.85}
\renewcommand\bottomfraction{0.85}
\renewcommand\textfraction{0.1}
\renewcommand\floatpagefraction{0.85}

\AtBeginDocument{%
 \abovedisplayskip=2pt
 \abovedisplayshortskip=0pt
 \belowdisplayskip=2pt
 \belowdisplayshortskip=0pt
}

\setlength{\floatsep}{10pt}
\setlength{\textfloatsep}{12pt}

\usepackage[compact]{titlesec}
\titlespacing*{\section}{0pt}{4pt}{2pt}
\titlespacing*{\subsection}{0pt}{4pt}{2pt}
\titlespacing*{\subsubsection}{0pt}{4pt}{2pt}
\titlespacing*{\paragraph}{0pt}{4pt}{2pt}
\setlength{\skip\footins}{3pt plus 1px minus 5px}

\usepackage[colorlinks=true,allcolors=blue,breaklinks,draft=false]{hyperref}   % hyperlinks, including DOIs and URLs in bibliography
% known bug: http://tex.stackexchange.com/questions/1522/pdfendlink-ended-up-in-different-nesting-level-than-pdfstartlink

\usepackage{enumitem}
\makeatletter
\def\thm@space@setup{%
  \thm@preskip=4px plus 2px minus 2px
  \thm@postskip=\thm@preskip % or whatever, if you don't want them to be equal
}
\makeatother

\usepackage{todonotes}
\usepackage{xcolor}
%\usepackage{adjustbox}

% !TEX root = hazelnut-dynamics.tex

% Violet hotdogs; highlight color helps distinguish them
\newcommand{\llparenthesiscolor}{\textcolor{violet}{\llparenthesis}}
\newcommand{\rrparenthesiscolor}{\textcolor{violet}{\rrparenthesis}}
% \newcommand{\llparenthesiscolor}{\textcolor{red}{\lfloor}}
% \newcommand{\rrparenthesiscolor}{\textcolor{red}{\rfloor}}

%% TODO if feeling really obsessive, use the following two in place of x and u
\newcommand{\varVar}{x}
\newcommand{\varHole}{u}

% HTyp and HExp
\newcommand{\hcomplete}[1]{#1~\mathsf{complete}}

% HTyp
\newcommand{\htau}{\tau}
\newcommand{\tarr}[2]{#1 \rightarrow #2}
\newcommand{\tprod}[2]{#1 \times #2}
\newcommand{\tnum}{\texttt{num}}
\newcommand{\tb}{\texttt{b}}
\newcommand{\tehole}{\llparenthesiscolor\rrparenthesiscolor}
\newcommand{\tsum}[2]{{#1} + {#2}}

\newcommand{\tconsistent}[2]{#1 \sim #2}
\newcommand{\tinconsistent}[2]{#1 \nsim #2}

% HExp
\newcommand{\hexp}{e}
\newcommand{\hlam}[2]{\lambda #1.#2}
\newcommand{\halam}[3]{\lambda #1{:}#2.#3}
\newcommand{\hap}[2]{#1(#2)}
\newcommand{\hapP}[2]{(#1)~(#2)} % Extra paren around function term
\newcommand{\hpair}[2]{(#1, #2)}
\newcommand{\hprj}[2]{\mathsf{prj}_{#1}(#2)}
\newcommand{\lblL}{\mathsf{L}}
\newcommand{\lblR}{\mathsf{R}}
\newcommand{\hnum}[1]{\underline{#1}}
\newcommand{\hadd}[2]{#1 + #2}
\newcommand{\hehole}[1]{\llparenthesiscolor\rrparenthesiscolor^{#1}}
% \newcommand{\hhole}[1]{\setlength{\fboxsep}{0pt}\fcolorbox{red}{white}{\vphantom{)}$#1$}}
\newcommand{\hhole}[2]{\llparenthesiscolor#1\rrparenthesiscolor^{#2}}
% \newcommand{\hhole}[1]{
  % \setlength{\fboxsep}{0pt}
  % \colorbox{violet!10!white!100}{\ensuremath{\llparenthesiscolor#1\rrparenthesiscolor}}}
\newcommand{\hindet}[1]{\lceil#1\rceil}
\newcommand{\hinj}[2]{\texttt{inj}_{#1}({#2})}
\newcommand{\hcase}[5]{\texttt{case}({#1},{#2}.{#3},{#4}.{#5})}

\newcommand{\hGamma}{\Gamma}
\newcommand{\domof}[1]{\text{dom}(#1)}
\newcommand{\hsyn}[3]{#1 \vdash #2 \Rightarrow #3}
\newcommand{\hana}[3]{#1 \vdash #2 \Leftarrow #3}

% ZTyp and ZExp
\newcommand{\zlsel}[1]{{\bowtie}{#1}}
\newcommand{\zrsel}[1]{{#1}{\bowtie}}

%\newcommand{\zwsel}[1]{\adjustbox{cframe=blue}{\ensuremath{{\textcolor{blue}{\triangleright}}{#1}{\textcolor{blue}{\triangleleft}}}}}
\newcommand{\zwsel}[1]{
  \setlength{\fboxsep}{0pt}
  \colorbox{green!10!white!100}{
    \ensuremath{{{\textcolor{Green}{{\hspace{-2px}\triangleright}}}}{#1}{\textcolor{Green}{\triangleleft{\vphantom{\tehole}}}}}}
}
%\newcommand{\zwsel}[1]{{\triangleright}{#1}{\triangleleft}}

\newcommand{\removeSel}[1]{#1^{\diamond}}

% ZTyp
\newcommand{\ztau}{\hat{\tau}}

% ZExp
\newcommand{\zexp}{\hat{e}}

% Direction
\newcommand{\dParent}{\mathtt{parent}}
\newcommand{\dChild}{\mathtt{firstChild}}
\newcommand{\dNext}{\mathtt{nextSib}}
\newcommand{\dPrev}{\mathtt{prevSib}}

% Action
\newcommand{\aMove}[1]{\mathtt{move}~#1}
	\newcommand{\zrightmost}[1]{\mathsf{rightmost}(#1)}
	\newcommand{\zleftmost}[1]{\mathsf{leftmost}(#1)}
\newcommand{\aSelect}[1]{\mathtt{sel}~#1}
\newcommand{\aDel}{\mathtt{del}}
\newcommand{\aReplace}[1]{\mathtt{replace}~#1}
\newcommand{\aConstruct}[1]{\mathtt{construct}~#1}
\newcommand{\aConstructx}[1]{#1}
\newcommand{\aFinish}{\mathtt{finish}}

\newcommand{\performAna}[5]{#1 \vdash #2 \xlongrightarrow{#4} #5 \Leftarrow #3}
\newcommand{\performAnaI}[5]{#1 \vdash #2 \xlongrightarrow{#4}\hspace{-3px}{}^{*}~ #5 \Leftarrow #3}
\newcommand{\performSyn}[6]{#1 \vdash #2 \Rightarrow #3 \xlongrightarrow{#4} #5 \Rightarrow #6}
\newcommand{\performSynI}[6]{#1 \vdash #2 \Rightarrow #3 \xlongrightarrow{#4}\hspace{-3px}{}^{*}~ #5 \Rightarrow #6}
\newcommand{\performTyp}[3]{#1 \xlongrightarrow{#2} #3}
\newcommand{\performTypI}[3]{#1 \xlongrightarrow{#2}\hspace{-3px}{}^{*}~#3}

\newcommand{\performMove}[3]{#1 \xlongrightarrow{#2} #3}
\newcommand{\performDel}[2]{#1 \xlongrightarrow{\aDel} #2}

% Form
\newcommand{\farr}{\mathtt{arrow}}
\newcommand{\fnum}{\mathtt{num}}
\newcommand{\fsum}{\mathtt{sum}}

\newcommand{\fasc}{\mathtt{asc}}
\newcommand{\fvar}[1]{\mathtt{var}~#1}
\newcommand{\flam}[1]{\mathtt{lam}~#1}
\newcommand{\fap}{\mathtt{ap}}
\newcommand{\farg}{\mathtt{arg}}
\newcommand{\fnumlit}[1]{\mathtt{lit}~#1}
\newcommand{\fplus}{\mathtt{plus}}
\newcommand{\fhole}{\mathtt{hole}}
\newcommand{\fnehole}{\mathtt{nehole}}

\newcommand{\finj}[1]{\mathtt{inj}~#1}
\newcommand{\fcase}[2]{\mathtt{case}~#1~#2}

% Talk about formal rules in example
\newcommand{\refrule}[1]{\textrm{Rule~(#1)}}

\newcommand{\herase}[1]{\left|#1\right|_\textsf{erase}}

\newcommand{\arrmatch}[2]{#1 \blacktriangleright_{\rightarrow} #2}
\newcommand{\prodmatch}[2]{#1 \blacktriangleright_{\times} #2}
\newcommand{\summatch}[2]{#1 \blacktriangleright_{+} #2}


\newcommand{\TABperformAna}[5]{#1 \vdash & #2                & \xlongrightarrow{#4} & #5 & \Leftarrow #3}
\newcommand{\TABperformSyn}[6]{#1 \vdash & #2 \Rightarrow #3 & \xlongrightarrow{#4} & #5 \Rightarrow #6}
\newcommand{\TABperformTyp}[3]{& #1 & \xlongrightarrow{#2} & #3}

\newcommand{\TABperformMove}[3]{#1 & \xlongrightarrow{#2} & #3}
\newcommand{\TABperformDel}[2]{#1 \xlongrightarrow{\aDel} #2}

\newcommand{\sumhasmatched}[2]{#1 \mathrel{\textcolor{black}{\blacktriangleright_{+}}} #2}

%%%% DYNAMICS %%%%
%% TODO remove these macros
%% marks for eval
\newcommand{\unevaled}{\times}
\newcommand{\evaled}{\checkmark}
\newcommand{\markname}{m}

\newcommand{\mvar}[0]{u}
\newcommand{\subst}[0]{\sigma}
\newcommand{\dexp}[0]{d}
\newcommand{\dconst}[0]{c}
\newcommand{\dval}[0]{\ddot{v}}
%% TODO remove this macro
\newcommand{\dcast}[2]{\langle #1 \rangle ~ #2}
%% TODO make the following two look better
\newcommand{\dcasttwo}[3]{#1 \langle{#2}\Rightarrow{#3}\rangle}
\newcommand{\dcastthree}[4]{#1 \langle{#2}\Rightarrow{#3}\Rightarrow{#4}\rangle}
\newcommand{\dcastfail}[4]{#1 \langle{#2}\Rightarrow{#3}\not\Rightarrow{#4}\rangle}
\newcommand{\dlam}[3]{\lambda #1:#2.#3}
\newcommand{\dap}[2]{#1(#2)}
\newcommand{\dapP}[2]{(#1)(#2)} % Extra paren around function term
\newcommand{\dnum}[1]{\underline{#1}}
\newcommand{\dadd}[2]{#1 + #2}
\newcommand{\dehole}[3]{\leftidx{^{#3}}{\llparenthesiscolor\rrparenthesiscolor}{^{#1}_{#2}}}
\newcommand{\dhole}[4]{\leftidx{^{#4}}{\llparenthesiscolor#1\rrparenthesiscolor}{^{#2}_{#3}}}
\newcommand{\dindet}[1]{\lceil#1\rceil}
\newcommand{\dinj}[2]{\texttt{inj}_{#1}({#2})}
\newcommand{\dcase}[5]{\texttt{case}({#1},{#2}.{#3},{#4}.{#5})}

\newcommand{\expandAna}[6]{#1 \vdash #2 \Leftarrow #3 \leadsto #4 : #5 \dashv #6}
\newcommand{\expandSyn}[5]{#1 \vdash #2 \Rightarrow #3 \leadsto #4 \dashv #5}
\newcommand{\hasType}[4]{#1; #2 \vdash #3 : #4}
\newcommand{\isValue}[1]{#1~\mathsf{val}}
\newcommand{\isIndet}[1]{#1~\mathsf{indet}}
\newcommand{\isFinal}[1]{#1~\mathsf{final}}
\newcommand{\isErr}[2]{#1 \vdash #2~\mathsf{err}}
%% \newcommand{\stepsTo}[2]{#1 \mapsto_{\Delta} #2}
\newcommand{\stepsToD}[3]{#1 \vdash #2 \mapsto #3}

%% TODO if feeling obsessive, replace direct uses of \Delta
\newcommand{\hDelta}{\Delta}
\newcommand{\Dunion}[2]{#1 \cup #2}
\newcommand{\idof}[1]{\mathsf{id}(#1)}
\newcommand{\Dbinding}[3]{#1 :: [#2]#3}

% Contextual dynamics
\newcommand{\evalctx}{\mathcal{E}}
\newcommand{\evalhole}{\circ}
\newcommand{\isevalctx}[1]{#1~\mathsf{evalCtx}}
\newcommand{\reducesE}[3]{#1 \vdash #2 \longrightarrow #3}
\newcommand{\selectEvalCtxR}[2]{#1\{#2\}}
\newcommand{\selectEvalCtx}[3]{#1=\selectEvalCtxR{#2}{#3}}

\begin{document}

\conferenceinfo{-}{-}
\copyrightyear{-}
\copyrightdata{[to be supplied]}

%\titlebanner{}        % These are ignored unless
\preprintfooter{Draft}   % 'preprint' option specified.

\title{Hazelnut: A Bidirectionally Typed \\ Structure Editor
 Calculus}
% \title{Hazelnut:\\A Bidirectionally Typed Structure Editor Calculus}

%\authorinfo{\vspace{-2px}}{}{}
%\authorinfo{~}{~}{\vspace{-10px}}
%\authorinfo{Cyrus Omar \and Jonathan Aldrich}
%         {Carnegie Mellon University}
%         {\{comar, aldrich\}@cs.cmu.edu}
\iftr
\authorinfo{
        Cyrus Omar$^1$
        \and Ian Voysey$^1$
        \and Michael Hilton$^2$
        \and Jonathan Aldrich$^1$
        \and Matthew A. Hammer$^3$
}
{
 $^1$~Carnegie Mellon University
 \and
 $^2$~Oregon State University
 \and
 $^3$~University of Colorado Boulder\vspace{-10px}
}
{
%\email{comar@cs.cmu.edu}\\
%\texttt{http://www.cs.cmu.edu/\homedir comar/}
%\email{hiltonm@eecs.oregonstate.edu}\\
}
\else
\authorinfo{~}{~}{\vspace{-10px}}
\fi

%
%\authorrunning{Omar et al.} % abbreviated author list (for running head)
%
%%%% list of authors for the TOC (use if author list has to be modified)
%\tocauthor{Ivar Ekeland, Roger Temam, Jeffrey Dean, David Grove,
%Craig Chambers, Kim B. Bruce, and Elisa Bertino}
%
%\email{Matthew.Hammer@colorado.edu}


\maketitle
\begin{abstract}
% Programmers typically construct and manipulate well-typed expressions  only indirectly, as text that must pass through a parser and type\-checker. Not all text survives this journey. In particular, text that arises transiently, or when the programmer makes a mistake, is often malformed or ill-typed.
% Not all text that arises during the programming process survives the journey through a parser and typechecker. In particular, text that arises transiently during the editing process, or when the programmer has made a mistake, is often malformed or ill-typed.
% Contending with malformed program text or well-formed but meaningless syntax trees is difficult for programmers and their tools alike.

%  \emph{Structure editors} have long promised to alleviate these burdens by exposing only edit actions that  cause sensible changes to the program structure.
% Existing designs for structure editors, however, are complex and somewhat \emph{ad hoc}. They also focus primarily on syntactic well-formedness, so programs can still be left semantically meaningless as they are being constructed.
% Structure editors (e.g. Scratch) eliminate the possibility of syntax errors. 

\emph{Structure editors} allow programmers to edit the tree structure of a program directly. This can have cognitive benefits, particularly for novice and end-user programmers (as evidenced by the popularity of structure editors like Scratch.) It also simplifies matters for tool designers, because they do not need to contend with malformed program text.

This paper defines Hazelnut, a {structure editor} based on a small bidirectionally typed lambda calculus extended with \emph{holes} and a \emph{cursor} (\emph{ala} Huet's zipper.) Hazelnut goes one step beyond syntactic well-formedness: it's {edit actions} operate over statically meaningful (i.e. well-typed) terms.  
Na\"ively, this prohibition on ill-typed edit states would force the programmer to construct terms in a rigid ``outside-in'' manner. To avoid this problem, the {action semantics} automatically places terms assigned a type that is inconsistent with the expected type {inside} a {hole}. This safely defers the type consistency check until the term inside the hole is \emph{finished}.

Hazelnut is a foundational type-theoretic account of typed structure editing, rather than an end-user tool itself. To that end, we describe how Hazelnut's rich metatheory, which we have mechanized in Agda, guides the definition of an extension to the calculus. We also discuss various plausible evaluation strategies for terms with holes, and in so doing reveal connections with gradual typing and contextual modal type theory (the Curry-Howard interpretation of contextual modal logic.) Finally, we  discuss how Hazelnut's semantics lends itself to implementation as a functional reactive program. Our reference implementation is written using \lstinline{js_of_ocaml}.

%Formally, Hazelnut is a bidirectionally typed lambda calculus extended with \emph{holes}, a \emph{focus model} (based on Huet's zipper) and an \emph{action model}.
\end{abstract}

%\category{D.3.2}{Programming Languages}{Language Classifications}[Extensible Languages]
%\category{D.3.4}{Programming Languages}{Processors}[Compilers]
%\category{F.3.1}{Logics \& Meanings of Programs}{Specifying and Verifying and Reasoning about Programs}[Specification Techniques]
%\keywords
%extensible languages; module systems; type abstraction; typed compilation; type-level computation

\section{Introduction}\label{sec:introduction}
% !TEX root = hazelnut-popl17.tex

%% There are some benefits to this approach, to be sure, but the structural
%% mismatch between programs and their textual representations also imposes
%% various burdens.  For example, the primitive edit actions available in a
%% text editor (e.g. inserting or deleting a character or word) do not
%% always correspond to sensible structural transformations.

% spj: describe the problem; state our contributions; STOP. one page max.

% When constructing a program or proof in a language with rich type
% structure, skilled programmers generally follow a \emph{type discipline}
% where they first determine the type of the expression that they are
% constructing in order to constrain the mental search space that they are
% operating within.

% For example, if the programmer knows that an expression of type
% $\tarr{\tnum}{\tnum}$ is needed, then it is often the case (though, of course, not
% necessarily the case) that the expression will take the form $$\hlam{\mathit{x}}{e}$$
% for some variable $x$ and function body $e$. If the programmer chooses this
% form, then after picking a suitable variable name, her focus will be on
% constructing a suitable body, $e$. Following the type discipline, $e$ must
% be of type $\tnum$, and so this process can begin anew.

% The problem is that when using a text
% editor to construct a program, it is easy, and indeed necessary, to deviate from this disciplined process.  Rather, text editors operate on sequences of
% characters (i.e. \emph{text}.) 

%Although programs can be represented as text, most text does not correspond to a syntactically well-formed and semantically well-defined program. 
% Programming languages, and therefore programs themselves, are rich mathematical structures.
Programmers typically construct and manipulate well-typed expressions only indirectly, by editing text that is first parsed according to a textual syntax and then typechecked according to a static semantics. This indirection has practical benefits, to be sure -- text editors and other text-based tools benefit from decades of development effort -- but it also introduces some fundamental complexity into the programming process. 

% Every day, programmers use text-based tools to construct and manipulate programs.  
% These programs are written in languages which defined by a textual syntax. 
% While textual syntax has proved to have practical utility, it also introduces some fundamental complexity into the programming process.

% It also complicates matters for tool designers because tools are often confronted with text that does not correspond to a well-formed, meaningful program\todo{mention YoungSeok's data here}. This may be because the programmer is in the midst of a sequence of 
% edit actions that leaves the text temporarily malformed or ill-typed, or because the programmer has made a mistake. The language definition is silent about these situations, so it is difficult for tools to help programmers determine and execute a corrective course of action (e.g. by providing \todo{cite study on benefits of syntax highlighting}syntax highlighting and semantics-aware code completion services\todo{cite something?}.)
%Doesn't really start out with a bang, IMHO. I like to start papers out with a "big" problem.
%Most programming languages define a textual syntax. This allows programmers to use text editors and other standard text-based tools to construct and manipulate programs. While this is of substantial practical utility, it also introduces some fundamental complexity into the programming process. 

First, it requires that programmers learn the subtleties of the textual syntax (e.g. the precedence of the various forms.) This can be particularly challenging for novices \cite{Altadmri:2015:MCI:2676723.2677258}. %For example, one study of novice programmers found syntax errors (e.g. unbalanced parentheses) to be the most common class of errors.
% Too much detail? 

The fact that not every sequence of characters corresponds to a meaningful program also complicates matters for tool designers. In particular, program editors must contend with meaningless text on a regular basis. In a dataset analyzed by Yoon and Myers consisting of 1460 hours of edit logs, 44.2\% of edit states were syntactically malformed \cite{6883030}. Some additional percentage of edit states were well-formed but ill-typed (the dataset did not contain enough information to determine the exact percentage.)  
Some of these meaningless edit states arose because the programmer was in the midst of a sequence of edits that left the edit state transiently meaningless, while others arose because the programmer made a logical mistake. Because the language definitions is silent about these edit states, it is difficult to design useful editor services, e.g. syntax highlighting~\cite{sarkar2015impact}, type-aware code completion~\cite{Mooty:2010:CCC:1915084.1916348,Omar:2012:ACC:2337223.2337324}, and refactoring support \cite{mens2004survey}. Editors must either disable these editor services when they encounter meaningless edit states or develop \emph{ad hoc} heuristics, which can be misleading. %, or because the programmer has made a mistake. 
% The syntax definition is silent about these situations, so it is difficult for tools to help programmers determine and execute a corrective course of action 

% Many editors have developed \emph{ad hoc} workarounds for this problem, e.g. they might attempt to use regular expressions to highlight malformed program text, insert closing delimiters automatically, use whitespace to guess where a delimiter is likely to appear, or continue past a type error by pretending that the type was as expected (if, indeed, an expected type can be determined.) These heuristic methods are often complex, and they can confuse or mislead the programmer. % This may also help explain why the error messages emitted by parsers and typecheckers are often quite baroque\todo{citations}.

% stuck editing a representation of the program instead of the structures
% themselves. The editor does not restrict what the programmer may do: you
% can delete characters that belong, insert ones that don't, forget things
% that were needed, and so. There's nothing stopping us from accidentally
% writing $$\lambda \mathit{x:num}.\mathit{(x,x)}$$ even though it's obvious
% that building a pair can't hope to form a natural number.

% The type structure of the language makes this sort of error obvious: it's
% not that you're adding characters that make your program incorrect, or even
% malformed; you're adding characters that can't possibly create a structure
% you want because of the type. Simply put, the primitive operations
% available in text editors do not always correspond to sensible
% transformations on the structure of the program.
These complications have motivated a long line of research into \emph{structure editors}, i.e. program editors where every edit state is a program structure. % Eliminating text eliminates the possibility of syntax errors.

Most structure editors are \emph{syntactic structure editors}, i.e. the edit state is a syntax tree with \emph{holes} that stand for branches of the tree that have yet to be constructed, and the edit actions are context-free tree transformations. For example, Scratch is a syntactic structure editor that has achieved success as a tool for teaching children how to program \cite{Resnick:2009:SP:1592761.1592779}. 

The Scratch language has a trivial static semantics, but researchers have also designed syntactic structure editors for  languages with a non-trivial static semantics. For example, \texttt{mbeddr} is an editor for a C-like language \cite{voelter_mbeddr:_2012}, TouchDevelop is an editor for an object-oriented language \cite{tillmann_touchdevelop:_2011} and Lamdu is an editor for a functional language similar to Haskell \cite{lamdu}. Each of these editors presents an innovative user interface, but the non-trivial type and binding structure of the underlying language complicates its design. The reason is that a syntactic structure editor does not guarantee that every edit state is statically meaningful -- only that it is syntactically well-formed. As in textual program editors, these syntactic structure editors must either disable key editor services when they encounter meaningless edit states or deploy \emph{ad hoc} heuristics.
%Indeed, it is not entirely clear what it should mean for terms with holes to be statically meaningful. 
% Consequently, heuristics analagous to those that pervade textual program editors also pervade these syntactic structure editors.

% \begin{itemize}
% \item 

% First, there is an all too familiar problem: these languages do not have a clear formal semantics and metatheory, i.e. ``the implementation is the specification''. This makes it difficult to reason methodically about types and binding, so, in turn, it is difficult to design reliable editor services.

% Even if a formal semantics and metatheory for the underlying language were forthcoming (e.g. following Standard ML \cite{Harper00atype-theoretic,mthm97-for-dart}), it would go only part of the way towards addressing the challenges faced by structure editor designers. First, a standard semantics assigns no formal meaning to \emph{incomplete terms}, i.e. terms with holes. Moreover, a syntactic structure editor  -- so incorporating holes into the static semantics would in any case be only a minor salve. Heuristics analagous to those that pervade textual program editors would still be needed to contend with the ill-typed edit states.

This paper develops a principled solution to this problem. We introduce Hazelnut,  a \emph{typed structure editor}  based on a bidirectionally typed lambda calculus extended to assign static meaning to expressions and types with {holes}, which we call \textbf{H-expressions} and \textbf{H-types}. Hazelnut's formal \emph{action semantics} maintains the invariant that every edit state is a statically meaningful (i.e. well-typed) H-expression with a single superimposed \emph{cursor}. We call H-expressions and H-types with a cursor \textbf{Z-expressions} and \textbf{Z-types} (so prefixed because our encoding follows Huet's \emph{zipper} pattern \cite{JFP::Huet1997}.) %ctions act relative to the H-expression or H-type under the cursor.% (which need not be a hole.)  %More specifically, actions are context-aware,  acting on an expression whose type is determined by its surroundings (e.g. a function argument), only actions consistent with that type are permitted. 

Na\"ively, enforcing an injunction on ill-typed edit states would force programmers to construct programs in a rigid ``outside-in'' manner. For example, the programmer would often need to construct the outer function application form before identifying the intended function. To address this problem, Hazelnut leaves newly constructed expressions \emph{inside} a hole if the expression's type is inconsistent with the expected type. This meaningfully defers the type consistency check until the expression inside the hole is \emph{finished}. In other words, holes appear both at the leaves and at the internal nodes of an H-expression that remain under construction. %In short, Hazelnut is a \emph{bidirectionally typed structure editor calculus}. %Actions act at a programmer-indicate subtree, called the \emph{focus} (which is defined following Huet's zipper pattern.) 

% By defining \emph{holes} as a language construct, for both expressions as types, Hazelnut enables type-aware actions that always leave the program in both a structurally and semantically well-defined state. 
%In Hazelnut, expressions and types with \emph{holes} have a well-defined static semantics. Edit actions are type-aware and leave the program in both a structurally and semantically well-defined state. 



%However, these syntactically correct states can be semantically meaningless (i.e. undefined), because the language definitions generally only give meaning to complete, well-typed terms. (e.g., when a branching element is introduced, the program does not become semantically correct until both branches are complete)
% This makes it difficult for humans and tools to reason about types and binding during the development process, even when using a structure editor.
% Similar to text editors, structure editors also can develop workarounds to try to help users, but these efforts can only extend so far, because of the underlying language definition.
%Some structure editors attempt stuff, but it is not clear what invariants are being maintained (e.g. Unison ; Scratch seems not to allow literals of the wrong type, but variables are not typed.)\todo{revise} %More sophisticated semantic reasoning principles thus remain .


 



The remainder of this paper is organized as follows:
\begin{itemize}[itemsep=0px,partopsep=2px,topsep=2px]
\item We begin in Sec. 
    \ref{sec:example} with two examples of edit sequences to develop the reader's intuitions.  
\item We then give a detailed overview of Hazelnut's semantics and metatheory, and our mechanization in the Agda proof assistant, in Sec.  \ref{sec:hazel}.
\item Hazelnut is a {foundational} calculus, i.e. a calculus that language and editor designers are expected to extend with higher level constructs. We extend Hazelnut with simple sum types in Sec.  \ref{sec:extending} to demonstrate how Hazelnut's rich metatheory guides such a development. 
\item In Sec.  \ref{sec:impl}, we briefly describe how Hazelnut's action semantics lends itself to efficient implementation as a functional reactive program. Our reference implementation is written using \lstinline{js_of_ocaml} and the OCaml \lstinline{React} library.
 
% \item However, we consider the space of possible evaluation strategies for incomplete expressions in Section \ref{sec:dynamics}. 
% so doing, we discover interesting connections with gradual typing and contextual modal type theory. The former provides an interpretation of type holes, and the latter provides a logical interpretation of expression holes. \todo{revise this one} 

\item In Sec.  \ref{sec:rw}, we summarize related work. In particular, much of the technical machinery needed to handle type holes coincides with machinery developed in the study of gradual type systems. Similarly, expression holes can be interpreted as the closures of contextual modal type theory, which, by its correspondence with contextual modal logic, lays logical foundations beneath our work. %This suggests a principled design for an ``edit and resume'' feature.

\item We conclude in Sec.  \ref{sec:future} by summarizing our vision of a principled science of structure editor design rooted in type theory, and suggest a number of future directions.
\end{itemize} 
The supplemental material includes 1) a typeset listing of Hazelnut's rules in definitional order; 2) the formalization of Hazelnut in Agda; and 3) our reference implementation of Hazelnut, both in source form and pre-compiled to JavaScript for use in a web browser.

\section{Programming in Hazelnut}\label{sec:example}
% !TEX root = hazelnut-popl17.tex
The reader is encouraged to follow along with the examples that we will discuss in this section using the implementation.

%
\subsection{Example 1: Constructing the Increment Function}

Figure~\ref{fig:first-example} shows an edit sequence that constructs the increment function, of type $\tarr{\tnum}{\tnum}$, starting from the empty hole via the indicated sequence of {actions}. We will introduce Hazelnut's formal syntax and define the referenced rules in Sec. \ref{sec:hazel}. First, let us build some high-level intuitions. 

%In the first task (Lines 1-9), the user constructs the identity function over numbers. In the second task (Lines 10-19), the user applies this function (assumed to be bound to a variable, $id$), to the number expression $\hnum{3}$.
% Each of these tasks is carried out interactively, through the sequence of \emph{actions} shown in the  column labeled \textbf{Next Action}. For reference, we cite the relevant rules from Sec. \ref{sec:hazel} in the final column.

% The second and third columns of the
% table show the program as it is being constructed in two forms. The second column shows it as an \textbf{H-expression}, which is an expression that can contain \emph{holes}, delimited by $\llparenthesis$ and $\rrparenthesis$. The third column shows a corresponding \textbf{Z-expression}. Z-expressions are H-expressions with a single focus on some sub-term, delimited by $\triangleright$ and $\triangleleft$. The focus need not be on a hole.
% on working on filling just one of the holes.
% Each action produces a new Z-expression, but this may or may not correspond to a new H-expression (in particular, some actions only move the focus, without changing the structure of the term.)
% to the hole in
%focus in the Z-Expression to produce the next line, which may or may not
%produce a substantively different H-Expression.

The edit state in Hazelnut is a {Z-expression}, $\zexp$. Every Z-expression has a single {H-expression}, $\hexp$, or {H-type}, $\htau$, under the {cursor}, typeset $\zwsel{\hexp}$ or $\zwsel{\htau}$, respectively. For example, on Line 1, the empty expression hole, $\hhole{}$, is under the cursor. 

Actions act relative to the cursor. The first action that we perform is $\aConstruct{\flam{x}}$, which instantiates the hole with a lambda abstraction binding the variable $x$. This results in the Z-expression on Line 2, consisting of a lambda abstraction with an arrow type ascription. The cursor is placed on the argument type hole. Type holes are typeset like empty expression holes for visual consistency. 

The actions on Lines 2-4 complete the type ascription. In particular, the $\aConstruct{\fnum}$ action constructs the $\tnum$ type at the cursor and the $\aMove{\dNext}$ action moves the cursor to the next hole.



\begin{figure}[t!]
\begin{center}
$\arraycolsep=4px
\begin{array}{|r||l|l||l|l|}
\hline
\# & \textbf{Z-Expression} & 
%\textbf{H-Expression} & 
% \textbf{Type} & 
\textbf{Next Action} & \textbf{Rule}
\\
\hline
1 &
\zwsel{\hhole{}} & 
% \hhole{} &
% \tehole 
% &
\aConstruct{\flam{x}} & 
\text{(\ref{r:conelamhole})}
\\ 2 &
\hlam{x}{\hhole{}} : \tarr{\zwsel{\hhole{}}}{\hhole{}} & 
% \hlam{x}{\hhole{}} : \tarr{\hhole{}}{\hhole{}} &
% \tarr{\tehole}{\tehole} &
\aConstruct{\fnum{}} &
\text{(\ref{r:contnum})}
\\ 3 &
\hlam{x}{\hhole{}} : \tarr{\zwsel{\tnum{}}}{\hhole{}} &
% \hlam{x}{\hhole{}} : \tarr{\tnum{}}{\hhole{}} &
% \tarr{\tnum}{\tehole} & 
\aMove{\dNext{}} & 
\text{({\ref{rule:move-nextSib-arr}})}
\\ 4 &
\hlam{x}{\hhole{}} : \tarr{\tnum}{\zwsel{\hhole{}}}
&
% \text{\textquotedbl}&
% \tarr{\tnum}{\tehole} & 
\aConstruct{\fnum{}} & 
\text{(\ref{r:contnum})}
\\ 5 &
\hlam{x}{\hhole{}} : \tarr{\tnum{}}{\zwsel{\tnum{}}} & 
% \hlam{x}{\hhole{}} : \tarr{\tnum{}}{\tnum{}} &
% \tarr{\tnum}{\tnum} &
\aMove{\dParent{}} & 
\text{(\ref{rule:move-parent-arr-right})}
\\ 6 &
\hlam{x}{\hhole{}} : \zwsel{\tarr{\tnum{}}{\tnum{}}}
&
% \text{\textquotedbl}&
% \tarr{\tnum}{\tnum} & 
\aMove{\dParent{}} & 
\text{(\ref{rule:move-parent-asc-right})}
\\ 7 & 
\zwsel{\hlam{x}{\hhole{}} : \tarr{\tnum}{\tnum}} & 
\aMove{\dChild} & 
\text{(\ref{r:movefirstchild})}
\\ 8 &
% &
\zwsel{\hlam{x}{\hhole{}}} : \tarr{\tnum{}}{\tnum{}} & 
% \tarr{\tnum}{\tnum} &
% &
\aMove{\dChild{}} & 
\text{(\ref{r:movefirstchild-lam})}
\\ 9 &
% &
\hlam{x}{\zwsel{\hhole{}}} : \tarr{\tnum{}}{\tnum{}} &
% \tarr{\tnum}{\tnum} & 
\aConstruct{\fvar{x}} & 
\text{(\ref{r:conevar})}
\\ 10 &
% \hlam{x}{{x}} : \tarr{\tnum{}}{\tnum{}}
% &
\hlam{x}{\zwsel{{x}}} : \tarr{\tnum{}}{\tnum{}} &
% \tarr{\tnum}{\tnum} & 
{\aConstruct{\fplus}}
&
\text{(\ref{rule:construct-plus-compat})}
\\ 11 & 
\hlam{x}{\hadd{x}{\zwsel{\hhole{}}}} : \tarr{\tnum{}}{\tnum{}} & 
\aConstruct{\fnumlit{1}} & 
\text{(\ref{r:conenumnum})}
\\ 12 & 
\hlam{x}{\hadd{x}{\zwsel{\hnum{1}}}} : \tarr{\tnum}{\tnum} & 
\textrm{---} & 
{\textrm{---}}
\\ \hline
\end{array}
$\end{center}\vspace{-6px}
\caption{Constructing an increment function in Hazelnut.}
\label{fig:first-example}
\end{figure}

The actions on Lines 5-8 move the cursor to the function body. For simplicity, Hazelnut defines only a small collection of primitive movement actions. In practice, an editor would also define compound movement actions, e.g. an action that moves the cursor directly to the next remaining hole, in terms of these primitive movement actions.

Lines 9-11 complete the function body by first constructing the variable $x$, then  constructing the plus form, and finally constructing the number $\hnum{1}$. Notice that we did not need to construct the function body in an ``outside-in'' manner, i.e. we constructed $x$ before constructing the outer plus form. Luckily, the transient function bodies, $x$ and $x + \hhole{}$, could be checked against type $\tnum$ (as we will detail in Sec. \ref{sec:holes}.)

\subsection{Example 2: Applying the Increment Function}

\begin{figure}[t!]
\begin{center}
\colorbox{light-gray}{\hspace{53px} now assume $incr : \tarr{\tnum}{\tnum}$ \hspace{54px}}
% \vspace{-6px}
$\arraycolsep=4px
\begin{array}{|r||l|l||l|l|}
% \\
\hline
\# & \textbf{Z-Expression} & 
%\textbf{H-Expression} & 
% \textbf{Type} & 
\textbf{Next Action} & \textbf{Rule}
\\
\hline
13 &
\zwsel{\hhole{}} & 
% \hhole{} &
% \tehole 
% &
\aConstruct{\fvar{incr}} \hphantom{~\,\,~~}& 
\text{(\ref{r:conevar})}
\\ 14 & 
\zwsel{incr} & 
\aConstruct{\fap} & 
\text{(\ref{r:coneapfn})}
\\ 15 & 
incr(\zwsel{\hhole{}}) & 
\aConstruct{\fvar{incr}} & 
\text{(\ref{r:conevar2})}
\\ 16 & 
incr(\hhole{\zwsel{incr}}) & 
\aConstruct{\fap} & 
\text{(\ref{r:coneapfn})}
\\ 17 & 
incr(\hhole{incr(\zwsel{\hhole{}})}) \hphantom{~~~~} & 
\aConstruct{\fnumlit{3}} & 
\text{(\ref{r:conenumnum})}
\\ 18 & 
incr(\hhole{incr(\zwsel{\hnum{3}})}) & 
\aMove{\dParent}& 
\text{(\ref{r:moveparent-ap2})}
\\ 19 & 
incr(\hhole{\zwsel{incr(\hnum{3})}}) & 
\aMove{\dParent} & 
\text{(\ref{r:moveparent-hole})}
\\ 20 &
incr(\zwsel{\hhole{incr(\hnum{3})}})& 
\aFinish & 
\text{(\ref{r:finishana})}
\\ 21 & 
incr(\zwsel{incr(\hnum{3})}) & 
\textrm{---} & 
{\textrm{---}}
\\ \hline
\end{array}
$\end{center}\vspace{-6px}
\caption{Appyling the increment function.}
\label{fig:second-example}
\end{figure}

Figure \ref{fig:second-example} shows an edit sequence that constructs the expression $incr(incr(\hnum{3}))$, where $incr$ is assumed bound to the increment function from Figure \ref{fig:first-example}.

We begin on Line 13 by constructing the variable $incr$. Line 14 then constructs the application form with $incr$ in function position, leaving the cursor on a hole in the argument position. Notice that we did not need to construct the outer application form before identifying the function being applied. Again, the transient edit state happens to be well-typed, so we needed to make no special allowances for this order of actions.

We now need to apply $incr$ again, so we perform the same action on Line 15 as we did on Line 13, i.e. $\aConstruct{\fvar{incr}}$. In a syntactic structure editor, performing such an action would result in the following edit state: 
\[
incr(\zwsel{incr})
\]
This edit state is ill-typed (after \emph{cursor erasure}): the argument of $incr$ must be of type $\tnum$ but here it is of type $\tarr{\tnum}{\tnum}$. Hazelnut cannot allow such an edit state to arise. 

The programmer could alternatively have performed the $\aConstruct{\fap}$ action on Line 15. This would result in the following edit state, which is well-typed according to the static semantics that we will define in the next section:
\[
incr(\hhole{}(\zwsel{\hhole{}}))
\]
The problem is that the programmer is not able to identify the intended function before constructing the function application form. This stands in contrast to Lines 13-14. % This is qualitatively unnatural.

Hazelnut's action semantics addresses this problem: rather than disallowing the $\aConstruct{\fvar{incr}}$ action on Line 15, it leaves $incr$ inside a hole:
\[
incr(\hhole{\zwsel{incr}})
\]
This defers the type consistency check, exactly as an empty hole in the same position does. One way to think about non-empty holes is as an internalization of the ``squiggly underline'' that text or syntactic structure editors display to indicate a type inconsistency. By internalizing this concept, the presence of a type inconsistency does not leave the entire program formally meaningless.

The expression inside a non-empty hole must itself be well-typed, so the programmer can continue to edit it. Lines 16-17 proceed to apply the inner mention of $incr$ to a number literal, $\hnum{3}$. Finally, Lines 18-19 move the cursor to the non-empty hole and Line 20 performs the $\aFinish$ action. The $\aFinish$ action removes the hole if the type of the expression inside the hole is consistent with the expected type, as it now is. This results in the final edit state on Line 20, as desired. In practice, the editor might automatically perform the $\aFinish$ action as soon as it becomes possible, but for simplicity, Hazelnut formally requires that it be performed explicitly.

% So far, editing has proceeded in an essentially type-directed, outside-in fashion -- the user first specified the type of the function, then produced a body of that type by the action on Line 8. Lines 10-12 similarly begin in a type-directed manner with the user giving an explicit type ascription, indicating that the expression that they are constructing will have type $\tnum$. 

% However, on Line 13, the user performs the $\aConstruct{\fvar{id}}$ action. Notice that $id$ has type $\tarr{\tnum}{\tnum}$, which is not consistent with the type $\tnum$ given in the ascription. Na\"ively, this would produce a type error, leaving the program in a well-formed but semantically undefined state. One way to avoid this state is to simply not make this action available in the program configuration on Line 12. This is inflexible, forcing an outside-in approach to program construction (i.e. the user would need to construct the function application form before constructing the variable $id$.) Instead, Hazelnut permits this action, but places the variable $id$ inside a hole. This defers the consistency check that would normally occur: a hole can be checked against any type, as long as its contents have some type. The cursor is placed inside the hole. The user then proceeds to apply $id$ to the number expression $\hnum{3}$. At this point, the expression inside the hole has a type consistent with the ascription, so the user can \emph{finish} the hole. In our simple formalism, this requires moving the cursor to the hole (in practice, the system might find the nearest parent of hole form.) The result is the complete, well-typed program shown on Line 19 (notice that \emph{complete} is distinct from \emph{closed} -- the variable $id$ is free on Line 19, so this is not a closed program.)

%% The third column~(\textbf{Next Action}) lists the first user action:
%% Constructing a lambda abstraction using variable~$x$.
%% %
%% The final column~(\textbf{Semantics}) indicates the semantic rule for this
%% action, Rule (\ref{r:conelamhole}), which gives general semantics for
%% introducing lambda terms into holes.
%% %
%% In Section~\ref{sec:hazel}, we list this rule, and the other rules used in
%% this final column. In total, these rules give a formal semantics to the
%% user actions, which relate each line's Z-Expression to the Z-Expression on
%% the subsequent line.

%% In addition to introducing the lambda term, and its variable, the
%% first user action~$\aConstruct{\flam{x}}$ also introduces a type
%% ascription for this function, as an arrow type, with holes for the
%% type of its domain and codomain.
%% %
%% The actions for Lines~2--5 consist of the user filling these holes
%% with the basetype $\tnum{}$.
%% %
%% To do so, the user constructs the type constructor twice (Lines 2 and
%% 4), and navigates between the holes with a move action (Line~3).
%% %
%% Generally, the move action~$\dNext$ moves the focus from one
%% sub-structure to the next sibling sub-structure of the (common) parent
%% structure; in this case, it moves from the domain type of the arrow
%% type to the codomain of the arrow type.
%% %

\section{Hazelnut, Formally}
\label{sec:hazel}
% !TEX root = hazelnut-popl17.tex

The previous section introduced Hazelnut by example. In this section, we systematically introduce the following  structures:
\begin{itemize}[itemsep=0px,partopsep=2px,topsep=2px]
\item \textbf{H-types} and \textbf{H-expressions} (Sec. \ref{sec:holes}), which are types and expressions with {holes}. H-types classify H-expressions according to Hazelnut's \textbf{bidirectional static semantics}.
\item \textbf{Z-types} and \textbf{Z-expressions} (Sec. \ref{sec:cursors}), which superimpose\- a \emph{cursor} onto H-types and H-expressions, respectively (following Huet's \emph{zipper pattern} \cite{JFP::Huet1997}.) Every Z-type (resp. Z-expression) corresponds to an H-type (resp. H-expression) by \emph{cursor erasure}.
\item \textbf{Actions} (Sec. \ref{sec:actions}), which act relative to the cursor according to Hazelnut's \textbf{bidirectional action semantics}. The action semantics enjoys a rich metatheory. Of particular note, the \emph{sensibility theorem} establishes that every edit state is well-typed after cursor erasure.
\end{itemize}

Our overview below omits certain ``uninteresting'' details. The supplement includes the complete collection of rules, in definitional order. These rules, along with the proofs of two key metatheorems, have been mechanized in Agda \cite{norell:thesis}, also in the supplement. We will summarize this effort in Sec. \ref{sec:mech}.% The supplement contains the Agda sources and additional technical details.

\subsection{H-types and H-expressions}\label{sec:holes}
Figure \ref{fig:hexp-syntax} defines the syntax of H-types, $\htau$, and H-expressions, $\hexp$. Most forms correspond directly to those of the simply typed lambda calculus (STLC) extended with a single base type, $\tnum$, of numbers (cf. \cite{pfpl}.) The number expression corresponding to the mathematical number $n$ is drawn $\hnum{n}$, and for simplicity, we define only a single arithmetic operation, $\hadd{\hexp}{\hexp}$. The form $\hexp : \htau$ is an explicit \emph{type ascription}. 

In addition to these standard forms, \emph{type holes} and \emph{empty expression holes} are both drawn $\hehole$ and \emph{non-empty expression holes} are drawn $\hhole{\hexp}$. We do not need non-empty type holes because every H-type is a valid classifier of H-expressions. 

Types and expressions that contain no holes are \emph{complete types} and \emph{complete expressions}, respectively. Formally, we can derive $\hcomplete{\htau}$ when $\htau$ is complete, and $\hcomplete{\hexp}$ when $\hexp$ is complete (see supplement.) We are not concerned here with defining a dynamic semantics for Hazelnut, but the dynamics for complete H-expressions would be entirely standard (in Sec. \ref{sec:rw} we discuss the multi-dimensional design space around the question of evaluating incomplete H-expressions.)% The complete types and expressions correspond immediately to the types and expressions of the STLC with numbers. %Holes mark subterms that are, notionally, ``under construction.'' We will see what this formally corresponds to in a moment.

\begin{figure}[t]
$\arraycolsep=4pt\begin{array}{lllllll}
\mathsf{HTyp} & \htau & ::= &
  \tarr{\htau}{\htau} ~\vert~
  \tnum ~\vert~
  \tehole\\
\mathsf{HExp} & \hexp & ::= &
  \hexp : \htau ~\vert~
  x ~\vert~
  \hlam{x}{\hexp} ~\vert~
  \hap{\hexp}{\hexp} ~\vert~
  \hnum{n} ~\vert~
  \hadd{\hexp}{\hexp} ~\vert~
  \hehole ~\vert~
  \hhole{\hexp}
\end{array}$
%\textbf{Sort} & & & \textbf{Operational Form} & \textbf{Stylized Form} & \textbf{Description}\\
\caption{Syntax of H-types and H-expressions. Metavariable $x$ ranges over variables and $n$ ranges over numerals.}
\label{fig:hexp-syntax}
\end{figure}
\begin{figure}
\noindent\fbox{$\tcompat{\htau}{\htau'}$}~~\text{$\tau$ and $\tau'$ are consistent}
% \begin{subequations}%\label{rules:tcompat}
% \begin{equation}%\label{rule:tcompat-comm}
% \inferrule
% %[TCSym]
% {
%   \tcompat{\htau}{\htau'}
% }{
%   \tcompat{\htau'}{\htau}
% }
% \end{equation}
\begin{subequations}\label{eqns:consistency}
\begin{mathpar}
% \begin{equation}
\inferrule{ }{
  \tcompat{\tehole}{\htau}
}%~\text{(1a)}
% \end{equation}
% \end{minipage}
% \begin{minipage}{0.15\linewidth}
% \begin{equation}%\label{rule:tcompat-hole}
~~~~~~~~
\inferrule{ }{
  \tcompat{\htau}{\tehole}
}%~\text{(1b)}
% \end{equation}
% \end{minipage}
% \begin{minipage}{0.15\linewidth}
% \begin{equation}
~~~~~~~~
\inferrule{ }{
  \tcompat{\htau}{\htau}
}%~\text{(1c)}
% \end{equation}
% \end{minipage}
% \begin{minipage}{0.15\linewidth}
% \begin{equation}
~~~~~~~~
\inferrule{
  \tcompat{\htau_1}{\htau_1'}\\
  \tcompat{\htau_2}{\htau_2'}
}{
  \tcompat{\tarr{\htau_1}{\htau_2}}{\tarr{\htau_1'}{\htau_2'}}
}~~~~~~~~\text{(\ref*{eqns:consistency}a-d)}
% \end{equation}
% \end{minipage}
\end{mathpar}
\end{subequations}
% \begin{equation}%\label{rule:tcompat-num}
% \end{equation}
% \begin{equation}%\label{rule:tcompat-arr}
% \end{equation}
% \end{subequations}
\fbox{$\arrmatch{\htau}{\tarr{\htau_1}{\htau_2}}$}~~\text{$\tau$ has matched arrow type $\tarr{\htau_1}{\htau_2}$}
\begin{subequations}
\begin{minipage}{0.5\linewidth}
\begin{equation}
\inferrule{ }{
  \arrmatch{\tarr{\htau_1}{\htau_2}}{\tarr{\htau_1}{\htau_2}}
}
\end{equation}
\end{minipage}
\begin{minipage}{0.5\linewidth}
\begin{equation}
\inferrule{ }{
  \arrmatch{\tehole}{\tarr{\tehole}{\tehole}}
}
\end{equation}
\end{minipage}
% \end{subequations}
% \end{mathpar}
% \noindent\fbox{$\tincompat{\htau}{\htau'}$}
% \begin{subequations}
%   % \begin{equation}
%   %   \inferrule{
%   %     \tincompat{\htau}{\htau'}
%   %   }{
%   %     \tincompat{\htau'}{\htau}
%   %   }
%   % \end{equation}
%   \begin{equation}
%     \inferrule{ }{
%       \tincompat{\tarr{\htau_1}{\htau_2}}{\tnum}
%     }
%   \end{equation}
%   \begin{equation}
%     \inferrule{ }{
%       \tincompat{\tnum}{\tarr{\htau_1}{\htau_2}}
%     }
%   \end{equation}
%   \begin{equation}
%     \inferrule{
%       \tincompat{\htau_1}{\htau_1'}
%     }{
%       \tincompat{\tarr{\htau_1}{\htau_2}}{\tarr{\htau_1'}{\htau_2'}}
%     }
%   \end{equation}
%   \begin{equation}
%     \inferrule{
%       \tincompat{\htau_2}{\htau_2'}
%     }{
%       \tincompat{\tarr{\htau_1}{\htau_2}}{\tarr{\htau_1'}{\htau_2'}}
%     }
%   \end{equation}
\end{subequations}
\caption{H-type consistency and matched arrow types.}
\label{fig:type-consistency}
\end{figure}

\begin{figure}
\fbox{$\hana{\hGamma}{\hexp}{\htau}$}~~\text{$\hexp$ analyzes against $\htau$}
\begin{subequations}
\begin{equation}\label{rule:ana-subsume}
\inferrule{
  \hsyn{\hGamma}{\hexp}{\htau'}\\
  \tcompat{\htau}{\htau'}
}{
  \hana{\hGamma}{\hexp}{\htau}
}
\end{equation}
\begin{equation}\label{rule:syn-lam}
\inferrule{
  \arrmatch{\htau}{\tarr{\htau_1}{\htau_2}}\\
  \hana{\hGamma, x : \htau_1}{\hexp}{\htau_2}
}{
  \hana{\hGamma}{\hlam{x}{\hexp}}{\htau}
}
\end{equation}
\end{subequations}
\fbox{$\hsyn{\hGamma}{\hexp}{\htau}$}~~\text{$\hexp$ synthesizes $\htau$}
\begin{subequations}
\begin{equation}\label{rule:syn-asc}
\inferrule{
  \hana{\hGamma}{\hexp}{\htau}
}{
  \hsyn{\hGamma}{\hexp : \htau}{\htau}
}
\end{equation}
\begin{equation}\label{rule:syn-var}
\inferrule{ }{
  \hsyn{\hGamma, x : \htau}{x}{\htau}
}
\end{equation}
\begin{equation}\label{rule:syn-ap}
\inferrule{
  \hsyn{\hGamma}{\hexp_1}{\htau}\\
  \arrmatch{\htau}{\tarr{\htau_2}{\htau'}}\\
  \hana{\hGamma}{\hexp_2}{\htau_2}
}{
  \hsyn{\hGamma}{\hap{\hexp_1}{\hexp_2}}{\htau'}
}
\end{equation}
\begin{equation}\label{rule:syn-num}
\inferrule{ }{
  \hsyn{\hGamma}{\hnum{n}}{\tnum}
}
\end{equation}
\begin{equation}\label{rule:syn-plus}
\inferrule{
  \hana{\hGamma}{\hexp_1}{\tnum}\\
  \hana{\hGamma}{\hexp_2}{\tnum}
}{
  \hsyn{\hGamma}{\hadd{\hexp_1}{\hexp_2}}{\tnum}
}
\end{equation}
\begin{equation}\label{rule:syn-ehole}
\inferrule{ }{
  \hsyn{\hGamma}{\hehole}{\tehole}
}
\end{equation}
\begin{equation}\label{rule:syn-hole}
\inferrule{
  \hsyn{\hGamma}{\hexp}{\htau}
}{
  \hsyn{\hGamma}{\hhole{\hexp}}{\tehole}
}
\end{equation}
\end{subequations}
\caption{Analysis and synthesis.}
\label{fig:ana-syn}
\end{figure}

Hazelnut's static semantics is organized as a \emph{bidirectional type system} \cite{Pierce:2000:LTI:345099.345100,DBLP:conf/icfp/DaviesP00,DBLP:conf/tldi/ChlipalaPH05,bidi-tutorial} around the two mutually defined judgements in Figure \ref{fig:ana-syn}. Typing contexts, $\hGamma$, map each variable $x \in \domof{\hGamma}$ to a hypothesis $x : \htau$ in the usual manner. We identify the context up to exchange and adopt the standard identification convention for formal structures that differ only up to alpha-equivalence. 

Derivations of the type analysis judgement, $\hana{\hGamma}{\hexp}{\htau}$, establish that $\hexp$ can appear where an expression of type $\htau$ is expected. Derivations of the type synthesis judgement, $\hsyn{\hGamma}{\hexp}{\htau}$, infer a type from $\hexp$, which is necessary in positions where an expected type is not available (e.g. at the top level.) Algorithmically, the type is an ``input'' of the type analysis judgement, but an ``output'' of the type synthesis judgement. %The rules describe a \emph{local type inference} scheme, i.e. type ascriptions are unnecessary when an expression is being analyzed against a known type.
Making a judgemental distinction between these two notions will be essential in our action semantics (Sec. \ref{sec:actions}.)


 %We use the metavariable $\Gamma$ for \emph{complete typing contexts}, i.e. typing contexts where each hypothesis mentions only complete types.




\begin{subequations}\label{rules:syn-ana}
Type synthesis is stronger than type analysis in that if an expression is able to synthesize a type, it can also be analyzed against that type, or any other \emph{consistent} type, according to the \emph{subsumption rule}, Rule (\ref{rule:ana-subsume}).

The \emph{H-type consistency judgement}, $\tcompat{\htau}{\htau'}$, that appears as a premise in the subsumption rule is a reflexive and symmetric (but not transitive) relation between H-types defined judgementally in Figure \ref{fig:type-consistency}. This relation coincides with equality for complete H-types. Two incomplete H-types are consistent if they differ only at positions where a hole appears in either type. The type hole is therefore consistent with every type. This notion of H-type consistency coincides with the notion of type consistency that Siek and Taha discovered in their foundational work on gradual type systems, if we interpret the type hole as the $?$ (i.e. unknown) type \cite{Siek06a}.

Rule (\ref{rule:syn-lam}) defines analysis for lambda abstractions, $\hlam{x}{\hexp}$. There is no type synthesis rule that applies to this form, so lambda abstractions can appear only in analytic position, i.e. where an expected type is known.\footnote{It is possible to also define a ``half-annotated'' synthetic lambda form, $\lambda x{:}\tau.e$, but for simplicity, we leave it out \cite{DBLP:conf/tldi/ChlipalaPH05}.} Rule (\ref{rule:syn-lam}) is not quite the standard rule, reproduced below:
\begin{equation*}
\inferrule{
  \hana{\hGamma, x : \htau_1}{\hexp}{\htau_2}
}{
  \hana{\hGamma}{\hlam{x}{\hexp}}{\tarr{\htau_1}{\htau_2}}
}
\end{equation*}
The problem is that this standard rule alone leaves us unable to analyze lambda abstractions against the type hole, because the type hole is not immediately of the form $\tarr{\htau_1}{\htau_2}$. There are two plausible solutions to this problem. One solution would be to define a second rule specifically for this case:
\begin{equation*}
\inferrule{
  \hana{\hGamma, x : \tehole}{\hexp}{\tehole}
}{
  \hana{\hGamma}{\hlam{x}{\hexp}}{\tehole}
}
\end{equation*}
Intead, we combine these two possible rules into a single rule through the simple auxiliary \emph{matched arrow type} judgement, $\arrmatch{\htau}{\tarr{\htau_1}{\htau_2}}$, defined in Figure \ref{fig:type-consistency}. This judgement leaves arrow types alone and assigns the type hole the matched arrow type $\tarr{\tehole}{\tehole}$. It is easy to see that the two rules above are admissible by appeal to Rule (\ref{rule:syn-lam}) and the matched arrow type judgement. Encouragingly, this judgement also arises in the study of gradual type systems \cite{DBLP:conf/popl/CiminiS16,DBLP:conf/popl/GarciaC15,DBLP:conf/popl/RastogiCH12}.

Rule (\ref{rule:syn-asc}) defines type synthesis of expressions of ascription form, $\hexp : \htau$. This allows the user to explicitly state  a type for the ascribed expression to be analyzed against.

Rule (\ref{rule:syn-var}) is the standard rule for variables. 

Rule (\ref{rule:syn-ap}) is again nearly the standard rule for function application. It also makes use of the matched function type judgement to combine what would otherwise need to be two rules for function application -- one for when $e_1$ synthesizes an arrow type, and another for when $e_1$ synthesizes the type hole. Indeed, Siek and Taha needed two application rules for the same fundamental reason \cite{Siek06a}. Later work on gradual typing introduced this notion of type matching to resolve this redundancy.

Rule (\ref{rule:syn-num}) states that numbers synthesize the $\tnum$ type. Rule (\ref{rule:syn-plus}) states that $\hexp_1 + \hexp_2$ behaves like a function over numbers. 

The rules described so far are sufficient to type complete H-expressions. The two remaining rules give H-expressions with holes a well-defined static semantics.

Rule (\ref{rule:syn-ehole}) states that the empty expression hole synthesizes the type hole.


A non-empty hole contains an H-expression that is ``under construction'', as described in Sec. \ref{sec:example}. According to Rule (\ref{rule:syn-hole}), this inner  expression must synthesize some type, but, like the empty hole,  non-empty holes synthesize the type hole. 


% \begin{equation}\label{rule:tcompat-comm}
% \inferrule{
%   \tcompat{\htau}{\htau'}
% }{
%   \tcompat{\htau'}{\htau}
% }
% \end{equation}
% \begin{equation}\label{rule:tcompat-num}
% \inferrule{ }{
%   \tcompat{\tnum}{\tnum}
% }
% \end{equation}
% \begin{equation}\label{rule:tcompat-arr}
% \inferrule{
%   \tcompat{\htau_1}{\htau_1'}\\
%   \tcompat{\htau_2}{\htau_2'}
% }{
%   \tcompat{\tarr{\htau_1}{\htau_2}}{\tarr{\htau_1'}{\htau_2'}}
% }
% \end{equation}

Given these rules, it is instructive to derive the following (by subsumption):
\[\hana{incr : \tarr{\tnum}{\tnum}}{\hhole{incr}}{\tnum}\]
% This is the key premise necessary to synthesize the $\tnum$ type for the H-expression that results from cursor erasure -- defined next -- of the Z-expression on Line 16 of Fig. \ref{fig:second-example}. %In other words, this mechanism is essential if  users are to able to construct a program in anything but an ``outside in'' fashion.

% The final rule handles function applications where the expression in function position synthesizes a hole type, rather than an arrow type. We treat it as if it had instead synthesized $\tarr{\tehole}{\tehole}$:
% \begin{equation}\label{rule:syn-ap-2}
% \inferrule{
%   \hsyn{\hGamma}{\hexp_1}{\tehole}\\
%   \hana{\hGamma}{\hexp_2}{\tehole}
% }{
%   \hsyn{\hGamma}{\hap{\hexp_1}{\hexp_2}}{\tehole}
% }
% \end{equation}

% \todo{redundant text w/ citation}
% The hole type behaves much like the type $?$ in prior work by Siek and Taha on gradual types for functional languages \cite{Siek06a}. Their system (which was not bidirectionally typed nor an editor model) also needed to define two rules for function application. In general, when a premise requires that a synthesized type be of a particular form, we need a special case where the synthesized hole type is treated instead as if it were the ``holey-est'' type of that form.\footnote{Alternatively, we might add a rule that allows expressions that synthesize hole type to then non-deterministically synthesize any other type, but maintaining determinism is useful in practice, so we avoid this approach.}

\end{subequations}
\subsection{Z-types and Z-expressions}\label{sec:cursors}
\newcommand{\cvert}{{\,{\vert}\,}}
\begin{figure}[t]
\hspace{-3px}$\arraycolsep=2pt\begin{array}{lllllll}
\mathsf{ZTyp} & \ztau & ::= &
  %\zlsel{\htau} ~\vert~
  \zwsel{\htau} \cvert
  %\zrsel{\htau} ~\vert~
  \tarr{\ztau}{\htau} \cvert
  \tarr{\htau}{\ztau} \\
\mathsf{ZExp} & \zexp & ::= &
  %\zlsel{\hexp} ~\vert~
  \zwsel{\hexp} \cvert
  %\zrsel{\hexp} ~\vert~
  \zexp : \htau \cvert
  \hexp : \ztau \cvert
  \hlam{x}{\zexp} \cvert
  \hap{\zexp}{\hexp} \cvert
  \hap{\hexp}{\zexp} \cvert
  \hadd{\zexp}{\hexp} \cvert
  \hadd{\hexp}{\zexp} \cvert
  \hhole{\zexp}
\end{array}$
%\textbf{Sort} & & & \textbf{Operational Form} & \textbf{Stylized Form} & \textbf{Description}\\
\caption{Syntax of Z-types and Z-expressions.}
\label{fig:zexp-syntax}
\end{figure}

Hazelnut's action semantics operates over Z-types, $\ztau$, and Z-expressions, $\zexp$. Figure \ref{fig:zexp-syntax} defines the syntax of Z-types and Z-expressions. A Z-type (resp. Z-expression) represents an H-type (resp. H-expression) with a single superimposed \emph{cursor}.

The only base cases in these inductive grammars are $\zwsel{\htau}$ and $\zwsel{\hexp}$, which identify the H-type or H-expression that the cursor is on. All of the other forms correspond to the recursive forms in the syntax of H-types and H-expressions, and contain exactly one ``hatted'' subterm that identifies the subtree where the cursor will be found. Any other sub-term is ``dotted'', i.e. it is either an H-type or H-expression. Taken together, every Z-type and Z-expression contains exactly one selected H-type or H-expression by construction. This can be understood as an instance of Huet's \emph{zipper pattern} \cite{JFP::Huet1997} (which, coincidentally, Huet encountered while implementing a structure editor.)

We write $\removeSel{\ztau}$ for the H-type constructed by erasing the cursor from $\ztau$, which we refer to as the \emph{cursor erasure} of $\ztau$. This straightforward metafunction is defined as follows:
\begin{align*}
%\removeSel{(\zlsel{\htau})} & = \htau\\
\removeSel{(\zwsel{\htau})} & = \htau\\
%\removeSel{(\zrsel{\htau})} & = \htau\\
\removeSel{(\tarr{\ztau}{\htau})} & = \tarr{\removeSel{\ztau}}{\htau}\\
\removeSel{(\tarr{\htau}{\ztau})} & = \tarr{\htau}{\removeSel{\ztau}}
\end{align*}

Similarly, we write $\removeSel{\zexp}$ for the H-expression constructed by erasing the cursor from the Z-expression $\zexp$, i.e. the cursor erasure of $\zexp$. The definition of this metafunction is analogous, so we omit it for concision (see supplement.)
% \begin{align*}
% %\removeSel{(\zlsel{\hexp})} & = \hexp\\
% \removeSel{(\zwsel{\hexp})} & = \hexp\\
% %\removeSel{(\zrsel{\hexp})} & = \hexp\\
% \removeSel{(\zexp : \htau)} & = \removeSel{\zexp} : \htau\\
% \removeSel{(\hexp : \ztau)} & = \hexp : \removeSel{\ztau}\\
% \removeSel{(\hlam{x}{\zexp})} & = \hlam{x}{\removeSel{\zexp}}\\ogoo
% \removeSel{(\hap{\zexp}{\hexp})} & = \hap{\removeSel{\zexp}}{\hexp}\\
% \removeSel{(\hap{\hexp}{\zexp})} & = \hap{\hexp}{\removeSel{\zexp}}\\
% \removeSel{(\hadd{\zexp}{\hexp})} & = \hadd{\removeSel{\zexp}}{\hexp}\\
% \removeSel{(\hadd{\hexp}{\zexp})} & = \hadd{\hexp}{\removeSel{\zexp}}\\
% \removeSel{\hhole{\zexp}} &= \hhole{\removeSel{\zexp}}
% \end{align*}

\subsection{Actions}\label{sec:actions}


We now arrive at the heart of Hazelnut: its \emph{bidirectional action semantics}.  
Figure \ref{fig:action-syntax} defines the syntax of \emph{actions}, $\alpha$, some of which involve \emph{directions}, $\delta$, and \emph{shapes}, $\psi$. 

Expression actions are governed by two mutually defined judgements, 1) the \emph{synthetic action judgement}:
\[
\performSyn{\hGamma}{\zexp}{\htau}{\alpha}{\zexp'}{\htau'}
\]
and 2) \emph{the analytic action judgement}:
\[
\performAna{\hGamma}{\zexp}{\htau}{\alpha}{\zexp'}
\]

In some Z-expressions, the cursor is in a type ascription, so we also need a \emph{type action judgement}, pronounced ``performing $\alpha$ on $\ztau$ results in $\ztau'$'':
\[
\performTyp{\ztau}{\alpha}{\ztau'}
\]


% The type action judgement appears as a premise in the rules that handle Z-expressions where the cursor is inside a . 
% \[\begin{array}{ll}
% %\textbf{Judgement Form} & \textbf{Description}\\
% & 
% \end{array}\]
% \[
% \begin{array}{l}
% \fbox{$\performTyp{\ztau}{\alpha}{\ztau'}$} ~~ \text{Performing $\alpha$ on $\ztau$ produces $\ztau'$}\\
% \fbox{$\performSyn{\hGamma}{\zexp}{\htau}{\alpha}{\zexp'}{\htau'}$} \\
% \quad \text{Performing $\alpha$ on $\zexp$ when $\removeSel{\zexp}$ synthesizes type $\htau$}\\
% \quad \text{produces $\zexp'$ such that $\removeSel{\zexp'}$ synthesizes type $\htau'$}\\
% \fbox{$\performAna{\hGamma}{\zexp}{\htau}{\alpha}{\zexp'}$} \\
% \quad \text{Performing $\alpha$ on $\zexp$ when analyzing $\removeSel{\zexp}$ against $\htau$}\\
% \quad \text{produces $\zexp'$, such that $\removeSel{\zexp'}$ can also be analyzed}\\
% \quad \text{against $\htau$}
% \end{array}\]
\subsubsection{Sensibility}


Before giving the rules for these judgements, let us state the key metatheorem: \emph{sensibility}. This metatheorem deeply informs the design of the rules, given starting in Sec. \ref{sec:action-subsumption}. Its proof, which is mechanized in Agda, is by straightforward induction, so the reader is encouraged to think about the relevant cases of these proofs as we present the rules. 
\begin{theorem}[Action Sensibility]\label{thrm:actsafe} Both of the following hold:\begin{enumerate}[itemsep=0px,partopsep=0px,topsep=0px]
\item If $\hsyn{\hGamma}{\removeSel{\zexp}}{\htau}$ and $\performSyn{\hGamma}{\zexp}{\htau}{\alpha}{\zexp'}{\htau'}$ 
   then
  $\hsyn{\hGamma}{\removeSel{\zexp'}}{\htau'}$.
\item If $\hana{\hGamma}{\removeSel{\zexp}}{\htau}$ and
   $\performAna{\hGamma}{\zexp}{\htau}{\alpha}{\zexp'}$ then
  $\hana{\hGamma}{\removeSel{\zexp'}}{\htau}$.
\end{enumerate}
\end{theorem}
\noindent In other words, if an edit state (i.e. a Z-expression) is statically meaningful, i.e. its cursor erasure is well-typed, then performing an action on it leaves the resulting edit state statically meaningful. In particular, the first clause of Theorem \ref{thrm:actsafe} establishes that when an action is performed on an edit state whose cursor erasure synthesizes an H-type, the result is an edit state whose cursor erasure also synthesizes some (possibly different) H-type. The second clause establishes that when an action is performed using the analytic action judgement on an edit state whose cursor erasure analyzes against some H-type, the result is a Z-expression whose cursor erasure also analyzes against the same H-type. % Non-empty holes allow us to avoid top-down program construction becau but rather can construct fragments of the program inside a hole until ready to ``expose'' them to type analysis.


\begin{figure}[t]
\hspace{-3px}$\arraycolsep=3pt\begin{array}{llcllll}
\mathsf{Action} & \alpha & ::= &
  \aMove{\delta} ~\vert~
  %\aSelect{\delta} ~\vert~
  \aDel ~\vert~
  %\aReplace{\htau} ~\vert~
  %\aReplace{\hexp} ~\vert~
  \aConstruct{\psi} ~\vert~
  \aFinish\\
\mathsf{Dir} & \delta & ::= &
  \dChild ~\vert~
  \dParent ~\vert~
  \dNext \\ % ~\vert~
  % \dPrev\\
\mathsf{Shape} & \psi & ::= &
  \farr ~\vert~
  \fnum \\
& & \vert &
  \fasc ~\vert~
  \fvar{x} ~\vert~
  \flam{x} ~\vert~
  \fap ~\vert~
  \farg ~\vert~
  \fnumlit{n} ~\vert~
  \fplus\\
& & \vert & 
  {\color{gray}\fnehole}
\end{array}$
%\textbf{Sort} & & & \textbf{Operational Form} & \textbf{Stylized Form} & \textbf{Description}\\
\caption{Syntax of actions.}
\label{fig:action-syntax}
\end{figure}

% \subsubsection{Determinism}
% Another useful property that we maintain is \emph{action determinism}, i.e. that performing an action should produce a unique result. Formally, this is established as follows:
% \begin{theorem}[Action Determinism] All of the following hold:
% \label{thrm:actdet}
% \begin{enumerate}[itemsep=0px,partopsep=0px,topsep=0px]
% \item If $\performTyp{\ztau}{\alpha}{\ztau'}$ and $\performTyp{\ztau}{\alpha}{\ztau''}$ then $\ztau'=\ztau''$.
% \item If $\hsyn{\hGamma}{\removeSel{\zexp}}{\htau}$ and
%   $\performSyn{\hGamma}{\zexp}{\htau}{\alpha}{\zexp'}{\htau'}$ and
%   $\performSyn{\hGamma}{\zexp}{\htau}{\alpha}{\zexp''}{\htau''}$ then
%   $\zexp' = \zexp''$ and $\htau' = \htau''$.
% % \item If all of

%   \begin{quote}
%     \begin{enumerate}
%     \item $\hsyn{\hGamma}{\removeSel{\zexp}}{\htau}$, and
%     \item $\performSyn{\hGamma}{\zexp}{\htau}{\alpha}{\zexp'}{\htau'}$, and
%     \item $\tcompat{\htau}{\htau'}$, and
%     \item either $\performAna{\hGamma}{\zexp}{\htau}{\alpha}{\zexp''}$ or
%       $\performAna{\hGamma}{\zexp}{\htau'}{\alpha}{\zexp''}$
%     \end{enumerate}
%   \end{quote}
%   hold, then $\zexp' = \zexp''$.
% % \item If $\hana{\hGamma}{\removeSel{\zexp}}{\htau}$ and
% %   $\performAna{\hGamma}{\zexp}{\htau}{\alpha}{\zexp'}$ and
% %   $\performAna{\hGamma}{\zexp}{\htau}{\alpha}{\zexp''}$ then $\zexp' =
% %   \zexp''$.
% % \end{enumerate}
% \end{theorem}

\subsubsection{Type Inconsistency}
In some of the rules below, we will need to supplement our definition of type consistency from Figure \ref{fig:type-consistency} with a definition of \emph{type inconsistency}, written $\tincompat{\htau}{\htau'}$. One can define this notion either directly as the metatheoretic negation of type consistency, or as a separate inductively defined judgement. The supplement does the latter. The key rule establishes that arrow types are inconsistent with $\tnum$:
% \begin{subequations}
  % \begin{equation}
  %   \inferrule{
  %     \tincompat{\htau}{\htau'}
  %   }{
  %     \tincompat{\htau'}{\htau}
  %   }
  % \end{equation}
  \begin{equation*}
    \inferrule{ }{
      \tincompat{\tnum}{\tarr{\htau_1}{\htau_2}}
    }
  \end{equation*}
  % \begin{equation}
  %   \inferrule{
  %     \tincompat{\htau_1}{\htau_1'}
  %   }{
  %     \tincompat{\tarr{\htau_1}{\htau_2}}{\tarr{\htau_1'}{\htau_2'}}
  %   }
  % \end{equation}
  % \begin{equation}
  %   \inferrule{
  %     \tincompat{\htau_2}{\htau_2'}
  %   }{
  %     \tincompat{\tarr{\htau_1}{\htau_2}}{\tarr{\htau_1'}{\htau_2'}}
  %   }
  % \end{equation}
% \end{subequations}
% The remaining rules, given in the appendix, establish that type incompatibility is symmetric.
The mechanization proves that this judgemental definition of type inconsistency is indeed the negation of type consistency.


\subsubsection{Action Subsumption}\label{sec:action-subsumption}

The action semantics includes a subsumption rule similar to the subsumption rule, Rule (\ref{rule:ana-subsume}), in the statics:
\begin{equation}\label{rule:action-subsume}
  \inferrule{
    \hsyn{\hGamma}{\removeSel{\zexp}}{\htau'}\\
    \performSyn{\hGamma}{\zexp}{\htau'}{\alpha}{\zexp'}{\htau''}\\
    \tcompat{\htau}{\htau''}%\\\\
    % \zexp, \alpha~\mathsf{subsumes}
  }{
    \performAna{\hGamma}{\zexp}{\htau}{\alpha}{\zexp'}
  }
\end{equation}
In other words, if the cursor erasure of the edit state synthesizes a type, then we defer to the synthetic action judgement, as long as the type of the resulting cursor erasure is consistent with the type provided for analysis.  The case involving Rule (\ref{rule:action-subsume}) in the proof of Theorem \ref{thrm:actsafe} goes through by induction and static subsumption, i.e. Rule (\ref{rule:ana-subsume}). Algorithmically, subsumption should be the rule of last resort.

% We need to make some exceptions in this rule to maintain determinism, because certain construction actions, described in Sec. \ref{sec:construction}, behave slightly differently depending on whether a type is known. We do so by defining a simple judgement $\alpha~\mathsf{subsumes}$ that excludes these action forms:\todo{put into supp}
% \begin{subequations}
% \begin{equation}\label{rule:subsumes}
% \inferrule{
%   \alpha \neq \aConstruct{\fasc}\\
%   (\forall x) \alpha \neq \aConstruct{\flam{x}} 
% }{
%   \zexp, \alpha~\mathsf{subsumes}
% }
% \end{equation}
% \begin{equation}\label{rule:subsumes}
% \inferrule{
%   \zexp \neq \zwsel{\hexp}
% }{
%   \zexp, \alpha~\mathsf{subsumes}
% }
% \end{equation}
% \end{subequations}
% Algorithmically, this captures the notion that subsumption should be the case of last resort.

\subsubsection{Relative Movement}\label{sec:movement} 
The rules below define relative movement within Z-types. They should be self-explanatory:
\begin{subequations}
\begin{equation}
  \inferrule{ }{
    \performTyp{
      \zwsel{\tarr{\htau_1}{\htau_2}}
    }{
      \aMove{\dChild}
    }{
      \tarr{\zwsel{\htau_1}}{\htau_2}
    }
  }
\end{equation}
\begin{equation}
  \inferrule{ }{
    \performTyp{
      \tarr{\zwsel{\htau_1}}{\htau_2}
    }{
      \aMove{\dParent}
    }{
      \zwsel{\tarr{\htau_1}{\htau_2}}
    }
  }
\end{equation}
\begin{equation}\label{rule:move-parent-arr-right}
  \inferrule{ }{
    \performTyp{
      \tarr{{\htau_1}}{\zwsel{\htau_2}}
    }{
      \aMove{\dParent}
    }{
      \zwsel{\tarr{\htau_1}{\htau_2}}
    }
  }
\end{equation}
\begin{equation}\label{rule:move-nextSib-arr}
  \inferrule{ }{
    \performTyp{
      \tarr{\zwsel{\htau_1}}{{\htau_2}}
    }{
      \aMove{\dNext}
    }{
      {\tarr{\htau_1}{\zwsel{\htau_2}}}
    }
  }
\end{equation}
% \begin{equation}
%   \inferrule{ }{
%     \performTyp{
%       \tarr{{\htau_1}}{\zwsel{\htau_2}}
%     }{
%       \aMove{\dPrev}
%     }{
%       {\tarr{\zwsel{\htau_1}}{{\htau_2}}}
%     }
%   }
% \end{equation}
% \begin{equation}
% \inferrule{
%   \performTyp{
%     \ztau
%   }{
%     \aMove{\delta}
%   }{
%     \ztau'
%   }
% }{
%   \performTyp{
%     \tarr{\ztau}{\htau}
%   }{
%     \aMove{\delta}
%   }{
%     \tarr{\ztau'}{\htau}
%   }
% }
% \end{equation}
% \begin{equation}
%   \inferrule{
%     \performTyp{
%       \ztau
%     }{
%       \aMove{\delta}
%     }{
%       \ztau'
%     }
%   }{
%     \performTyp{
%       \tarr{\htau}{\ztau}
%     }{
%       \aMove{\delta}
%     }{
%       \tarr{\htau}{\ztau}
%     }
%   }
% \end{equation}
\end{subequations}
Two more rules are needed to recurse into the zipper structure. We define these zipper rules in an action-independent manner in Sec. \ref{sec:zipper-cases}. 
% The final two rules above recurse into the zipper structure.

The rules for relative movement within Z-expressions are similarly straightforward. Movement is type-independent, so we defer to an auxiliary expression movement judgement in both the analytic and synthetic case:
\begin{subequations}
\begin{equation}
\inferrule{
  \performMove{\zexp}{\aMove{\delta}}{\zexp'}
}{
  \performSyn{\hGamma}{\zexp}{\htau}{\aMove{\delta}}{\zexp'}{\htau}
}
\end{equation}
\begin{equation}
  \inferrule{
  \performMove{\zexp}{\aMove{\delta}}{\zexp'}
}{
  \performAna{\hGamma}{\zexp}{\htau}{\aMove{\delta}}{\zexp'}
}
\end{equation}
\end{subequations}
% \begin{figure}
% \judgbox{\performMove{\zexp}{\aMove{\delta}}{\zexp'}}{(Additional movement-action rules:)}
% \begin{displaymath}
% \begin{array}{@{}rcl}
%   %\multicolumn{3}{l}{\textbf{Movement Actions for Sum Type Forms}:}
%   %\\
%   \TABperformMove
%       {\zwsel{\hexp : \htau}}
%       {\aMove{\dChild}}
%       {      \zwsel{\hexp} : \htau}
%   \\
%   \TABperformMove
%       {
%       \zwsel{\hexp} : \htau
%     }{
%       \aMove{\dParent}
%     }{
%       \zwsel{\hexp : \htau}
%     }
%   \\
%   \TABperformMove
%     {
%       \hexp : \zwsel{\htau}
%     }{
%       \aMove{\dParent}
%     }{
%       \zwsel{\hexp : \htau}
%     }
%   \\
%   \TABperformMove
%       {
%       \zwsel{\hexp} : \htau
%     }{
%       \aMove{\dNext}
%     }{
%       \hexp : \zwsel{\htau}
%     }
%   \\[2mm]
%   \TABperformMove
%       {      {\tsum{{\htau_1}}{{\zwsel{\htau_2}}}}}
%       {\aMove{\dParent}}
%       {\zwsel{\tsum{{\htau_1}}{\htau_2}}}
%   \\[2mm]
%   \TABperformMove
%       {\zwsel{\hinj{i}{\hexp}}}
%       {\aMove{\dChild}}
%       {\hinj{i}{\zwsel{\hexp}}}      
%   \\
%   \TABperformMove
%       {\hinj{i}{\zwsel{\hexp}}}      
%       {\aMove{\dParent}}
%       {\zwsel{\hinj{i}{\hexp}}}
%   \\[2mm]
%   \TABperformMove
%       {\zwsel{\hcase{\hexp}{x}{\hexp_1}{y}{\hexp_2}}}
%       {\aMove{\dChild}}
%       {      {\hcase{\zwsel{\hexp}}{x}{\hexp_1}{y}{\hexp_2}}}
%   \\
%   \TABperformMove
%       {      {\hcase{\zwsel{\hexp}}{x}{\hexp_1}{y}{\hexp_2}}}
%       {\aMove{\dNext}}
%       {      {\hcase{{\hexp}}{x}{\zwsel{\hexp_1}}{y}{\hexp_2}}}
%   \\
%   \TABperformMove
%       {      {\hcase{{\hexp}}{x}{\zwsel{\hexp_1}}{y}{\hexp_2}}}
%       {\aMove{\dNext}}
%       {      {\hcase{{\hexp}}{x}{{\hexp_1}}{y}{\zwsel{\hexp_2}}}}
%   \\
%   \TABperformMove
%       {      {\hcase{{\hexp}}{x}{{\hexp_1}}{y}{\zwsel{\hexp_2}}}}
%       {\aMove{\dPrev}}
%       {      {\hcase{{\hexp}}{x}{\zwsel{\hexp_1}}{y}{\hexp_2}}}
%   \\
%   \TABperformMove
%       {      {\hcase{{\hexp}}{x}{\zwsel{\hexp_1}}{y}{\hexp_2}}}
%       {\aMove{\dPrev}}
%       {      {\hcase{{\zwsel{\hexp}}}{x}{{\hexp_1}}{y}{{\hexp_2}}}}
%   \\
%   \TABperformMove
%       {      {\hcase{\zwsel{\hexp}}{x}{\hexp_1}{y}{\hexp_2}}}
%       {\aMove{\dParent}}
%       {\zwsel{\hcase{{\hexp}}{x}{\hexp_1}{y}{\hexp_2}}}
%   \\
%   \TABperformMove
%       {      {\hcase{{\hexp}}{x}{\zwsel{\hexp_1}}{y}{\hexp_2}}}
%       {\aMove{\dParent}}
%       {\zwsel{\hcase{{\hexp}}{x}{\hexp_1}{y}{\hexp_2}}}
%   \\
%   \TABperformMove
%       {      {\hcase{{\hexp}}{x}{\hexp_1}{y}{\zwsel{\hexp_2}}}}
%       {\aMove{\dParent}}
%       {\zwsel{\hcase{{\hexp}}{x}{\hexp_1}{y}{\hexp_2}}}
% \end{array}
% \end{displaymath}
% \caption{Movement action semantics for sum type forms.}
% \end{figure}
The expression movement judgement is defined as follows.

\paragraph{Ascription}
\begin{subequations}
  \begin{equation}
    \label{r:movefirstchild}
  \inferrule{ }{
    \performTyp{
      \zwsel{\hexp : \htau}
    }{
      \aMove{\dChild}
    }{
      \zwsel{\hexp} : \htau
    }
  }
\end{equation}
\begin{equation}
  \label{r:moveparent}
  \inferrule{ }{
    \performTyp{
      \zwsel{\hexp} : \htau
    }{
      \aMove{\dParent}
    }{
      \zwsel{\hexp : \htau}
    }
  }
\end{equation}
\begin{equation}\label{rule:move-parent-asc-right}
  \inferrule{ }{
    \performTyp{
      \hexp : \zwsel{\htau}
    }{
      \aMove{\dParent}
    }{
      \zwsel{\hexp : \htau}
    }
  }
\end{equation}
\begin{equation}
  \label{r:movenextsib}
  \inferrule{ }{
    \performTyp{
      \zwsel{\hexp} : \htau
    }{
      \aMove{\dNext}
    }{
      \hexp : \zwsel{\htau}
    }
  }
\end{equation}
% \begin{equation}
%   \label{r:moveprevsib}
%   \inferrule{ }{
%     \performTyp{
%       \hexp : \zwsel{\htau}
%     }{
%       \aMove{\dPrev}
%     }{
%       \zwsel{\hexp} : \htau
%     }
%   }
% \end{equation}
% \begin{equation}
% \inferrule{
%   \performTyp{
%     \zexp
%   }{
%     \aMove{\delta}
%   }{
%     \zexp'
%   }
% }{
%   \performTyp{
%     \zexp : \htau
%   }{
%     \aMove{\delta}
%   }{
%     \zexp' : \htau
%   }
% }
% \end{equation}
% \begin{equation}
%   \inferrule{
%     \performTyp{
%       \ztau
%     }{
%       \aMove{\delta}
%     }{
%       \ztau'
%     }
%   }{
%     \performTyp{
%       \hexp : \ztau
%     }{
%       \aMove{\delta}
%     }{
%       \hexp : \ztau'
%     }
%   }
% \end{equation}


\paragraph{Lambda}\vspace{-3px}
\begin{equation}\label{r:movefirstchild-lam}
\inferrule{ }{
  \performMove{
    \zwsel{\hlam{x}{\hexp}}
  }{
    \aMove{\dChild}
  }{
    \hlam{x}{\zwsel{\hexp}}
  }
}
\end{equation}
\begin{equation}
  \inferrule{ }{
    \performMove{
      \hlam{x}{\zwsel{\hexp}}
    }{
      \aMove{\dParent}
    }{
      \zwsel{\hlam{x}{\hexp}}
    }
  }
\end{equation}
\paragraph{Application}\vspace{-5px}
\begin{equation}
  \inferrule{ }{
    \performMove{
      \zwsel{\hap{\hexp_1}{\hexp_2}}
    }{
      \aMove{\dChild}
    }{
      \hap{\zwsel{\hexp_1}}{\hexp_2}
    }
  }
\end{equation}
\begin{equation}
  \inferrule{ }{
    \performMove{
      \hap{\zwsel{\hexp_1}}{\hexp_2}
    }{
      \aMove{\dParent}
    }{
      \zwsel{\hap{\hexp_1}{\hexp_2}}
    }
  }
\end{equation}
\begin{equation}\label{r:moveparent-ap2}
  \inferrule{ }{
    \performMove{
      \hap{{\hexp_1}}{\zwsel{\hexp_2}}
    }{
      \aMove{\dParent}
    }{
      \zwsel{\hap{\hexp_1}{\hexp_2}}
    }
  }
\end{equation}
\begin{equation}
  \inferrule{ }{
    \performMove{
      \hap{\zwsel{\hexp_1}}{\hexp_2}
    }{
      \aMove{\dNext}
    }{
      \hap{\hexp_1}{\zwsel{\hexp_2}}
    }
  }
\end{equation}
% \begin{equation}
%   \inferrule{ }{
%     \performMove{
%       \hap{\hexp_1}{\zwsel{\hexp_2}}
%     }{
%       \aMove{\dPrev}
%     }{
%       \hap{\zwsel{\hexp_1}}{\hexp_2}
%     }
%   }
% \end{equation}

\paragraph{Plus}
\begin{equation}
  \inferrule{ }{
    \performMove{
      \zwsel{\hadd{\hexp_1}{\hexp_2}}
    }{
      \aMove{\dChild}
    }{
      \hadd{\zwsel{\hexp_1}}{\hexp_2}
    }
  }
\end{equation}
\begin{equation}
  \inferrule{ }{
    \performMove{
      \hadd{\zwsel{\hexp_1}}{\hexp_2}
    }{
      \aMove{\dParent}
    }{
      \zwsel{\hadd{\hexp_1}{\hexp_2}}
    }
  }
\end{equation}
\begin{equation}
  \inferrule{ }{
    \performMove{
      \hadd{{\hexp_1}}{\zwsel{\hexp_2}}
    }{
      \aMove{\dParent}
    }{
      \zwsel{\hadd{\hexp_1}{\hexp_2}}
    }
  }
\end{equation}
\begin{equation}
  \inferrule{ }{
    \performMove{
      \hadd{\zwsel{\hexp_1}}{\hexp_2}
    }{
      \aMove{\dNext}
    }{
      \hadd{\hexp_1}{\zwsel{\hexp_2}}
    }
  }
\end{equation}
% \begin{equation}
%   \inferrule{ }{
%     \performMove{
%       \hadd{\hexp_1}{\zwsel{\hexp_2}}
%     }{
%       \aMove{\dPrev}
%     }{
%       \hadd{\zwsel{\hexp_1}}{\hexp_2}
%     }
%   }
% \end{equation}

\paragraph{Non-Empty Hole}
\begin{equation}
\inferrule{ }{
  \performMove{
    \zwsel{\hhole{\hexp}}
  }{
    \aMove{\dChild}
  }{
    \hhole{\zwsel{\hexp}}
  }
}
\end{equation}
\begin{equation}\label{r:moveparent-hole}
  \inferrule{ }{
    \performMove{
      \hhole{\zwsel{\hexp}}
    }{
      \aMove{\dParent}
    }{
      \zwsel{\hhole{\hexp}}
    }
  }
\end{equation}
Again, additional rules are needed to recurse into the zipper structure, but we will define these zipper rules in an action-independent manner in Sec. \ref{sec:zipper-cases}. 
\end{subequations}

The rules above are numerous and fairly uninteresting. That makes them quite hazardous -- we might make a mistake absent-mindedly. One check against this is to prove a lemma that establishes that movement actions move the cursor but do not change the cursor erasure.

\begin{theorem}[Movement Erasure Invariance]\label{lemma:movement-erasure} ~
\begin{enumerate}[itemsep=0px,partopsep=0px,topsep=0px] 
\item If $\performMove{\ztau}{\aMove{\delta}}{\ztau'}$ then $\removeSel{\ztau}=\removeSel{\ztau'}$.
\item If $\performMove{\zexp}{\aMove{\delta}}{\zexp'}$ then $\removeSel{\zexp}=\removeSel{\zexp'}$.
\end{enumerate}
\end{theorem}
\noindent Theorem \ref{lemma:movement-erasure} is useful also in that the relevant cases of Theorem \ref{thrm:actsafe} are straightforward by its application.

Another useful check is to establish \emph{reachability}, i.e. that it is possible, through a sequence of movement actions, to move the cursor from any position to any other position within a well-typed H-expression. 

This requires developing machinery for reasoning about sequences of actions. There are two possibilities: we can either add a sequencing action, $\alpha; \alpha$, directly to the syntax of actions, or we can define a syntax for lists of actions, $\bar{\alpha}$, together with iterated action judgements. To keep the core of the action semantics small, we take the latter approach in Figure \ref{fig:multistep}.


A simple auxiliary judgement, $\bar\alpha~\mathsf{movements}$, defined in the supplement, establishes that $\bar\alpha$ consists only of actions of the form $\aMove{\delta}$.

With these definitions, we can state reachability as follows:

\begin{theorem}[Reachability]\label{thrm:reachability} ~
\begin{enumerate}[itemsep=0px,partopsep=0px,topsep=0px]
\item If $\removeSel{\ztau}=\removeSel{\ztau'}$ then there exists some $\bar\alpha$ such that $\bar{\alpha}~\mathsf{movements}$ and $\performTypI{\ztau}{\bar\alpha}{\ztau'}$.
\item If $\hana{\hGamma}{\removeSel{\zexp}}{\htau}$ and $\removeSel{\zexp}=\removeSel{\zexp'}$ then there exists some $\bar{\alpha}$ such that $\bar{\alpha}~\mathsf{movements}$ and $\performAnaI{\hGamma}{\zexp}{\htau}{\bar{\alpha}}{\zexp'}$. 
\item If $\hsyn{\hGamma}{\removeSel{\zexp}}{\htau}$ and $\removeSel{\zexp}=\removeSel{\zexp'}$ then there exists some $\bar{\alpha}$ such that $\bar{\alpha}~\mathsf{movements}$ and $\performSynI{\hGamma}{\zexp}{\htau}{\bar\alpha}{\zexp'}{\htau}$.
\end{enumerate}
\end{theorem}


\begin{figure}
$\mathsf{ActionList}$~~$\bar{\alpha} ::= \cdot ~\vert~ \alpha; \bar{\alpha}$\vspace{4px}\\
\fbox{$\performTypI{\ztau}{\bar{\alpha}}{\ztau'}$}

\vspace{-10px}\begin{subequations}
\begin{minipage}{0.35\linewidth}
\begin{equation}
\inferrule{ }{
    \performTypI{\ztau}{\cdot}{\ztau}
}
\end{equation}
\end{minipage}
\begin{minipage}{0.65\linewidth}
\begin{equation}
\inferrule{
  \performTyp{\ztau}{\alpha}{\ztau'}\\
  \performTypI{\ztau'}{\bar{\alpha}}{\ztau''}
}{
  \performTypI{\ztau}{\alpha; \bar{\alpha}}{\ztau''}
}
\end{equation}
\end{minipage}
\end{subequations}

\fbox{$\performSynI{\hGamma}{\zexp}{\htau}{\bar{\alpha}}{\zexp'}{\htau'}$}
\vspace{-10px}

\begin{subequations}\label{rules:iterated-syn}
\hspace{-12px}
% \begin{minipage}{0.88\linewidth}
\begin{mathpar}
\inferrule{ }{
  \performSynI{\hGamma}{\zexp}{\htau}{\cdot}{\zexp}{\htau}
}
~~~~~~
\inferrule{
  \performSyn{\hGamma}{\zexp}{\htau}{\alpha}{\zexp'}{\htau'}\\\\
  \performSynI{\hGamma}{\zexp'}{\htau'}{\bar{\alpha}}{\zexp''}{\htau''}
}{
  \performSynI{\hGamma}{\zexp}{\htau}{\alpha; \bar{\alpha}}{\zexp''}{\htau''}
}
~~~~~
\text{(\ref*{rules:iterated-syn}a-b)}
\end{mathpar}
% \end{minipage}
% \begin{minipage}{0.12\linewidth}
% (11a-b)
% \end{minipage}
\end{subequations}

\fbox{$\performAna{\hGamma}{\zexp}{\htau}{\bar{\alpha}}{\zexp'}$}
\vspace{-12px}
\begin{subequations}\label{rules:iterated-ana}
\begin{mathpar}
~~~~~~~\inferrule{ }{
  \performAnaI{\hGamma}{\zexp}{\htau}{\cdot}{\zexp}
}
~~~~~~~~~~~~~~
\inferrule{
  \performAna{\hGamma}{\zexp}{\htau}{\alpha}{\zexp'}\\\\
  \performAnaI{\hGamma}{\zexp'}{\htau}{\bar\alpha}{\zexp''}
}{
  \performAnaI{\hGamma}{\zexp}{\htau}{\alpha; \bar\alpha}{\zexp''}
}
~~~~~~~~~~
\text{(\ref*{rules:iterated-ana}a-b)}
\end{mathpar}
\end{subequations}
\caption{Iterated Action Judgements}
\label{fig:multistep}
\end{figure}

The simplest way to prove Theorem \ref{thrm:reachability} is to  break it into two lemmas. Lemma \ref{lemma:reach-up} establishes that you can always move the cursor to the outermost position in an expression. This serves as a check on our $\aMove{\dParent}$ rules.
\begin{lemma}[Reach Up]\label{lemma:reach-up} ~
\begin{enumerate}[itemsep=0px,partopsep=0px,topsep=0px]
\item If $\removeSel{\ztau}=\htau$ then there exists some $\bar\alpha$ such that $\bar\alpha~\mathsf{movements}$ and $\performTypI{\ztau}{\bar\alpha}{\zwsel{\htau}}$.
\item If $\hana{\hGamma}{\hexp}{\htau}$ and $\removeSel{\zexp}=\hexp$ then there exists some $\bar\alpha$ such that $\bar\alpha~\mathsf{movements}$ and $\performAnaI{\hGamma}{\zexp}{\htau}{\bar\alpha}{\zwsel{\hexp}}$. 
\item If $\hsyn{\hGamma}{\hexp}{\htau}$ and $\removeSel{\zexp}=\hexp$ then there exists some $\bar\alpha$ such that $\bar\alpha~\mathsf{movements}$ and $\performSynI{\hGamma}{\zexp}{\htau}{\bar\alpha}{\zwsel{\hexp}}{\htau}$.
\end{enumerate}
\end{lemma}
Lemma \ref{lemma:reach-down} establishes that you can always move the cursor from the outermost position to any other position. This serves as a check on our $\aMove{\dChild}$ and $\aMove{\dNext}$ rules. 
\begin{lemma}[Reach Down]\label{lemma:reach-down} ~
\begin{enumerate}[itemsep=0px,partopsep=0px,topsep=0px]
\item If $\removeSel{\ztau}=\htau$ then there exists some $\bar\alpha$ such that $\bar\alpha~\mathsf{movements}$ and $\performTypI{\zwsel{\htau}}{\bar\alpha}{\ztau}$.
\item If $\hana{\hGamma}{\hexp}{\htau}$ and $\removeSel{\zexp}=\hexp$ then there exists some $\bar\alpha$ such that $\bar\alpha~\mathsf{movements}$ and $\performAnaI{\hGamma}{\zwsel{\hexp}}{\htau}{\bar\alpha}{\zexp}$.
\item If $\hsyn{\hGamma}{\hexp}{\htau}$ and $\removeSel{\zexp}=\hexp$ then there exists some $\bar\alpha$ such that $\bar\alpha~\mathsf{movements}$ nand $\performSynI{\hGamma}{\zwsel{\hexp}}{\htau}{\bar\alpha}{\zexp}{\htau}$.
\end{enumerate}
\end{lemma}
Theorem \ref{thrm:reachability} follows by straightforward composition of these two lemmas. %(The witness to the existential is not, of course, the shortest path between any two cursor locations.)

% \paragraph{Reachability}

% We give the following theorem to demonstrate that the rules for movement do
% indeed capture the process of moving the focus to any editable position
% within a term. Intuitively, given two zippered expressions that are the
% same up to the erasure of focus, its proof gives a list of composable
% actions that, when applied to the first term produces the second.

% As a consequence of applying Theorem \ref{thrm:actsafe} inductively at
% every action in such a list, it is possible consider moving to an ill-typed
% term. Therefore the statement here matches the mutually recursive structure
% of the bidirectional typing judgements.

% \begin{theorem}[Reachability]\todo{why $\htau$ and not $\ztau$?}
% \label{thrm:reach}
% Both of the following hold:
% \begin{enumerate}
%   \item If $\hsyn{\hGamma}{\removeSel{\zexp}}{\htau}$ and
%    $\hsyn{\hGamma}{\removeSel{\zexp'}}{\htau}$ and $\removeSel{\zexp}
%    = \removeSel{\zexp'}$ then $\exists L \in \mathit{list}
%    ~\mathtt{action}$ such that $\mathit{iterate}~ L~ \zexp
%    = \zexp'$.

%   \item If $\hana{\hGamma}{\removeSel{\zexp}}{\htau}$ and
%    $\hana{\hGamma}{\removeSel{\zexp'}}{\htau}$ and $\removeSel{\zexp}
%    = \removeSel{\zexp'}$ then $\exists L \in \mathit{list}
%    ~\mathtt{action}$ such that $\mathit{iterate}~ L~ \zexp
%    = \zexp'$.
% \end{enumerate}
% \end{theorem}


\subsubsection{Deletion} The $\aDel$ action inserts an empty hole at the cursor, deleting what was there before.

The type action rule for $\aDel$ is self-explanatory:
\begin{equation}
  \inferrule{ }{
    \performTyp{
      \zwsel{\htau}
    }{
      \aDel
    }{
      \zwsel{\tehole}
    }
  }
\end{equation}
% \begin{equation}
%   \inferrule{
%     \performTyp{\ztau}{\aDel}{\ztau'}
%   }{
%     \performTyp{\tarr{\ztau}{\htau}}{\aDel}{\tarr{\ztau'}{\htau}}
%   }
% \end{equation}
% \begin{equation}
%   \inferrule{
%     \performTyp{\ztau}{\aDel}{\ztau'}
%   }{
%     \performTyp{\tarr{\htau}{\ztau}}{\aDel}{\tarr{\htau}{\ztau'}}
%   }
% \end{equation}

Deletion within a Z-expression is similarly straightforward:
\begin{subequations}
\begin{equation}
  \inferrule{ }{
    \performSyn{\hGamma}{\zwsel{\hexp}}{\htau}{\aDel}{\zwsel{\hehole}}{\tehole}
  }
\end{equation}
\begin{equation}
  \inferrule{ }{
    \performAna{\hGamma}{\zwsel{\hexp}}{\htau}{\aDel}{\zwsel{\hehole}}
  }
\end{equation}
%\end{subequations}
% The base case turns into a hole:
%\begin{subequations}
% \begin{equation}
% \inferrule{ }{
%   \performDel{\zwsel{\hexp}}{\hehole}
% }
% \end{equation}
% The rules for the recursive ascription case is shown below. The other recursive cases are analagous:
% \begin{equation}
%   \inferrule{
%     \performDel{\zexp}{\zexp'}
%   }{
%     \performDel{\zexp : \htau}{\zexp' : \htau}
%   }
% \end{equation}
% \begin{equation}
%   \inferrule{
%     \performTyp{\ztau}{\aDel}{\ztau'}
%   }{
%     \performDel{\hexp : \ztau}{\hexp : \ztau'}
%   }
% \end{equation}
\end{subequations}

\subsubsection{Construction}\label{sec:construction} The construction actions, $\aConstruct{\psi}$, are used to construct terms of a shape indicated by $\psi$ at the cursor.

\paragraph{Types} The $\aConstruct{\farr}$ action constructs an arrow type. The H-type under the cursor becomes the argument type, and the cursor is placed on an empty return type hole:
\begin{subequations}
  \begin{equation}
    \label{r:contarr}
  \inferrule{ }{
    \performTyp{
      \zwsel{\htau}
    }{
      \aConstruct{\farr}
    }{
      \tarr{\htau}{\zwsel{\tehole}}
    }
  }
\end{equation}

The $\aConstruct{\fnum}$ action replaces an empty type hole under the cursor with the $\tnum$ type:
  \begin{equation}
    \label{r:contnum}
  \inferrule{ }{
    \performTyp{
      \zwsel{\tehole}
    }{
      \aConstruct{\fnum}
    }{
      \zwsel{\tnum}
    }
  }
\end{equation}

% Construction proceeds recursively down the zipper:
%   \begin{equation}
%     \label{r:contarrL}
%   \inferrule{
%     \performTyp{\ztau}{\aConstruct{\psi}}{\ztau'}
%   }{
%     \performTyp{
%       \tarr{\ztau}{\htau}
%     }{
%       \aConstruct{\psi}
%     }{
%       \tarr{\ztau'}{\htau}
%     }
%   }
% \end{equation}
%   \begin{equation}
%     \label{r:contarrR}
%   \inferrule{
%     \performTyp{\ztau}{\aConstruct{\psi}}{\ztau'}
%   }{
%     \performTyp{
%       \tarr{\htau}{\ztau}
%     }{
%       \aConstruct{\psi}
%     }{
%       \tarr{\htau}{\ztau'}
%     }
%   }
% \end{equation}
\end{subequations}

\begin{subequations}

\paragraph{Ascription} The $\aConstruct{\fasc}$ action operates differently depending on whether the H-expression under the cursor synthesizes a type or is being analyzed against a type. In the first case, the synthesized type appears in the ascription:
\begin{equation}
  \label{r:constructasc}
  \inferrule{ }{
    \performSyn{\hGamma}{\zwsel{\hexp}}{\htau}{\aConstruct{\fasc}}{\hexp : \zwsel{\htau}}{\htau}
  }
\end{equation}
In the second case, the type provided for analysis appears in the ascription:
\begin{equation}
  \inferrule{ }{
    \performAna{\hGamma}{\zwsel{\hexp}}{\htau}{\aConstruct{\fasc}}{\hexp : \zwsel{\htau}}
  }
\end{equation}
% This rule should supercede the subsumption rule, so we excluded $\aConstruct{\fasc}$ from subsumption in Rule (\ref{rule:subsumes}).

\paragraph{Variables} The $\aConstruct{\fvar{x}}$ action places the variable $x$ into an empty hole. If that hole is being asked to synthesize a type, then the result synthesizes the hypothesized type:
\begin{equation}
  \label{r:conevar}
  \inferrule{ }{
    \performSyn{\hGamma, x : \htau}{\zwsel{\hehole}}{\tehole}{\aConstruct{\fvar{x}}}{\zwsel{x}}{\htau}
  }
\end{equation}
If the hole is being analyzed against a type that is consistent with the hypothesized type, then the action semantics goes through the {action subsumption rule} described in Sec. \ref{sec:action-subsumption}. If the hole is being analyzed against a type that is inconsistent with the hypothesized type, $x$ is placed inside a hole:
\begin{equation}
 \label{r:conevar2}
  \inferrule{
    \tincompat{\htau}{\htau'}
  }{
    \performAna{\hGamma, x : \htau'}{\zwsel{\hehole}}{\htau}{\aConstruct{\fvar{x}}}{\hhole{\zwsel{x}}}
  }
\end{equation}
The rule above featured on Line 15 of Figure \ref{fig:second-example}.

\paragraph{Lambdas} The $\aConstruct{\flam{x}}$ action places a lambda abstraction binding $x$ into an empty hole. If the empty hole is being asked to synthesize a type, then the result of the action is a lambda ascribed the type $\tarr{\tehole}{\tehole}$, with the cursor on the argument type hole:
\begin{equation}
  \label{r:conelamhole}
  \inferrule{ }{
    \performSyn
      {\hGamma}
      {\zwsel{\hehole}}
      {\tehole}
      {\aConstruct{\flam{x}}}
      {\hlam{x}{\hehole} : \tarr{\zwsel{\tehole}}{\tehole}}
      {\tarr{\tehole}{\tehole}}
  }
\end{equation}
The type ascription is necessary because lambda expressions do not synthesize a type. If the empty hole is being analyzed against a type with matched arrow type, then no ascription is necessary:
\begin{equation}\label{rule:performAna-lam-1}
  \inferrule{
    \arrmatch{\htau}{\tarr{\htau_1}{\htau_2}}
  }{
    \performAna
      {\hGamma}
      {\zwsel{\hehole}}
      {\htau}
      {\aConstruct{\flam{x}}}
      {\hlam{x}{\zwsel{\hehole}}}
  }
\end{equation}
% Because the ascription is omitted here but included in Rule (\ref{r:conelamhole}), we must exclude all $\aConstruct{\flam{x}}$ actions from subsumption in Rule (\ref{rule:subsumes}).

Finally, if the empty hole is being analyzed against a type that has no matched arrow type, expressed in the premise as inconsistency with $\tarr{\tehole}{\tehole}$, then a lambda ascribed the type $\tarr{\tehole}{\tehole}$
is inserted inside a hole, which defers the type inconsistency as previously discussed:
\begin{equation}\label{rule:performAna-construct-lam-2}
  \inferrule{
    \tincompat{\htau}{\tarr{\tehole}{\tehole}}
  }{
    \performAna
      {\hGamma}
      {\zwsel{\hehole}}
      {\htau}
      {\aConstruct{\flam{x}}}
      {\hhole{
        \hlam{x}{\hehole} : \tarr{\zwsel{\tehole}}{\tehole}
      }}
  }
\end{equation}

\paragraph{Application} The $\aConstruct{\fap}$ action applies the expression under the cursor. The following rule handles the case where the synthesized type has matched function type:
\begin{equation}
  \label{r:coneapfn}
  \inferrule{
    \arrmatch{\htau}{\tarr{\htau_1}{\htau_2}}
  }{
    \performSyn
      {\hGamma}
      {\zwsel{\hexp}}
      {\htau}
      {\aConstruct{\fap}}
      {\hap{\hexp}{\zwsel{\hehole}}}
      {\htau_2}
  }
\end{equation}
If the expression under the cursor synthesizes a type that is inconsistent with an arrow type, then we must place that expression inside a hole to maintain Theorem \ref{sec:holes}:
\begin{equation}
  \inferrule{
    \tincompat{\htau}{\tarr{\tehole}{\tehole}}
  }{
    \performSyn
      {\hGamma}
      {\zwsel{\hexp}}
      {\htau}
      {\aConstruct{\fap}}
      {\hap{\hhole{\hexp}}{\zwsel{\hehole}}}
      {\tehole}
  }
\end{equation}

The $\aConstruct{\farg}$ action instead places the expression under the cursor in the argument position of an application form. Because the function position is always an empty hole in this situation, we need only a single rule:
\begin{equation}
  \inferrule{ }{
    \performSyn
      {\hGamma}
      {\zwsel{\hexp}}
      {\htau}
      {\aConstruct{\farg}}
      {\hap{\zwsel{\hehole}}{\hexp}}
      {\tehole}
  }
\end{equation}

\paragraph{Numbers} The $\aConstruct{\fnumlit{n}}$ action replaces an empty hole with the number expression $\hnum{n}$. If the empty hole is being asked to synthesize a type, then the rule is straightforward:
\begin{equation}
  \label{r:conenumnum}
  \inferrule{ }{
    \performSyn
      {\hGamma}
      {\zwsel{\hehole}}
      {\tehole}
      {\aConstruct{\fnumlit{n}}}
      {\zwsel{\hnum{n}}}
      {\tnum}
  }
\end{equation}
If the empty hole is being analyzed against a type that is inconsistent with $\tnum$, then we must place the number expression inside the hole:
\begin{equation}
  \inferrule{
    \tincompat{\htau}{\tnum}
  }{
    \performAna
      {\hGamma}
      {\zwsel{\hehole}}
      {\htau}
      {\aConstruct{\fnumlit{n}}}
      {\hhole{\zwsel{\hnum{n}}}}
  }
\end{equation}

The $\aConstruct{\fplus}$ action constructs a plus expression with the expression under the cursor as its first argument. If that expression synthesizes a type consistent with $\tnum$, then the rule is straightforward:
\begin{equation}\label{rule:construct-plus-compat}
  \inferrule{
    \tcompat{\htau}{\tnum}
  }{
    \performSyn
      {\hGamma}
      {\zwsel{\hexp}}
      {\htau}
      {\aConstruct{\fplus}}
      {\hadd{\hexp}{\zwsel{\hehole}}}
      {\tnum}
  }
\end{equation}
Otherwise, we must place that expression inside a hole:
\begin{equation}
  \inferrule{
    \tincompat{\htau}{\tnum}
  }{
    \performSyn
      {\hGamma}
      {\zwsel{\hexp}}
      {\htau}
      {\aConstruct{\fplus}}
      {\hadd{\hhole{\hexp}}{\zwsel{\hehole}}}
      {\tnum}
  }
\end{equation}

\paragraph{Non-Empty Holes} The final shape is $\fnehole$. This explicitly places the expression under the cursor inside a hole:
\begin{equation}
\inferrule{ }{
  \performSyn
    {\hGamma}
    {\zwsel{\hexp}}
    {\htau}
    {\aConstruct{\fnehole}}
    {\hhole{\zwsel{\hexp}}}
    {\tehole}
}
\end{equation}\end{subequations}

The $\fnehole$ shape is grayed out in Figure \ref{fig:action-syntax} because we do not expect the programmer to perform it explicitly -- other actions automatically insert holes when a type inconsistency would arise. Its inclusion is mainly to make it easier to state another ``checksum'' theorem: \emph{constructability}

\paragraph{Constructability}
To check that we have defined ``enough'' construct actions, we need to establish that we can start from an empty hole and arrive at any well-typed expression with, for simplicity, the cursor on the outside (Lemma \ref{lemma:reach-down} allows us to then move the cursor anywhere else.) As with reachability, we rely on the iterated action judgements defined in Figure \ref{fig:multistep}.
\begin{theorem}[Constructability]\label{thrm:constructability} ~
\begin{enumerate}[itemsep=0px,partopsep=0px,topsep=0px]
\item For every $\htau$ there exists $\bar\alpha$ such that $\performTypI{\zwsel{\tehole}}{\bar\alpha}{\zwsel{\htau}}$.
\item If $\hana{\hGamma}{\hexp}{\htau}$ then there exists $\bar\alpha$ such that: $$\performAnaI{\hGamma}{\zwsel{\hhole{}}}{\htau}{\bar\alpha}{\zwsel{\hexp}}$$
\item If $\hsyn{\hGamma}{\hexp}{\htau}$ then there exists $\bar\alpha$ such that: $$\performSynI{\hGamma}{\zwsel{\hhole{}}}{\tehole}{\bar\alpha}{\zwsel{\hexp}}{\htau}$$
\end{enumerate}
\end{theorem}

\subsubsection{Finishing}
The final action we need to consider is $\aFinish$, which finishes the non-empty hole under the cursor.

If the non-empty hole appears in synthetic position, then it can always be finished:
\begin{subequations}
  \begin{equation}
  \inferrule{
    \hsyn{\hGamma}{\hexp}{\htau'}
  }{
    \performSyn
      {\hGamma}
      {\zwsel{\hhole{\hexp}}}
      {\tehole}
      {\aFinish}
      {\zwsel{\hexp}}
      {\htau'}
  }
\end{equation}

If the non-empty hole appears in analytic position, then it can only be finished if the type synthesized for the enveloped expression is consistent with the type that the hole is being analyzed against. This amounts to analyzing the enveloped expression against the provided type (by subsumption):
\begin{equation}\label{r:finishana}
  \inferrule{
    \hana{\hGamma}{\hexp}{\htau}
  }{
    \performAna
      {\hGamma}
      {\zwsel{\hhole{\hexp}}}
      {\htau}
      {\aFinish}
      {\zwsel{\hexp}}
  }
\end{equation}
\end{subequations}

\subsubsection{Zipper Cases}\label{sec:zipper-cases} The rules defined so far handle the base cases, i.e. the cases where the action has ``reached'' the expression under the cursor. We also need to define the recursive cases, which propagate the action into the subtree where the cursor appears, as encoded by the zipper structure. For types, the zipper rules are straightforward:
\begin{subequations}
\begin{equation}
    %\label{r:contarrL}
  \inferrule{
    \performTyp{\ztau}{\alpha}{\ztau'}
  }{
    \performTyp{
      \tarr{\ztau}{\htau}
    }{
      \alpha
    }{
      \tarr{\ztau'}{\htau}
    }
  }
\end{equation}
  \begin{equation}
    %\label{r:contarrR}
  \inferrule{
    \performTyp{\ztau}{\alpha}{\ztau'}
  }{
    \performTyp{
      \tarr{\htau}{\ztau}
    }{
      \alpha
    }{
      \tarr{\htau}{\ztau'}
    }
  }
\end{equation}
\end{subequations}
For expressions, the zipper rules essentially follow the structure of the corresponding rules in the statics.

\begin{subequations}
In particular, when the cursor is in the body of a lambda expression, the zipper case mirrors Rule (\ref{rule:syn-lam}):
\begin{equation}
\inferrule{
  \arrmatch{\htau}{\tarr{\htau_1}{\htau_2}}\\
  \performAna
    {\hGamma, x : \htau_1}
    {\zexp}
    {\htau_2}
    {\alpha}
    {\zexp'}
}{
  \performAna
    {\hGamma}
    {\hlam{x}{\zexp}}
    {\htau}
    {\alpha}
    {\hlam{x}{\zexp'}}
}
\end{equation}

When the cursor is in the expression position of an ascription, we use the analytic  judgement, mirroring Rule (\ref{rule:syn-asc}):
\begin{equation}
\inferrule{
  \performAna
    {\hGamma}
    {\zexp}
    {\htau}
    {\alpha}
    {\zexp'}
}{
  \performSyn
    {\hGamma}
    {\zexp : \htau}
    {\htau}
    {\alpha}
    {\zexp' : \htau}
    {\htau}
}
\end{equation}


When the cursor is in the type position of an ascription, we must re-check the ascribed expression because the cursor erasure might have changed (in practice, one would optimize this check to only occur if the cursor erasure did change):
\begin{equation}\label{rule:zipper-asc}
\inferrule{
  \performTyp{\ztau}{\alpha}{\ztau'}\\
  \hana{\hGamma}{\hexp}{\removeSel{\ztau'}}
}{
  \performSyn
    {\hGamma}
    {\hexp : \ztau}
    {\removeSel{\ztau}}
    {\alpha}
    {\hexp : \ztau'}
    {\removeSel{\ztau'}}
}
\end{equation}


When the cursor is in the function position of an application, the rule mirrors Rule (\ref{rule:syn-ap}):
\begin{equation}
  \inferrule{
    \hsyn{\hGamma}{\removeSel{\zexp}}{\htau_2}\\
    \performSyn
      {\hGamma}
      {\zexp}
      {\htau_2}
      {\alpha}
      {\zexp'}
      {\htau_3}\\\\
    \arrmatch{\htau_3}{\tarr{\htau_4}{\htau_5}}\\
    \hana{\hGamma}{\hexp}{\htau_4}
  }{
    \performSyn
      {\hGamma}
      {\hap{\zexp}{\hexp}}
      {\htau_1}
      {\alpha}
      {\hap{\zexp'}{\hexp}}
      {\htau_5}
  }
\end{equation}
% \begin{equation}
%   \inferrule{
%     \hsyn{\hGamma}{\removeSel{\zexp}}{\htau_2}\\
%     \performSyn
%       {\hGamma}
%       {\zexp}
%       {\htau_2}
%       {\alpha}
%       {\zexp'}
%       {\tehole}\\
%     \hana{\hGamma}{\hexp}{\tehole}
%   }{
%     \performSyn
%       {\hGamma}
%       {\hap{\zexp}{\hexp}}
%       {\htau_1}
%       {\alpha}
%       {\hap{\zexp'}{\hexp}}
%       {\tehole}
%   }
% \end{equation}

The situation is similar when the cursor is in argument position:
\begin{equation}
  \inferrule{
    \hsyn{\hGamma}{\hexp}{\htau_2}\\
    \arrmatch{\htau_2}{\tarr{\htau_3}{\htau_4}}\\
    \performAna
      {\hGamma}
      {\zexp}
      {\htau_3}
      {\alpha}
      {\zexp'}
  }{
    \performSyn
      {\hGamma}
      {\hap{\hexp}{\zexp}}
      {\htau_1}
      {\alpha}
      {\hap{\hexp}{\zexp'}}
      {\htau_4}
  }
\end{equation}
% \begin{equation}
%   \inferrule{
%     \hsyn{\hGamma}{\hexp}{\tehole}\\
%     \performAna
%       {\hGamma}
%       {\zexp}
%       {\tehole}
%       {\alpha}
%       {\zexp'}
%   }{
%     \performSyn
%       {\hGamma}
%       {\hap{\hexp}{\zexp}}
%       {\tehole}
%       {\alpha}
%       {\hap{\hexp}{\zexp'}}
%       {\tehole}
%   }
% \end{equation}

The rules for the addition operator mirror Rule (\ref{rule:syn-plus}):
\begin{equation}
  \inferrule{
    \performAna
      {\hGamma}
      {\zexp}
      {\tnum}
      {\alpha}
      {\zexp'}
  }{
    \performSyn
      {\hGamma}
      {\hadd{\zexp}{\hexp}}
      {\tnum}
      {\alpha}
      {\hadd{\zexp'}{\hexp}}
      {\tnum}
  }
\end{equation}
\begin{equation}
  \inferrule{
    \performAna
      {\hGamma}
      {\zexp}
      {\tnum}
      {\alpha}
      {\zexp'}
  }{
    \performSyn
      {\hGamma}
      {\hadd{\hexp}{\zexp}}
      {\tnum}
      {\alpha}
      {\hadd{\hexp}{\zexp'}}
      {\tnum}
  }
\end{equation}

Finally, if the cursor is inside a non-empty hole, the relevant zipper rule mirrors Rule (\ref{rule:syn-ehole}):
\begin{equation}
  \inferrule{
    \hsyn{\hGamma}{\removeSel{\zexp}}{\htau}\\
    \performSyn
      {\hGamma}
      {\zexp}
      {\htau}
      {\alpha}
      {\zexp'}
      {\htau'}\\
  }{
    \performSyn
      {\hGamma}
      {\hhole{\zexp}}
      {\tehole}
      {\alpha}
      {\hhole{\zexp'}}
      {\tehole}
  }
\end{equation}
% \begin{equation}
%   \inferrule{
%     \hsyn{\hGamma}{\removeSel{\zexp}}{\htau}\\
%     \performSyn
%       {\hGamma}
%       {\zexp}
%       {\htau}
%       {\alpha}
%       {\zwsel{\hehole}}
%       {\tehole}\\
%   }{
%     \performSyn
%       {\hGamma}
%       {\hhole{\zexp}}
%       {\tehole}
%       {\alpha}
%       {\zwsel{\hehole}}
%       {\tehole}
%   }
% \end{equation}

Theorem \ref{thrm:actsafe} directly checks the correctness of these rules. Moreover, the zipper rules arise ubiquitously in derivations of edit steps, so the proofs of the other ``check'' theorems, e.g. Reachability and Constructability, serve as a check that none of these rules have been missed.
\end{subequations}


\subsection{Mechanization}
\label{sec:mech}\label{sec:mt}
% !TEX root = hazelnut-popl17.tex
So far, we have given an overview of the most important judgements and rules in
the semantics of Hazelnut, and stated the critical metatheorem, Sensibility, and several auxiliary ``checks''. In a few
cases, we have informally sketched out why these metatheorems will hold. In order to formally establish that our design meets our stated objectives, we
have mechanized the semantics of Hazelnut using the Agda proof assistant \cite{norell:thesis} (also see the Agda Wiki, hosted
at \url{http://wiki.portal.chalmers.se/agda/}.)

This development is available in the supplemental material. At the time of submission, the full set of rules is available and the proofs of Theorem \ref{thrm:actsafe}, Theorem \ref{lemma:movement-erasure} and a number of technical lemmas that we elided or mentioned informally in the text are complete. We also have nearly all of the cases of Reachability and Constructability complete, and we see no major barrier to the remaining cases -- we prioritized developing a clear exposition of our contributions after proving representative cases and finding that they were straightforward. The remaining cases will be completed in the repository linked in the non-anonymous supplementary material.

The documentation 
includes a more detailed discussion of the technical representation
decisions that we made. The main idea is standard: we encode each judgement as a
dependent type. The rules defining the judgements become the constructors of this
type, and derivations are terms of these type. This is a rich
setting that allows proofs to take advantage of pattern matching on the
shape of derivations, closely matching standard on-paper proofs. No proof automation was used, because the proof structure itself is likely to be interesting to researchers who plan to build upon our work. 
% The formalization differs from the calculus defined here in a few small 
% ways. Most interestingly, instead of giving separate movement rules for each 
% form of Z-expression, we abstract
% over different corresponding forms that happen to have the same arity, e.g. 
% additions and applications. Formalizing this intuition
% reduces the number of cases we need to consider 
% somewhat, but more importantly allows us to write a slightly more
% general calculus -- it will be easier to extend Hazelnut with more 
% interesting language features with this generic infrastructure in place.
We adopt Barendregt's convention for
bound variables \cite{urban}. Hazelnut's semantics does not need substitution, so we do not need to adopt more sophisticated encodings (e.g. \cite{lh09unibind,Pouillard11}.)

\subsection{Determinism}
A useful property would be \emph{action determinism}, i.e. that performing an action produces a unique result. Formally, this would be established as follows:
% \begin{theorem}[Action Determinism] All of the following hold:
% \label{thrm:actdet}
\begin{enumerate}[itemsep=0px,partopsep=0px,topsep=0px]
\item If $\performTyp{\ztau}{\alpha}{\ztau'}$ and $\performTyp{\ztau}{\alpha}{\ztau''}$ then $\ztau'=\ztau''$.
\item If $\hsyn{\hGamma}{\removeSel{\zexp}}{\htau}$ and
  $\performSyn{\hGamma}{\zexp}{\htau}{\alpha}{\zexp'}{\htau'}$ and
  $\performSyn{\hGamma}{\zexp}{\htau}{\alpha}{\zexp''}{\htau''}$ then
  $\zexp' = \zexp''$ and $\htau' = \htau''$.
% \item If all of

  % \begin{quote}
  %   \begin{enumerate}
  %   \item $\hsyn{\hGamma}{\removeSel{\zexp}}{\htau}$, and
  %   \item $\performSyn{\hGamma}{\zexp}{\htau}{\alpha}{\zexp'}{\htau'}$, and
  %   \item $\tcompat{\htau}{\htau'}$, and
  %   \item either $\performAna{\hGamma}{\zexp}{\htau}{\alpha}{\zexp''}$ or
  %     $\performAna{\hGamma}{\zexp}{\htau'}{\alpha}{\zexp''}$
  %   \end{enumerate}
  % \end{quote}
  % hold, then $\zexp' = \zexp''$.
\item If $\hana{\hGamma}{\removeSel{\zexp}}{\htau}$ and
  $\performAna{\hGamma}{\zexp}{\htau}{\alpha}{\zexp'}$ and
  $\performAna{\hGamma}{\zexp}{\htau}{\alpha}{\zexp''}$ then $\zexp' =
  \zexp''$.
\end{enumerate}
% \end{theorem}

This is not a theorem of the system as described in this paper. The reason is quite subtle: the semantics of two construction actions, $\aConstruct{\fasc}$ and $\aConstruct{\flam{x}}$, behave differently in the analytic case than they do in the synthetic case -- the difference has to do with what the generated ascription contains. Both cases can technically ``fire'' due to action subsumption. The mechanization includes a ``proof'' of determinism for all cases but these two. Resolving this ambiguity formally requires some rather inelegant machinery to exclude these particular base cases from action subsumption, so we are content to admonish implementors who apply action subsumption when another analytic rule would do. 

\clearpage

\section{Dynamics of Hazelnut}\label{sec:dynamics}

\subsection{Complete Expressions}
%... canonical forms and type safety with e complete in it ...

\subsection{Holes as TODOs}
%... add an e err and define new type safety ...

\subsection{Indeterminate Expressions}
%% Cyrus: We took a look at the dynamics. Overall it seems like the
%% right idea. We noticed:

%% Done: 1) The stepping rules are non-deterministic (i.e. you can
%% step the right or left of e1 + e2 in any order). Might be useful to
%% make them deterministic.

%% Done (mostly): 2) The premises that have disjunctions in them could
%% be broken out into two rules -- this would take a little more space
%% but follows the usual conventions more closely.

%% Done (mostly): 3) We need to add the ``ceil'' forms to the grammar
%% of \dot{e} (you just used e) and give them a static semantics.

%% TODO: 4) We need to figure out (the analogs of) canonical forms,
%% preservation and progress -- I guess we had decided on defining a
%% declarative statics to do that. Ian has started to prove the
%% correspondence (sans the ceil forms).

%% #4 seems like the most important next step.

\begin{figure}[htbp]
  \centering

  \judgbox{\hexp~\textsf{value}}{H-Expression $\hexp$ is a closed value}
  \begin{mathpar}
    \Infer{V-num}
          { }
          {\hnum{n}~\textsf{value}}
    \and
    \Infer{V-lam}
        {}
        {\hlam{x}{\hexp}~\textsf{value}}
  \end{mathpar}
  
  \caption{Value forms}
  \label{fig:judg-value}
\end{figure}

\begin{figure}[htbp]
  \centering

  \judgbox{\hexp~\textsf{final}}{H-Expression $\hexp$ is final}
  \begin{mathpar}
    \Infer{F-val}
          {\hexp~\textsf{value}}
          {\hexp~\textsf{final}}
    \and
    \Infer{F-filled}
          {\hexp~\textsf{final}}
          {\hhole{\hexp}~\textsf{final}}
    \and
    \Infer{F-unfilled}
          { }
          {\hhole{}~\textsf{final}}
    \and
    \Infer{F-indet}
        {\hexp~\textsf{indet}}
        {\hindet{\hexp}~\textsf{final}}
  \end{mathpar}
  
  \caption{Final forms}
  \label{fig:judg-value}
\end{figure}

\begin{figure}[htbp]
  \centering

  \judgbox{\hexp~\textsf{indet}}{H-Expression~$\hexp$ is indeterminate}
  \begin{mathpar}
    \Infer{I-plus$_1$}
          { \hexp_1~\textsf{final} \\
            \hexp_2~\textsf{final} \\
            \hexp_1 \ne \hnum{n_1}
          }
          {\hadd{e_1}{e_2}~\textsf{indet}}
    \and
    \Infer{I-plus$_2$}
          { \hexp_1~\textsf{final} \\
            \hexp_2~\textsf{final} \\
            \hexp_2 \ne \hnum{n_2}
          }
          {\hadd{\hexp_1}{\hexp_2}~\textsf{indet}}
    \and
    \Infer{I-app}
          { \hexp_1~\textsf{final} \\
            \hexp_2~\textsf{final} \\
            \hexp_1 \ne \hlam{x}{\hexp_1'}
          }
          {\hap{\hexp_1}{\hexp_2}~\textsf{indet}}
    \and
  \end{mathpar}
  
  \caption{Indeterminate forms}
  \label{fig:judg-value}
\end{figure}

\begin{figure}[htbp]
  \centering

  \judgbox{\hexp_1 \longrightarrow \hexp_2}{H-Expression $\hexp_1$ steps to~$\hexp_2$}
  \begin{mathpar}
    \Infer{S-plus$_1$}
          { \hexp_1 \longrightarrow \hexp_1' }
          { \hadd{\hexp_1}{\hexp_2} \longrightarrow \hadd{\hexp_1'}{\hexp_2} }

    \Infer{S-plus$_2$}
          { \hexp_1~\textsf{final}
            \\
            \hexp_2 \longrightarrow \hexp_2' }
          { \hadd{\hexp_1}{\hexp_2} \longrightarrow \hadd{\hexp_1}{\hexp_2'} }

    \Infer{S-plus$_3$}
          { n_1 + n_2 = n_3 }
          { \hadd{\hnum{n_1}}{\hnum{n_2}} \longrightarrow \hnum{n_3} }

   \Infer{S-plus$_{4a}$}
          { \hexp_1~\textsf{final}
            \\
            \hexp_2~\textsf{final} 
            \\
            \hexp_1 \ne \hnum{n_1}
          }
          { \hadd{\hexp_1}{\hexp_2} \longrightarrow \hindet{\hadd{\hexp_1}{\hexp_2}} }

   \Infer{S-plus$_{4b}$}
          { \hexp_1~\textsf{final}
            \\
            \hexp_2~\textsf{final} 
            \\
            \hexp_2 \ne \hnum{n_2}
          }
          { \hadd{\hexp_1}{\hexp_2} \longrightarrow \hindet{\hadd{\hexp_1}{\hexp_2}} }

    \Infer{S-ap$_1$}
          { \hexp_1 \longrightarrow \hexp_1' }
          { \hap{\hexp_1}{\hexp_2} \longrightarrow \hap{\hexp_1'}{\hexp_2} }

    \Infer{S-ap$_2$}
          { \hexp_2 \longrightarrow \hexp_2'
            \\
            \hexp_1~\textsf{final}
          }
          { \hap{\hexp_1}{\hexp_2} \longrightarrow \hap{\hexp_1}{\hexp_2'} }

    \Infer{S-ap$_3$}
          { \hexp_2~\textsf{final} }
          { \hap{\hlam{x}{\hexp_1}}{\hexp_2} \longrightarrow [\hexp_2/x]\hexp_1 }

   \Infer{S-ap$_4$}
          { \hexp_1~\textsf{final} ~~ ~~ ~~
            \hexp_2~\textsf{final}
            \\\\
            \hexp_1 \ne \hlam{x}{\hexp_1'} ~~ ~~ ~~ ~~ ~~ ~~ ~~ { }
          }
          { \hap{\hexp_1}{\hexp_2} \longrightarrow \hindet{\hap{\hexp_1}{\hexp_2}} }

  \end{mathpar}
  
  \caption{Small-step operational semantics.}
  \label{fig:judg-value}
\end{figure}

% Q: Are we running programs with ascriptions?
% No. We erase the ascriptions before running. Ascriptions below are
%
% Q: Are the ascriptions needed to type-check the programs?
% No. The ascriptions are needed to check the programs algorithmically, not to derive typing derivations.
% (Q: How do ascriptions, holes, and typing derivations interact?)
%
% Q: Does it ever make sense to run open programs? (i.e., run programs with ``free variables''?)
%  -- Yes, but only in the sense that we can bind holes to variables,
%     ascribe them types, and (attempt to) compute with them.
%
% Example: (where '?' means "empty hole")
%  let x = ? : int -> int in
%  let y = ? : int in
%  x y : int
% ==erase==>
%  let x = ? in
%  let y = ? in
%  x y
% -->
%  let y = ? in
%  ? y
% -->
%  ? ?
% --> 
%  (indet ? ?)

We extend the syntax of H-Expressions as follows:
\[
\hexp ::= \cdots~|~\hindet{\hexp}
\]

Here are some things that we want to prove:

\textbf{Def}~(Ascription erasure). 
\\
$\herase{\hexp}$ is the same as $\hexp$, but without its type ascriptions.
%
All cases are congruences, except for the ascription case, where $\herase{\hexp : \htau} = \hexp$.

\textbf{Def}~(Declarative typing).
\\
The judgement $\hGamma \vdash \hexp : \htau$ is a type assignment
system for erased terms.  It is declarative, and unlike the
bidirectional rules, is not algorithmic.
%
We employ it to relate bidirectionally-typed terms to (erased) terms
that enjoy type soundness with respect to the dynamics.

\textbf{Conjecture}~(Bidirectional implies declarative).
\\
(i) If $\hsyn{\hGamma}{\hexp}{\htau}$ then $\hGamma \vdash \herase{\hexp} : \htau$.
\\
(ii) If $\hana{\hGamma}{\hexp}{\htau}$ then $\hGamma \vdash \herase{\hexp} : \htau$.

\textbf{Conjecture}~(Substitution).
\\
If $\hGamma, x : \tau_x \vdash \hexp : \htau$
\\
and $\hexp'~\textsf{final}$~~~(do we actually need this condition? We do not have effects, yet.)
\\
and $\hGamma \vdash \hexp' : \htau_x$
\\
then $\hGamma \vdash \hexp[\hexp' / x] : \htau$

% Q: Do we need to enforce that hexp' is final?

\textbf{Conjecture}~(Canonical forms)
\\
If $\cdot \vdash \hexp : \htau$
\\
and $\hexp~\textsf{final}$ then
\begin{itemize}
\item if $\hexp = \hindet{\hexp'}$ then $\hexp'~\textsf{indet}$ and $\cdot \vdash \hexp' : \htau$
%\marginnote{Since $\hexp'~\textsf{indet}$, the conjecture applies to $\hexp'$.}
\item else:
\begin{itemize}
\item if $\htau = \tnum$ then exists $\hnum{n}$ such that $\hexp = \hnum{n}$.
\item else if $\htau = \tarr{\htau_1}{\htau_2}$ then exists $x$ and $\hexp'$ such that $\hexp = \hlam{x}{\hexp'}$
\item else if $\htau = \tehole$ then either:
\begin{itemize}
\item $\hexp = \hehole$, or
\item exists $\htau'$ and $\hexp'$ such that $\hexp = \hhole{\hexp'}$, $\hexp'~\textsf{final}$ and $\cdot \vdash \hexp' : \htau'$
\end{itemize}
\end{itemize}
\end{itemize}

\textbf{Conjecture}~(Progresss).
\\
If $\cdot \vdash \hexp_1 : \htau$
\\
then either $\hexp_1~\textsf{final}$
\\
or exists $\hexp_2$ such that $\hexp_1 \longrightarrow \hexp_2$.

\textbf{Conjecture}~(Preservation).
\\
If $\cdot\vdash \hexp_1 : \htau$ 
\\
and $\hexp_1 \longrightarrow \hexp_2$
\\
then $\cdot \vdash \hexp_2 : \htau$


\clearpage

\section{Extending Hazelnut}\label{sec:extending}
%% \todo{write this section} Add sum types.
%% H-expression
%% Z-expression + Erasure rules
%% Statics
%% Movement actions
%% Construct actions
% !TEX root = hazelnut-dynamics.tex

\subsection{Sum Types}


%%%%%%%% %%%%%%%%% %%%%%%%%% %%%%%%%%% %%%%%%%%% %%%%%%%%% %%%%%%%%% %%%%%%%%% %%%%%%%%% %%%%%%%%%
%%%%%%% Syntax
We extend the syntax for sum types as follows:
\[
\begin{array}{rllllll}
\mathsf{HTyp} & \htau & ::= \cdots ~\vert~ {\htau + \htau} &
\\
\mathsf{HExp} & \hexp & ::= \cdots
~\vert~ \hinL{\hexp}
~\vert~ \hinR{\hexp}
%~\vert~ \hcase{\hexp}{\hinL{x}}{\hexp}{\hinR{x}}{\hexp}
~\vert~ \hcase{\hexp}{x}{\hexp}{y}{\hexp}
\\
\mathsf{IHExp} & \dexp & ::= \cdots
~\vert~ \dinL{\htau}{\dexp}
~\vert~ \dinR{\htau}{\dexp}
%~\vert~ \dcase{\dexp}{\hinL{x}}{\dexp}{\hinR{x}}{\dexp}
~\vert~ \dcase{\dexp}{x}{\dexp}{y}{\dexp}
\end{array}
\]

%%%%%%%% %%%%%%%%% %%%%%%%%% %%%%%%%%% %%%%%%%%% %%%%%%%%% %%%%%%%%% %%%%%%%%% %%%%%%%%% %%%%%%%%%
%%%%%%% Definition of Join: a meta-level, binary function over types.

%\newcommand{\JoinTypes}[2]{\textsf{join}~~#1~~#2}
%\newcommand{\JoinTypes}[2]{\textsf{join}(#1,#2)}

%\fbox{$\JoinTypes{\htau_1}{\htau_2} = \htau_3$}
\judgbox
 {\JoinTypes{\htau_1}{\htau_2} = \htau_3}
 {Types~$\htau_1$ and $\htau_2$ join consistently, forming type~$\htau_3$}
\[
\begin{array}{lcl}
\JoinTypes{\htau}{\htau} &=&  \htau
\\
\JoinTypes{\tehole}{\htau} &=&  \htau
\\
\JoinTypes{\htau}{\tehole} &=&  \htau
\\
\JoinTypes
{\tarr{\htau_1}{\htau_2}}
{\tarr{\htau_1}{\htau_2}}
&=&
\tarr{\JoinTypes{\htau_1}{\htau_2}}
     {\JoinTypes{\htau_1}{\htau_2}}
\\
\JoinTypes
{\tsum{\htau_1}{\htau_2}}
{\tsum{\htau_1}{\htau_2}}
&=&
\tsum{\JoinTypes{\htau_1}{\htau_2}}
     {\JoinTypes{\htau_1}{\htau_2}}
\end{array}
\]

\begin{thm}[Joins]
If $\JoinTypes{\tau_1}{\tau_2} = \tau$
%
then types $\tau_1$, $\tau_2$ and $\tau$ are pair-wise consistent, i.e.,
%
$\tconsistent{\tau_1}{\tau_2}$,
$\tconsistent{\tau_1}{\tau}$ and
$\tconsistent{\tau_2}{\tau}$.
\begin{proof}
By induction on the derivation of $\JoinTypes{\tau_1}{\tau_2} = \tau$.
\end{proof}
\end{thm}

%%%%%%%% %%%%%%%%% %%%%%%%%% %%%%%%%%% %%%%%%%%% %%%%%%%%% %%%%%%%%% %%%%%%%%% %%%%%%%%% %%%%%%%%%
\vsepRule

\judgbox
 {\summatch{\htau}{\tsum{\htau_1}{\htau_2}}}
 {Type~$\htau$ has matched sum type~$\tsum{\htau_1}{\htau_2}$}
\begin{mathpar}
\inferrule[]{~}{\summatch{\tsum{\tau_1}{\tau_2}}{\tsum{\tau_1}{\tau_2}}}
\and
\inferrule[]{~}{\summatch{\tehole}{\tsum{\tehole}{\tehole}}}
\end{mathpar}

%%%%%%%%% %%%%%%%%% %%%%%%%%% %%%%%%%%% %%%%%%%%% %%%%%%%%% %%%%%%%%% %%%%%%%%% %%%%%%%%% %%%%%%%%%
\vsepRule

\judgbox
  {\hana{\hGamma}{\hexp}{\htau}}
  {$\hexp$ analyzes against type $\htau$}
\begin{mathpar}
\inferrule[]{
  \summatch{\htau}{\tsum{\htau_1}{\htau_2}}\\
  \hana{\hGamma}{\hexp}{\htau_1}
}{
  \hana{\hGamma}{\hinL{\hexp}}{\htau}
}

\inferrule[]{
  \summatch{\htau}{\tsum{\htau_1}{\htau_2}}\\
  \hana{\hGamma}{\hexp}{\htau_2}
}{
  \hana{\hGamma}{\hinR{\hexp}}{\htau}
}

\inferrule[]{
  \hsyn{\hGamma}{\hexp_1}{\htau_1}\\
  \summatch{\htau_1}{\tsum{\htau_{11}}{\htau_{12}}}\\\\
  \hana{\hGamma, x : \htau_{11}}{\hexp_2}{\htau}\\\\
  \hana{\hGamma, y : \htau_{12}}{\hexp_3}{\htau}
}{
  \hana{\hGamma}{
    \hcase{\hexp_1}{x}{\hexp_2}{y}{\hexp_3}
  }{
    \htau
  }
}
\end{mathpar}

%%%%%%%%% %%%%%%%%% %%%%%%%%% %%%%%%%%% %%%%%%%%% %%%%%%%%% %%%%%%%%% %%%%%%%%% %%%%%%%%% %%%%%%%%%
\vsepRule

\judgbox
  {\elabAna{\hGamma}{\hexp}{\htau_1}{\dexp}{\htau_2}{\Delta}}
  {$\hexp$ analyzes against type $\htau_1$ and
   elaborates to $\dexp$ of consistent type $\htau_2$}
\begin{mathpar}
\inferrule[]{
  \summatch{\htau}{\tsum{\htau_1}{\htau_2}}
  \\
  \elabAna{\hGamma}{\hexp}{\htau_1}{\dexp}{\htau'_1}{\Delta}
}{
  \elabAna{\hGamma}{\hinL{\hexp}}{\htau}{\dinL{\tau_2}{\dexp}}{\tsum{\htau_1'}{\htau_2}}{\Delta}
}

\inferrule[]{
  \summatch{\htau}{\tsum{\htau_1}{\htau_2}}
  \\
  \elabAna{\hGamma}{\hexp}{\htau_2}{\dexp}{\htau'_2}{\Delta}
}{
  \elabAna{\hGamma}{\hinR{\hexp}}{\htau}{\dinL{\tau_1}{\dexp}}{\tsum{\htau_1}{\htau'_2}}{\Delta}
}

\inferrule[]{
  \elabSyn{\hGamma}{\hexp_1}{\htau_1}{\dexp_1}{\Delta_1}
  \\
  \summatch{\htau_1}{\tsum{\htau_{11}}{\htau_{12}}}
  \\\\
  \elabAna{\hGamma, x:\htau_{11}}{\hexp_2}{\htau}{\dexp_2}{\htau_2}{\Delta_2}
  \\
  \elabAna{\hGamma, y:\htau_{12}}{\hexp_3}{\htau}{\dexp_3}{\htau_3}{\Delta_3}
  \\\\
  \JoinTypes{\htau_2}{\htau_3} = {\htau'}
  \\
  \Delta = \Delta_1 \cup \Delta_2 \cup \Delta_3
}{
  \elabAna{\hGamma}
            {\hcase{e_1}{x}{e_2}{y}{e_3}}
            {\htau}
            {\dcase
                {\dcasttwo{d_1}{\htau_1}{\tsum{\htau_{11}}{\htau_{12}}}}
                {x}{\dcasttwo{d_2}{\htau_2}{\htau'}}
                {y}{\dcasttwo{d_3}{\htau_3}{\htau'}}
            }
            {\htau'}
            {\Delta}
            }
\end{mathpar}

%%%%%%%%% %%%%%%%%% %%%%%%%%% %%%%%%%%% %%%%%%%%% %%%%%%%%% %%%%%%%%% %%%%%%%%% %%%%%%%%% %%%%%%%%% %%%%%%%%%

\judgbox{\hasType{\Delta}{\hGamma}{\dexp}{\htau}}{$\dexp$ is assigned type $\htau$}
\begin{mathpar}
\inferrule[]{
  \hasType{\Delta}{\hGamma}{\dexp}{\htau_1}
}{
  \hasType{\Delta}{\hGamma}{\dinL{\tau_2}{\dexp}}{\tsum{\htau_1}{\htau_2}}
}

\inferrule[]{
  \hasType{\Delta}{\hGamma}{\dexp}{\htau_2}
}{
  \hasType{\Delta}{\hGamma}{\dinR{\tau_1}{\dexp}}{\tsum{\htau_1}{\htau_2}}
}

\inferrule[]{
  \hasType{\Delta}{\hGamma}{\dexp_1}{\tsum{\htau_1}{\htau_2}}
  \\
  \hasType{\Delta}{\hGamma,x:\htau_1}{\dexp_2}{\htau}
  \\
  \hasType{\Delta}{\hGamma,y:\htau_2}{\dexp_3}{\htau}
}{
  \hasType{\Delta}{\hGamma}{\dcase{\dexp_1}{x}{\dexp_2}{y}{\dexp_3}}{\htau}
}
\end{mathpar}

%%%%%%%%% %%%%%%%%% %%%%%%%%% %%%%%%%%% %%%%%%%%% %%%%%%%%% %%%%%%%%% %%%%%%%%% %%%%%%%%% %%%%%%%%%

\judgbox{\isValue{\dexp}}{$\dexp$ is a value}
\begin{mathpar}
\inferrule[]
{\isValue{\dexp}}
{\isValue{\dinL{\htau}{\dexp}}}

\inferrule[]
{\isValue{\dexp}}
{\isValue{\dinR{\htau}{\dexp}}}
\end{mathpar}

\vsepRule

\judgbox{\isGround{\htau}}{$\htau$ is a ground type}
\begin{mathpar}
\inferrule[]{~}{
  \isGround{\tsum{\tehole}{\tehole}}
}
\end{mathpar}

\judgbox{\groundmatch{\htau}{\htau'}}{$\htau$ has matched ground type $\htau'$}
\begin{mathpar}
\inferrule[]{
  \tsum{\htau_1}{\htau_2}\neq\tsum{\tehole}{\tehole}
}{
  \groundmatch{\tsum{\htau_1}{\htau_2}}{\tsum{\tehole}{\tehole}}
}
\end{mathpar}

\vsepRule

\judgbox{\isBoxedValue{\dexp}}{$\dexp$ is a boxed value}
\begin{mathpar}
\inferrule[]
{\isBoxedValue{\dexp}}
{\isBoxedValue{\dinL{\htau}{\dexp}}}

\inferrule[]
{\isBoxedValue{\dexp}}
{\isBoxedValue{\dinR{\htau}{\dexp}}}

\inferrule[]
{\tsum{\htau_1}{\htau_2} \ne
 \tsum{\htau_1'}{\htau_2'}
 \\
 \isBoxedValue{\dexp}
}
{\isBoxedValue{\dcasttwo{\dexp}
    {\tsum{\htau_1}{\htau_2}}
    {\tsum{\htau_1'}{\htau_2'}}
}}
\end{mathpar}

\vsepRule

\judgbox{\isIndet{\dexp}}{$\dexp$ is indeterminate}
\begin{mathpar}
\inferrule[]
{\isIndet{\dexp}}
{\isIndet{\dinL{\htau}{\dexp}}}

\inferrule[]
{\isIndet{\dexp}}
{\isIndet{\dinR{\htau}{\dexp}}}

\inferrule[]
{\tsum{\htau_1}{\htau_2} \ne
 \tsum{\htau_1'}{\htau_2'}
 \\
 \isIndet{\dexp}
}
{\isIndet{\dcasttwo{\dexp}
    {\tsum{\htau_1}{\htau_2}}
    {\tsum{\htau_1'}{\htau_2'}}
}}

\inferrule[]
{
  \dexp_1 \ne \dinL{\tau}{\dexp_1'}
  \\
  \dexp_1 \ne \dinR{\tau}{\dexp_1'}
  \\
  \dexp_1 \ne \dcasttwo{\dexp_1'}{\tsum{\htau_1}{\htau_2}}{\tsum{\htau_1'}{\htau_2'}}
  \\
  \isIndet{\dexp_1}
}
{
  \dcase{\dexp_1}{x}{\dexp_2}{y}{\dexp_3}
}
\end{mathpar}

%%%%%%%% %%%%%%%%% %%%%%%%%% %%%%%%%%% %%%%%%%%% %%%%%%%%% %%%%%%%%% %%%%%%%%% %%%%%%%%% %%%%%%%%%

\begin{mathpar}
\arraycolsep=4pt\begin{array}{rllllll}
\mathsf{EvalCtx} & \evalctx & ::= & \cdots ~\vert~
  \dinL{\htau}{\evalctx}
  ~\vert~
  \dinR{\htau}{\evalctx}
  ~\vert~
  \dcase{\evalctx}{x}{\dexp_1}{y}{\dexp_2}
  %% \evalhole ~\vert~
  %% \hap{\evalctx}{\dexp} ~\vert~
  %% \hap{\dexp}{\evalctx} ~\vert~
  %% \dhole{\evalctx}{\mvar}{\subst}{} ~\vert~
  %% \dcasttwo{\evalctx}{\htau}{\htau} ~\vert~
  %% \dcastfail{\evalctx}{\htau}{\htau}
\end{array}
\end{mathpar}

%% \judgbox{\isevalctx{\evalctx}}{$\evalctx$ is an evaluation context}
%% \begin{mathpar}
%% \inferrule[]{\isevalctx{\evalctx}}{\isevalctx{\dinL{\htau}{\evalctx}}}
%% \and
%% \inferrule[]{\isevalctx{\evalctx}}{\isevalctx{\dinR{\htau}{\evalctx}}}
%% \and
%% \inferrule[]{\isevalctx{\evalctx}}{\isevalctx{\dcase{\evalctx}{x}{\dexp_1}{y}{\dexp_2}}}
%% \end{mathpar}

\judgbox{\selectEvalCtx{\dexp}{\evalctx}{\dexp'}}{$\dexp$ is obtained by placing $\dexp'$ at the mark in $\evalctx$}
\begin{mathpar}
\inferrule[]
{\selectEvalCtx{\dexp_1}{\evalctx}{\dexp_1'}}
{\selectEvalCtx{\dcase{\dexp_1}{x}{\dexp_2}{y}{\dexp_3}}
               {\dcase{\evalctx}{x}{\dexp_2}{y}{\dexp_3}}{\dexp_1'}}
\end{mathpar}

\judgbox{\reducesE{\Delta}{\dexp_1}{\dexp_2}}{$\dexp_1$ transitions to $\dexp_2$}
\begin{mathpar}
\inferrule[]
{\maybePremise{\isFinal{\dexp_1}}}
{\reducesE{\Delta}
  {\dcase{\dinL{\htau}{\dexp_1}}{x}{\dexp_2}{y}{\dexp_3}}
  {\DoSubst{\dexp_1}{x}{\dexp_2}}}

\inferrule[]
{\maybePremise{\isFinal{\dexp_1}}}
{\reducesE{\Delta}
  {\dcase{\dinR{\htau}{\dexp_1}}{x}{\dexp_2}{y}{\dexp_3}}
  {\DoSubst{\dexp_1}{y}{\dexp_3}}}

\inferrule[]
{\maybePremise{\isFinal{\dexp_1}}}
{\reducesE{\Delta}
  {\dcase
    {\dcasttwo{\dexp_1}{\tsum{\tau_1}{\tau_2}}{\tsum{\tau_1'}{\tau_2'}}}
    {x}{\dexp_2}{y}{\dexp_3}}
  {\dcase
    {\dexp_1}
    {x}{\DoSubst{\dcasttwo{x}{\tau_1}{\tau_1'}}{x}{\dexp_2}}
    {y}{\DoSubst{\dcasttwo{y}{\tau_2}{\tau_2'}}{y}{\dexp_3}}}
}
\end{mathpar}


%% \begin{thm}[Canonical value forms--Sums]
%% If $\hasType{\Delta}{\EmptyhGamma}{\dexp}{\tsum{\htau}{\htau}}$
%% and $\isValue{\dexp}$ then either
%% \begin{enumerate}
%% \item
%% $\dexp = \dinL{\tau_2}{\dexp'}$
%% where $\isValue{\dexp'}$
%% and $\hasType{\Delta}{\EmptyhGamma}{\dexp'}{\htau_1}$
%% \item
%% $\dexp = \dinR{\tau_1}{\dexp'}$
%% where $\isValue{\dexp'}$
%% and $\hasType{\Delta}{\EmptyhGamma}{\dexp'}{\htau_1}$
%% \end{enumerate}
%% \end{thm}

%% \begin{thm}[Canonical boxed value forms--Sums]
%% If $\hasType{\Delta}{\EmptyhGamma}{\dexp}{\tsum{\htau}{\htau}}$
%% and $\isValue{\dexp}$ then either
%% \begin{enumerate}
%% \item
%% $\dexp = \dinL{\tau_2}{\dexp'}$
%% where $\isValue{\dexp'}$
%% and $\hasType{\Delta}{\EmptyhGamma}{\dexp'}{\htau_1}$
%% \item
%% $\dexp = \dinR{\tau_1}{\dexp'}$
%% where $\isValue{\dexp'}$
%% and $\hasType{\Delta}{\EmptyhGamma}{\dexp'}{\htau_1}$
%% \end{enumerate}
%% \end{thm}

%% %%%%%%%


\section{Implementation}
\label{sec:impl}
% !TEX root = hazelnut-dynamics.tex

\newcommand{\implementationSec}{Implementation}
\section{\protect\implementationSec}
\label{sec:implementation}

\rkc{there should be at least a brief section about the implementation, right?}

\rkc{describe the UI that allows a mix of text/structured edits that ensure
well-formed edit states}

\rkc{
The programs in \autoref{sec:examples}
are written using minor syntactic conveniences that
are not currently supported in our implementation.
%
For example, the implementation of recursive types in our current implementation
requires explicit \texttt{fold} and \texttt{unfold} expressions; these can be
inferred in standard ways (by pairing them with constructors and deconstructors
for algebraic datatypes) in future work.
%
More details about other minor syntactic differences are described in
\suppMaterials{}.
}


\section{Related Work and Discussion}\label{sec:rw}
%\subsection{Structure Editors}
\subsection{Structure Editors}
Syntactic structure editors have a long history -- the Cornell Program Synthesizer~\cite{teitelbaum_cornell_1981} was first introduced in 1981. Novice programmers have been a common target for structure editors. For example, 
GNOME~\cite{garlan_gnome:_1984} was developed to teach programming to undergraduates. 
Alice~\cite{Conway:2000:ALL:332040.332481} is a 3-D programming language with an integrated structure editor for teaching novice CS undergraduate students. Scratch~\cite{Resnick:2009:SP:1592761.1592779} is a structure editor targeted at children ages 8 to 16.  TouchDevelop \cite{tillmann_touchdevelop:_2011} incorporates a structure editor for programming on touch-based devices, and is used to teach high school students. An implementation of Hazelnut might be useful in teaching students about the typed lambda calculus, though that has not been our explicit aim with this work.

Not all structure editors are for educational purposes. For example,
mbeddr \cite{voelter_mbeddr:_2012} is a structure editor for a C-based programming language (nominally, for programming embedded systems.)  
Lamdu~\cite{lamdu}, like Hazelnut, is a structure editor for a statically typed functional language. It is designed for use by professional programmers. 

The examples given so far either define a language with a trivial static semantics, or do not attempt to maintain well-typedness as an edit invariant. This can pose problems, for reasons discussed in the Introduction. One apparent exception is Unison~\cite{unison}, a structure editor for a typed functional language similar to Haskell. Like Hazelnut, it seems to define some notion of well-typedness for expressions with holes (though there is no technical documentation on virtually any aspect of its design.) Unlike Hazelnut, it does not have a notion analagous to Hazelnut's notion of a non-empty hole. As such, programmers must construct programs in a rigid outside-in manner, as discussed in Sec. \ref{sec:example}. Another system with the same problem is Polymorphic Blocks, a block-based user interface where the structure of block connectors encodes a type \cite{DBLP:conf/chi/LernerFG15}.

We fundamentally differ from these projects in our design philosophy: we consider it essential to start by building type theoretic foundations, which are independent of nearly all decisions about the user interface. In contrast, these editors have developed innovative user interfaces (e.g. see the discussion in \cite{DBLP:conf/sle/VolterSBK14}) but lack a principled foundational calculus. In this respect, we follow the philosophical approach taken by languages that are rooted in the type theoretic tradition and have gone to great effort to develop a clear metatheory, like Standard ML \cite{mthm97-for-dart,Harper00atype-theoretic}.  In the future, we hope that these lines of research will merge to produce a human-usable typed structure editor with sound formal foundations. Our contribution, then, is in defining and analyzing the theoretical properties of a small foundational calculus that could be extended to achieve this vision.

Some structure editor generators do operate on formal or semi-formal definitions of an underlying language. For example, the Synthesizer Generator~\cite{Reps:1984:SG:390010.808247} allows the user to define an attribute grammar-based language implementation that then can be used to generate a structured editor. CENTAUR~\cite{Borras:1988:CS:64140.65005} produces a language specific environment from a user defined formal specification of a language. Barista is a programmatic toolkit for building structure editors \cite{ko_barista:_2006}. mbeddr is built on top of the commercial JetBrains MPS framework for constructing structure editors \cite{voelter2011language,DBLP:journals/software/VoelterWK15}. These systems do not give a semantics to the edit states of the structure editor itself, or maintain non-trivial edit invariants, as Hazelnut does. 
%These early systems were developed  of the systems are rooted in the type-theoretic tradition.

Related to structure editors are value editors, which operate directly on simple values (but not generally expressions or functions) of a programming language. For example, Eros is a typed value editor based in Haskell \cite{DBLP:conf/icfp/Elliott07}.


\subsection{Gradual Type Systems}
A significant contribution of this paper is the discovery of a clear technical relationship between typed structure editing and gradual typing. In particular, the machinery necessary to give a reasonable semantics to type holes -- i.e. type consistency and type matching -- coincides with that developed in the study of gradual type systems for functional languages. The pioneering work of Siek and Taha \cite{Siek06a} introduced type consistency. Subsequent work developed the concept of type matching \cite{DBLP:conf/popl/RastogiCH12,DBLP:conf/popl/GarciaC15}. In retrospect, this relationship is perhaps unsurprising: gradual typing is, notionally, also motivated by the idea of iterated development of a program where every intermediate state is well-defined in some sense. %Typed structure editing endows this intuitive notion with technical force.

Recent work has discovered a systematic procedure for generating a ``gradual version'' of a standard type system \cite{DBLP:conf/popl/CiminiS16}. This system, called the Gradualizer, operates on a logic program that specifies a simple type assignment system with some additional annotations to generate a corresponding specification of a gradual type system. The authors leave support for working with bidirectional type systems as future work. This suggests the possibility of an analagous ``Editorializer'' that generates a specification of a typed structure editor from a simple language definition. Our exposition in Sec. \ref{sec:extending} certainly suggests that many of the necessary definitions follow seemingly mechanically from the definition of the static semantics, and the relationship with gradual typing suggests that many of the technical details of this transformation may already exist in the Gradualizer. One possibility we have explored informally is to use Agda's reflection features to implement such a system.

An aspect of gradual typing that we did not touch on directly here is its concern with  assigning a dynamics to programs where type information is not known, by inserting dynamic type casts. This would correspond to assigning a dynamics to Hazelnut expressions with type holes such that a run-time error occurs when a type hole is found to be unfillable through evaluation.

\subsection{Type Holes as Unification Variables}
Another possible interpretation of type holes is as explicit unification variables. In other words, we might define the dynamics of Hazelnut such that a program cannot be run if some automated type inference procedure cannot statically fill in any type holes that arise. For a simple calculus, e.g. the STLC upon which Hazelnut is based, type inference is known to be decidable \cite{damas1982principal}. In more complex settings, e.g. in a dependently typed language, a partial decision procedure may still be useful in this regard, both at edit-time and (just prior to) run-time. Indeed, text editor modes for proof assistants, e.g. for Agda, attempt to do exactly this for indicated ``type holes'' (and do not always succeed.)

\subsection{Exceptions}
Expression holes, too, could dynamically be interpreted in several ways. One straightforward approach would be to dynamically raise an exception if evaluation encounters an expression hole. Indeed, placing \textt{raise Unimplemented} or similar in portions of an expression that are under construction is a common practice across programming languages.

\subsection{Type-Directed Program Synthesis}
Some text editor modes, e.g. those for proof assistants like Agda \cite{norell:thesis} and Idris \cite{brady2013idris}, support a more explicit hole-based programming model where indicated expression holes are treated as sites where the system can be asked to automatically generate an expression of an appropriate type. These systems are not statically well-defined themselves (though see below.)

The Graphite system used an informal model of typed holes in Java to allow library providers to associate interactive code generation interfaces with types \cite{Omar:2012:ACC:2337223.2337324}. %RThis system was limited in that these interfaces could not themselves manipulate terms.

More generally, the topic of type-directed program synthesis an active area of research, e.g. \cite{DBLP:conf/pldi/OseraZ15}. By maintaining static well-definedness throughout the editing process, Hazelnut provides researchers interested in editor-integrated type-directed program synthesis  with a formal foundation upon which to build.% such systems could be integrated directly into editors.

\subsection{Contextual Modal Type Theory}
Expression holes can also be understood by invoking the notion of a \emph{metavariable} as found in contextual modal type theory (CMTT) \cite{DBLP:journals/tocl/NanevskiPP08}. In particular, expression holes have types and are surrounded by contexts, just as metavariables in CMTT are associated with types and contexts. This helps to clarify the logical meaning of a typing derivation in Hazelnut -- it conveys well-typedness relative to an (implicit) modal context that extracts each expression hole's type and context. The modal context must be emptied -- i.e. the expression holes must be instantiated with expressions of the proper type in the proper context -- before the expression can be considered complete. This corresponds to the notion of modal necessity in contextual modal logic.

Making the modal context explicit in our semantics is not technically useful given our stated purpose -- interactive program editing is not merely hole filling in Hazelnut (i.e. the cursor need not be on a hole). Moreover, the hole's type and context become apparent as our action semantics traverses the zipper structure on each step. Some interactive proof assistants, however, support a tactic model based directly on hole filling, so the connection to CMTT and similar systems is more useful. For example,  Beluga \cite{DBLP:conf/flops/Pientka10} is based on dependent CMTT and aspects of Idris' editor support \cite{brady2013idris} are based on McBride's OLEG \cite{mcbride2000dependently} and Lee and Friedman have explored a lambda calculus with contexts for a similar purpose \cite{DBLP:conf/icfp/LeeF96}.

One interesting avenue of future work is to elaborate expression holes to CMTT's closures, i.e. CMTT terms of the form $\mathsf{clo}(u; \text{id}(\Gamma))$ where $u$ is a unique metavariable associated with each hole and $\text{id}(\Gamma)$ is the explicit identity substitution. This would allow us to evaluate expressions with holes such that the closure ``accumulates'' substitutions explicitly. When evaluation gets ``stuck'' (as it can, for CMTT does not define a dynamics equipped with a notion of progress under a non-empty modal context), it would then be possible for the programmer to choose holes from amongst the visible holes (which may have been duplicated) to edit in their original context. Once finished, the CMTT hole instantiation operation, together with a metatheorem that establishes that reduction commutes with instantiation, would enable an ``edit and resume'' feature with a clear formal basis. This notion of reduction commuting with instantiation has also been studied in other calculi \cite{DBLP:journals/entcs/Sands97}. Being able to edit a running program also has connections to less formal work on ``live programming'' interfaces \cite{burckhardt2013s,lamdu}. %, e.g. 

\subsection{Evaluation Strategies: A High-Dimensional Space}
To summarize, we have discussed three different evaluation strategies in the presence of type holes:
\begin{enumerate}[noitemsep]
\item ...as preventing evaluation (the standard approach.)
\item ...as unknown types, in the gradual sense.
\item ...as unification variables.
\end{enumerate}
In addition, we have discussed four different evaluation strategies in the presence of expression holes:
\begin{enumerate}[noitemsep]
\item ...as preventing evaluation (the standard approach.)
\item ...as causing exceptions.
\item ...as sites for automatic program synthesis.
\item ...as the closures of CMTT.
\end{enumerate}

Every combination of these choices could well be considered in the design of a full-scale programming system in the spirit of Hazelnut. Indeed, the user might be given several options from among these combinations, depending on their usage scenario. Many of these warrant further inquiry.

% \todo{Eclipse parsing partial programs // other heuristics}


% \paragraph{Gradual typing: Has Matched Relation}
% %
% A. Rastogi, A. Chaudhuri, and B. Hosmer. The ins and outs of gradual type inference.
% \\
% Definition 3.1 is similar to our ``has matched'' relations:
% \\
% \url{https://www.cs.umd.edu/~avik/papers/iogti.pdf}

% This paper defines ``has matched'' relation for arrow, just like us:
% \\
% \url{http://cimini.info/publications/Gradualizer_Draft.pdf}
% \\
% They credit this technique to the following paper:
% \\
% \url{http://www.cs.ubc.ca/~rxg/ptsgp.pdf}
% \\
% But, the Rastogi paper came earlier.


% \todo{TODO: \url{http://research.nii.ac.jp/~hu/pub/hosc07.pdf}}


% Agda and Idris are two dependently typed languages that attempt to simulate a structured editor from within a rich text editor (e.g. Emacs.) These systems also have notions of holes and use types to guide the user toward filling these holes. These  systems are also, to our knowledge, not formally well-defined but rather exist only as part of system implementations.


% Our work differs from all of these in that we begin with a formal editor calculus and build from there, rather than starting with an implementation and leaving many of the formal details formally unspecified. For example, while Lamdu has many interesting features, there is no theoretical basis presented for their work -- it is a rather large body of Haskell code with an unclear (and indeed, often somewhat perplexing, in our experience) action model. Unison is also a rather large body of Haskell code, though its action model appears superficially more similar to ours. We maintain what we believe to be a stronger action sensibility invariant than Unison (i.e. in Unison, one must construct expressions from the outside-in.) These systems are rich sources of interesting ideas, however -- there is room enough for many different approaches in this (re-)emerging space.

% \todo{TODO: cite \url{http://cseweb.ucsd.edu/~lerner/pb.html}}

% \todo{TODO: cite} \url{http://delivery.acm.org/10.1145/1060000/1056965/p1557-ko.pdf?ip=73.154.143.34&id=1056965&acc=AUTHOR-IZED&key=4D4702B0C3E38B35%2E4D4702B0C3E38B35%2E4D4702B0C3E38B35%2EC2CC0A83071F7605&CFID=780613712&CFTOKEN=33786941&__acm__=1462392463_f29fc98965004c61bfb1291d07756a23}

%Drag-and-drop / for novices: lots of examples, e.g. Alice and others
%
%Contemporary: Lamdu, MPS/Mbeddr, TouchDevelop
%
%Hybrid: Cyrus' active code completion paper

%\subsection{Refactoring Models}
%(Michael, can you fill this section out?)

%\subsection{Formal Editor Models}
%Need to do a search to see what else has been done...

\section{Conclusion}
\label{sec:future}
This paper presented Hazelnut, a type theoretic structure editor calculus. Our aim was to take a principled approach to its design by formally defining its static semantics as well as its action semantics and developing a rich metatheory. Moreover, we have mechanized substantial portions of the metatheory, including the crucial Sensibility theorem that establishes that every edit state is statically meaningful.

In addition to simplifying the job of an editor designer, typed structure editors also promise to increase the speed of development by eliminating redundant syntax and supporting higher-level primitive actions. However, we did not discuss such ``edit costs'' here, because they depend on particular implementation details, e.g. whether a keyboard or a mouse is in use. Indeed, we consider it a virtue of this work that such implementation details do not enter into our design.

% By keeping the program in both a structurally and semantically well-defined state at all times, Hazelnut allows users to avoid premature commitment~\cite{green1996usability}.  
% By inserting a hole, the user can leave certain parts of the program unfinished, \todo{What is the best word to define this concept?  In progress, unfinished, incomple?} and yet still in a well defined type state. 
% This also enables progressive evaluation~\cite{green1996usability}, because unfinished solutions are also are well-defined at all times, thus enabling tools to provide evaluation of unfinished solutions.\todo{integrate this} 

\subsection{Future Work}
\subsubsection{Richer Languages}
Hazelnut is, obviously, a very limited language at its core. So the most obvious avenue for future work is to increase the expressive power of this language by extension. Our plan is to simultaneously maintain a mechanization and implementation (following, for example, Standard ML) as we proceed, ultimately producing the first large-scale, formally verified bidirectionally typed structure editor.

It is interesting to note that the demarcation between the language and the editor is fuzzy (indeed, non-existent) in Hazelnut. There may well be interesting opportunities in language design when the language is being codesigned with a typed structure editor. It may be that certain language features are unnecessary given a sufficiently advanced type-aware structure editor (e.g. SML's \texttt{open}?), while other features may only be practical with editor support. We intend to use Hazelnut and derivative systems thereof as a platform for rigorously exploring such questions.

\subsubsection{Editor Services}
There are various aspects of the editor model that we have not yet formalized. For example, our action model does not consider how actions are actually entered using, for example, key combinations or chords. In practice, we would want also to rank available actions in some reasonable manner (perhaps based on usage data gathered from other users or code repositories.) Designing a rigorous typed probability model over actions and H-expressions is one avenue of research that we have started to explore, with intriguing initial results.

\subsubsection{Programmable Actions}
Our language of actions is intentionally primitive. However, even now it acts much like a simple imperative command language. This suggests future expansion to, for example, a true tactic language. Alternatively, it may be more useful to develop the notion of an \emph{action macro}, whereby functional programs could themselves be lifted to the level of actions to compute non-trivial compound actions. 

\subsubsection{Views}
Another research direction is in exploring how types can be used to control the presentation of expressions in the editor. For example, following an approach developed in a textual setting of developing \emph{type-specific languages} (TSLs), it should be possible to have the type that an expression is being analyzed against define alternative display forms and interaction modes \cite{TSLs}.

It should also be possible to develop a semantics of semantic comments, i.e. comments that mention semantic structures. These would be subject to the same operations, e.g. renaming, as other structures, helping to address the problem of comments becoming inconsistent with code.

\subsubsection{Collaborative Programming}
Finally, we did not consider any aspects of \emph{collaborative programming}, such as a packaging system, a differencing algorithm for use in a source control system, support for multiple simultaneous focii for different users, and so on. These are all interesting avenues for future work.

\subsubsection{Empirical Evaluation}
Although we make few empirical claims in this paper, it is ultimately an empirical question as to whether structure editors, and typed structure editors, are practical. We hope to conduct user studies once a richer semantics has been developed.

\subsubsection{More Theory}
Connections with gradual type systems and CMTT, discussed in the previous section, seem likely to continue to be revealing. 

The notion of having one of many possible locations within a term under a cursor has a very strong intuitive connection to the proof theoretic notion of focusing \cite{Simmons11tr}. Building closer connections with proof theory (and category theory) is likely to be a fruitful avenue of further inquiry. 





% We already discussed a connection to gradual typing \cite{Siek06a}. We hope to explore this connection more thoroughly. In particular, it may be possible to better support exploratory and live programming by allowing even programs with holes in them to execute as long as those holes are only in the type portions, by deferring to the semantics given in work on gradual typing.

% It may also be possible to give a dynamics to incomplete expressions. Prior work on staged evaluation suggests that there may be a connection to modal logic, viewing holes as quantifying over all possible terms that may fill them \cite{DBLP:journals/jacm/DaviesP01}. In developing a dynamic semantics, we will also need to handle terms like $\hhole{\hehole}$ and
% $\hhole{\hhole{\hexp}}$. In our semantics given here, we eliminated them as they came up in a somewhat \emph{ad hoc} manner. We have not yet
% explored an equational theory for terms with holes, but intend to once our
% formalization effort is more mature.\todo{fix bib}

% \todo{new action form that makes actions extensible given proofs of
% admissiblity of a derived form, like prevSib}

\begin{quote}
\emph{In any case, these are but steps toward more graphical program-description
systems, for we will not forever stay confined to mere strings of symbols.}

--- Marvin Minsky, Turing Award lecture \cite{DBLP:journals/jacm/Minsky70}
\end{quote}

\clearpage
% We recommend abbrvnat bibliography style.
\bibliographystyle{abbrvnat}

% The bibliography should be embedded for final submission.
\bibliography{bibliography}
%\softraggedright
%P. Q. Smith, and X. Y. Jones. ...reference text...

\iftr
\clearpage
%\onecolumn
\appendix
% !TEX root = hazelnut-popl17.tex

\section{Hazelnut}
The full collection of rules defining the semantics of Hazelnut are reproduced here in their definitional order for reference.
\subsection{H-Types and H-Expressions}
\subsubsection{Type Compatibility and Incompatibility}

\noindent\fbox{$\tcompat{\htau}{\htau'}$}
\begin{subequations}%%\label{rules:tcompat}
% \begin{equation}%%\label{rule:tcompat-comm}
% \inferrule
% %[TCSym]
% {
%   \tcompat{\htau}{\htau'}
% }{
%   \tcompat{\htau'}{\htau}
% }
% \end{equation}
\begin{equation}
\inferrule{ }{
	\tcompat{\tehole}{\htau}
}
\end{equation}
\begin{equation}%%\label{rule:tcompat-hole}
\inferrule{ }{
  \tcompat{\htau}{\tehole}
}
\end{equation}
\begin{equation}%%\label{rule:tcompat-num}
\inferrule{ }{
  \tcompat{\htau}{\htau}
}
\end{equation}
\begin{equation}%%\label{rule:tcompat-arr}
\inferrule{
  \tcompat{\htau_1}{\htau_1'}\\
  \tcompat{\htau_2}{\htau_2'}
}{
  \tcompat{\tarr{\htau_1}{\htau_2}}{\tarr{\htau_1'}{\htau_2'}}
}
\end{equation}
\end{subequations}

\noindent\fbox{$\tincompat{\htau}{\htau'}$}
\begin{subequations}
  % \begin{equation}
  %   \inferrule{
  %     \tincompat{\htau}{\htau'}
  %   }{
  %     \tincompat{\htau'}{\htau}
  %   }
  % \end{equation}
  \begin{equation}
  	\inferrule{ }{
  		\tincompat{\tarr{\htau_1}{\htau_2}}{\tnum}
  	}
  \end{equation}
  \begin{equation}
    \inferrule{ }{
      \tincompat{\tnum}{\tarr{\htau_1}{\htau_2}}
    }
  \end{equation}
  \begin{equation}
    \inferrule{
      \tincompat{\htau_1}{\htau_1'}
    }{
      \tincompat{\tarr{\htau_1}{\htau_2}}{\tarr{\htau_1'}{\htau_2'}}
    }
  \end{equation}
  \begin{equation}
    \inferrule{
      \tincompat{\htau_2}{\htau_2'}
    }{
      \tincompat{\tarr{\htau_1}{\htau_2}}{\tarr{\htau_1'}{\htau_2'}}
    }
  \end{equation}
\end{subequations}

\subsubsection{Function Type Matching}~

% \noindent\fbox{$\tcompat{\htau}{\htau'}$}~~\text{$\tau$ and $\tau'$ are consistent}
% % \begin{subequations}%%\label{rules:tcompat}
% % \begin{equation}%%\label{rule:tcompat-comm}
% % \inferrule
% % %[TCSym]
% % {
% %   \tcompat{\htau}{\htau'}
% % }{
% %   \tcompat{\htau'}{\htau}
% % }
% % \end{equation}
% \begin{equation}
% \inferrule{ }{
%   \tcompat{\tehole}{\htau}
% }
% \end{equation}
% \begin{equation}%%\label{rule:tcompat-hole}
% \inferrule{ }{
%   \tcompat{\htau}{\tehole}
% }

% \inferrule{ }{
%   \tcompat{\htau}{\htau}
% }

% \inferrule{
%   \tcompat{\htau_1}{\htau_1'}\\
%   \tcompat{\htau_2}{\htau_2'}
% }{
%   \tcompat{\tarr{\htau_1}{\htau_2}}{\tarr{\htau_1'}{\htau_2'}}
% }
% \end{mathpar}
% \begin{equation}%%\label{rule:tcompat-num}
% \end{equation}
% \begin{equation}%%\label{rule:tcompat-arr}
% \end{equation}
% \end{subequations}
\noindent
\fbox{$\arrmatch{\htau}{\tarr{\htau_1}{\htau_2}}$}~~\text{$\tau$ has matched arrow type $\tarr{\htau_1}{\htau_2}$}
% \begin{mathpar}
\begin{subequations}
\begin{equation}
\inferrule{ }{
  \arrmatch{\tarr{\htau_1}{\htau_2}}{\tarr{\htau_1}{\htau_2}}
}
\end{equation}
\begin{equation}
\inferrule{ }{
  \arrmatch{\tehole}{\tarr{\tehole}{\tehole}}
}
\end{equation}
\end{subequations}
% \end{mathpar}
% \noindent\fbox{$\tincompat{\htau}{\htau'}$}
% \begin{subequations}
%   % \begin{equation}
%   %   \inferrule{
%   %     \tincompat{\htau}{\htau'}
%   %   }{
%   %     \tincompat{\htau'}{\htau}
%   %   }
%   % \end{equation}
%   \begin{equation}
%     \inferrule{ }{
%       \tincompat{\tarr{\htau_1}{\htau_2}}{\tnum}
%     }
%   \end{equation}
%   \begin{equation}
%     \inferrule{ }{
%       \tincompat{\tnum}{\tarr{\htau_1}{\htau_2}}
%     }
%   \end{equation}
%   \begin{equation}
%     \inferrule{
%       \tincompat{\htau_1}{\htau_1'}
%     }{
%       \tincompat{\tarr{\htau_1}{\htau_2}}{\tarr{\htau_1'}{\htau_2'}}
%     }
%   \end{equation}
%   \begin{equation}
%     \inferrule{
%       \tincompat{\htau_2}{\htau_2'}
%     }{
%       \tincompat{\tarr{\htau_1}{\htau_2}}{\tarr{\htau_1'}{\htau_2'}}
%     }
%   \end{equation}
% \end{subequations}

\subsubsection{Synthesis and Analysis}
The judgements $\hsyn{\hGamma}{\hexp}{\htau}$ and
$\hana{\hGamma}{\hexp}{\htau}$ are defined mutually inductively by Rules
(\ref{Arules:hana}) and Rules (\ref{Arules:hsyn}), respectively.

\noindent\fbox{$\hana{\hGamma}{\hexp}{\htau}$}~~\text{$\hexp$ analyzes against $\htau$}
\begin{subequations}\label{Arules:hana}
\begin{equation}%\label{rule:ana-subsume}
\inferrule{
  \hsyn{\hGamma}{\hexp}{\htau'}\\
  \tcompat{\htau}{\htau'}
}{
  \hana{\hGamma}{\hexp}{\htau}
}
\end{equation}
\begin{equation}%\label{rule:syn-lam}
\inferrule{
  \arrmatch{\htau}{\tarr{\htau_1}{\htau_2}}\\
  \hana{\hGamma, x : \htau_1}{\hexp}{\htau_2}
}{
  \hana{\hGamma}{\hlam{x}{\hexp}}{\htau}
}
\end{equation}
\end{subequations}
\fbox{$\hsyn{\hGamma}{\hexp}{\htau}$}~~\text{$\hexp$ synthesizes $\htau$}
\begin{subequations}\label{Arules:hsyn}
\begin{equation}%\label{rule:syn-asc}
\inferrule{
  \hana{\hGamma}{\hexp}{\htau}
}{
  \hsyn{\hGamma}{\hexp : \htau}{\htau}
}
\end{equation}
\begin{equation}%\label{rule:syn-var}
\inferrule{ }{
  \hsyn{\hGamma, x : \htau}{x}{\htau}
}
\end{equation}
\begin{equation}%\label{rule:syn-ap}
\inferrule{
  \hsyn{\hGamma}{\hexp_1}{\htau}\\
  \arrmatch{\htau}{\tarr{\htau_2}{\htau'}}\\
  \hana{\hGamma}{\hexp_2}{\htau_2}
}{
  \hsyn{\hGamma}{\hap{\hexp_1}{\hexp_2}}{\htau'}
}
\end{equation}
\begin{equation}%\label{rule:syn-num}
\inferrule{ }{
  \hsyn{\hGamma}{\hnum{n}}{\tnum}
}
\end{equation}
\begin{equation}%\label{rule:syn-plus}
\inferrule{
  \hana{\hGamma}{\hexp_1}{\tnum}\\
  \hana{\hGamma}{\hexp_2}{\tnum}
}{
  \hsyn{\hGamma}{\hadd{\hexp_1}{\hexp_2}}{\tnum}
}
\end{equation}
\begin{equation}%\label{rule:syn-ehole}
\inferrule{ }{
  \hsyn{\hGamma}{\hehole}{\tehole}
}
\end{equation}
\begin{equation}%\label{rule:syn-hole}
\inferrule{
  \hsyn{\hGamma}{\hexp}{\htau}
}{
  \hsyn{\hGamma}{\hhole{\hexp}}{\tehole}
}
\end{equation}
\end{subequations}

\subsubsection{Complete H-Types and H-Expressions}
By convention, we use the metavariable $\tau$ rather than $\htau$ for
complete H-types, and $e$ rather than $\hexp$ for complete H-expressions.

\noindent\fbox{$\hcomplete{\tau}$}
\begin{subequations}
\begin{equation}
\inferrule{
  \hcomplete{\tau_1}\\
  \hcomplete{\tau_2}
}{
  \hcomplete{\tarr{\tau_1}{\tau_2}}
}
\end{equation}
\begin{equation}
\inferrule{ }{
  \hcomplete{\tnum}
}
\end{equation}
\end{subequations}

\noindent\fbox{$\hcomplete{e}$}
\begin{subequations}
\begin{equation}
  \inferrule{
    \hcomplete{\hexp}\\
    \hcomplete{\htau}
  }{
    \hcomplete{\hexp : \htau}
  }
\end{equation}
\begin{equation}
  \inferrule{ }{
    \hcomplete{x}
  }
\end{equation}
\begin{equation}
  \inferrule{
    \hcomplete{\hexp}
  }{
    \hcomplete{\hlam{x}{\hexp}}
  }
\end{equation}
\begin{equation}
  \inferrule{
    \hcomplete{\hexp_1}\\
    \hcomplete{\hexp_2}
  }{
    \hcomplete{\hap{\hexp_1}{\hexp_2}}
  }
\end{equation}
\begin{equation}
  \inferrule{ }{\hcomplete{\hnum{n}}}
\end{equation}
\begin{equation}
  \inferrule{
    \hcomplete{\hexp_1}\\
    \hcomplete{\hexp_2}
  }{
    \hcomplete{\hadd{\hexp_1}{\hexp_2}}
  }
\end{equation}
\end{subequations}

\subsection{Z-Types and Z-Expressions}
\subsubsection{Type Focus Erasure}
\noindent\fbox{$\removeSel{\ztau}=\htau$} is a metafunction defined as follows:
\begin{subequations}
\begin{align}
%\removeSel{(\zlsel{\htau})} & = \htau\\
\removeSel{(\zwsel{\htau})} & = \htau\\
%\removeSel{(\zrsel{\htau})} & = \htau\\
\removeSel{(\tarr{\ztau}{\htau})} & = \tarr{\removeSel{\ztau}}{\htau}\\
\removeSel{(\tarr{\htau}{\ztau})} & = \tarr{\htau}{\removeSel{\ztau}}
\end{align}
\end{subequations}

\subsubsection{Expression Focus Erasure}
\noindent\fbox{$\removeSel{\zexp}=\hexp$} is a metafunction defined as follows:
\begin{subequations}
\begin{align}
%\removeSel{(\zlsel{\hexp})} & = \hexp\\
\removeSel{(\zwsel{\hexp})} & = \hexp\\
%\removeSel{(\zrsel{\hexp})} & = \hexp\\
\removeSel{(\zexp : \htau)} & = \removeSel{\zexp} : \htau\\
\removeSel{(\hexp : \ztau)} & = \hexp : \removeSel{\ztau}\\
\removeSel{(\hlam{x}{\zexp})} & = \hlam{x}{\removeSel{\zexp}}\\
\removeSel{(\hap{\zexp}{\hexp})} & = \hap{\removeSel{\zexp}}{\hexp}\\
\removeSel{(\hap{\hexp}{\zexp})} & = \hap{\hexp}{\removeSel{\zexp}}\\
\removeSel{(\hadd{\zexp}{\hexp})} & = \hadd{\removeSel{\zexp}}{\hexp}\\
\removeSel{(\hadd{\hexp}{\zexp})} & = \hadd{\hexp}{\removeSel{\zexp}}\\
\removeSel{\hhole{\zexp}} &= \hhole{\removeSel{\zexp}}
\end{align}
\end{subequations}
\subsection{Action Model}
\subsubsection{Type Actions}
\noindent\fbox{$\performTyp{\ztau}{\alpha}{\ztau'}$}
\paragraph{Type Movement}
\begin{subequations}
\begin{equation}
  \inferrule{ }{
    \performTyp{
      \zwsel{\tarr{\htau_1}{\htau_2}}
    }{
      \aMove{\dChild}
    }{
      \tarr{\zwsel{\htau_1}}{\htau_2}
    }
  }
\end{equation}
\begin{equation}
  \inferrule{ }{
    \performTyp{
      \tarr{\zwsel{\htau_1}}{\htau_2}
    }{
      \aMove{\dParent}
    }{
      \zwsel{\tarr{\htau_1}{\htau_2}}
    }
  }
\end{equation}
\begin{equation}
  \inferrule{ }{
    \performTyp{
      \tarr{{\htau_1}}{\zwsel{\htau_2}}
    }{
      \aMove{\dParent}
    }{
      \zwsel{\tarr{\htau_1}{\htau_2}}
    }
  }
\end{equation}
\begin{equation}
  \inferrule{ }{
    \performTyp{
      \tarr{\zwsel{\htau_1}}{{\htau_2}}
    }{
      \aMove{\dNext}
    }{
      {\tarr{\htau_1}{\zwsel{\htau_2}}}
    }
  }
\end{equation}
% \begin{equation}
%   \inferrule{ }{
%     \performTyp{
%       \tarr{{\htau_1}}{\zwsel{\htau_2}}
%     }{
%       \aMove{\dPrev}
%     }{
%       {\tarr{\zwsel{\htau_1}}{{\htau_2}}}
%     }
%   }
% \end{equation}

\paragraph{Type Deletion}
\begin{equation}
  \inferrule{ }{
    \performTyp{
      \zwsel{\htau}
    }{
      \aDel
    }{
      \zwsel{\tehole}
    }
  }
\end{equation}

\paragraph{Type Construction}
\begin{equation}
    %%\label{r:contarr}
  \inferrule{ }{
    \performTyp{
      \zwsel{\htau}
    }{
      \aConstruct{\farr}
    }{
      \tarr{\htau}{\zwsel{\tehole}}
    }
  }
\end{equation}

  \begin{equation}
    %%\label{r:contnum}
  \inferrule{ }{
    \performTyp{
      \zwsel{\tehole}
    }{
      \aConstruct{\fnum}
    }{
      \zwsel{\tnum}
    }
  }
\end{equation}


\paragraph{Zipper Cases}
  \begin{equation}
    %%\label{r:contarrL}
  \inferrule{
    \performTyp{\ztau}{\alpha}{\ztau'}
  }{
    \performTyp{
      \tarr{\ztau}{\htau}
    }{
      \alpha
    }{
      \tarr{\ztau'}{\htau}
    }
  }
\end{equation}
  \begin{equation}
    %%\label{r:contarrR}
  \inferrule{
    \performTyp{\ztau}{\alpha}{\ztau'}
  }{
    \performTyp{
      \tarr{\htau}{\ztau}
    }{
      \alpha
    }{
      \tarr{\htau}{\ztau'}
    }
  }
\end{equation}
\end{subequations}

\subsubsection{Expression Movement Actions} 
\noindent\fbox{$\performMove{\zexp}{\aMove{\delta}}{\zexp'}$}

\begin{subequations}
\paragraph{Ascription}

\begin{equation}
  \inferrule{ }{
    \performMove{
      \zwsel{\hexp : \htau}
    }{
      \aMove{\dChild}
    }{
      \zwsel{\hexp} : \htau
    }
  }
\end{equation}
\begin{equation}
  \inferrule{ }{
    \performMove{
      \zwsel{\hexp} : \htau
    }{
      \aMove{\dParent}
    }{
      \zwsel{\hexp : \htau}
    }
  }
\end{equation}
\begin{equation}
  \inferrule{ }{
    \performMove{
      \hexp : \zwsel{\htau}
    }{
      \aMove{\dParent}
    }{
      \zwsel{\hexp : \htau}
    }
  }
\end{equation}
\begin{equation}
  \inferrule{ }{
    \performMove{
      \zwsel{\hexp} : \htau
    }{
      \aMove{\dNext}
    }{
      \hexp : \zwsel{\htau}
    }
  }
\end{equation}
% \begin{equation}
%   \inferrule{ }{
%     \performMove{
%       \hexp : \zwsel{\htau}
%     }{
%       \aMove{\dPrev}
%     }{
%       \zwsel{\hexp} : \htau
%     }
%   }
% \end{equation}
% \begin{equation}
% \inferrule{
%   \performMove{
%     \zexp
%   }{
%     \aMove{\delta}
%   }{
%     \zexp'
%   }
% }{
%   \performMove{
%     \zexp : \htau
%   }{
%     \aMove{\delta}
%   }{
%     \zexp' : \htau
%   }
% }
% \end{equation}
% \begin{equation}
%   \inferrule{
%     \performMove{
%       \ztau
%     }{
%       \aMove{\delta}
%     }{
%       \ztau'
%     }
%   }{
%     \performMove{
%       \hexp : \ztau
%     }{
%       \aMove{\delta}
%     }{
%       \hexp : \ztau'
%     }
%   }
% \end{equation}

\paragraph{Lambda}
\begin{equation}%\label{r:movefirstchild-lam}
\inferrule{ }{
  \performMove{
    \zwsel{\hlam{x}{\hexp}}
  }{
    \aMove{\dChild}
  }{
    \hlam{x}{\zwsel{\hexp}}
  }
}
\end{equation}
\begin{equation}
  \inferrule{ }{
    \performMove{
      \hlam{x}{\zwsel{\hexp}}
    }{
      \aMove{\dParent}
    }{
      \zwsel{\hlam{x}{\hexp}}
    }
  }
\end{equation}
\paragraph{Application}
\begin{equation}
  \inferrule{ }{
    \performMove{
      \zwsel{\hap{\hexp_1}{\hexp_2}}
    }{
      \aMove{\dChild}
    }{
      \hap{\zwsel{\hexp_1}}{\hexp_2}
    }
  }
\end{equation}
\begin{equation}
  \inferrule{ }{
    \performMove{
      \hap{\zwsel{\hexp_1}}{\hexp_2}
    }{
      \aMove{\dParent}
    }{
      \zwsel{\hap{\hexp_1}{\hexp_2}}
    }
  }
\end{equation}
\begin{equation}%\label{r:moveparent-ap2}
  \inferrule{ }{
    \performMove{
      \hap{{\hexp_1}}{\zwsel{\hexp_2}}
    }{
      \aMove{\dParent}
    }{
      \zwsel{\hap{\hexp_1}{\hexp_2}}
    }
  }
\end{equation}
\begin{equation}
  \inferrule{ }{
    \performMove{
      \hap{\zwsel{\hexp_1}}{\hexp_2}
    }{
      \aMove{\dNext}
    }{
      \hap{\hexp_1}{\zwsel{\hexp_2}}
    }
  }
\end{equation}
% \begin{equation}
%   \inferrule{ }{
%     \performMove{
%       \hap{\hexp_1}{\zwsel{\hexp_2}}
%     }{
%       \aMove{\dPrev}
%     }{
%       \hap{\zwsel{\hexp_1}}{\hexp_2}
%     }
%   }
% \end{equation}

\paragraph{Plus}
\begin{equation}
  \inferrule{ }{
    \performMove{
      \zwsel{\hadd{\hexp_1}{\hexp_2}}
    }{
      \aMove{\dChild}
    }{
      \hadd{\zwsel{\hexp_1}}{\hexp_2}
    }
  }
\end{equation}
\begin{equation}
  \inferrule{ }{
    \performMove{
      \hadd{\zwsel{\hexp_1}}{\hexp_2}
    }{
      \aMove{\dParent}
    }{
      \zwsel{\hadd{\hexp_1}{\hexp_2}}
    }
  }
\end{equation}
\begin{equation}
  \inferrule{ }{
    \performMove{
      \hadd{{\hexp_1}}{\zwsel{\hexp_2}}
    }{
      \aMove{\dParent}
    }{
      \zwsel{\hadd{\hexp_1}{\hexp_2}}
    }
  }
\end{equation}
\begin{equation}
  \inferrule{ }{
    \performMove{
      \hadd{\zwsel{\hexp_1}}{\hexp_2}
    }{
      \aMove{\dNext}
    }{
      \hadd{\hexp_1}{\zwsel{\hexp_2}}
    }
  }
\end{equation}
% \begin{equation}
%   \inferrule{ }{
%     \performMove{
%       \hadd{\hexp_1}{\zwsel{\hexp_2}}
%     }{
%       \aMove{\dPrev}
%     }{
%       \hadd{\zwsel{\hexp_1}}{\hexp_2}
%     }
%   }
% \end{equation}

\paragraph{Non-Empty Hole}
\begin{equation}
\inferrule{ }{
  \performMove{
    \zwsel{\hhole{\hexp}}
  }{
    \aMove{\dChild}
  }{
    \hhole{\zwsel{\hexp}}
  }
}
\end{equation}
\begin{equation}%\label{r:moveparent-hole}
  \inferrule{ }{
    \performMove{
      \hhole{\zwsel{\hexp}}
    }{
      \aMove{\dParent}
    }{
      \zwsel{\hhole{\hexp}}
    }
  }
\end{equation}

\end{subequations}
\subsubsection{Synthetic and Analytic Expression Actions}
The synthetic and analytic expression action performance judgements are defined mutually inductively by Rules (\ref{Arules:performSyn}) and Rules (\ref{Arules:performAna}), respectively.


\noindent\fbox{$\performSyn{\hGamma}{\zexp}{\htau}{\alpha}{\zexp'}{\htau'}$}

\begin{subequations}\label{Arules:performSyn}
\paragraph{Movement}
\begin{equation}
\inferrule{
  \performMove{\zexp}{\aMove{\delta}}{\zexp'}
}{
  \performSyn{\hGamma}{\zexp}{\htau}{\aMove{\delta}}{\zexp'}{\htau}
}
\end{equation}

\paragraph{Deletion}
\begin{equation}
  \inferrule{ }{
    \performSyn{\hGamma}{\zwsel{\hexp}}{\htau}{\aDel}{\zwsel{\hehole}}{\tehole}
  }
\end{equation}

\paragraph{Construction}
\begin{equation}
  \inferrule{ }{
    \performSyn{\hGamma}{\zwsel{\hexp}}{\htau}{\aConstruct{\fasc}}{\hexp : \zwsel{\htau}}{\htau}
  }
\end{equation}

\begin{equation}
  \inferrule{ }{
    \performSyn{\hGamma, x : \htau}{\zwsel{\hehole}}{\tehole}{\aConstruct{\fvar{x}}}{\zwsel{x}}{\htau}
  }
\end{equation}

\begin{equation}
  \inferrule{ }{
    \performSyn
      {\hGamma}
      {\zwsel{\hehole}}
      {\tehole}
      {\aConstruct{\flam{x}}}
      {\hlam{x}{\hehole} : \tarr{\zwsel{\tehole}}{\tehole}}
      {\tarr{\tehole}{\tehole}}
  }
\end{equation}

\begin{equation}
  %\label{r:coneapfn}
  \inferrule{
    \arrmatch{\htau}{\tarr{\htau_1}{\htau_2}}
  }{
    \performSyn
      {\hGamma}
      {\zwsel{\hexp}}
      {\htau}
      {\aConstruct{\fap}}
      {\hap{\hexp}{\zwsel{\hehole}}}
      {\htau_2}
  }
\end{equation}

% \begin{equation}
%   \inferrule{ }{
%     \performSyn
%       {\hGamma}
%       {\zwsel{\hexp}}
%       {\tehole}
%       {\aConstruct{\fap}}
%       {\hap{\hexp}{\zwsel{\hehole}}}
%       {\tehole}
%   }
% \end{equation}

\begin{equation}
  \inferrule{
    \tincompat{\htau}{\tarr{\tehole}{\tehole}}
  }{
    \performSyn
      {\hGamma}
      {\zwsel{\hexp}}
      {\htau}
      {\aConstruct{\fap}}
      {\hap{\hhole{\hexp}}{\zwsel{\hehole}}}
      {\tehole}
  }
\end{equation}

\begin{equation}
  \inferrule{ }{
    \performSyn
      {\hGamma}
      {\zwsel{\hexp}}
      {\htau}
      {\aConstruct{\farg}}
      {\hap{\zwsel{\hehole}}{\hexp}}
      {\tehole}
  }
\end{equation}

\begin{equation}
  \inferrule{ }{
    \performSyn
      {\hGamma}
      {\zwsel{\hehole}}
      {\tehole}
      {\aConstruct{\fnumlit{n}}}
      {\zwsel{\hnum{n}}}
      {\tnum}
  }
\end{equation}

\begin{equation}
  \inferrule{
    \tcompat{\htau}{\tnum}
  }{
    \performSyn
      {\hGamma}
      {\zwsel{\hexp}}
      {\htau}
      {\aConstruct{\fplus}}
      {\hadd{\hexp}{\zwsel{\hehole}}}
      {\tnum}
  }
\end{equation}

\begin{equation}
  \inferrule{
    \tincompat{\htau}{\tnum}
  }{
    \performSyn
      {\hGamma}
      {\zwsel{\hexp}}
      {\htau}
      {\aConstruct{\fplus}}
      {\hadd{\hhole{\hexp}}{\zwsel{\hehole}}}
      {\tnum}
  }
\end{equation}

\begin{equation}
\inferrule{ }{
  \performSyn
    {\hGamma}
    {\zwsel{\hexp}}
    {\htau}
    {\aConstruct{\fnehole}}
    {\hhole{\zwsel{\hexp}}}
    {\tehole}
}
\end{equation}
\paragraph{Finishing}
  \begin{equation}
    %% %\label{r:finishana} %% TODO; get labels right
  \inferrule{
    \hsyn{\hGamma}{\hexp}{\htau'}
  }{
    \performSyn
      {\hGamma}
      {\zwsel{\hhole{\hexp}}}
      {\tehole}
      {\aFinish}
      {\zwsel{\hexp}}
      {\htau'}
  }
\end{equation}


\paragraph{Zipper Cases}
\begin{equation}
\inferrule{
  \performAna
    {\hGamma}
    {\zexp}
    {\htau}
    {\alpha}
    {\zexp'}
}{
  \performSyn
    {\hGamma}
    {\zexp : \htau}
    {\htau}
    {\alpha}
    {\zexp' : \htau}
    {\htau}
}
\end{equation}
\begin{equation}
\inferrule{
  \performTyp{\ztau}{\alpha}{\ztau'}\\
  \hana{\hGamma}{\hexp}{\removeSel{\ztau'}}
}{
  \performSyn
    {\hGamma}
    {\hexp : \ztau}
    {\removeSel{\ztau}}
    {\alpha}
    {\hexp : \ztau'}
    {\removeSel{\ztau'}}
}
\end{equation}
\begin{equation}
  \inferrule{
    \hsyn{\hGamma}{\removeSel{\zexp}}{\htau_2}\\
    \performSyn
      {\hGamma}
      {\zexp}
      {\htau_2}
      {\alpha}
      {\zexp'}
      {\htau_3}\\\\
    \arrmatch{\htau_3}{\tarr{\htau_4}{\htau_5}}\\
    \hana{\hGamma}{\hexp}{\htau_4}
  }{
    \performSyn
      {\hGamma}
      {\hap{\zexp}{\hexp}}
      {\htau_1}
      {\alpha}
      {\hap{\zexp'}{\hexp}}
      {\htau_5}
  }
\end{equation}
\begin{equation}
  \inferrule{
    \hsyn{\hGamma}{\hexp}{\htau_2}\\
    \arrmatch{\htau_2}{\tarr{\htau_3}{\htau_4}}\\
    \performAna
      {\hGamma}
      {\zexp}
      {\htau_3}
      {\alpha}
      {\zexp'}
  }{
    \performSyn
      {\hGamma}
      {\hap{\hexp}{\zexp}}
      {\htau_1}
      {\alpha}
      {\hap{\hexp}{\zexp'}}
      {\htau_4}
  }
\end{equation}

\begin{equation}
  \inferrule{
    \performAna
      {\hGamma}
      {\zexp}
      {\tnum}
      {\alpha}
      {\zexp'}
  }{
    \performSyn
      {\hGamma}
      {\hadd{\zexp}{\hexp}}
      {\tnum}
      {\alpha}
      {\hadd{\zexp'}{\hexp}}
      {\tnum}
  }
\end{equation}

\begin{equation}
  \inferrule{
    \performAna
      {\hGamma}
      {\zexp}
      {\tnum}
      {\alpha}
      {\zexp'}
  }{
    \performSyn
      {\hGamma}
      {\hadd{\hexp}{\zexp}}
      {\tnum}
      {\alpha}
      {\hadd{\hexp}{\zexp'}}
      {\tnum}
  }
\end{equation}

\begin{equation}
  \inferrule{
    \hsyn{\hGamma}{\removeSel{\zexp}}{\htau}\\
    \performSyn
      {\hGamma}
      {\zexp}
      {\htau}
      {\alpha}
      {\zexp'}
      {\htau'}
  }{
    \performSyn
      {\hGamma}
      {\hhole{\zexp}}
      {\tehole}
      {\alpha}
      {\hhole{\zexp'}}
      {\tehole}
  }
\end{equation}
% \begin{equation}
%   \inferrule{
%     \hsyn{\hGamma}{\removeSel{\zexp}}{\htau}\\
%     \performSyn
%       {\hGamma}
%       {\zexp}
%       {\htau}
%       {\alpha}
%       {\zwsel{\hehole}}
%       {\tehole}\\
%   }{
%     \performSyn
%       {\hGamma}
%       {\hhole{\zexp}}
%       {\tehole}
%       {\alpha}
%       {\zwsel{\hehole}}
%       {\tehole}
%   }
% \end{equation}
\end{subequations}

\noindent\fbox{$\performAna{\hGamma}{\zexp}{\htau}{\alpha}{\zexp'}$}
\begin{subequations}\label{Arules:performAna}
\paragraph{Subsumption}
\begin{equation}
  \inferrule{
    \hsyn{\hGamma}{\removeSel{\zexp}}{\htau'}\\
    \performSyn{\hGamma}{\zexp}{\htau'}{\alpha}{\zexp'}{\htau''}\\
    \tcompat{\htau}{\htau''}%\\\\
    % \alpha \neq \aConstruct{\fasc}\\
    % \alpha \neq \aConstruct{\flam{x}}
  }{
    \performAna{\hGamma}{\zexp}{\htau}{\alpha}{\zexp'}
  }
\end{equation}

\paragraph{Movement}
\begin{equation}
  \inferrule{
  \performMove{\zexp}{\aMove{\delta}}{\zexp'}
}{
  \performAna{\hGamma}{\zexp}{\htau}{\aMove{\delta}}{\zexp'}
}
\end{equation}

\paragraph{Deletion}
\begin{equation}
  \inferrule{ }{
    \performAna{\hGamma}{\zwsel{\hexp}}{\htau}{\aDel}{\zwsel{\hehole}}
  }
\end{equation}

\paragraph{Construction}
\begin{equation}
  \inferrule{ }{
    \performAna{\hGamma}{\zwsel{\hexp}}{\htau}{\aConstruct{\fasc}}{\hexp : \zwsel{\htau}}
  }
\end{equation}

\begin{equation}
  \inferrule{
    \tincompat{\htau}{\htau'}
  }{
    \performAna{\hGamma, x : \htau'}{\zwsel{\hehole}}{\htau}{\aConstruct{\fvar{x}}}{\hhole{\zwsel{x}}}
  }
\end{equation}

\begin{equation}%\label{rule:performAna-lam-1}
  \inferrule{
    \arrmatch{\htau}{\tarr{\htau_1}{\htau_2}}
  }{
    \performAna
      {\hGamma}
      {\zwsel{\hehole}}
      {\htau}
      {\aConstruct{\flam{x}}}
      {\hlam{x}{\zwsel{\hehole}}}
  }
\end{equation}

\begin{equation}
  \inferrule{
    \tincompat{\htau}{\tarr{\tehole}{\tehole}}
  }{
    \performAna
      {\hGamma}
      {\zwsel{\hehole}}
      {\htau}
      {\aConstruct{\flam{x}}}
      {\hhole{
        \hlam{x}{\hehole} : \tarr{\zwsel{\tehole}}{\tehole}
      }}
  }
\end{equation}
\begin{equation}
  \inferrule{
    \tincompat{\htau}{\tnum}
  }{
    \performAna
      {\hGamma}
      {\zwsel{\hehole}}
      {\htau}
      {\aConstruct{\fnumlit{n}}}
      {\hhole{\zwsel{\hnum{n}}}}
  }
\end{equation}
\paragraph{Finishing}
\begin{equation}
  \inferrule{
    \hana{\hGamma}{\hexp}{\htau}
  }{
    \performAna
      {\hGamma}
      {\zwsel{\hhole{\hexp}}}
      {\htau}
      {\aFinish}
      {\zwsel{\hexp}}
  }
\end{equation}

\paragraph{Zipper Cases}
\begin{equation}
\inferrule{
  \arrmatch{\htau}{\tarr{\htau_1}{\htau_2}}\\
  \performAna
    {\hGamma, x : \htau_1}
    {\zexp}
    {\htau_2}
    {\alpha}
    {\zexp'}
}{
  \performAna
    {\hGamma}
    {\hlam{x}{\zexp}}
    {\htau}
    {\alpha}
    {\hlam{x}{\zexp'}}
}
\end{equation}

\end{subequations}
\subsubsection{Iterated Action Judgements} ~

\noindent $\mathsf{ActionList}$~~$\bar{\alpha} ::= \cdot ~\vert~ \alpha; \bar{\alpha}$\vspace{4px}\\
\fbox{$\performTypI{\ztau}{\bar{\alpha}}{\ztau'}$}
\begin{subequations}
\begin{equation}
\inferrule{ }{
    \performTypI{\ztau}{\cdot}{\ztau}
}
\end{equation}
\begin{equation}
\inferrule{
  \performTyp{\ztau}{\alpha}{\ztau'}\\
  \performTypI{\ztau'}{\bar{\alpha}}{\ztau''}
}{
  \performTypI{\ztau}{\alpha; \bar{\alpha}}{\ztau''}
}
\end{equation}
\end{subequations}
\begin{subequations}
\fbox{$\performSynI{\hGamma}{\zexp}{\htau}{\bar{\alpha}}{\zexp'}{\htau'}$}
\begin{equation}
\inferrule{ }{
  \performSynI{\hGamma}{\zexp}{\htau}{\cdot}{\zexp}{\htau}
}
\end{equation}
\begin{equation}
\inferrule{
  \performSyn{\hGamma}{\zexp}{\htau}{\alpha}{\zexp'}{\htau'}\\
  \performSynI{\hGamma}{\zexp'}{\htau'}{\bar{\alpha}}{\zexp''}{\htau''}
}{
  \performSynI{\hGamma}{\zexp}{\htau}{\alpha; \bar{\alpha}}{\zexp''}{\htau''}
}
\end{equation}
\end{subequations}
\begin{subequations}
\fbox{$\performAnaI{\hGamma}{\zexp}{\htau}{\bar{\alpha}}{\zexp'}$}
\begin{equation}
\inferrule{ }{
  \performAnaI{\hGamma}{\zexp}{\htau}{\cdot}{\zexp}
}
\end{equation}
\begin{equation}
\inferrule{
  \performAna{\hGamma}{\zexp}{\htau}{\alpha}{\zexp'}\\
  \performAnaI{\hGamma}{\zexp'}{\htau}{\bar\alpha}{\zexp''}
}{
  \performAnaI{\hGamma}{\zexp}{\htau}{\alpha; \bar\alpha}{\zexp''}
}
\end{equation}
\end{subequations}
\noindent \fbox{$\bar\alpha~\mathsf{movements}$}
\begin{subequations}
\begin{equation}
\inferrule{ }{
	\cdot~\mathsf{movements}
}
\end{equation}

\begin{equation}
\inferrule{
	\bar\alpha~\mathsf{movements}
}{
	\aMove{\delta}; \bar\alpha~\mathsf{movements}
}
\end{equation}
\end{subequations}

\else
% No Appendix Here
\fi

\end{document}
