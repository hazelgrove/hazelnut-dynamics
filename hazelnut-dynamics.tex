\PassOptionsToPackage{svgnames,dvipsnames,svgnames}{xcolor}
%% For double-blind review submission
\documentclass[acmlarge,review,anonymous]{acmart}\settopmatter{printfolios=true}
%% For single-blind review submission
%\documentclass[acmlarge,review]{acmart}\settopmatter{printfolios=true}
%% For final camera-ready submission
%\documentclass[acmlarge]{acmart}\settopmatter{}

%% Note: Authors migrating a paper from PACMPL format to traditional
%% SIGPLAN proceedings format should change 'acmlarge' to
%% 'sigplan,10pt'.


%% Some recommended packages.
\usepackage{booktabs}   %% For formal tables:
                        %% http://ctan.org/pkg/booktabs
\usepackage{subcaption} %% For complex figures with subfigures/subcaptions
                        %% http://ctan.org/pkg/subcaption

%% Cyrus packages
\usepackage{microtype}
\usepackage{todonotes}
\usepackage{mdframed}
\usepackage{colortab}
\usepackage{mathpartir}
\usepackage{enumitem}
\usepackage{bbm}

%% Joshua Dunfield macros
\def\OPTIONConf{1}%
\usepackage{joshuadunfield}


% \newtheorem{theorem}{Theorem}[chapter]
% \newtheorem{lemma}[theorem]{Lemma}
% \newtheorem{corollary}[theorem]{Corollary}
% \newtheorem{definition}[theorem]{Definition}
\newtheorem{assumption}[theorem]{Assumption}
\newtheorem{condition}[theorem]{Condition}


%% Listings
\makeatletter\if@ACM@journal\makeatother
%% Journal information (used by PACMPL format)
%% Supplied to authors by publisher for camera-ready submission
\acmJournal{PACMPL}
\acmVolume{1}
\acmNumber{1}
\acmArticle{1}
\acmYear{2017}
\acmMonth{1}
\acmDOI{10.1145/nnnnnnn.nnnnnnn}
\startPage{1}
\else\makeatother
%% Conference information (used by SIGPLAN proceedings format)
%% Supplied to authors by publisher for camera-ready submission
\acmConference[PL'17]{ACM SIGPLAN Conference on Programming Languages}{January 01--03, 2017}{New York, NY, USA}
\acmYear{2017}
\acmISBN{978-x-xxxx-xxxx-x/YY/MM}
\acmDOI{10.1145/nnnnnnn.nnnnnnn}
\startPage{1}
\fi


%% Copyright information
%% Supplied to authors (based on authors' rights management selection;
%% see authors.acm.org) by publisher for camera-ready submission
\setcopyright{none}             %% For review submission
%\setcopyright{acmcopyright}
%\setcopyright{acmlicensed}
%\setcopyright{rightsretained}
%\copyrightyear{2017}           %% If different from \acmYear


%% Bibliography style
\bibliographystyle{ACM-Reference-Format}
%% Citation style
%% Note: author/year citations are required for papers published as an
%% issue of PACMPL.
\citestyle{acmauthoryear}   %% For author/year citations

% !TEX root = hazelnut-dynamics.tex

% Violet hotdogs; highlight color helps distinguish them
\newcommand{\llparenthesiscolor}{\textcolor{violet}{\llparenthesis}}
\newcommand{\rrparenthesiscolor}{\textcolor{violet}{\rrparenthesis}}
% \newcommand{\llparenthesiscolor}{\textcolor{red}{\lfloor}}
% \newcommand{\rrparenthesiscolor}{\textcolor{red}{\rfloor}}

%% TODO if feeling really obsessive, use the following two in place of x and u
\newcommand{\varVar}{x}
\newcommand{\varHole}{u}

% HTyp and HExp
\newcommand{\hcomplete}[1]{#1~\mathsf{complete}}

% HTyp
\newcommand{\htau}{\tau}
\newcommand{\tarr}[2]{#1 \rightarrow #2}
\newcommand{\tprod}[2]{#1 \times #2}
\newcommand{\tnum}{\texttt{num}}
\newcommand{\tb}{\texttt{b}}
\newcommand{\tehole}{\llparenthesiscolor\rrparenthesiscolor}
\newcommand{\tsum}[2]{{#1} + {#2}}

\newcommand{\tconsistent}[2]{#1 \sim #2}
\newcommand{\tinconsistent}[2]{#1 \nsim #2}

% HExp
\newcommand{\hexp}{e}
\newcommand{\hlam}[2]{\lambda #1.#2}
\newcommand{\halam}[3]{\lambda #1{:}#2.#3}
\newcommand{\hap}[2]{#1(#2)}
\newcommand{\hapP}[2]{(#1)~(#2)} % Extra paren around function term
\newcommand{\hpair}[2]{(#1, #2)}
\newcommand{\hprj}[2]{\mathsf{prj}_{#1}(#2)}
\newcommand{\lblL}{\mathsf{L}}
\newcommand{\lblR}{\mathsf{R}}
\newcommand{\hnum}[1]{\underline{#1}}
\newcommand{\hadd}[2]{#1 + #2}
\newcommand{\hehole}[1]{\llparenthesiscolor\rrparenthesiscolor^{#1}}
% \newcommand{\hhole}[1]{\setlength{\fboxsep}{0pt}\fcolorbox{red}{white}{\vphantom{)}$#1$}}
\newcommand{\hhole}[2]{\llparenthesiscolor#1\rrparenthesiscolor^{#2}}
% \newcommand{\hhole}[1]{
  % \setlength{\fboxsep}{0pt}
  % \colorbox{violet!10!white!100}{\ensuremath{\llparenthesiscolor#1\rrparenthesiscolor}}}
\newcommand{\hindet}[1]{\lceil#1\rceil}
\newcommand{\hinj}[2]{\texttt{inj}_{#1}({#2})}
\newcommand{\hcase}[5]{\texttt{case}({#1},{#2}.{#3},{#4}.{#5})}

\newcommand{\hGamma}{\Gamma}
\newcommand{\domof}[1]{\text{dom}(#1)}
\newcommand{\hsyn}[3]{#1 \vdash #2 \Rightarrow #3}
\newcommand{\hana}[3]{#1 \vdash #2 \Leftarrow #3}

% ZTyp and ZExp
\newcommand{\zlsel}[1]{{\bowtie}{#1}}
\newcommand{\zrsel}[1]{{#1}{\bowtie}}

%\newcommand{\zwsel}[1]{\adjustbox{cframe=blue}{\ensuremath{{\textcolor{blue}{\triangleright}}{#1}{\textcolor{blue}{\triangleleft}}}}}
\newcommand{\zwsel}[1]{
  \setlength{\fboxsep}{0pt}
  \colorbox{green!10!white!100}{
    \ensuremath{{{\textcolor{Green}{{\hspace{-2px}\triangleright}}}}{#1}{\textcolor{Green}{\triangleleft{\vphantom{\tehole}}}}}}
}
%\newcommand{\zwsel}[1]{{\triangleright}{#1}{\triangleleft}}

\newcommand{\removeSel}[1]{#1^{\diamond}}

% ZTyp
\newcommand{\ztau}{\hat{\tau}}

% ZExp
\newcommand{\zexp}{\hat{e}}

% Direction
\newcommand{\dParent}{\mathtt{parent}}
\newcommand{\dChild}{\mathtt{firstChild}}
\newcommand{\dNext}{\mathtt{nextSib}}
\newcommand{\dPrev}{\mathtt{prevSib}}

% Action
\newcommand{\aMove}[1]{\mathtt{move}~#1}
	\newcommand{\zrightmost}[1]{\mathsf{rightmost}(#1)}
	\newcommand{\zleftmost}[1]{\mathsf{leftmost}(#1)}
\newcommand{\aSelect}[1]{\mathtt{sel}~#1}
\newcommand{\aDel}{\mathtt{del}}
\newcommand{\aReplace}[1]{\mathtt{replace}~#1}
\newcommand{\aConstruct}[1]{\mathtt{construct}~#1}
\newcommand{\aConstructx}[1]{#1}
\newcommand{\aFinish}{\mathtt{finish}}

\newcommand{\performAna}[5]{#1 \vdash #2 \xlongrightarrow{#4} #5 \Leftarrow #3}
\newcommand{\performAnaI}[5]{#1 \vdash #2 \xlongrightarrow{#4}\hspace{-3px}{}^{*}~ #5 \Leftarrow #3}
\newcommand{\performSyn}[6]{#1 \vdash #2 \Rightarrow #3 \xlongrightarrow{#4} #5 \Rightarrow #6}
\newcommand{\performSynI}[6]{#1 \vdash #2 \Rightarrow #3 \xlongrightarrow{#4}\hspace{-3px}{}^{*}~ #5 \Rightarrow #6}
\newcommand{\performTyp}[3]{#1 \xlongrightarrow{#2} #3}
\newcommand{\performTypI}[3]{#1 \xlongrightarrow{#2}\hspace{-3px}{}^{*}~#3}

\newcommand{\performMove}[3]{#1 \xlongrightarrow{#2} #3}
\newcommand{\performDel}[2]{#1 \xlongrightarrow{\aDel} #2}

% Form
\newcommand{\farr}{\mathtt{arrow}}
\newcommand{\fnum}{\mathtt{num}}
\newcommand{\fsum}{\mathtt{sum}}

\newcommand{\fasc}{\mathtt{asc}}
\newcommand{\fvar}[1]{\mathtt{var}~#1}
\newcommand{\flam}[1]{\mathtt{lam}~#1}
\newcommand{\fap}{\mathtt{ap}}
\newcommand{\farg}{\mathtt{arg}}
\newcommand{\fnumlit}[1]{\mathtt{lit}~#1}
\newcommand{\fplus}{\mathtt{plus}}
\newcommand{\fhole}{\mathtt{hole}}
\newcommand{\fnehole}{\mathtt{nehole}}

\newcommand{\finj}[1]{\mathtt{inj}~#1}
\newcommand{\fcase}[2]{\mathtt{case}~#1~#2}

% Talk about formal rules in example
\newcommand{\refrule}[1]{\textrm{Rule~(#1)}}

\newcommand{\herase}[1]{\left|#1\right|_\textsf{erase}}

\newcommand{\arrmatch}[2]{#1 \blacktriangleright_{\rightarrow} #2}
\newcommand{\prodmatch}[2]{#1 \blacktriangleright_{\times} #2}
\newcommand{\summatch}[2]{#1 \blacktriangleright_{+} #2}


\newcommand{\TABperformAna}[5]{#1 \vdash & #2                & \xlongrightarrow{#4} & #5 & \Leftarrow #3}
\newcommand{\TABperformSyn}[6]{#1 \vdash & #2 \Rightarrow #3 & \xlongrightarrow{#4} & #5 \Rightarrow #6}
\newcommand{\TABperformTyp}[3]{& #1 & \xlongrightarrow{#2} & #3}

\newcommand{\TABperformMove}[3]{#1 & \xlongrightarrow{#2} & #3}
\newcommand{\TABperformDel}[2]{#1 \xlongrightarrow{\aDel} #2}

\newcommand{\sumhasmatched}[2]{#1 \mathrel{\textcolor{black}{\blacktriangleright_{+}}} #2}

%%%% DYNAMICS %%%%
%% TODO remove these macros
%% marks for eval
\newcommand{\unevaled}{\times}
\newcommand{\evaled}{\checkmark}
\newcommand{\markname}{m}

\newcommand{\mvar}[0]{u}
\newcommand{\subst}[0]{\sigma}
\newcommand{\dexp}[0]{d}
\newcommand{\dconst}[0]{c}
\newcommand{\dval}[0]{\ddot{v}}
%% TODO remove this macro
\newcommand{\dcast}[2]{\langle #1 \rangle ~ #2}
%% TODO make the following two look better
\newcommand{\dcasttwo}[3]{#1 \langle{#2}\Rightarrow{#3}\rangle}
\newcommand{\dcastthree}[4]{#1 \langle{#2}\Rightarrow{#3}\Rightarrow{#4}\rangle}
\newcommand{\dcastfail}[4]{#1 \langle{#2}\Rightarrow{#3}\not\Rightarrow{#4}\rangle}
\newcommand{\dlam}[3]{\lambda #1:#2.#3}
\newcommand{\dap}[2]{#1(#2)}
\newcommand{\dapP}[2]{(#1)(#2)} % Extra paren around function term
\newcommand{\dnum}[1]{\underline{#1}}
\newcommand{\dadd}[2]{#1 + #2}
\newcommand{\dehole}[3]{\leftidx{^{#3}}{\llparenthesiscolor\rrparenthesiscolor}{^{#1}_{#2}}}
\newcommand{\dhole}[4]{\leftidx{^{#4}}{\llparenthesiscolor#1\rrparenthesiscolor}{^{#2}_{#3}}}
\newcommand{\dindet}[1]{\lceil#1\rceil}
\newcommand{\dinj}[2]{\texttt{inj}_{#1}({#2})}
\newcommand{\dcase}[5]{\texttt{case}({#1},{#2}.{#3},{#4}.{#5})}

\newcommand{\expandAna}[6]{#1 \vdash #2 \Leftarrow #3 \leadsto #4 : #5 \dashv #6}
\newcommand{\expandSyn}[5]{#1 \vdash #2 \Rightarrow #3 \leadsto #4 \dashv #5}
\newcommand{\hasType}[4]{#1; #2 \vdash #3 : #4}
\newcommand{\isValue}[1]{#1~\mathsf{val}}
\newcommand{\isIndet}[1]{#1~\mathsf{indet}}
\newcommand{\isFinal}[1]{#1~\mathsf{final}}
\newcommand{\isErr}[2]{#1 \vdash #2~\mathsf{err}}
%% \newcommand{\stepsTo}[2]{#1 \mapsto_{\Delta} #2}
\newcommand{\stepsToD}[3]{#1 \vdash #2 \mapsto #3}

%% TODO if feeling obsessive, replace direct uses of \Delta
\newcommand{\hDelta}{\Delta}
\newcommand{\Dunion}[2]{#1 \cup #2}
\newcommand{\idof}[1]{\mathsf{id}(#1)}
\newcommand{\Dbinding}[3]{#1 :: [#2]#3}

% Contextual dynamics
\newcommand{\evalctx}{\mathcal{E}}
\newcommand{\evalhole}{\circ}
\newcommand{\isevalctx}[1]{#1~\mathsf{evalCtx}}
\newcommand{\reducesE}[3]{#1 \vdash #2 \longrightarrow #3}
\newcommand{\selectEvalCtxR}[2]{#1\{#2\}}
\newcommand{\selectEvalCtx}[3]{#1=\selectEvalCtxR{#2}{#3}}



\begin{document}

%% Title information
\title{Hazelnut Live: Type-Theoretic Foundations for Running Incomplete Programs}         %% [Short Title] is optional;
                                        %% when present, will be used in
                                        %% header instead of Full Title.
% \titlenote{with title note}             %% \titlenote is optional;
                                        %% can be repeated if necessary;
                                        %% contents suppressed with 'anonymous'
% \subtitle{Subtitle}                     %% \subtitle is optional
% \subtitlenote{with subtitle note}       %% \subtitlenote is optional;
                                        %% can be repeated if necessary;
                                        %% contents suppressed with 'anonymous'


%% Author information
%% Contents and number of authors suppressed with 'anonymous'.
%% Each author should be introduced by \author, followed by
%% \authornote (optional), \orcid (optional), \affiliation, and
%% \email.
%% An author may have multiple affiliations and/or emails; repeat the
%% appropriate command.
%% Many elements are not rendered, but should be provided for metadata
%% extraction tools.

%% Author with single affiliation.
\author{First1 Last1}
\authornote{with author1 note}          %% \authornote is optional;
                                        %% can be repeated if necessary
\orcid{nnnn-nnnn-nnnn-nnnn}             %% \orcid is optional
\affiliation{
  \position{Position1}
  \department{Department1}              %% \department is recommended
  \institution{Institution1}            %% \institution is required
  \streetaddress{Street1 Address1}
  \city{City1}
  \state{State1}
  \postcode{Post-Code1}
  \country{Country1}
}
\email{first1.last1@inst1.edu}          %% \email is recommended

%% Author with two affiliations and emails.
\author{First2 Last2}
\authornote{with author2 note}          %% \authornote is optional;
                                        %% can be repeated if necessary
\orcid{nnnn-nnnn-nnnn-nnnn}             %% \orcid is optional
\affiliation{
  \position{Position2a}
  \department{Department2a}             %% \department is recommended
  \institution{Institution2a}           %% \institution is required
  \streetaddress{Street2a Address2a}
  \city{City2a}
  \state{State2a}
  \postcode{Post-Code2a}
  \country{Country2a}
}
\email{first2.last2@inst2a.com}         %% \email is recommended
\affiliation{
  \position{Position2b}
  \department{Department2b}             %% \department is recommended
  \institution{Institution2b}           %% \institution is required
  \streetaddress{Street3b Address2b}
  \city{City2b}
  \state{State2b}
  \postcode{Post-Code2b}
  \country{Country2b}
}
\email{first2.last2@inst2b.org}         %% \email is recommended


%% Paper note
%% The \thanks command may be used to create a "paper note" ---
%% similar to a title note or an author note, but not explicitly
%% associated with a particular element.  It will appear immediately
%% above the permission/copyright statement.
% \thanks{with paper note}                %% \thanks is optional
                                        %% can be repeated if necesary
                                        %% contents suppressed with 'anonymous'


%% Abstract
%% Note: \begin{abstract}...\end{abstract} environment must come
%% before \maketitle command
\begin{abstract}
\todo{abstract}
\end{abstract}


%% 2012 ACM Computing Classification System (CSS) concepts
%% Generate at 'http://dl.acm.org/ccs/ccs.cfm'.
% \begin{CCSXML}
% <ccs2012>
% <concept>
% <concept_id>10011007.10011006.10011008</concept_id>
% <concept_desc>Software and its engineering~General programming languages</concept_desc>
% <concept_significance>500</concept_significance>
% </concept>
% <concept>
% <concept_id>10003456.10003457.10003521.10003525</concept_id>
% <concept_desc>Social and professional topics~History of programming languages</concept_desc>
% <concept_significance>300</concept_significance>
% </concept>
% </ccs2012>
% \end{CCSXML}

% \ccsdesc[500]{Software and its engineering~General programming languages}
% \ccsdesc[300]{Social and professional topics~History of programming languages}
%% End of generated code


%% Keywords
%% comma separated list
% \keywords{keyword1, keyword2, keyword3}  %% \keywords is optional


%% \maketitle
%% Note: \maketitle command must come after title commands, author
%% commands, abstract environment, Computing Classification System
%% environment and commands, and keywords command.
\maketitle

% !TEX root = hazelnut-popl17.tex

%% There are some benefits to this approach, to be sure, but the structural
%% mismatch between programs and their textual representations also imposes
%% various burdens.  For example, the primitive edit actions available in a
%% text editor (e.g. inserting or deleting a character or word) do not
%% always correspond to sensible structural transformations.

% spj: describe the problem; state our contributions; STOP. one page max.

% When constructing a program or proof in a language with rich type
% structure, skilled programmers generally follow a \emph{type discipline}
% where they first determine the type of the expression that they are
% constructing in order to constrain the mental search space that they are
% operating within.

% For example, if the programmer knows that an expression of type
% $\tarr{\tnum}{\tnum}$ is needed, then it is often the case (though, of course, not
% necessarily the case) that the expression will take the form $$\hlam{\mathit{x}}{e}$$
% for some variable $x$ and function body $e$. If the programmer chooses this
% form, then after picking a suitable variable name, her focus will be on
% constructing a suitable body, $e$. Following the type discipline, $e$ must
% be of type $\tnum$, and so this process can begin anew.

% The problem is that when using a text
% editor to construct a program, it is easy, and indeed necessary, to deviate from this disciplined process.  Rather, text editors operate on sequences of
% characters (i.e. \emph{text}.) 

%Although programs can be represented as text, most text does not correspond to a syntactically well-formed and semantically well-defined program. 
% Programming languages, and therefore programs themselves, are rich mathematical structures.
Programmers typically construct and manipulate well-typed expressions only indirectly, by editing text that is first parsed according to a textual syntax and then typechecked according to a static semantics. This indirection has practical benefits, to be sure -- text editors and other text-based tools benefit from decades of development effort -- but it also introduces some fundamental complexity into the programming process. 

% Every day, programmers use text-based tools to construct and manipulate programs.  
% These programs are written in languages which defined by a textual syntax. 
% While textual syntax has proved to have practical utility, it also introduces some fundamental complexity into the programming process.

% It also complicates matters for tool designers because tools are often confronted with text that does not correspond to a well-formed, meaningful program\todo{mention YoungSeok's data here}. This may be because the programmer is in the midst of a sequence of 
% edit actions that leaves the text temporarily malformed or ill-typed, or because the programmer has made a mistake. The language definition is silent about these situations, so it is difficult for tools to help programmers determine and execute a corrective course of action (e.g. by providing \todo{cite study on benefits of syntax highlighting}syntax highlighting and semantics-aware code completion services\todo{cite something?}.)
%Doesn't really start out with a bang, IMHO. I like to start papers out with a "big" problem.
%Most programming languages define a textual syntax. This allows programmers to use text editors and other standard text-based tools to construct and manipulate programs. While this is of substantial practical utility, it also introduces some fundamental complexity into the programming process. 

First, it requires that programmers learn the subtleties of the textual syntax (e.g. the precedence of the various forms.) This can be particularly challenging for novices \cite{Altadmri:2015:MCI:2676723.2677258}. %For example, one study of novice programmers found syntax errors (e.g. unbalanced parentheses) to be the most common class of errors.
% Too much detail? 

The fact that not every sequence of characters corresponds to a meaningful program also complicates matters for tool designers. In particular, program editors must contend with meaningless text on a regular basis. In a dataset analyzed by Yoon and Myers consisting of 1460 hours of edit logs, 44.2\% of edit states were syntactically malformed \cite{6883030}. Some additional percentage of edit states were well-formed but ill-typed (the dataset did not contain enough information to determine the exact percentage.)  
Some of these meaningless edit states arose because the programmer was in the midst of a sequence of edits that left the edit state transiently meaningless, while others arose because the programmer made a logical mistake. Because the language definitions is silent about these edit states, it is difficult to design useful editor services, e.g. syntax highlighting~\cite{sarkar2015impact}, type-aware code completion~\cite{Mooty:2010:CCC:1915084.1916348,Omar:2012:ACC:2337223.2337324}, and refactoring support \cite{mens2004survey}. Editors must either disable these editor services when they encounter meaningless edit states or develop \emph{ad hoc} heuristics, which can be misleading. %, or because the programmer has made a mistake. 
% The syntax definition is silent about these situations, so it is difficult for tools to help programmers determine and execute a corrective course of action 

% Many editors have developed \emph{ad hoc} workarounds for this problem, e.g. they might attempt to use regular expressions to highlight malformed program text, insert closing delimiters automatically, use whitespace to guess where a delimiter is likely to appear, or continue past a type error by pretending that the type was as expected (if, indeed, an expected type can be determined.) These heuristic methods are often complex, and they can confuse or mislead the programmer. % This may also help explain why the error messages emitted by parsers and typecheckers are often quite baroque\todo{citations}.

% stuck editing a representation of the program instead of the structures
% themselves. The editor does not restrict what the programmer may do: you
% can delete characters that belong, insert ones that don't, forget things
% that were needed, and so. There's nothing stopping us from accidentally
% writing $$\lambda \mathit{x:num}.\mathit{(x,x)}$$ even though it's obvious
% that building a pair can't hope to form a natural number.

% The type structure of the language makes this sort of error obvious: it's
% not that you're adding characters that make your program incorrect, or even
% malformed; you're adding characters that can't possibly create a structure
% you want because of the type. Simply put, the primitive operations
% available in text editors do not always correspond to sensible
% transformations on the structure of the program.
These complications have motivated a long line of research into \emph{structure editors}, i.e. program editors where every edit state is a program structure. % Eliminating text eliminates the possibility of syntax errors.

Most structure editors are \emph{syntactic structure editors}, i.e. the edit state is a syntax tree with \emph{holes} that stand for branches of the tree that have yet to be constructed, and the edit actions are context-free tree transformations. For example, Scratch is a syntactic structure editor that has achieved success as a tool for teaching children how to program \cite{Resnick:2009:SP:1592761.1592779}. 

The Scratch language has a trivial static semantics, but researchers have also designed syntactic structure editors for  languages with a non-trivial static semantics. For example, \texttt{mbeddr} is an editor for a C-like language \cite{voelter_mbeddr:_2012}, TouchDevelop is an editor for an object-oriented language \cite{tillmann_touchdevelop:_2011} and Lamdu is an editor for a functional language similar to Haskell \cite{lamdu}. Each of these editors presents an innovative user interface, but the non-trivial type and binding structure of the underlying language complicates its design. The reason is that a syntactic structure editor does not guarantee that every edit state is statically meaningful -- only that it is syntactically well-formed. As in textual program editors, these syntactic structure editors must either disable key editor services when they encounter meaningless edit states or deploy \emph{ad hoc} heuristics.
%Indeed, it is not entirely clear what it should mean for terms with holes to be statically meaningful. 
% Consequently, heuristics analagous to those that pervade textual program editors also pervade these syntactic structure editors.

% \begin{itemize}
% \item 

% First, there is an all too familiar problem: these languages do not have a clear formal semantics and metatheory, i.e. ``the implementation is the specification''. This makes it difficult to reason methodically about types and binding, so, in turn, it is difficult to design reliable editor services.

% Even if a formal semantics and metatheory for the underlying language were forthcoming (e.g. following Standard ML \cite{Harper00atype-theoretic,mthm97-for-dart}), it would go only part of the way towards addressing the challenges faced by structure editor designers. First, a standard semantics assigns no formal meaning to \emph{incomplete terms}, i.e. terms with holes. Moreover, a syntactic structure editor  -- so incorporating holes into the static semantics would in any case be only a minor salve. Heuristics analagous to those that pervade textual program editors would still be needed to contend with the ill-typed edit states.

This paper develops a principled solution to this problem. We introduce Hazelnut,  a \emph{typed structure editor}  based on a bidirectionally typed lambda calculus extended to assign static meaning to expressions and types with {holes}, which we call \textbf{H-expressions} and \textbf{H-types}. Hazelnut's formal \emph{action semantics} maintains the invariant that every edit state is a statically meaningful (i.e. well-typed) H-expression with a single superimposed \emph{cursor}. We call H-expressions and H-types with a cursor \textbf{Z-expressions} and \textbf{Z-types} (so prefixed because our encoding follows Huet's \emph{zipper} pattern \cite{JFP::Huet1997}.) %ctions act relative to the H-expression or H-type under the cursor.% (which need not be a hole.)  %More specifically, actions are context-aware,  acting on an expression whose type is determined by its surroundings (e.g. a function argument), only actions consistent with that type are permitted. 

Na\"ively, enforcing an injunction on ill-typed edit states would force programmers to construct programs in a rigid ``outside-in'' manner. For example, the programmer would often need to construct the outer function application form before identifying the intended function. To address this problem, Hazelnut leaves newly constructed expressions \emph{inside} a hole if the expression's type is inconsistent with the expected type. This meaningfully defers the type consistency check until the expression inside the hole is \emph{finished}. In other words, holes appear both at the leaves and at the internal nodes of an H-expression that remain under construction. %In short, Hazelnut is a \emph{bidirectionally typed structure editor calculus}. %Actions act at a programmer-indicate subtree, called the \emph{focus} (which is defined following Huet's zipper pattern.) 

% By defining \emph{holes} as a language construct, for both expressions as types, Hazelnut enables type-aware actions that always leave the program in both a structurally and semantically well-defined state. 
%In Hazelnut, expressions and types with \emph{holes} have a well-defined static semantics. Edit actions are type-aware and leave the program in both a structurally and semantically well-defined state. 



%However, these syntactically correct states can be semantically meaningless (i.e. undefined), because the language definitions generally only give meaning to complete, well-typed terms. (e.g., when a branching element is introduced, the program does not become semantically correct until both branches are complete)
% This makes it difficult for humans and tools to reason about types and binding during the development process, even when using a structure editor.
% Similar to text editors, structure editors also can develop workarounds to try to help users, but these efforts can only extend so far, because of the underlying language definition.
%Some structure editors attempt stuff, but it is not clear what invariants are being maintained (e.g. Unison ; Scratch seems not to allow literals of the wrong type, but variables are not typed.)\todo{revise} %More sophisticated semantic reasoning principles thus remain .


 



The remainder of this paper is organized as follows:
\begin{itemize}[itemsep=0px,partopsep=2px,topsep=2px]
\item We begin in Sec. 
    \ref{sec:example} with two examples of edit sequences to develop the reader's intuitions.  
\item We then give a detailed overview of Hazelnut's semantics and metatheory, and our mechanization in the Agda proof assistant, in Sec.  \ref{sec:hazel}.
\item Hazelnut is a {foundational} calculus, i.e. a calculus that language and editor designers are expected to extend with higher level constructs. We extend Hazelnut with simple sum types in Sec.  \ref{sec:extending} to demonstrate how Hazelnut's rich metatheory guides such a development. 
\item In Sec.  \ref{sec:impl}, we briefly describe how Hazelnut's action semantics lends itself to efficient implementation as a functional reactive program. Our reference implementation is written using \lstinline{js_of_ocaml} and the OCaml \lstinline{React} library.
 
% \item However, we consider the space of possible evaluation strategies for incomplete expressions in Section \ref{sec:dynamics}. 
% so doing, we discover interesting connections with gradual typing and contextual modal type theory. The former provides an interpretation of type holes, and the latter provides a logical interpretation of expression holes. \todo{revise this one} 

\item In Sec.  \ref{sec:rw}, we summarize related work. In particular, much of the technical machinery needed to handle type holes coincides with machinery developed in the study of gradual type systems. Similarly, expression holes can be interpreted as the closures of contextual modal type theory, which, by its correspondence with contextual modal logic, lays logical foundations beneath our work. %This suggests a principled design for an ``edit and resume'' feature.

\item We conclude in Sec.  \ref{sec:future} by summarizing our vision of a principled science of structure editor design rooted in type theory, and suggest a number of future directions.
\end{itemize} 
The supplemental material includes 1) a typeset listing of Hazelnut's rules in definitional order; 2) the formalization of Hazelnut in Agda; and 3) our reference implementation of Hazelnut, both in source form and pre-compiled to JavaScript for use in a web browser.
% !TEX root = hazelnut-dynamics.tex

\clearpage
\newcommand{\calculusSec}{Hazelnut Live, Formally}
\section{\protect\calculusSec}
\label{sec:calculus}

We will now make the intuitions developed in the previous section formally precise by specifying the \HazelnutLive core calculus and  its accompanying metatheory. 
We have mechanized these formal developments using the Agda proof assistant \cite{norell:thesis,norell2009dependently} 
(see Sec.~\ref{sec:agda-mechanization} for some additional details, and the supplemental material for the full mechanization).

The syntax of the core calculus, specified in Fig.~\ref{fig:hazelnut-live-syntax}, consists of types and expressions with holes. 
We distinguish between \emph{external} expressions, $e$, and \emph{internal} expressions, $d$. 
External expressions correspond to programs as entered by the programmer 
(see Sec.~\ref{sec:implementation}\todo{or maybe Sec. 2?}{} for discussion of manual, semi-automated and fully automated hole entry methods). 
Each well-typed external expression (as specified in Sec.~\ref{sec:external-statics} below) expands to a well-typed internal expression (see Sec.~\ref{sec:expansion}) before it is evaluated (see Sec.~\ref{sec:evaluation}). 
We distinguish the external and internal languages because 
(1) the external language supports type inference and explicit type ascriptions, $\hexp : \htau$, but it is  formally simpler to eliminate ascriptions and specify a type assignment system when defining the dynamic semantics\todo{cite frank's notes}; and 
(2) we need additional syntactic machinery during evaluation for tracking hole closures and dynamic type casts. 
This machinery is inserted by the expansion step, rather than entered explicitly by the programmer. 
In this regard, the internal language is analagous to the cast calculus in the gradually typed lambda calculus \cite{DBLP:conf/snapl/SiekVCB15,Siek06a}, though as we will see the \HazelnutLive internal language goes beyond the cast calculus in several respects.

\begin{figure}[t]
$\arraycolsep=4pt\begin{array}{rllllll}
\mathsf{HTyp} & \htau & ::= &
  b ~\vert~
  \tarr{\htau}{\htau} ~\vert~
  % \tprod{\htau}{\htau} ~\vert~
  % \tsum{\htau}{\htau} ~\vert~
  \tehole\\
\mathsf{HExp} & \hexp & ::= &
  c ~\vert~
  x ~\vert~
  \halam{x}{\htau}{\hexp} ~\vert~
  \hap{\hexp}{\hexp} ~\vert~
  % \hpair{\hexp}{\hexp} ~\vert~
  % \hprj{i}{\hexp} ~\vert~
  % \hinj{i}{\hexp} ~\vert~
  % \hcase{\hexp}{x}{\hexp}{x}{\hexp} ~\vert~
  % \hadd{\hexp}{\hexp} ~\vert~
  \hehole{u} ~\vert~
  \hhole{\hexp}{u} ~\vert~
  {\hlam{x}{\hexp}} ~\vert~
  \hexp : \htau\\
% \mathsf{Mark} & \markname{} & ::= &
%   \evaled{} ~\vert~  \unevaled{}\\
 \mathsf{DHExp} & \dexp  & ::= &
  c ~\vert~
  x ~\vert~
  {\halam{x}{\htau}{\dexp}} ~\vert~
  \hap{\dexp}{\dexp} ~\vert~
  % \hpair{\dexp}{\dexp} ~\vert~
  % \hprj{i}{\dexp} ~\vert~
  % \hinj{i}{\dexp} ~\vert~
  % \hcase{\dexp}{x}{\dexp}{x}{\dexp} ~\vert~
  % \hadd{\dexp}{\dexp} ~\vert~
  \dehole{\mvar}{\subst}{} ~\vert~
  \dhole{\dexp}{\mvar}{\subst}{} ~\vert~
  \dcasttwo{\dexp}{\htau}{\htau} ~\vert~
  \dcastfail{\dexp}{\htau}{\htau}\\
\end{array}$
\caption{Syntax of H-types, H-expressions and dynamic H-expressions.
We write $x$ to range over variables,
$u$ over hole names, and
$\sigma$ over finite substitutions (i.e., environments) from
variables to dynamic H-expressions.}
\label{fig:HTyp}
\label{fig:HExp}
\end{figure}


% \rkc{this syntactic sugar is used in four places: ITCastSucceed, ITCastFail,
% ITGround, and ITExpand. that's not many, and those rules don't look much more
% cluttered without the sugar, so consider eliminating it. if so, just toggle the
% definition of the dcastthree macro to the unsugared option.}

\subsection{Static Semantics of the External Language}
\label{sec:external-statics}

% !TEX root = hazelnut-dynamics.tex

\begin{figure}[t]
\judgbox{\hsyn{\hGamma}{\hexp}{\htau}}{$\hexp$ synthesizes type $\htau$}
\begin{mathpar}
\inferrule[SConst]{ }{
  \hsyn{\hGamma}{c}{b}
}

\inferrule[SVar]{
  x : \htau \in \hGamma
}{
  \hsyn{\hGamma}{x}{\htau}
}

\inferrule[SLam]{
  \hsyn{\hGamma, x : \htau_1}{\hexp}{\htau_2}
}{
  \hsyn{\hGamma}{\halam{x}{\htau_1}{\hexp}}{\tarr{\htau_1}{\htau_2}}
}

\inferrule[SAp]{
  \arrayenvbr{
    \hsyn{\hGamma}{\hexp_1}{\htau_1}    
    \\
    \hana{\hGamma}{\hexp_2}{\htau_2}
  }
  \\
    \arrmatch{\htau_1}{\tarr{\htau_2}{\htau}}
}{
  \hsyn{\hGamma}{\hap{\hexp_1}{\hexp_2}}{\htau}
}

\inferrule[SEHole]{ }{
  \hsyn{\hGamma}{\hehole{u}}{\tehole}
}

\inferrule[SNEHole]{
  \hsyn{\hGamma}{\hexp}{\htau}
}{
  \hsyn{\hGamma}{\hhole{\hexp}{u}}{\tehole}
}

\inferrule[SAsc]{
  \hana{\hGamma}{\hexp}{\htau}
}{
  \hsyn{\hGamma}{\hexp : \htau}{\htau}
}
\end{mathpar}

\vsepRule

\judgbox{\hana{\hGamma}{\hexp}{\htau}}{$\hexp$ analyzes against type $\htau$}
\begin{mathpar}
\inferrule[ALam]{
  \arrmatch{\htau}{\tarr{\htau_1}{\htau_2}}\\
  \hana{\hGamma, x : \htau_1}{\hexp}{\htau_2}
}{
  \hana{\hGamma}{\hlam{x}{\hexp}}{\htau}
}

\inferrule[ASubsume]{
  \hsyn{\hGamma}{\hexp}{\htau}\\
  \tconsistent{\htau}{\htau'}
}{
  \hana{\hGamma}{\hexp}{\htau'}
}
\end{mathpar}
\CaptionLabel{Bidirectional Typing of External Expressions}{fig:bidirectional-typing}
\end{figure}


We start with the type system of the \HazelnutLive
external language, which closely follows the \Hazelnut type
system \cite{popl-paper}; we discuss the minor differences as they come up below.

\Figref{fig:bidirectional-typing} defines the type system in the \emph{bidirectional} style 
%
with two mutually defined judgements \cite{Pierce:2000ve,bidi-tutorial,DBLP:conf/icfp/DunfieldK13,Chlipala:2005da}. The type synthesis
judgement~$\hsyn{\hGamma}{\hexp}{\htau}$ synthesizes a type~$\htau$
for external expression~$\hexp$ under typing context $\hGamma$, which tracks typing
assumptions of the form $x : \htau$ in the usual
manner \cite{pfpl,tapl}.
%
The type analysis judgement~$\hana{\hGamma}{\hexp}{\htau}$ checks
expression~$\hexp$ against a given type~$\htau$.
%
Algorithmically, analysis accepts a type as input, and synthesis gives
a type as output.
%
We start with synthesis for the programmer's ``top level'' external
exression.

% Algorithmically, the type is an output of type synthesis but an input of type analysis.

The primary benefit of specifying the \HazelnutLive external language 
bidirectionally is that the programmer does not need to annotate each hole with a type. 
%
An empty hole is
written $\hehole{u}$, where $u$ is a name for the hole, which we tacitly assume is unique. 
Holes did not have names in \Hazelnut. 
%
Rule {SEHole}\todo{rule name macros}{} specifies that an empty hole synthesizes hole type, written $\tehole$.
%
If an empty hole appears where an expression of some other type is
expected, e.g. under an explicit ascription (governed by Rule {SAsc})
or in the argument position of a function application (governed by
Rule {SAp}, discussed below), we apply the \emph{subsumption rule},
Rule {ASubsume}, which specifies that if an expression $e$ synthesizes
type $\htau$, then it may be checked against any \emph{consistent}
type, $\htau'$.

% !TEX root = hazelnut-dynamics.tex

\begin{figure}[t]
\judgbox{\tconsistent{\htau_1}{\htau_2}}{$\htau_1$ is consistent with $\htau_2$}
\begin{mathpar}
\inferrule[TCHole1]{ }{
  \tconsistent{\tehole}{\htau}
}

\inferrule[TCHole2]{ }{
  \tconsistent{\htau}{\tehole}
}

\inferrule[TCRefl]{ }{
  \tconsistent{\htau}{\htau}
}

\inferrule[TCArr]{
  \tconsistent{\htau_1}{\htau_1'}\\
  \tconsistent{\htau_2}{\htau_2'}
}{
  \tconsistent{\tarr{\htau_1}{\htau_2}}{\tarr{\htau_1'}{\htau_2'}}
}
%
% \inferrule{
%   \tconsistent{\htau_1}{\htau_1'}\\
%   \tconsistent{\htau_2}{\htau_2'}
% }{
%   \tconsistent{\tprod{\htau_1}{\htau_2}}{\tprod{\htau_1'}{\htau_2'}}
% }
%
% \inferrule{
%   \tconsistent{\htau_1}{\htau_1'}\\
%   \tconsistent{\htau_2}{\htau_2'}
% }{
%   \tconsistent{\tsum{\htau_1}{\htau_2}}{\tsum{\htau_1'}{\htau_2'}}
% }
\end{mathpar}

% \vsepRule

% \judgbox{\tinconsistent{\htau_1}{\htau_2}}{$\htau_1$ is inconsistent with $\htau_2$}
% \begin{mathpar}
%     \inferrule[ICBaseArr1]{ }{
%       \tinconsistent{\tb}{\tarr{\htau_1}{\htau_2}}
%     }

%     \inferrule[ICBaseArr2]{ }{
%       \tinconsistent{\tarr{\htau_1}{\htau_2}}{\tb}
%     }

%     \inferrule[ICArr1]{
%       \tinconsistent{\htau_1}{\htau_3}
%     }{
%       \tinconsistent{\tarr{\htau_1}{\htau_2}}{\tarr{\htau_3}{\htau_4}}
%     }

%     \inferrule[ICArr2]{
%       \tinconsistent{\htau_2}{\htau_4}
%     }{
%       \tinconsistent{\tarr{\htau_1}{\htau_2}}{\tarr{\htau_3}{\htau_4}}
%     }
% \end{mathpar}

\vsepRule

\judgbox{\arrmatch{\htau}{\tarr{\htau_1}{\htau_2}}}{$\htau$ has matched arrow type $\tarr{\htau_1}{\htau_2}$}
\begin{mathpar}
\inferrule[MAHole]{ }{
  \arrmatch{\tehole}{\tarr{\tehole}{\tehole}}
}

\inferrule[MAArr]{ }{
  \arrmatch{\tarr{\htau_1}{\htau_2}}{\tarr{\htau_1}{\htau_2}}
}
\end{mathpar}

% \judgbox{\prodmatch{\htau}{\tprod{\htau_1}{\htau_2}}}{$\htau$ has matched product type $\tprod{\htau_1}{\htau_2}$}
% \begin{mathpar}
% \inferrule{ }{
%   \prodmatch{\tehole}{\tprod{\tehole}{\tehole}}
% }

% \inferrule{ }{
%   \prodmatch{\tprod{\htau_1}{\htau_2}}{\tprod{\htau_1}{\htau_2}}
% }
% \end{mathpar}

% \judgbox{\summatch{\htau}{\tsum{\htau_1}{\htau_2}}}{$\htau$ has matched sum type $\tsum{\htau_1}{\htau_2}$}
% \begin{mathpar}
% \inferrule{ }{
%   \summatch{\tehole}{\tsum{\tehole}{\tehole}}
% }

% \inferrule{ }{
%   \summatch{\tsum{\htau_1}{\htau_2}}{\tsum{\htau_1}{\htau_2}}
% }
% \end{mathpar}
\caption{Type Consistency and Matching}
\Label{fig:tconsistent}
\Label{fig:arrmatch}
\end{figure}


Fig.~\ref{fig:tconsistent} specifies the type consistency relation, written $\tconsistent{\htau}{\htau'}$, which specifies that two types are consistent if they differ only up to type holes in corresponding positions.
%
The hole type is consistent with every type, and so, by the subsumption rule, expression holes may appear where an expression of any type is expected. The type consistency relation here coincides with the type consistency relation from gradual type theory by identifying the hole type with the unknown type~\cite{Siek06a}.
%
Type consistency is reflexive and symmetric, but it is \emph{not} transitive.
%
This relation stands in contrast to subtyping, which is anti-symmetric and transitive; subtyping may be integrated into a gradual type system following \citet{Siek:2007qy}).

Non-empty expression holes, written $\hhole{\hexp}{u}$, behave similarly to empty holes.
%
Rule {SNEHole} specifies that a non-empty expression hole also synthesizes hole type as long as the expression inside the hole, $\hexp$, synthesizes some (arbitrary) type.
%
Non-empty expression holes therefore internalize the ``red squiggles'' that many editors display under or around type inconsistencies in a program.\todo{example?}\todo{briefly say something about binding inconsistencies?}

For the familiar forms of the lambda calculus, the rules again follow prior work.
%
For simplicity, the core calculus includes only a single base type~$b$ with a single constant~$c$, governed by Rule {SConst} (i.e. $b$ is the unit type).
%
By contrast, \Hazelnut instead defines a number type with a single operation, which we include in Appendix~\ref{sec:extensions} alongside various other standard extensions to the core calculus\todo{do this, say more?}. 
%
Rule {SVar} specifies that variables synthesize the corresponding type from $\hGamma$. 

For the sake of exposition, \HazelnutLive includes ``half-annotated'' lambdas, $\halam{x}{\htau}{\hexp}$, in addition to the unannotated lambdas, $\hlam{x}{\hexp}$, from \Hazelnut.
%
Half-annotated lambdas may appear in synthetic position according to Rule {SLam}, which is standard \cite{Chlipala:2005da}.
%
Unannotated lambdas may only appear where the expected type is known to be either an arrow type or the hole type, which is treated as if it were $\tarr{\tehole}{\tehole}$.
%
To avoid the need for two separate rules, Rule {ALam} uses the auxiliary relation $\arrmatch{\htau}{\tarr{\htau_1}{\htau_2}}$ in \Figref{fig:arrmatch}, which produces the matched arrow type $\tarr{\tehole}{\tehole}$ given the hole type, and operates as the identity on arrow types \cite{DBLP:conf/snapl/SiekVCB15,DBLP:conf/popl/GarciaC15}.
%
\Secref{sec:related} dicusses how \HazelnutLive might be enriched with
with ML-style type reconstruction~\cite{damas1982principal}, perhaps via
the approach outlined by~\citet{DBLP:conf/icfp/DunfieldK13}.
%
%%%%%%%%%%%%%%%%%%%%%%%%%%%%%%%%%%%%%%%%%%%%%%%%%%%%%%%%%%%%%%%%%%%%%%%%%%%%%%%%%%%%%%%%%%%%%%%%%%%%%%%%%%%%%%%%%%%%%
% To Cyrus from Matt:
%
% Why say the following here? --- It's discussing a different design that we didn't pursue here.  We should move such discussion to related work.
%
%Note that a system supporting ML-style type reconstruction \cite{damas1982principal} might include a synthetic rule for unannotated lambdas, e.g. as outlined by \citet{DBLP:conf/icfp/DunfieldK13}.
%%

The rule governing function application, Rule {SAp}, similarly treats an expression of hole type in function position as if it were of type $\tarr{\tehole}{\tehole}$ using the same matched arrow type judgement.

\vspace{-4px}
\subsection{Expansion}
\label{sec:expansion}
\vspace{-1px}

% !TEX root = hazelnut-dynamics.tex

\begin{figure}[p]
\judgbox
  {\expandSyn{\hGamma}{\hexp}{\htau}{\dexp}{\Delta}}
  {$\hexp$ synthesizes type $\htau$ and expands to $\dexp$}
\begin{mathpar}
\inferrule[ESConst]{ }{
  \expandSyn{\hGamma}{c}{b}{c}{\emptyset}
}

\inferrule[ESVar]{
  x : \htau \in \hGamma
}{
  \expandSyn{\hGamma}{x}{\htau}{x}{\emptyset}
}

\inferrule[ESLam]{
  \expandSyn{\hGamma, x : \htau_1}{\hexp}{\htau_2}{\dexp}{\Delta}
}{
  \expandSyn{\hGamma}{\halam{x}{\htau_1}{\hexp}}{\tarr{\htau_1}{\htau_2}}{\halam{x}{\htau_1}{\dexp}}{\Delta}
}

\inferrule[ESAp]{
  \hsyn{\hGamma}{\hexp_1}{\htau_1}\\
  \arrmatch{\htau_1}{\tarr{\htau_2}{\htau}}
  \\\\
  \expandAna{\hGamma}{\hexp_1}{\tarr{\htau_2}{\htau}}{\dexp_1}{\htau_1'}{\Delta_1}\\
  \expandAna{\hGamma}{\hexp_2}{\htau_2}{\dexp_2}{\htau_2'}{\Delta_2}
}{
  \expandSyn
    {\hGamma}
    {\hap{\hexp_1}{\hexp_2}}
    {\htau}
    {\hap{(\dcasttwo{\dexp_1}{\htau_1'}{\tarr{\htau_2}{\htau}})}
         {\dcasttwo{\dexp_2}{\htau_2'}{\htau_2}}}
    {\Dunion{\Delta_1}{\Delta_2}}
}
%
%% \inferrule[ESAp1]{
%%   \hsyn{\hGamma}{\hexp_1}{\tehole}\\
%%   \expandAna{\hGamma}{\hexp_1}{\tarr{\htau_2}{\tehole}}{\dexp_1}{\htau_1}{\Delta_1}\\
%%   \expandAna{\hGamma}{\hexp_2}{\tehole}{\dexp_2}{\htau_2}{\Delta_2}
%% }{
%%   \expandSyn{\hGamma}{\hap{\hexp_1}{\hexp_2}}{\tehole}{\hap{(\dcast{\tarr{\htau_2}{\tehole}}{\dexp_1})}{\dexp_2}}{\Dunion{\Delta_1}{\Delta_2}}
%% }
%% 
%% \inferrule[ESAp2]{
%%   \expandSyn{\hGamma}{\hexp_1}{\tarr{\htau_2}{\htau}}{\dexp_1}{\Delta_1}\\
%%   \expandAna{\hGamma}{\hexp_2}{\htau_2}{\dexp_2}{\htau'_2}{\Delta_2}\\
%%   \htau_2 \neq \htau'_2
%% }{
%%   \expandSyn{\hGamma}{\hap{\hexp_1}{\hexp_2}}{\htau}{\hap{\dexp_1}{\dcast{\htau_2}{\dexp_2}}}{\Dunion{\Delta_1}{\Delta_2}}
%% }
%% 
%% \inferrule[ESAp3]{
%%   \expandSyn{\hGamma}{\hexp_1}{\tarr{\htau_2}{\htau}}{\dexp_1}{\Delta_1}\\
%%   \expandAna{\hGamma}{\hexp_2}{\htau_2}{\dexp_2}{\htau_2}{\Delta_2}
%% }{
%%   \expandSyn{\hGamma}{\hap{\hexp_1}{\hexp_2}}{\htau}{\hap{\dexp_1}{\dexp_2}}{\Dunion{\Delta_1}{\Delta_2}}
%% }\\
%
%
% \inferrule[expand-pair]{
%   \expandSyn{\hGamma}{\hexp_1}{\htau_1}{\dexp_1}{\Delta_1}\\
%   \expandSyn{\hGamma}{\hexp_2}{\htau_2}{\dexp_2}{\Delta_2}
% }{
%   \expandSyn{\hGamma}{\hpair{\hexp_1}{\hexp_2}}{\tprod{\htau_1}{\htau_2}}{\hpair{\dexp_1}{\dexp_2}}{\Dunion{\Delta_1}{\Delta_2}}
% }
%
% \inferrule[expand-prj]{
%   a
% }{
%   b
% }
%
% (inj)
%
%
% \inferrule[expand-plus]{ }{
%   \expandSyn{\hGamma}{\hadd{\hexp_1}{\hexp_2}}{\tnum}{\hadd{\dexp_1}{\dexp_2}}{\Dunion{\Delta_1}{\Delta_2}}
% }

\inferrule[ESEHole]{ }{
  \expandSyn{\hGamma}{\hehole{u}}{\tehole}{\dehole{u}{\idof{\hGamma}}{}}{\Dbinding{u}{\hGamma}{\tehole}}
}

\inferrule[ESNEHole]{
  \expandSyn{\hGamma}{\hexp}{\htau}{\dexp}{\Delta}
}{
  \expandSyn{\hGamma}{\hhole{\hexp}{u}}{\tehole}{\dhole{\dexp}{u}{\idof{\hGamma}}{}}{\Delta, \Dbinding{u}{\hGamma}{\tehole}}
}\\
%

\inferrule[ESAsc]{
  \expandAna{\hGamma}{\hexp}{\htau}{\dexp}{\htau'}{\Delta}
}{
  \expandSyn{\hGamma}{\hexp : \htau}{\htau}{\dcasttwo{\dexp}{\htau'}{\htau}}{\Delta}
}

%% \inferrule[ESAsc1]{
%%   \expandAna{\hGamma}{\hexp}{\htau}{\dexp}{\htau'}{\Delta}\\
%%   \htau \neq \htau'
%% }{
%%   \expandSyn{\hGamma}{\hexp : \htau}{\htau}{\dcast{\htau}{\dexp}}{\Delta}
%% }
%% 
%% \inferrule[ESAsc2]{
%%   \expandAna{\hGamma}{\hexp}{\htau}{\dexp}{\htau}{\Delta}
%% }{
%%   \expandSyn{\hGamma}{\hexp : \htau}{\htau}{\dexp}{\Delta}
%% }
\end{mathpar}

\vsepRule

\judgbox
  {\expandAna{\hGamma}{\hexp}{\htau_1}{\dexp}{\htau_2}{\Delta}}
  {$\hexp$ analyzes against type $\htau_1$ and
   expands to $\dexp$ of consistent type $\htau_2$}
\begin{mathpar}
\inferrule[EALam]{
  \arrmatch{\htau}{\tarr{\htau_1}{\htau_2}}\\
  \expandAna{\hGamma, x : \htau_1}{\hexp}{\htau_2}{\dexp}{\htau'_2}{\Delta}
}{
  \expandAna{\hGamma}{\hlam{x}{\hexp}}{\htau}{\halam{x}{\htau_1}{\dexp}}{\tarr{\htau_1}{\htau_2'}}{\Delta}
}

%% \inferrule[EALam]{
%%   \expandAna{\hGamma, x : \htau_1}{\hexp}{\htau_2}{\dexp}{\htau'_2}{\Delta}
%% }{
%%   \expandAna{\hGamma}{\hlam{x}{\hexp}}{\tarr{\htau_1}{\htau_2}}{\halam{x}{\htau_1}{\dexp}}{\tarr{\htau_1}{\htau_2'}}{\Delta}
%% }
%% 
%% \inferrule[EALamHole]{
%%   \expandAna{\hGamma, x : \tehole}{\hexp}{\tehole}{\dexp}{\htau}{\Delta}
%% }{
%%   \expandAna{\hGamma}{\hlam{x}{\hexp}}{\tehole}{\halam{x}{\tehole}{\dexp}}{\tarr{\tehole}{\htau}}{\Delta}
%% }
%% 
\inferrule[EASubsume]{
  \hexp \neq \hehole{u}\\
  \hexp \neq \hhole{\hexp'}{u}\\\\
  \expandSyn{\hGamma}{\hexp}{\htau'}{\dexp}{\Delta}\\
  \tconsistent{\htau}{\htau'}
}{
  \expandAna{\hGamma}{\hexp}{\htau}{\dexp}{\htau'}{\Delta}
}

\inferrule[EAEHole]{ }{
  \expandAna{\hGamma}{\hehole{u}}{\htau}{\dehole{u}{\idof{\hGamma}}{}}{\htau}{\Dbinding{u}{\hGamma}{\htau}}
}

\inferrule[EANEHole]{
  \expandSyn{\hGamma}{\hexp}{\htau'}{\dexp}{\Delta}\\
}{
  \expandAna{\hGamma}{\hhole{\hexp}{u}}{\htau}{\dhole{\dexp}{u}{\idof{\hGamma}}{}}{\htau}{\Delta, \Dbinding{u}{\hGamma}{\htau}}
}
\end{mathpar}
\caption{Expansion}
\label{fig:expansion}
\label{fig:expandSyn}
\label{fig:expandAna}
\end{figure}

% !TEX root = hazelnut-dynamics.tex

\begin{figure}[p]
\judgbox{\hasType{\Delta}{\hGamma}{\dexp}{\htau}}{$\dexp$ is assigned type $\htau$}
\begin{mathpar}
\inferrule[TAConst]{ }{
  \hasType{\Delta}{\hGamma}{c}{b}
}

\inferrule[TAVar]{
  x : \htau \in \hGamma
}{
	\hasType{\Delta}{\hGamma}{x}{\htau}
}

\inferrule[TALam]{
  \hasType{\Delta}{\hGamma, x : \htau_1}{\dexp}{\htau_2}
}{
  \hasType{\Delta}{\hGamma}{\halam{x}{\htau_1}{\dexp}}{\tarr{\htau_1}{\htau_2}}
}

\inferrule[TAAp]{
  \hasType{\Delta}{\hGamma}{\dexp_1}{\tarr{\htau_2}{\htau}}\\
  \hasType{\Delta}{\hGamma}{\dexp_2}{\htau_2}
}{
  \hasType{\Delta}{\hGamma}{\hap{\dexp_1}{\dexp_2}}{\htau}
}

\inferrule[TAEHole]{
  \Dbinding{u}{\hGamma'}{\htau} \in \Delta\\
  \hasType{\Delta}{\hGamma}{\sigma}{\hGamma'}
}{
  \hasType{\Delta}{\hGamma}{\dehole{u}{\sigma}{}}{\htau}
}

\inferrule[TANEHole]{
  \hasType{\Delta}{\hGamma}{\dexp}{\htau'}\\\\
  \Dbinding{u}{\hGamma'}{\htau} \in \Delta\\
  \hasType{\Delta}{\hGamma}{\sigma}{\hGamma'}
}{
  \hasType{\Delta}{\hGamma}{\dhole{\dexp}{u}{\sigma}{}}{\htau}
}

\inferrule[TACast]{
  \hasType{\Delta}{\Gamma}{\dexp}{\htau_1}\\
  \tconsistent{\htau_1}{\htau_2}
}{
  \hasType{\Delta}{\hGamma}{\dcasttwo{\dexp}{\htau_1}{\htau_2}}{\htau_2}
}

\inferrule[TAFailedCast]{
  \hasType{\Delta}{\Gamma}{\dexp}{\htau_1}\\
  \isGround{\htau_1}\\
  \isGround{\htau_2}\\
  \htau_1\neq\htau_2
}{
  \hasType{\Delta}{\hGamma}{\dcastfail{\dexp}{\htau_1}{\htau_2}}{\htau_2}
}
\end{mathpar}
\caption{Type Assignment for Internal Expressions}
\Label{fig:hasType}
\end{figure}


Each well-typed external expression~$e$ expands to a well-typed internal expression~$d$, for evaluation.
%
\Figref{fig:expansion} gives the rules governing expansion, and \Figref{fig:hasType} gives the rules governing type assignment for internal expressions.
%
\Secref{sec:evaluation} discusses internal expression evaluation.

As with the type system for the external language (above), 
we specify expansion in the bidirectional style, via two mutually defined judgements.
%
The synthetic expansion judgement~$\expandSyn{\hGamma}{\hexp}{\htau}{\dexp}{\Delta}$ synthesizes a type~$\htau$ from~$\hexp$.
%
Synthetic expansion produces an internal expression~$d$ and a hole context~$\hDelta$.
%
%We say more about hole contexts, which are used in the type assignment judgement, $\hasType{\Delta}{\hGamma}{d}{\htau}$, below.
We discuss hole contexts further below, including their role in the type assignment judgement~$\hasType{\Delta}{\hGamma}{d}{\htau}$.

The analytic expansion judgement~$\expandAna{\hGamma}{\hexp}{\htau}{\dexp}{\htau'}{\Delta}$, checks~$\hexp$ against~$\htau$ and produces an expansion~$d$ of type~$\htau'$, and a hole context~$\hDelta$.
%
The governing theorem below establishes that expansions are well-typed and in the analytic case that the type $\htau'$ is necessarily consistent with $\htau$.
%
% \rkc{minor, but the syntax of type assignment isn't mentioned before use. it's not
% entirely standard, because of the hole context, so perhaps worth mentioning.}
%
\begin{thm}[Typed Expansion]\label{thm:typed-expansion} ~
  \begin{enumerate}[nolistsep]
    \item
      If $\expandSyn{\hGamma}{\hexp}{\htau}{\dexp}{\Delta}$
      then $\hasType{\Delta}{\hGamma}{\dexp}{\htau}$.
    \item
      If $\expandAna{\hGamma}{\hexp}{\htau}{\dexp}{\htau'}{\Delta}$
      then $\tconsistent{\htau}{\htau'}$ and $\hasType{\Delta}{\hGamma}{\dexp}{\htau'}$.
  \end{enumerate}
\end{thm}
\noindent

%The reason analytic expansion produces an expansion of consistent
%type is because the subsumption rule, as previously discussed, allows
%us to check an external expression against any type consistent with
%the type the expression actually synthesizes, whereas every internal
%expression can be assigned at most one type, i.e. the following
%standard unicity property holds of the type assignment system.
%
Analytic expansion produces an expansion of consistent type:
%
Inuitively, the subsumption rule, as previously discussed, allows us
to check an external expression against any type consistent with the
type the expression actually synthesizes.


\begin{thm}[Type Assignment Unicity]
  If $\hasType{\Delta}{\hGamma}{\dexp}{\htau}$
  and $\hasType{\Delta}{\hGamma}{\dexp}{\htau'}$
  then $\htau=\htau'$.
\end{thm}
\noindent
Consequently, analytic expansion reports the type actually assigned to the expansion it produces.
%
For example, we can derive that $\expandAna{\hGamma}{c}{\tehole}{c}{b}{\emptyset}$.% where $\emptyset$ is the empty hole context.

Before describing the rules in detail, let us state a few other useful theorems. 
The following theorem establishes that an expansion exists for every well-typed external expression.
 \begin{thm}[Expandability] \label{thm:expandability}~
  \begin{enumerate}[nolistsep]
    \item
      If $\hsyn{\hGamma}{\hexp}{\htau}$
      then $\expandSyn{\hGamma}{\hexp}{\htau}{\dexp}{\Delta}$
      for some $\dexp$ and $\Delta$.
    \item
      If $\hana{\hGamma}{\hexp}{\htau}$
      then $\expandAna{\hGamma}{\hexp}{\htau}{\dexp}{\htau'}{\Delta}$
      for some $\dexp$ and $\htau'$ and $\Delta$.
  \end{enumerate}
\end{thm}
\noindent
The following theorem establishes that when an expansion exists, it is unique.
\begin{thm}[Expansion Unicity] \label{thm:expansion-unicity}~
  \begin{enumerate}[nolistsep]
    \item
      If $\expandSyn{\hGamma}{\hexp}{\htau}{\dexp}{\Delta}$
      and $\expandSyn{\hGamma}{\hexp}{\htau'}{\dexp'}{\Delta'}$
      then $\htau=\htau'$ and $\dexp=\dexp'$ and $\Delta=\Delta'$.
    \item
      If $\expandAna{\hGamma}{\hexp}{\htau_1}{\dexp}{\htau_2}{\Delta}$
      and $\expandAna{\hGamma}{\hexp}{\htau_1}{\dexp'}{\htau_2'}{\Delta'}$
      then $\dexp=\dexp'$ and $\htau_2=\htau_2'$ and $\Delta=\Delta'$.
  \end{enumerate}
\end{thm}
\noindent
The following theorem establishes that expansion generalizes external typing.\todo{rename correspondence to generality}
\begin{thm}[Expansion Generality] \label{thm:expansion-generality}~
  \begin{enumerate}[nolistsep]
    \item
      If $\expandSyn{\hGamma}{\hexp}{\htau}{\dexp}{\Delta}$
      then $\hsyn{\hGamma}{\hexp}{\htau}$.
    \item
      If $\expandAna{\hGamma}{\hexp}{\htau}{\dexp}{\htau'}{\Delta}$
      then $\hana{\hGamma}{\hexp}{\htau}$.
  \end{enumerate}
\end{thm}

%%%%%%%%%%%%%%%%%%%%%%%%%%%%%%%%%%%%%%%%%%%%%%%%%% <<<<<<<<<<<<< Matt is here (latest pass over everything above is done -- 2018.04.11)

The rules governing expansion of constants, variables and lambda expressions --- Rules {ESConst}, {ESVar}, {ESLam} and {EALam} --- and the corresponding type assignment rules --- Rules {TAConst}, {TAVar} and {TALam} --- mirror the typing rules from Fig.~\ref{fig:bidirectional-typing} 
(so the corresponding cases of Theorem~\ref{thm:typed-expansion}, Theorem~\ref{thm:expandability} and Theorem~\ref{thm:expansion-generality} are straightforward). 
Note that in the internal language, all lambdas are half-annotated, again to support type assignment---Rule {EALam} inserts the annotation when expanding an unannotated external lambda based on the given type. 
The rules governing holes and the rules governing function applications and ascriptions, which both perform \emph{cast insertion}, are more interesting, so let us consider both cases in turn.

\subsubsection{Hole Expansion}\label{sec:hole-expansion} 
Rules {ESEHole}, {ESNEHole}, {EAEHole} and {EANEHole} govern the expansion of empty and non-empty expression holes to empty and non-empty \emph{hole closures}, $\dehole{u}{\sigma}{}$ and $\dhole{\dexp}{u}{\sigma}{}$. 
%

The hole name, $u$, on a hole closure identifies the external hole that the hole closure corresponds to. 
Note that while we assume each hole name to be unique in the external language, there can be multiple hole closures with the same name during evaluation due to substitution. 
For example, the result from Fig.~\ref{fig:grades-example} showed four closures for the hole named 1\todo{update with actual number (four?) and hole name later}. 
There, we numbered each hole closure for a given hole sequentially, \li{1:1}, \li{1:2} and so on, but this is strictly for the sake of presentation so we omit hole closure numbers from the core calculus.

The hole expansion rules are the only rules that introduce hypotheses, of the form $\Dbinding{u}{\hGamma}{\htau}$, into the hole context, $\Delta$. 
The purpose of the hole context is to record a type, $\tau$, and a typing context, $\Gamma$, for each hole name, $u$.\footnote{
We use a hole context, rather than recording the typing context and type directly on each hole closure, to ensure that all closures for a hole name have the same typing context and type.} 
This notation for hole contexts is taken from contextual modal type theory (CMTT) \cite{Nanevski2008}, identifying hole names with metavariables and hole contexts with modal contexts (we say more about the connection with CMTT below). 
In all four hole expansion rules, the typing context recorded in the hole context is simply the current typing context when the hole is expanded. 
In the synthetic hole expansion rules, {ESEHole} and {ESNEHole}, the generated hole context assigns the hole type, $\tehole$, to $u$, as in the typing rules. 
However, the first two premises of the expansion subsumption rule, Rule EASubsume, disallow the use of subsumption for holes in analytic position. 
Instead, we have separate analytic rules, {EAEHole} and {EANEHole}, which record the type that the hole is being checked against into the hole context. 
This is again so that we can use type assignment for the internal language --- the type assignment rules TAEHole and TANEHole in Fig.~\ref{fig:hasType} assign a hole closure for $u$ the corresponding type from the hole context.

Each hole closure also has an associated environment, $\sigma$, which is a finite substitution of the form $[d_1/x_1, ~\cdots, d_n/x_n]$ for $n \geq 0$. 
The purpose of the closure environment is to keep a record of the substitutions that occur around the hole as evaluation occurs. 
Initially, no evaluation has occurred, so the initial environment generated by the hole expansion rules is the identity substitution for the typing context associated with $u$ in $\Delta$, which we notate $\idof{\hGamma}$ and define as follows.
\begin{defn}[Identity Substitution] $\idof{x_1 : \tau_1, ~\cdots, x_n : \tau_n} = [x_1/x_1, ~\cdots, x_n/x_n]$
\end{defn}
\noindent
The type assignment rules for hole closures, Rules TAEHole and TANEHole, require that we be able to check the environment of each hole closure against the corresponding typing context, written $\hasType{\Delta}{\hGamma}{\sigma}{\hGamma'}$ and defined as follows:\todo{check that the definition in the Agda corresponds}
\begin{defn}[Substitution Typing]
$\hasType{\Delta}{\hGamma}{\sigma}{\hGamma'}$ iff $\domof{\sigma} = \domof{\hGamma'}$ and for each $x : \htau \in \hGamma'$ we have that $d/x \in \sigma$ and $\hasType{\Delta}{\hGamma}{d}{\tau}$.
\end{defn}
\noindent
It is easy to verify that the identity substitution satisfies this requirement, i.e. that $\hasType{\Delta}{\hGamma}{\idof{\hGamma}}{\hGamma}$. 

Empty hole closures, $\dehole{u}{\sigma}{}$,  correspond to the metavariable closures (a.k.a. deferred substitutions) from CMTT, $\cmttclo{u}{\sigma}$.\todo{change notation from CMTT}{} 
We will see how closure environments evolve during evaluation in Sec.~\ref{sec:evaluation}. 
Non-empty hole closures, $\dhole{d}{u}{\sigma}{}$, do not directly correspond to a notion from CMTT (see Sec.~\ref{sec:resumption}).

\subsubsection{Cast Insertion}\label{sec:cast-insertion} 
%
Consider the following example: $\hap{(\halam{x}{\tehole}{\hap{x}{c}})}{c}$. 
The type synthesized for this example viewed as an external expression is $\tehole$, because the hole type annotation on $x$ allows us to apply it as a function of type $\tarr{\tehole}{\tehole}$, as previously discussed, and $c$ can be checked against type $\tehole$ by subsumption. 
However, viewed as an internal expression, this example is not well-typed---we do not have subsumption in the type assignment system defined in Fig.~\ref{fig:hasType}. 
Indeed, it would violate type safety if we could assign a type to this example in the internal language, because beta reduction of this example viewed as an internal expression would result in $c(c)$, which is clearly not well-typed.
The difficulty is that leaving the argument type unknown leaves how the argument is being used (in this case, as a function) also unknown.\footnote{In a system where type reconstruction is first used to try to fill in type holes, we could express a similar example by using $x$ at two or more different types, thereby causing type reconstruction to fail.
%On the other hand, if it is acceptable to arbitrarily choose one of the possible types, and type reconstruction is complete, then type holes will never appear in the internal language and the cast insertion machinery described in this section can be omitted entirely, leaving only the hole closure machinery described previously.
}
\todo{talk about elsewhere? maybe do type-hole-free version of calculus in appendix if there is time}{} 
By our  interpretation of hole types as unknown types from gradual type theory, we can address the problem by performing cast insertion. 
%

The cast form in Hazelnut Live is $\dcasttwo{\dexp}{\htau_1}{\htau_2}$, which, as specified by Rule TACast in Fig.~\ref{fig:hasType}, serves to ``box'' an expression of type $\htau_1$ for treatment as an expression of a consistent type $\htau_2$.%
\footnote{In the earliest work on gradual type theory, the cast form only gave the target type, $\htau_2$ \cite{Siek06a}, but it simplifies matters to include the assigned type, $\htau_1$, in the syntax \cite{DBLP:conf/snapl/SiekVCB15}.}

Casts are inserted during the expansion of function applications and ascriptions. 
The latter is more straightforward: 
Rule~{ESAsc} in Fig.~\ref{fig:expandSyn} inserts a cast from the assigned type to the ascribed  type. 
Theorem~\ref{thm:typed-expansion} inductively ensures that the two types are consistent.  
Note that we included ascription only for the sake of exposition---it can be defined using application together with the half-annotated identity, $e : \tau = \hap{(\halam{x}{\htau}{x})}{e}$, so application expansion, discussed next, is more general.

Cast insertion by the application expansion rule, Rule~{ESAp}, requires a bit more care. 
To understand the rule, consider the expansion for the example above:
\[\hap{\dcasttwo{(\halam{x}{\tehole}{\underline{\hap{\dcasttwo{x}{\tehole}{\tarr{\tehole}{\tehole}}}{\dcasttwo{c}{b}{\tehole}}}})}{\tarr{\tehole}{\tehole}}{\tarr{\tehole}{\tehole}}}{\dcasttwo{c}{b}{\tehole}}
\]
Let us focus on the expansion of the function body, underlined\todo{shaded?}, first. 
A cast has been inserted on both the function expression, $x$, and the the argument, $c$. 

The cast on $x$ allows us to treat the variable $x$, which is of type $\tehole$, as being of the matched arrow type $\tarr{\tehole}{\tehole}$, as described in Sec.~\ref{sec:external-statics}. 
The first three premises of Rule~{ESAp} accomplish this by first synthesizing a type for the function expression, here $\tehole$, then determining the matched arrow type, $\tarr{\tehole}{\tehole}$, and then performing analytic expansion on the function expression with this matched arrow type.
The resulting expansion will have some type $\tau_1'$ consistent with the matched arrow type. 
In this case, because $x$ is of variable form, analytic expansion goes through subsumption so $\tau_1'$ is simply $\tehole$. 
The conclusion of the rule inserts the corresponding cast. 
We go through type synthesis then analytic expansion so that the hole context records the matched arrow type for holes in function position, rather than the type $\tehole$ for all such holes as would be the case in a variant of this rule using synthetic expansion for the function expression\todo{put this in the appendix?}{}.

The conclusion of Rule~{ESAp} inserts the cast on the argument's expansion, from the type it is assigned by the final premise of the rule, here $b$ as described at the start of Sec.~\ref{sec:expansion}, to the argument type of the matched arrow type of the function expression, here $\tehole$.

Observe that together, these casts allow us assign a type to the function body according to the rules in Fig.~\ref{fig:hasType}, where we could not do so under the same context without casts.

The outer application in the example above goes through the same rule. 
In this case, the cast on the function is the identity cast for $\tarr{\tehole}{\tehole}$. 
We do not attempt to avoid the insertion of identity casts in the core calculus for simplicity (these will simply never fail during evaluation), 
but it is safe to eliminate these during expansion (and some formal accounts of gradual typing do so by defining three application rules, including the original account of \cite{Siek06a}).


\subsection{Dynamic Semantics}
\label{sec:evaluation}

To recap, the result of expansion is a well-typed internal expression with hole closures and casts, 
where the former corresponds to metavariable closures from CMTT \cite{Nanevski2008}, 
and the latter corresponds to casts from gradual type theory \cite{Siek06a,DBLP:conf/snapl/SiekVCB15}. 
However, the dynamic semantics for \HazelnutLive does not simply ``fall out'' from these observations. 

The problem is first that \citet{Nanevski2008} defined only the logical reductions for CMTT, viewing it as a proof system for intuitionistic contextual modal logic via the propositions-as-types (Curry-Howard) principle. 
The paper therefore proved only a subject reduction property (which is closely related to type preservation). 
This is not a full dynamic semantics, and in particular, there is no notion of \emph{progress}, i.e. that well-typed terms cannot get ``stuck'' in an undefined state \cite{wright94:_type_soundness}. 
In any case, a conventional dynamic semantics for CMTT would not be immediately relevant to our goal of evaluating incomplete programs because, by our interpretation of hole closures, we would need a dynamic semantics for terms with free metavariables. 
\citet{Nanevski2008} sketched an interpretation of CMTT into the simply-typed lambda calculus with sums under permutation conversion%
\footnote{Permutation conversions are necessary to encode the commuting reductions of CMTT, which in turn are necessary to prove a strong normalization property. These issues are not relevant in \HazelnutLive because, as in the gradually typed lambda calculus, type holes admit non-termination: we can express the Y combinator as $(\halam{x}{\tehole}{x(x)}) (\halam{x}{\tehole}{x(x)})$.}, 
which has been studied by \citet{DBLP:journals/iandc/Groote02}, 
but under this interpretation an analagous problem arises---metavariables become variables of a function type, so again we cannot rely on the standard notion of progress on closed terms.\todo{citations}{}% We also cannot rely on, for example, weak head normalization because \HazelnutLive admits non-termination (due to casts).\todo{citation} 

Furthermore, we need to integrate casts into the dynamic semantics. 
Fortunately, the dynamic semantics for the cast calculus from the gradually typed lambda calculus provides most, but not all, of the necessary machinery. 
The first problem is again with progress: in the cast calculus, the only irreducible terms of hole type are casts, which are accounted for by the progress theorem, but in $\HazelnutLive$, holes induce additional irreducible terms. 
The second missing piece is that in prior work on casts, evaluation aborts when cast failure occurs. 
Our goal, as discussed in Sec.~\ref{sec:failed-cast-example}\todo{where?}{}, is for cast failure to instead insert a membrane around the dynamic type error, 
much like a non-empty hole serves as a membrane around a static type error, 
allowing in both cases for evaluation to safely and meaningfully continue past the error when desired.

The dynamic semantics for \HazelnutLive specified in Figures~\ref{fig:isGround}-\ref{fig:step} addresses the difficulties just outlined, 
resulting in a system capable of running incomplete programs without aborting when a hole is encountered. 
This semantics is equipped with a meaningful notion of type safety (involving both preservation and progress). 
We also establish that the standard notion of type safety falls out when running complete terms.

The dynamic semantics is specified as a ``small-step'' transition system. 
The cast-related machinery is based closely on the cast calculus from the ``refined'' account of the gradually typed lambda calculus by \citet{DBLP:conf/snapl/SiekVCB15}, which is known to be theoretically well-behaved. 
In particular, Fig.~\ref{fig:isGround} defines the judgement $\isGround{\htau}$, which distinguishes the base type, $b$, and the least specific arrow type, $\tarr{\tehole}{\tehole}$, as \emph{ground types}, which play a role in simplifying the treatment of function casts. 
Fig.~\ref{fig:isFinal} defines the judgement $\isFinal{d}$, which distinguishes the final forms of the transition system. 
There are two classes of final forms: (possibly-)boxed values and indeterminate forms.%
\footnote{In most accounts of the cast calculus, values and ground types are distinguished with separate grammars together with an implicit identification convention. 
Our judgemental formulation is more faithful to the mechanization and cleaner for our purposes, because we are distinguishing several classes of final forms.}

% !TEX root = hazelnut-dynamics.tex

\begin{figure}
\begin{subfigure}[t]{0.5\textwidth}
\judgbox{\isGround{\htau}}{$\htau$ is a ground type}
\begin{mathpar}
\inferrule[GBase]{ }{
  \isGround{b}
}

\inferrule[GHole]{ }{
  \isGround{\tarr{\tehole}{\tehole}}
}
\end{mathpar}
\end{subfigure}
\hfill
\begin{subfigure}[t]{0.46\textwidth}
\judgbox{\groundmatch{\htau}{\htau'}}{$\htau$ has matched ground type $\htau'$}
\begin{mathpar}
\inferrule[MGArr]{
  \tarr{\htau_1}{\htau_2}\neq\tarr{\tehole}{\tehole}
}{
  \groundmatch{\tarr{\htau_1}{\htau_2}}{\tarr{\tehole}{\tehole}}
}
\end{mathpar}
\end{subfigure}
\CaptionLabel{Ground Types}{fig:isGround}
\label{fig:groundmatch}
\end{figure}

% !TEX root = hazelnut-dynamics.tex
\begin{figure}[t]

\begin{tabular}[t]{cc}

\begin{minipage}{0.5\textwidth}
\judgbox{\isFinal{\dexp}}{$\dexp$ is final}
\begin{mathpar}
%% \inferrule[FVal]
%% {\isValue{\dexp}}{\isFinal{\dexp}}
\inferrule[FBoxedVal]
{\isBoxedValue{\dexp}}{\isFinal{\dexp}}
\and
\inferrule[FIndet]
{\isIndet{\dexp}}{\isFinal{\dexp}}
\end{mathpar}
\end{minipage}

& 

\begin{minipage}{0.5\textwidth}
    
\judgbox{\isValue{\dexp}}{$\dexp$ is a value}
\begin{mathpar}
\inferrule[VConst]{ }{
  \isValue{c}
}

\inferrule[VLam]{ }{
  \isValue{\halam{x}{\htau}{\dexp}}
}
\end{mathpar}
\end{minipage}

\end{tabular}

\vsepRule

\judgbox{\isBoxedValue{\dexp}}{$\dexp$ is a boxed value}
\begin{mathpar}
\inferrule[BVVal\rkc{name?}]{
  \isValue{\dexp}
}{
  \isBoxedValue{\dexp}
}

\inferrule[BVCastArr\rkc{name?}]{
  \tarr{\htau_1}{\htau_2} \neq \tarr{\htau_3}{\htau_4}\\
  \isBoxedValue{\dexp}
}{
  \isBoxedValue{\dcasttwo{\dexp}{\tarr{\htau_1}{\htau_2}}{\tarr{\htau_3}{\htau_4}}}
}

\inferrule[BVHoleCast\rkc{name?}]{
  \isBoxedValue{\dexp}\\
  \isGround{\htau}
}{
  \isBoxedValue{\dcasttwo{\dexp}{\htau}{\tehole}}
}
\end{mathpar}

\vsepRule

\judgbox{\isIndet{\dexp}}{$\dexp$ is indeterminate}
\begin{mathpar}
\inferrule[IEHole]
{ }
{\isIndet{\dehole{\mvar}{\subst}{}}}

\inferrule[INEHole]
{\isFinal{\dexp}}
{\isIndet{\dhole{\dexp}{\mvar}{\subst}{}}}

\inferrule[IAp]
{\dexp_1\neq
   \dcasttwo{\dexp_1'}
            {\tarr{\htau_1}{\htau_2}}
            {\tarr{\htau_3}{\htau_4}}\\
 \isIndet{\dexp_1}\\
% \isFinal{\dexp_2}~\text{\cy{??}}}
 \isFinal{\dexp_2}}
{\isIndet{\dap{\dexp_1}{\dexp_2}}}

\inferrule[ICastGroundHole] {
  \isIndet{\dexp}\\
  \isGround{\htau}
}{
  \isIndet{\dcasttwo{\dexp}{\htau}{\tehole}}
}

\inferrule[ICastHoleGround] {
  \dexp\neq\dcasttwo{\dexp'}{\htau'}{\tehole}\\
  \isIndet{\dexp}\\
  \isGround{\htau}
}{
  \isIndet{\dcasttwo{\dexp}{\tehole}{\htau}}
}

\inferrule[ICastArr]{
  \tarr{\htau_1}{\htau_2} \neq \tarr{\htau_3}{\htau_4}\\
  \isIndet{\dexp}
}{
  \isIndet{\dcasttwo{\dexp}{\tarr{\htau_1}{\htau_2}}{\tarr{\htau_3}{\htau_4}}}
}

\inferrule[IFailedCast] {
  \isFinal{\dexp}\\
  \isGround{\htau_1}\\
  \isGround{\htau_2}\\
  \htau_1\neq\htau_2
}{
  \isIndet{\dcastfail{\dexp}{\htau_1}{\htau_2}}
}

%% \inferrule[ICast]
%% {\isIndet{\dexp}}
%% {\isIndet{\dcast{\htau}{\dexp}}}

\end{mathpar}

%\vsepRule

\CaptionLabel{Final Forms}{fig:isFinal}
\label{fig:isValue}
\label{fig:isIndet}
\end{figure}


The judgement $\isBoxedValue{d}$ distinguishes (possibly-)boxed values, which correspond to the values from the cast calculus and include the classic values from the lambda calculus, distinguished by $\isValue{d}$, as well as two cast forms: 
casts between function types (except identity casts) 
and casts from a ground type to the hole type. In both cases, the cast must be on a boxed value.

The judgement $\isIndet{d}$ distinguishes \emph{indeterminate} forms, 
so named because they arise from the presence of expression holes and failed casts in a program, and so, conceptually, their ultimate value awaits programmer action (see Sec.~\ref{sec:resumption}). 
The first two rules specify that hole closures are indeterminate (in the case of non-empty hole closures, when the inner expression is final). 
The rules defining the remaining indeterminate forms are explained simultaneously with the corresponding transition rules below.\todo{relationship to weak head normal forms}

% !TEX root = hazelnut-dynamics.tex
\begin{comment}
\begin{figure}[t]

\begin{comment}
\vsepRule

\judgbox{\isevalctx{\evalctx}}{$\evalctx$ is an evaluation context}
\begin{mathpar}
\inferrule[ECDot]{ }{
  \isevalctx{\evalhole}
}

%% \inferrule[ECLam]{
%%   \isevalctx{\evalctx}
%% }{
%%   \isevalctx{\halam{x}{\htau}{\evalctx}}
%% }

\inferrule[ECAp1]{
  \isevalctx{\evalctx}
}{
  \isevalctx{\hap{\evalctx}{\dexp}}
}

\inferrule[ECAp2]{
  \maybePremise{\isFinal{\dexp}}\\
  \isevalctx{\evalctx}
}{
  \isevalctx{\hap{\dexp}{\evalctx}}
}

\inferrule[ECNEHole]{
  \isevalctx{\evalctx}
}{
  \isevalctx{\dhole{\evalctx}{\mvar}{\subst}{}}
}

\inferrule[ECCast]{
  \isevalctx{\evalctx}
}{
  \isevalctx{\dcasttwo{\evalctx}{\htau_1}{\htau_2}}
}

\inferrule[ECFailedCast]{
  \isevalctx{\evalctx}
}{
  \isevalctx{\dcastfail{\evalctx}{\htau_1}{\htau_2}}
}
\end{mathpar}
% \end{comment}
\vsepRule


\caption{Evaluation Contexts}
\label{fig:eval-contexts}
\end{figure}
\end{comment}

%% \vsepRule

\begin{figure}
\judgbox{\reducesE{}{\dexp}{\dexp'}}{$\dexp$ takes an instruction transition to $\dexp'$}
\begin{mathpar}
\inferrule[ITBeta]{
  \maybePremise{\isFinal{\dexp_2}}
}{
  \reducesE{}{\hap{(\halam{x}{\htau}{\dexp_1})}{\dexp_2}}{[\dexp_2/x]\dexp_1}
}

\inferrule[ITApCast]{
  \maybePremise{\isFinal{\dexp_1}}\\
  \maybePremise{\isFinal{\dexp_2}}\\
  \tarr{\htau_1}{\htau_2} \neq \tarr{\htau_1'}{\htau_2'}
}{
  \reducesE{}
    {\hap{\dcasttwo{\dexp_1}{\tarr{\htau_1}{\htau_2}}{\tarr{\htau_1'}{\htau_2'}}}{\dexp_2}}
    {\dcasttwo{(\hap{\dexp_1}{\dcasttwo{\dexp_2}{\htau_1'}{\htau_1}})}{\htau_2}{\htau_2'}}
}

\inferrule[ITCastId]{
  \maybePremise{\isFinal{\dexp}}
}{
  \reducesE{}{\dcasttwo{\dexp}{\htau}{\htau}}{\dexp}
}

\inferrule[ITCastSucceed]{
  \maybePremise{\isFinal{\dexp}}\\
  \isGround{\htau}
}{
  \reducesE{}{\dcastthree{\dexp}{\htau}{\tehole}{\htau}}{\dexp}
}

\inferrule[ITCastFail]{
  \maybePremise{\isFinal{\dexp}}\\
  \htau_1\neq\htau_2\\\\
  \isGround{\htau_1}\\
  \isGround{\htau_2}
}{
  \reducesE{}
    {\dcastthree{\dexp}{\htau_1}{\tehole}{\htau_2}}
    {\dcastfail{\dexp}{\htau_1}{\htau_2}}
}

\inferrule[ITGround]{
  \maybePremise{\isFinal{\dexp}}\\
  \groundmatch{\htau}{\htau'}
}{
  \reducesE{}
    {\dcasttwo{\dexp}{\htau}{\tehole}}
    {\dcastthree{\dexp}{\htau}{\htau'}{\tehole}}
}

\inferrule[ITExpand]{
  \maybePremise{\isFinal{\dexp}}\\
  \groundmatch{\htau}{\htau'}
}{
  \reducesE{}
    {\dcasttwo{\dexp}{\tehole}{\htau}}
    {\dcastthree{\dexp}{\tehole}{\htau'}{\htau}}
}

%% \inferrule[ITCast]{
%%   \isFinal{d}\\
%%   \hasType{\Delta}{\emptyset}{d}{\tau_2}\\
%%   \tconsistent{\tau_1}{\tau_2}
%% }{
%%   \reducesE{\Delta}{\dcast{\htau_1}{d}}{d}
%% }
%% 
%% \inferrule[ITEHole]{ }{
%%   \reducesE{\Delta}{\dehole{\mvar}{\subst}{\unevaled}}{\dehole{\mvar}{\subst}{\evaled}}
%% }
%% 
%% \inferrule[ITNEHole]{
%%   \isFinal{d}
%% }{
%%   \reducesE{\Delta}{\dhole{d}{\mvar}{\subst}{\unevaled}}{\dhole{d}{\mvar}{\subst}{\evaled}}
%% }
\end{mathpar}
\CaptionLabel{Instruction Transitions}{fig:instruction-transitions}
\end{figure}

\begin{figure}
$\arraycolsep=4pt\begin{array}{rllllll}
\mathsf{EvalCtx} & \evalctx & ::= &
  \evalhole ~\vert~
  \hap{\evalctx}{\dexp} ~\vert~
  \hap{\dexp}{\evalctx} ~\vert~
  \dhole{\evalctx}{\mvar}{\subst}{} ~\vert~
  \dcasttwo{\evalctx}{\htau}{\htau} ~\vert~
  \dcastfail{\evalctx}{\htau}{\htau}
\end{array}$

\vsepRule

\judgbox{\selectEvalCtx{\dexp}{\evalctx}{\dexp'}}{$\dexp$ is obtained by placing $\dexp'$ at the mark in $\evalctx$}
\begin{mathpar}
\inferrule[FHOuter]{ }{
  \selectEvalCtx{\dexp}{\evalhole}{\dexp}
}

%% \inferrule[FLam]{
%%   \selectEvalCtx{d}{\evalctx}{d'}
%% }{
%%   \selectEvalCtx{\halam{x}{\htau}{d}}{\halam{x}{\htau}{\evalctx}}{d'}
%% }

\inferrule[FHAp1]{
  \selectEvalCtx{\dexp_1}{\evalctx}{\dexp_1'}
}{
  \selectEvalCtx{\hap{\dexp_1}{\dexp_2}}{\hap{\evalctx}{\dexp_2}}{\dexp_1'}
}

\inferrule[FHAp2]{
  \maybePremise{\isFinal{\dexp_1}}\\
  \selectEvalCtx{\dexp_2}{\evalctx}{\dexp_2'}
}{
  \selectEvalCtx{\hap{\dexp_1}{\dexp_2}}{\hap{\dexp_1}{\evalctx}}{\dexp_2'}
}

%% \inferrule[FHEHole]{ }{
%%   \selectEvalCtx{\dehole{\mvar}{\subst}{}}{\evalhole}{\dehole{\mvar}{\subst}{}}
%% }
%% 
%% \inferrule[FHNEHoleEvaled]{ }{
%%   \selectEvalCtx{\dhole{d}{\mvar}{\subst}{\evaled}}{\evalhole}{\dhole{d}{\mvar}{\subst}{\evaled}}
%% }

\inferrule[FHNEHoleInside]{
  \selectEvalCtx{\dexp}{\evalctx}{\dexp'}
}{
  \selectEvalCtx{\dhole{\dexp}{\mvar}{\subst}{}}{\dhole{\evalctx}{\mvar}{\subst}{}}{\dexp'}
}

%% \inferrule[FHNEHoleFinal]{
%%   \isFinal{d}
%% }{
%%   \selectEvalCtx{\dhole{d}{\mvar}{\subst}{\unevaled}}{\evalhole}{\dhole{d}{\mvar}{\subst}{\unevaled}}
%% }

\inferrule[FHCastInside]{
  \selectEvalCtx{\dexp}{\evalctx}{\dexp'}
}{
  \selectEvalCtx{\dcasttwo{\dexp}{\htau_1}{\htau_2}}
                {\dcasttwo{\evalctx}{\htau_1}{\htau_2}}
                {\dexp'}
}

\inferrule[FHFailedCast]{
  \selectEvalCtx{\dexp}{\evalctx}{\dexp'}
}{
  \selectEvalCtx{\dcastfail{\dexp}{\htau_1}{\htau_2}}
                {\dcastfail{\evalctx}{\htau_1}{\htau_2}}
                {\dexp'}
}

%% \inferrule[FHCastFinal]{
%%   \isFinal{d}
%% }{
%%   \selectEvalCtx{\dcast{\htau}{d}}{\evalhole}{\dcast{\htau}{d}}
%% }
\end{mathpar}

\vsepRule

\judgbox{\stepsToD{}{\dexp}{\dexp'}}{$\dexp$ steps to $\dexp'$}
\vspace{-10px}
\begin{mathpar}
\inferrule[Step]{
  \selectEvalCtx{d}{\evalctx}{\dexp_0}\\
  \reducesE{}{\dexp_0}{\dexp_0'}\\
  \selectEvalCtx{\dexp'}{\evalctx}{\dexp_0'}
}{
  \stepsToD{}{\dexp}{\dexp'}
}
\end{mathpar}
\CaptionLabel{Evaluation Contexts and Steps}{fig:step}
\end{figure}

The transition rules are defined in Fig.~\ref{fig:instruction-transitions}-\ref{fig:step}. 
Top-level transitions, i.e. \emph{steps}, $\stepsToD{}{d}{d'}$, are governed by a single rule, Rule {Step} in Fig.~\ref{fig:step}, which 
(1) decomposes $d$ into an evaluation context, $\evalctx$, and a selected sub-term, $d_0$; 
(2) takes an \emph{instruction transition}, $\reducesE{}{d_0}{d_0'}$, as specified in Fig.~\ref{fig:instruction-transitions}; 
and (3) places $d_0'$ back in the selected position, which is marked%
\footnote{In the literature, the form $\evalhole$ in the grammar of evaluation contexts is referred to as the \emph{hole}, but this hole is a technical device entirely orthogonal to the holes of this paper, so we use the term ``mark'' instead.} 
in the evaluation context by $\evalhole$, to obtain $d'$.
This approach was originally developed in the reduction semantics of \citet{DBLP:journals/tcs/FelleisenH92} and is the predominant style of operational semantics in the literature on gradual typing. 
Because we distinguish final forms judgementally, rather than syntactically, we use a judgemental formulation of this approach called a \emph{contextual dynamics} by \citet{pfpl}. 
It would be straightforward to derive an equivalent structural operational semantics \cite{DBLP:journals/jlp/Plotkin04a} by using search rules instead of evaluation contexts (\citet{pfpl} relates the two approaches).
\todo{Put the search rules in the appendix?}{}



%% \begin{figure}[t]
\judgbox{\stepsToD{\Delta}{\dexp_1}{\dexp_2}}{$\dexp_1$
steps to $\dexp_2$}
\begin{mathpar}
\inferrule[STEHoleEvaled]
{ }
{\stepsToD{\Delta}{\dehole{\mvar}{\subst}{\unevaled}}{\dehole{\mvar}{\subst}{\evaled}}}

\inferrule[STNEHoleStep]
{\stepsToD{\Delta}{\dexp_1}{\dexp_2} }
{\stepsToD{\Delta}{\dhole{\dexp_1}{\mvar}{\subst}{\unevaled}}{\dhole{\dexp_2}{\mvar}{\subst}{\unevaled}}}

\inferrule[STNEHoleEvaled]
{\isFinal{\dexp}}
{\stepsToD{\Delta}{\dhole{\dexp}{\mvar}{\subst}{\unevaled}}{\dhole{\dexp}{\mvar}{\subst}{\evaled}}}

\inferrule[STCast]
{
\isValue{\dexp}\\
\hasType{\Delta}{\emptyset}{\dexp}{\htau_2} \\
\tconsistent{\tau_1}{\tau_2}}
{\stepsToD{\Delta}{\dcast{\htau_1}{\dexp}}{\dexp}}

\inferrule[STApStep1]
{\stepsToD{\Delta}{\dexp_1}{\dexp_1'}}
{\stepsToD{\Delta}{\dap{\dexp_1}{\dexp_2}}{\dap{\dexp_1'}{\dexp_2}}}

\inferrule[STApStep2]
{ \isFinal{\dexp_1} \\ \stepsToD{\Delta}{\dexp_2}{\dexp_2'}}
{\stepsToD{\Delta}{\dap{\dexp_1}{\dexp_2}}{\dap{\dexp_1}{\dexp_2'}}}

\inferrule[STApBeta]
{ \isFinal{\dexp_2} }
{\stepsToD{\Delta}{\dapP{\dlam{x}{\htau}{\dexp_1}}{\dexp_2}}{ [\dexp_2/x]\dexp_1 }}
\end{mathpar}
\caption{Structural Dynamics}
\label{fig:stepsTo}
\end{figure}


\subsubsection{Application and Substitution} 
Rule {ITBeta} in Fig.~\ref{fig:instruction-transitions} specifies the standard beta reduction transition. 
The bracketed premises of the form $\maybePremise{\isFinal{\dexp}}$ in Fig.~\ref{fig:instruction-transitions}-\ref{fig:step} can be included to specify an eager, left-to-right evaluation strategy, 
or excluded to leave the evaluation order unspecified. 
We do the latter in our metatheory, both for the sake of generality and for reasons we return to in Sec.~\ref{sec:resumption}.

Substitution, $[d/x]d'$, operates in the standard capture-avoiding manner \cite{pfpl}. 
The only cases of special interest arise when substitution reaches a hole closure:
\[
\begin{array}{rcl}
  [d/x]\dehole{u}{\sigma}{} & = & \dehole{u}{[d/x]\sigma}{} \\%
  \substitute{d}{x}{\dhole{d'}{u}{\sigma}{}} & = & \dhole{[d/x]d'}{u}{[d/x]\sigma}{}
\end{array}
\]
In both cases, we write $[d/x]\sigma$ to perform substitution on each expression in the hole environment, $\sigma$. 
For example, $\stepsToD{}{\hap{(\halam{x}{b}{\halam{y}{b}{\dehole{u}{[x/x, y/y]}{}}})}{c}}{\halam{y}{b}{\dehole{u}{[c/x, y/y]}{}}}$. 
Hole closures can be duplicated like any other term by substitution, and the environments of different closures for the same hole name can differ, e.g. because a function with a hole closure in its body is applied multiple times.%
\todo{show example here or refer back to sec 2?}{} 
Hole closures can also appear within the environments of other hole closures (this gives rise to the closure paths described in Sec.~\ref{sec:paths}\todo{where?}{}).

The ITBeta rule is not the only rule we need to handle function application, because lambdas are not the only final forms of arrow type. There are two other classes of possibilities. 

The expression in function position might be a cast between arrow types, in which case we apply the arrow cast conversion rule, Rule {ITApCast}, to rewrite the application form, obtaining an equivalent application where the expression under the function cast, $d_1$, is exposed. 
We know from inverting the typing rules that $d_1$ is of type $\tarr{\htau_1}{\htau_2}$ and that $d_2$ is of type $\htau_1'$ where $\tconsistent{\htau_1}{\htau_1'}$, so to maintain type safety, we must place a cast on $d_2$ from  $\htau_1'$ to $\htau_1$. 
The result of this application will be of type $\htau_2$, but the original cast promised that the result would be of consistent type $\htau_2'$, so we also need a cast on the result from $\htau_2$ to $\htau_2'$.

The other possibility is that the expression in function position is indeterminate and arrow cast conversion is not applicable, e.g. $\hap{(\dehole{u}{\sigma}{})}{c}$. 
In this case, no further progress can be made, so the function application as a whole is also deemed indeterminate by Rule {IAp} in Fig.~\ref{fig:isFinal}. 
Note that we are careful to maintain the following property, which establishes that expressions that step are not final, and that final expressions do not step.%
\todo{prove this as stated from Ian's disjointness proofs}{}%
\todo{do we need a typing premise?}{}
\begin{thm}[Finality] There does not exist $d$ such that both $\isFinal{d}$ and $\stepsToD{}{d}{d'}$ for some $d'$.
\end{thm}

\subsubsection{Casts}
Rule {ITCastId} drops identity casts. The remaining instruction transition rules assign meaning to non-trivial casts. 
As discussed in Sec.~\ref{sec:cast-insertion}, the structure of a term cast to hole type is statically obscure, 
so we need to wait until it is actually used at some other type, via a cast away from hole type, to be able to determine whether there is a problem. 
Rules {ITCastSucceed} and {ITCastFail} handle this situation when the two types involved are ground types (Fig.~\ref{fig:isGround}). 
If the two ground types are equal, then the cast succeeds and the cast can be dropped. 
If they are not equal, then the cast fails and the failed cast form, $\dcastfail{\dexp}{\htau_1}{\htau_2}$, arises. 
Rule {TAFailedCast} specifies that a failed cast is well-typed exactly when $d$ has ground type $\tau_1$ and $\tau_2$ is a ground type disequal to $\tau_1$. 
Rule {IFailedCast} specifies that a failed cast operates as an indeterminate form once $d$ is final. 
This allows evaluation to proceed after cast failure, much as with hole closures.

The two rules just described operate at ground type. 
The two remaining instruction transition rules, Rule {ITGround} and {ITExpand}, operate as technical devices that allow us to restrict our interest to ground types exclusively by inserting intermediate casts from non-ground type to  a consistent ground type, and \emph{vice versa}. 
In this core calculus, the only non-ground types are the arrow types, so the grounding judgement $\groundmatch{\tau_1}{\htau_2}$ defined in Fig.~\ref{fig:groundmatch}, produces the ground arrow type, $\tarr{\tehole}{\tehole}$. 
More generally, this judgement is governed by the following invariant:
\begin{lem}[Grounding] 
  If $\groundmatch{\htau}{\htau'}$
  then $\isGround{\htau'}$
  and $\tconsistent{\htau}{\htau'}$
  and $\htau\neq\htau'$.
\end{lem}

In all other cases, casts are either boxed values or indeterminate according to the remaining rules in Fig.~\ref{fig:isFinal}. 
Note in particular Rule {ICastHoleGround}, which handles casts from hole to ground type that are not of the form $\dcastthree{\dexp}{\htau_1}{\tehole}{\htau_2}$. 
In the cast calculus, there are no such irreducible terms because the only canonical form of type $\tehole$ is $\dcasttwo{\dexp}{\htau_1}{\tehole}$.

\subsubsection{Type Safety} 
The purpose of establishing type safety is to ensure that the static and dynamic semantics of a language cohere. 
We follow the approach developed by \citet{wright94:_type_soundness}, now standard \cite{pfpl}, which distinguishes two type safety properties, preservation and progress. 
In order to allow for the evaluation of incomplete programs, we establish these properties for terms typed under arbitrary hole context $\Delta$. 
We can assume $\hGamma$ is empty (in order to run programs with free variables, the system can treat them as empty holes with a corresponding name).

The preservation theorem establishes that type assignment is preserved by transitions, i.e. that the type of an expression accurately predicts the type of the result of evaluating that expression.

\begin{thm}[Preservation]
  If $\hasType{\Delta}{\emptyset}{\dexp}{\htau}$ and
  $\stepsToD{\Delta}{\dexp}{\dexp'}$ then
  $\hasType{\Delta}{\emptyset}{\dexp'}{\htau}$.
\end{thm}
\noindent
The proof relies on an analagous preservation lemma for instruction transitions and a standard substitution lemma (stated in Appendix~\ref{sec:additional-defns}). 
Hole closures can disappear during evaluation, so we rely on weakening of $\Delta$.

The progress theorem establishes that the dynamic semantics accounts for every well-typed term, i.e. that we have not forgotten some necessary rules or premises.
\begin{thm}[Progress]
  If $\hasType{\Delta}{\emptyset}{\dexp}{\htau}$ then either
  (a) $\stepsToD{}{\dexp}{\dexp'}$ or
  (b) $\isBoxedValue{\dexp}$ or 
  (c) $\isIndet{\dexp}$.
\end{thm}
\noindent
The key to being able to establish the progress theorem under a non-empty hole context lies in the fact that we have explicitly accounted for indeterminate forms, i.e. those rooted at either a hole closure or a failed cast. 
The proof relies on canonical forms lemmas stated in Appendix~\ref{sec:additional-defns}.

\subsubsection{Complete Programs} 
Although our focus in this paper is on running incomplete programs, it is helpful to know that the necessary machinery does not interfere with reasoning about the behavior of complete programs, i.e. those with no holes. 
Appendix~\ref{sec:additional-defns} defines complete programs with the judgements $\isComplete{e}$ and $\isComplete{d}$. 
Failed casts cannot appear in complete internal expressions. 
The following theorem establishes that expansion preserves program completeness.
\begin{thm}[Complete Expansion] ~
  \begin{enumerate}[nolistsep]
    \item
      If $\isComplete{\hexp}$
      and $\expandSyn{\hGamma}{\hexp}{\htau}{\dexp}{\Delta}$
      then $\isComplete{\dexp}$.
    \item
      If $\isComplete{\hexp}$
      and $\expandAna{\hGamma}{\hexp}{\htau}{\dexp}{\htau'}{\Delta}$
      then $\isComplete{\dexp}$.
  \end{enumerate}
\end{thm}

The following preservation theorem establishes that stepping preserves program completeness.\todo{can we do complete preservation without the typing assumption?}
\begin{thm}[Complete Preservation]
  If $\hasType{\hDelta}{\emptyset}{\dexp}{\htau}$
  and $\isComplete{\dexp}$
  and $\stepsToD{}{\dexp}{\dexp'}$
  then $\hasType{\hDelta}{\emptyset}{\dexp'}{\htau}$
  and $\isComplete{\dexp'}$.
\end{thm}

The following progress theorem establishes that only classic values result from evaluating a complete program. There are no boxed values or indeterminate forms. 
\begin{thm}[Complete Progress]
  If $\hasType{\hDelta}{\emptyset}{\dexp}{\htau}$
  and $\isComplete{\dexp}$
  then either $\stepsToD{}{\dexp}{\dexp'}$
  or $\isValue{\dexp}$.
\end{thm}

%% \begin{figure}[!ht]
%%   \begin{definition}
%%     $\hasType{\Delta}{\hGamma}{\sigma}{\hGamma'}$ iff for each $\dexp/x \in \sigma$, we have $x : \htau \in \hGamma'$ and $\hasType{\Delta}{\hGamma}{\dexp}{\htau}$.
%%   \end{definition}
%%   \caption{substitution type assignment}
%%   \label{fig:subassign}
%% \end{figure}


%% \begin{figure}[!ht]
%%   \caption{substitution type assignment}
%% \end{figure}


\subsection{Agda Mechanization}
\label{sec:agda-mechanization}
\vspace{-3px}

The supplemental material includes our mechanization of the semantics and metatheory of \HazelnutLive, 
including proofs of all of the theorems stated above and all necessary lemmas\todo{make this so ian!}{}. We chose the Agda proof assistant \cite{norell2009dependently,norell:thesis} 
(as did the mechanization of \Hazelnut by \citet{popl-paper}, though only a few definitions are common). 
Agda is a good choice because it is designed to explicitly communicate a proof's structure, as is our goal, rather than relying on proof automation. Agda itself was also an inspiration for this work because it supports holes, albeit in a more limited form than described here (cf. Sec.~\ref{sec:intro}). 
The basic approach is standard: we model judgements as 
inductive datatypes, and rules as dependently typed constructors of these judgements\todo{cite somtehing?}{}. 
We adopt Barendregt's convention for bound variables \cite{urban,barendregt84:_lambda_calculus} and encode typing 
contexts, hole contexts and hole environments using metafunctions.\todo{say something about axioms?}{}

\subsection{Implementation}\label{sec:implementation}

The supplemental material also includes a browser-based implementation of \HazelnutLive. 
The implementation is a functional reactive program \cite{DBLP:conf/pldi/CzaplickiC13} 
written using the Reason toolchain for OCaml \cite{reason-what,leroy03:_ocaml} 
together with the OCaml \lismall{React} library \cite{OcamlReact} 
and the \lismall{js_of_ocaml} compiler and its associated libraries \cite{vouillon2014bytecode}. 
Appendix~\ref{sec:ref-impl-screenshots}\todo{do this}{} provides a screenshot of the user interface, which implements all of the live programming features described in Sec.~\ref{sec:examples}, but for the more austere language of this section.

The editor component of the implementation is derived from \Hazelnut. 
It exposes a language of structured edit actions (summarized in the sidebar on the left of Fig.~\ref{fig:screenshot}) that insert holes automatically to guarantee that every edit state has some (possibly incomplete) type. 
Formally,
this corresponds to the Sensibility property from the prior work \cite{popl-paper}. 
By composing this property with the Expandability, Typed Expansion and Progress properties from this paper, we establish a uniquely powerful 
invariant: that every edit state is both statically and dynamically meaningful. 
In other words, live feedback does not ``flicker in and out'' or reflect past edit states. 
The primary purpose of the included implementation is to serve as a proof-of-concept of this invariant, and to be of use to researchers studying the calculus as presented in this section.
% Consistent with this goal, it closely follows the theoretical account in this section, rather than including advanced language features.

We are also separately integrating the mechanisms developed in this paper into the \Hazel programming environment being developed by \citet{HazelnutSNAPL} with the goal of realizing a full-scale hole-driven development experience rooted in the type theoretic foundations established by \Hazelnut. 
Ongoing development is focused on extending the language of \Hazelnut following, in large part, the design of the popular Elm language \cite{Elm,czaplicki2012elm}, and on adding various syntactic and edit-time conveniences that are orthogonal to the topics of this paper. 
For the sake of exposition, the examples in Sec.~\ref{sec:examples} assumed some of these orthogonal features and conveniences, all of which are available in Elm and standard in ML-like languages.
We plan to keep our interpreter for \Hazel, which is an extension of the interpreter in the included implementation, up-to-date as \Hazel evolves.%
\todo{say that we'll use existing results from gradual typing to handle polymorphism and so on}{} 
The included implementation provides a link to \Hazel for interested readers.\todo{bad idea?}{}

Sec.~\ref{sec:discussion} discusses integration of the ideas from \HazelnutLive into other programming systems.\todo{do this}

%%   \halam{x}{\htau}{\evalctx} ~\vert~



%%%%%%%%%%%%%%%%%%%%%%%%%%%%%%%%%%%%%%%%%%%%%%%%%%%%%%%%%%%%%%%%%%%%%%%%%%%

\newcommand{\commutativitySec}{A Contextual Modal Interpretation of Fill-and-Resume}
\section{\protect\commutativitySec}
\label{sec:resumption}


%The result of evaluation is a final internal expression with hole closures, each with an associated hole environment, $\sigma$. These hole environments can be reported directly to the programmer, e.g. via the sidebar shown in Fig.~X\todo{fig}, to help them as they think about how to fill in the corresponding hole in the external expression. Hole environments might also be useful indirectly, e.g. by informing an edit action synthesis and suggestion system. In any case, 
When the programmer performs one or more edit actions to fill in a hole in the program, a new result must be computed. Na\"ively, the system would need to compute the result ``from scratch'' on each such edit. A more efficient approach supported by the structure of \HazelnutLive is to resume evaluation from where it left off after an edit that amounts to hole filling. We call this feature \emph{fill-and-resume}, by analogy to the ``edit-and-resume'' features available in some systems, e.g. Visual Studio (see below for a comparison)\todo{cite and compare}{}.

\todo{fix definition of instantiation in the agda}The key is to interpret the hole environments as delayed substitutions, and indeed this is exactly the interpretation suggested for closures in contextual modal type theory by \citet{Nanevski2008}. We can derive the hole filling operation, $\instantiate{d}{u}{d'}$, defined in Fig.~\ref{fig:substitution}, from the contextual substitution operation of CMTT. Notice that unlike standard substitution, which is capture-avoiding, hole filling imposes no condition on the binder when passing into the body of a lambda expression---the expression that fills a hole can, of course, refer to variables in scope where the hole appears. When hole filling encounters an empty closure for the hole being instantiated, $\instantiate{d}{u}{\dehole{u}{\sigma}{}}$, the result is $[\instantiate{d}{u}{\sigma}]d$, i.e. the delayed substitution is applied to the fill expression, $d$, after first recursively filling any instances of hole $u$ that may appear in $\sigma$. Hole filling for non-empty closures is analagous (the enveloped expression can be discarded). Note that this is the reason why we cannot interpret a non-empty hole as an empty hole of arrow type applied to the enveloped expression---the hole filling operation would not operate as expected under this interpretation.

\begin{figure}[t]
\judgbox
  {[\dexp_1 / x] \dexp_2 = \dexp_3}
  {$\dexp_3$ is the result of substituting $\dexp_1$ for $x$ in $\dexp_2$}
\[
\begin{array}{lcll}
[\dexp_1 / x] c
&=&
c\\
{[\dexp_1 / x] x}
&=&
\dexp_1\\
{[\dexp_1 / x] y}
&=&
y & (y \neq x)\\
{[\dexp_1 / x] \dlam{y}{\htau}{\dexp_2}}
&=&
{\dlam{y}{\htau}{[\dexp_1 / x] \dexp_2}}
& (y \neq x)
\\
{[\dexp_1 / x] \dap{d_2}{d_3}}
&=&
{\dapP{[\dexp_1 / x] d_2}{[\dexp_1 / x] d_3}}
\\
% {[\dexp_1 / x] \dinj{i}{e_2}}
% &=&
% {\dinj{i}{[\dexp_1 / x] e_2}}
% \\
% {[\dexp_1 / x] \dcase{e}{x}{e_x}{y}{e_y}}
% &=&
% {\dcase{[\dexp_1 / x]e}{x}{[\dexp_1 / x]e_x}{y}{[\dexp_1 / x]e_y}}
% \\
{[\dexp_1 / x]} \dehole{\mvar}{\subst}{}
&=&
\dehole{\mvar}{[\dexp_1 / x] \subst}{}
\\
{[\dexp_1 / x] \dhole{\dexp_2}{\mvar}{\subst}{}}
&=&
\dhole{ [ \dexp_1 / x ] \dexp_2 }{\mvar}{[\dexp_1 / x] \subst}{}
\\
{[\dexp_1 / x] \dcast{\htau}{\dexp}}
&=&
\dcast{\htau}{[\dexp_1 / x] \dexp}
\end{array}
\]
\caption{Substitution}
\end{figure}


The following theorem characterizes the static behavior of hole filling.
\begin{thm}[Filling]
  If $\hasType{\hDelta, \Dbinding{u}{\hGamma'}{\htau'}}{\hGamma}{\dexp}{\tau}$
  and $\hasType{\hDelta}{\hGamma'}{\dexp'}{\htau'}$
  then $\hasType{\hDelta}{\hGamma}{\instantiate{\dexp'}{u}{\dexp}}{\tau}$.
\end{thm}
\begin{proof}
We prove a stronger version of the theorem which accounts for both terms and finite substitutions (see \cite{Nanevski2008}). In each case, we proceed by rule induction on the first assumption, appealing to the finite substitution lemma in the corresponding cases.
\end{proof}

For the fill-and-resume feature to be reasonable, we need the following commutativity property, where we write $\multiStepsTo{\dexp_1}{\dexp_2}$ for the reflexive, transitive closure of  stepping (see Fig.~\ref{fig:multi-step} in Appendix~\ref{sec:additional-defns}). In words, if there is some sequence of steps that go from $d_1$ to $d_2$, then you can fill the hole either at the beginning or at the end of that step sequence because if you do the former, you can take some sequence of steps to the latter. 
\begin{thm}[Commutativity]
  If $\hasType{\hDelta, \Dbinding{u}{\hGamma'}{\htau'}}{\emptyset}{\dexp_1}{\tau}$
  and $\hasType{\hDelta}{\hGamma'}{\dexp'}{\htau'}$ and $\multiStepsTo{\dexp_1}{\dexp_2}$
  then $\multiStepsTo{\instantiate{\dexp'}{u}{\dexp_1}}
                     {\instantiate{\dexp'}{u}{\dexp_2}}$.
\end{thm}
Note that this property follows because the specification in Sec.~\ref{sec:calculus} left evaluation order unspecified (i.e. formally, we omitted the bracketed premises). It is not the case in general that resuming from $\instantiate{\dexp'}{u}{d_2}$ will cause sub-expressions to be evaluated in the same order as they would have been if we had restarted from $\instantiate{\dexp'}{u}{d_1}$ and selected sub-expressions in, for example, an eager, left-to-right manner, because the hole that is being filled may have been ``skipped past''. In other words, this simple fill-and-resume protocol only works for languages where evaluation order ``does not matter''. There are various ways to encode this intuition. One way to do so is by establishing a confluence property (which is closely related to the Church-Rosser property \cite{church1936some}). The most general confluence property does not hold for the dynamic semantics in Sec.~\ref{sec:calculus} for the usual reason: we do not reduce under binders (\citet{DBLP:conf/birthday/BlancLM05} discuss the standard counterexample). We could recover confluence by specifying reduction under binders, either generally or in a more restricted form where only closed sub-expressions are reduced \cite{DBLP:journals/tcs/CagmanH98,DBLP:conf/birthday/BlancLM05,levy1999explicit}, but reduction under binders is at odds with the standard approaches used in implementing programming languages \cite{DBLP:conf/birthday/BlancLM05}. A more satisfying approach is to consider confluence modulo equality \cite{Huet:1980ng}. The simplest approach is to restrict our interest to top-level expressions of base type that result in values, in which case the following special case of confluence does hold (trivially when the only base type has a single value, but also more generally for other base types).
\begin{lem}[Base Confluence]
  If $\hasType{\Delta}{\emptyset}{\dexp}{b}$ and 
  $\multiStepsTo{\dexp}{\dexp_1}$
  and $\isValue{\dexp_1}$
  and $\multiStepsTo{\dexp}{\dexp_2}$
  then $\multiStepsTo{\dexp_2}{\dexp_1}$.
\end{lem}
We can then prove the following property, which establishes that fill-and-resume is sound.
\begin{thm}[Resumption]
  If $\hasType{\hDelta, \Dbinding{u}{\hGamma'}{\htau'}}{\emptyset}{\dexp}{b}$
  and $\hasType{\hDelta}{\hGamma'}{\dexp'}{\htau'}$ 
  and $\multiStepsTo{\dexp}{\dexp_1}$
  and $\multiStepsTo{\instantiate{\dexp'}{u}{\dexp}}{\dexp_2}$
  and $\isValue{\dexp_2}$
  then $\multiStepsTo{\instantiate{\dexp'}{u}{\dexp_1}}{\dexp_2}$.
  \begin{proof}
    By Commutativity,
    $\multiStepsTo{\instantiate{\dexp'}{u}{\dexp}}
                  {\instantiate{\dexp'}{u}{\dexp_1}}$.
    By Base Confluence, we can conclude.
  \end{proof}
\end{thm}

The proofs are outlined in Appendix~\ref{sec:hole-filling}, along with a number of other necessary lemmas and definitions. In particular, we need additional machinery to properly handle the situation where we have filled a non-empty hole where a step had taken place in the original evaluation. These proofs and definitions are not included in the mechanization for two reasons. First, fill-and-resume is an optimization, and as just discussed, these properties, unlike those in Sec.~\ref{sec:calculus}, would not be conserved by some reasonable extensions of the core calculus. Second, it would require a significantly more complex representation of hole environments to encode the hole filling operation. Under the simple representation that suffices for the developments in Sec.~\ref{sec:calculus}, Agda cannot be convinced that the definition is well-founded (\citet{Nanevski2008} establish that it is in fact well-founded).

Hole filling in the external language is understood in terms of hole filling in its expansion. Note that in practice, it may be useful to cache results from several previous expansions. For example, consider two edits, the first filling a hole $u$ with the number $2$, and the next applying the $+$ operator, resulting in $2 + \dehole{v}{\sigma}{}$. This second edit is not a hole filling edit with respect to the immediately preceding edit state, but it can be understood as filling hole $u$ with $2 + \dehole{v}{\sigma}{}$. Another related case arises in lab notebook interfaces like that of Jupyter/IPython \cite{PER-GRA:2007}. In this case, each cell can be understood as a series of \li{let} bindings ending in a hole, which is filled by the next cell. The result from each cell can be cached to avoid recomputing the environment from preceding cells. This contextual modal interpretation of lab notebook cells (and read-eval-print loops as a restricted case where edits to previous cells are impossible) is novel.

So far, we have understood hole filling as an extralinguistic operation. However, a key feature CMTT that we have not yet touched on is the \emph{internalization} of metavariable binding and contextual substitution via the contextual modal types, $[\hGamma]\htau$, which are introduced by the operation $\mathsf{box}(\hGamma.d)$ and eliminated by the operation $\mathsf{letbox}(d_1, u.d_2)$. A hole filling can be interpreted as an expression of contextual modal type, and the act of hole filling followed by evaluation to the next possibly-indeterminate edit state as evaluation under the binder of a suitable $\mathsf{letbox}$ construct, which is enabled by the dynamic semantics in Sec.~\ref{sec:calculus}. This interpretation allows us to \emph{compute} hole fillings, rather than simply stating them, by specifying non-trivial expressions of modal type. This could serve as the basis for a \emph{live} computational hole refinement system, extending the capabilities of purely static hole refinement systems like those available in some proof assistants, e.g. the elaborator reflection system of Idris \cite{brady2013idris,DBLP:conf/icfp/ChristiansenB16} and the tactic system\todo{right? read a bit more}{} of Beluga \cite{DBLP:conf/flops/Pientka10,pientka2015inductive}. Each applied hole filling serves as a boundary between dynamic \emph{edit stages}. This contextual modal interpretation of live staged hole refinement nicely mirrors the modal interpretation of staged argument evaluation \cite{Davies:2001op}. We leave further development of this interpretation of hole refinement to future work.

\begin{comment}

\begin{theorem}[Maximum Informativity]
If the expansion produces $t1$, and there exists another possible type choice
$t2$, then $t1 \sim t2$ and $t1 JOIN t2 = t1$
\end{theorem}\footnote{idea is that special casing the holes in EANEHole gives you ``the
most descriptive hole types'' for some sense of what that means -- they'd
all just be hole other wise. from Matt:
\begin{quote}
It sounds like we need a something akin to an abstract domain (a lattice),
where hole has the least information, and a fully-defined type (without
holes) has the most information.  You can imagine that this lattice really
expands the existing definition we have of type consistency, which is
merely the predicate that says whether two types are comparable
(“join-able”) in this lattice.  lattice join is the operation that goes
through the structure of two (consistent) types, and chooses the structure
that is more defined (i.e., non-hole, if given the choice between hole and
non-hole).

The rule choosenonhole below is the expansion of this consistency rule that
we already have (hole consistent with everything)
\end{quote}}

\begin{verbatim}
t not hole
-------------------- :: choose-non-hole
hole JOIN t  = t
\end{verbatim}
\begin{verbatim}
------------ :: hole-consistent-with-everything
hole ~ t
\end{verbatim}

\end{comment}


\end{document}
