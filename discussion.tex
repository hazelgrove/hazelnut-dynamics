% !TEX root = hazelnut-dynamics.tex
\newcommand{\discussionSection}{Conclusion}
\section{\protect\discussionSection} % don't like the all-caps thing that the template does, so protecting it from that
\label{sec:discussion}

\begin{comment}
To conclude, we quote Weinberg from The Psychology of Computer Programming (1998): ``how truly sad it is that just at the very moment
when the computer has something important to tell us, it starts
speaking gibberish.''
\end{comment}

\begin{quote}
``[H]ow truly sad it is that just at the very moment
when the computer has something important to tell us, it starts
speaking gibberish.''

\vspace{3pt}

\hfill{}---Gerald Weinberg, The Psychology of Computer Programming (1998)
\end{quote}

\vspace{3pt}

\noindent
%
Weinberg's sentiment applies to countless situations that programmers routinely face---deciphering static error messages, deciphering run-time error messages, mentally simulating large parts of the evaluation process, and so on.
%
Efforts to remediate this state of affairs have been increasingly common among programming languages, software engineering, and human-computer interaction researchers.
%
Yet essentially all of this work towards next-generation programming environments assumes that the programmer has already written a complete program.
%
It is our hope that, by providing a foundation for how to reason about and run incomplete programs, future live programming environments can provide meaningful feedback to users, without temporal and perceptive gaps, throughout the program editing process.


\begin{comment}
\newcommand{\acksSection}{Acknowledgments}
\section*{\protect\acksSection} 
\label{sec:acks}

The authors would like to thank Brigitte Pientka, Jonathan Aldrich, Joshua Sunshine, Michael Hilton, Claire LeGoues,
the participants of the PL reading group at UChicago,
and the referees and participants of the LIVE 2017 and Off the Beaten Track (OBT)
2017 workshops for their insights and feedback on various iterations of this work.
%
This material was supported by a gift from
Facebook, from the National Science Foundation under grant
number CCF-1619282, and from AFRL and DARPA under agreement \#FA8750-16-2-0042. 
%
Any opinions, findings, and conclusions or recommendations expressed
in this material are those of the authors and do not necessarily
reflect the views of Facebook, NSF, DARPA or AFRL.
%
\end{comment}

\clearpage
% \todo{discussion}

% \rkc{something about UI choices for displaying large indeterminate expressions}

% \rkc{something about UI choices for displaying hole environment traces}

% \rkc{something about patterns of use for debugging, such as inserting holes
% around all expressions in tail position}

% \rkc{something about visualizing hole environment traces}

% \rkc{\citet{Seidel2016} define a partial evaluation with a form of holes. compare.
% compare ``flattened'' sumList indeterminate result to their trace. could be
% useful to study whether evaluating the type-correct parts of the evaluation
% trace help students focus on the part that has the error.}

% \rkc{quick mention that lazy evaluation with run-time exceptions for holes is
% not enough. this would delay the failure a bit longer, but not indefinitely.}


% live mobile application development systems like Flutter \cite{flutter};
