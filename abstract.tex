\begin{abstract}

\emph{Live programming environments} aim to provide programmers (and, in some
cases, audiences) with continuous feedback about a program's dynamic behavior
throughout the editing process.
%
Unfortunately, because programming languages typically assign dynamic meaning
only to programs that are \emph{complete}---syntactically well-formed and free
of static type errors---live programming features are often severely restricted,
if available at all, when programs are incomplete.
%
Incomplete edit states are frequently encountered when writing programs, so they
are, arguably, when live programming services would be most valuable.
%
Defining a semantics for incomplete functional programs would help bridge the
gap between well-formed states and allowing live programming environments to
provide continuous, uninterrupted, and meaningful feedback.

Recent work by \citet{popl-paper} makes progress toward this goal by developing
a type system for lambda terms with \emph{holes} in both expression and type
position, with empty holes standing for missing expressions or types, and
non-empty holes operating as ``membranes'' around type inconsistencies.
%
The contribution of this paper is a corresponding \emph{dynamic semantics for
functional programs with holes}.
%
Rather than aborting with an exception when evaluation encounters a hole (as in
several existing systems), evaluation proceeds ``around'' expression holes,
performing as much computation as is possible and tracking the dynamic
environment around each hole as it flows through the program.  These accumulated
hole environments can be reported directly to the programmer by the editor as
they work to fill holes.  They also enable an ``edit-and-resume'' feature that
avoids the need to restart evaluation after edits that amount to hole filling.

Formally, our semantics combines features from gradual type theory (to handle
incomplete types) and contextual modal type theory (CMTT, which provides a
logical foundation for hole environment tracking) with novel structure necessary
to prove key metatheoretic properties, including (1) a type safety property that
accounts for expression holes (i.e. free metavariables, from the perspective of
CMTT), and (2) commutativity and confluence properties that guarantee that the
``edit-and-resume'' feature is sound.
%
We have mechanized our formal development using the Agda proof assistant.
%
We additionally describe a simple reference implementation, called Hazelnut
Live, that exposes a language of structured edit actions that insert holes as
necessary to guarantee that every edit state has some (possibly incomplete)
type, based on the edit action calculus in the prior work by \citet{popl-paper}.
%
Taken together, the result is a proof-of-concept live functional programming
environment where dynamic feedback is truly continuous, i.e. it is available for
every possible edit state.
% i.e. \emph{every edit state is guaranteed to have
% non-trivial static and dynamic meaning}.

%In our system, evaluation treats failed casts much like it treats expression holes (rather than immediately failing with a cast error, as in gradual type theory). Prior work on contextual modal type theory did not develop an operational semantics for programs with free metavariables, which correspond to programs with holes in our formulation. 


% These incomplete edit states are sometimes transient, but at other times, they persist for extended periods of time, e.g. when the programmer is filling in the branches of a large case analysis one-by-one, or when an edit causes type errors to appear throughout a program. %This gap can cause live programming services to lag behind the programmer's edits, in many cases for an extended period of time.
\end{abstract}
