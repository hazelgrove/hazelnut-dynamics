%% \subsection{Live Programming with Type Errors}
\subsection{Live Programming with Static Type Errors}

In the previous section, we described example programs that were incomplete
because of \emph{missing} expressions.
%
Next, we describe example programs that are incomplete because of
\emph{type-inconsistent} expressions.
%
For example, consider the following two definitions.

\begin{lstlisting}
let bad_bool : bool = ?? 0 ??_bad_bool;

let bad_int : int = 1 + ?? true ??_bad_int_second_argument;
\end{lstlisting}

\noindent
%
These two definitions are ill-typed under standard typing disciplines.
%
In contrast, \citet{popl-paper} present a bidirectional type system that assigns
types to both, by wrapping type-inconsistent expressions (\li{0} on line
\rkc{XXX} and \li{true} on line \rkc{XXX} above) in \emph{non-empty} holes.
%
Non-empty holes prevent local type inconsistencies from polluting the rest of
the program surrounding it, which may or may not itself contain additional
inconsistencies.

Understanding and debugging static type errors is notoriously difficult,
particularly for novices.
%
A variety of approaches have been
proposed~\cite{Seminal,ChenErwig2014,Pavlinovic2015,sherrloc} to better localize
and explain type errors.
%
One of these approaches~\cite{Seidel2016} proposes to generate a dynamic witness
that demonstrates a run-time failure, and then displays a compressed execution
trace to the user as a graph.
%
In \HazelnutLive{}, we can run programs with type errors (\ie{}, programs with
non-empty holes) as far as possible, in particular, until they
would go wrong.
%
Thus, our operational semantics provides a similar benefit to the approach of
\citet{Seidel2016}.

\overviewExample{3}{Sum List}
%
Consider the following buggy program (observed during an undergraduate
functional programming course~\cite{Seidel2016}) that attempts to sum a list
integers.
%
The error is that the base case produces a list rather than an integer.

\begin{lstlisting}
sum_list : list(int) -> int
sum_list [] = ?[]?
sum_list (n:ns) = n + sum_list ns
\end{lstlisting}

\noindent
%
Because the list expression on line 2 does not have type \li{int} as required,
it is wrapped in a (non-empty) hole by the bidirectional type
checker~\cite{popl-paper}.
%
Rather than trying to debug the error based on the static error, the programmer
may wish to trying running the function anyway by calling, say, \li{sum_list(2)}.
%
\HazelnutLive{} runs and produces the indeterminate expression \li{3 + ?[]?}.
%
By observing that the hole expression is being added to the integer \li{3}, he
realizes that it needs to be an integer, specifically, \li{0}.
%
Compared to the trace displayed by \citet{Seidel2016}, the indeterminate result
produced by \HazelnutLive{} is ``flattened'' because the expression \li{1 + 2}
successfully proceeded to evaluate despite the error elsewhere.


%% TODO fold error from Erwig paper.
%% %
%% see that final call on stack does have the right answer, but
%% it's wrapped in a singleton list when the expected type is not
%% a list.
%% %
%% fix is to remove the list, the rest of the computation remains
%% the same, but b/c they were all wrapped in holes, need to re-run.
%% %
%% (add some mechanism for type-consistent non-empty-holes...)

\overviewExample{4}{Stutter}
%
Consider the following function which attempts to produce a
list where every element is repeated twice (borrowed from \citet{Osera2015}).
%
The combiner function to \li{List.foldr} needs to produce a \li{list(int)}, but
it produces a \li{list(list(int)} instead.

\begin{lstlisting}
stutter : list(int) -> list(int)
stutter xs = List.foldr (\x acc -> ?[x,x]? : acc) [] xs
\end{lstlisting}

\noindent
%
The bidirectional type checker of \citet{popl-paper} wraps the expression
\li{[x,x]} inside a non-empty hole.
%
%% The editor has a choice about which expression to ``blame'' for the error; the
%% entire application that forms the body of the lambda is analyzed against the
%% return type \li{list(int)}, so that is a reasonable choice for the editor to
%% make; another would be to assume that the arguments \li{[x,x]} and \li{acc} are
%% both as intended and that only the function \li{(:)} is type-consistent.
%% %
%% Although one could imagine a setting in which a user would perform this
%% reasoning, let's assume the simplest approach for marking consistencies that
%% wraps the entire application.
%
Running this on \li{stutter [1,2,3]} produces the indeterminate result

\begin{lstlisting}
?  [1,1] : (? [2,2] : (? [[3:3]] ?) ?) ?,
\end{lstlisting}

\noindent
%
which shows the unfolding of \li{List.foldr}.
%
We refer to nested indeterminate computations like this as \rkc{\emph{hole
environment traces} or \emph{hole environment trees}}.
%
The result of the innermost indeterminate expression is \li{[[3,3]]}.
%
The user realizes that there are too many levels of nesting, so
he replaces the \li{(:)} with \li{(@)}, which addresses the type inconsistency
and, when reevaluated, produces the desired result.
