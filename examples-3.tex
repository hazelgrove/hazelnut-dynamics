% !TEX root = hazelnut-dynamics.tex

\vspace{-4px}
\subsection{Example 4: Gradual Type Errors}
\vspace{-2px}
So far, we have only discussed programs where a hole appears within an expression.
\Hazel also allows holes to appear in types.
Fortunately, \citet{popl-paper} confirms that the substantial literature on \emph{gradual type systems} \cite{Siek06a,DBLP:conf/snapl/SiekVCB15} is directly relevant to the problem of reasoning statically around type holes.
Unsurprisingly, it is also relevant to the problem of running 
programs with incomplete types. Indeed, that is the very purpose of gradual typing. As such, let us consider only a small synthetic example to demonstrate what is different in \Hazel.
\begin{lstlisting}
f : ? -> int
f x = if x then 1 else 2
(f true, f 3, f false)
\end{lstlisting}
Evaluation produces the following result:
\begin{lstlisting}[numbers=none]
(1, if SSTR3<int => ? =/> bool>ESTR then 1 else 2, 2)
\end{lstlisting}
Here, we have left a hole, \li{?}, in the type of \li{f}. On Line 3, we call \li{f} three times, with two different types of arguments. The static semantics of gradual typing permits all of these calls (see next section) and indeed, the first call, with \li{true} as the argument, evaluates unproblematically to \li{1}. 

The second call, however, is problematic: \li{3} is of type \li{int} but it flows into boolean guard position. The usual semantics for gradual typing would simply abort evaluation here with a dynamic type error. However, this deprives the programmer of live feedback about other parts of the program that might not depend at all on this problematic function call---in this case, the third call to \li{f}. Instead, we follow the approach we took above for static type inconsistencies: we treat a failed cast like a hole. A failed cast, notated textually and in red \li{SSTR3<int => ? =/> bool>ESTR} (see next section), reports both the expected type, here \li{bool}, and the assigned type, here \li{int}. Consequently, the second call to \li{f} stops as if the guard were an empty hole, but the third call can progress safely.

\begin{comment}
\emph{Gradual type systems}~(\eg{}~\cite{XXX,XXX,XXX}) can statically accept
programs that would otherwise be rejected by static type systems---either
because type inference cannot reconstruct a valid type assignment, or because
there may not be a unique valid type assignment at all.
%
In either case, gradual type systems allow types to contain \emph{unknown
types}, and partially unknown types can be used in contexts that require
different types, so long as they are \emph{consistent} (intuitively, equal
modulo the unknown parts).
%
Dynamic casts are then inserted to ensure that these remaining static
discrepancies are not violated at run-time.

\HazelnutLive{} inherits the notion of \emph{type holes} from
\citet{popl-paper}, which serves a similar static purpose as the unknown type in
gradual type systems.
%
In contrast to prior gradually typed languages, however, \HazelnutLive{} can
evaluate ``around'' cast errors in the same way as it does for expression holes.

\overviewExample{3}{Dynamically Typed Negation}

Consider the \li{negate} function (adapted from \cite{ChughPOPL2012}), which
cannot be assigned a static type in a conventional type system---either a
bidirectional one, like in \HazelnutLive{}, or one with ML-style,
unification-based inference---because the argument \li{x} is used at type
\li{int} on line \rkc{XXX} and \li{bool} on line \rkc{XXX}.
%
Therefore, the declared type of \li{x} is the hole type \li{??}, which allows
\li{x} to be used at the conflicting types, with each use expanded into an
expression wrapped in a \emph{cast} that will dynamically check for safety.
%
The return type also uses the hole type \li{??}, leading to additional casts in
the expansion.

\begin{lstlisting}
negate : bool -> ?? -> ??
negate b x =
  if b
    then 0 - x     // expanded to: (0 - (x<?? => int>))<int => ??>
    else not x     // expanded to: (not (x<?? => bool>))<bool => ??>

(negate false 1) + 2 + 3 + (negate true 4)
\end{lstlisting}

\noindent
%
As in prior gradually typed languages~\cite{XXX},
evaluating the expression \li{negate false 1} on line \rkc{XXX} leads to
\li{(not (1<int => ??><?? => bool>))<bool => ??>}, where the inner casts
lead to the \emph{cast error} \li{1<?? =/=> bool>} because \li{1} cannot be
safely treated as having type \li{bool}.
%
Unlike in prior gradually typed languages, however, \HazelnutLive{} evaluates
around the cast error, making progress on the \li{2 + 3 + negate false 4}
expression that surrounds the error; the final indeterminate result is
\li{(not (1<?? =/=> bool>))<bool => ??> + 1}.
%
Just like it is useful for static type checkers to report multiple errors, our
approach allows us to report multiple dynamic cast errors, and otherwise make
progress on expressions that do not depend on failed casts.
%
This approach can be incorporated into existing gradually typed languages.
\end{comment}