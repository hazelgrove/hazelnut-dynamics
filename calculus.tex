% !TEX root = hazelnut-dynamics.tex

\clearpage
\newcommand{\calculusSec}{Hazelnut Live, Formally}
\section{\protect\calculusSec}
\label{sec:calculus}

% \def \TirNameStyle \Vtexttt{#1}{{#1}}

\begin{figure}[t]
$\arraycolsep=4pt\begin{array}{rllllll}
\mathsf{HTyp} & \htau & ::= &
  b ~\vert~
  \tarr{\htau}{\htau} ~\vert~
  % \tprod{\htau}{\htau} ~\vert~
  % \tsum{\htau}{\htau} ~\vert~
  \tehole\\
\mathsf{HExp} & \hexp & ::= &
  c ~\vert~
  x ~\vert~
  \halam{x}{\htau}{\hexp} ~\vert~
  \hap{\hexp}{\hexp} ~\vert~
  % \hpair{\hexp}{\hexp} ~\vert~
  % \hprj{i}{\hexp} ~\vert~
  % \hinj{i}{\hexp} ~\vert~
  % \hcase{\hexp}{x}{\hexp}{x}{\hexp} ~\vert~
  % \hadd{\hexp}{\hexp} ~\vert~
  \hehole{u} ~\vert~
  \hhole{\hexp}{u} ~\vert~
  {\hlam{x}{\hexp}} ~\vert~
  \hexp : \htau\\
% \mathsf{Mark} & \markname{} & ::= &
%   \evaled{} ~\vert~  \unevaled{}\\
 \mathsf{DHExp} & \dexp  & ::= &
  c ~\vert~
  x ~\vert~
  {\halam{x}{\htau}{\dexp}} ~\vert~
  \hap{\dexp}{\dexp} ~\vert~
  % \hpair{\dexp}{\dexp} ~\vert~
  % \hprj{i}{\dexp} ~\vert~
  % \hinj{i}{\dexp} ~\vert~
  % \hcase{\dexp}{x}{\dexp}{x}{\dexp} ~\vert~
  % \hadd{\dexp}{\dexp} ~\vert~
  \dehole{\mvar}{\subst}{} ~\vert~
  \dhole{\dexp}{\mvar}{\subst}{} ~\vert~
  \dcasttwo{\dexp}{\htau}{\htau} ~\vert~
  \dcastfail{\dexp}{\htau}{\htau}\\
\end{array}$
\caption{Syntax of H-types, H-expressions and dynamic H-expressions.
We write $x$ to range over variables,
$u$ over hole names, and
$\sigma$ over finite substitutions (i.e., environments) from
variables to dynamic H-expressions.}
\label{fig:HTyp}
\label{fig:HExp}
\end{figure}

% !TEX root = hazelnut-dynamics.tex

\begin{figure}[t]
\judgbox{\hsyn{\hGamma}{\hexp}{\htau}}{$\hexp$ synthesizes type $\htau$}
\begin{mathpar}
\inferrule[SConst]{ }{
  \hsyn{\hGamma}{c}{b}
}

\inferrule[SVar]{
  x : \htau \in \hGamma
}{
  \hsyn{\hGamma}{x}{\htau}
}

\inferrule[SLam]{
  \hsyn{\hGamma, x : \htau_1}{\hexp}{\htau_2}
}{
  \hsyn{\hGamma}{\halam{x}{\htau_1}{\hexp}}{\tarr{\htau_1}{\htau_2}}
}

\inferrule[SAp]{
  \arrayenvbr{
    \hsyn{\hGamma}{\hexp_1}{\htau_1}    
    \\
    \hana{\hGamma}{\hexp_2}{\htau_2}
  }
  \\
    \arrmatch{\htau_1}{\tarr{\htau_2}{\htau}}
}{
  \hsyn{\hGamma}{\hap{\hexp_1}{\hexp_2}}{\htau}
}

\inferrule[SEHole]{ }{
  \hsyn{\hGamma}{\hehole{u}}{\tehole}
}

\inferrule[SNEHole]{
  \hsyn{\hGamma}{\hexp}{\htau}
}{
  \hsyn{\hGamma}{\hhole{\hexp}{u}}{\tehole}
}

\inferrule[SAsc]{
  \hana{\hGamma}{\hexp}{\htau}
}{
  \hsyn{\hGamma}{\hexp : \htau}{\htau}
}
\end{mathpar}

\vsepRule

\judgbox{\hana{\hGamma}{\hexp}{\htau}}{$\hexp$ analyzes against type $\htau$}
\begin{mathpar}
\inferrule[ALam]{
  \arrmatch{\htau}{\tarr{\htau_1}{\htau_2}}\\
  \hana{\hGamma, x : \htau_1}{\hexp}{\htau_2}
}{
  \hana{\hGamma}{\hlam{x}{\hexp}}{\htau}
}

\inferrule[ASubsume]{
  \hsyn{\hGamma}{\hexp}{\htau}\\
  \tconsistent{\htau}{\htau'}
}{
  \hana{\hGamma}{\hexp}{\htau'}
}
\end{mathpar}
\CaptionLabel{Bidirectional Typing of External Expressions}{fig:bidirectional-typing}
\end{figure}


We will now make the intuitions developed in the previous section formally precise by specifying a core calculus, which we call \HazelnutLive, and its accompanying metatheory.

\parahead{Overview} The syntax of the core calculus given in Fig.~\ref{fig:hazelnut-live-syntax} consists of types and expressions with holes. 
We distinguish between {external} expressions, $e$, and {internal} expressions, $d$. 
External expressions correspond to programs as entered by the programmer 
(see Sec.~\ref{sec:intro} for discussion of implicit, manual, semi-automated and fully automated hole entry methods). 
Each well-typed external expression (as specified in Sec.~\ref{sec:external-statics} below) expands to a well-typed internal expression (see Sec.~\ref{sec:expansion}) before it is evaluated (see Sec.~\ref{sec:evaluation}). 
We take this approach, notably also taken in the ``redefinition'' of Standard ML by \citet{Harper00atype-theoretic}, because (1) the external language supports type inference and explicit type ascriptions, $\hexp : \htau$, but it is formally simpler to eliminate ascriptions and specify a type assignment system when defining the dynamic semantics; and 
(2) we need additional syntactic machinery during evaluation for tracking hole closures and dynamic type casts. 
This machinery is inserted by the expansion step, rather than entered explicitly by the programmer. 
In this regard, the internal language is analogous to the cast calculus in the gradually typed lambda calculus \cite{DBLP:conf/snapl/SiekVCB15,Siek06a}, though as we will see the \HazelnutLive internal language goes beyond the cast calculus in several respects. We have mechanized these formal developments using the Agda proof assistant \cite{norell:thesis,norell2009dependently} 
(see Sec.~\ref{sec:agda-mechanization}). Rule names, written \rulename{SVar}, correspond to the names used in the mechanization. We have also implemented \HazelnutLive in a manner that maintains a novel end-to-end well-definedness invariant (see Sec.~\ref{sec:implementation}).

% \rkc{this syntactic sugar is used in four places: ITCastSucceed, ITCastFail,
% ITGround, and ITExpand. that's not many, and those rules don't look much more
% cluttered without the sugar, so consider eliminating it. if so, just toggle the
% definition of the dcastthree macro to the unsugared option.}

\subsection{Static Semantics of the External Language}
\label{sec:external-statics}


We start with the type system of the \HazelnutLive external language,
which closely follows the \Hazelnut type system \cite{popl-paper}; we summarize the minor differences as they come up.

% except that (1) holes in \HazelnutLive each have a \emph{unique name};
% (2) for brevity of exposition, we removed numbers in favor of a simpler base type, $b$, with one value, $c$ (i.e. $b$ is the unit type); and 
% (3) we include both unannotated lambdas, $\hlam{x}{\hexp}$, and half-annotated lambdas, $\halam{x}{\htau}{\hexp}$.  which we discuss
% (along with other systematic extensions) in
% Appendix~\ref{sec:extensions}.

\Figref{fig:bidirectional-typing} defines the type system in the \emph{bidirectional} style 
%
with two mutually defined judgements \cite{Pierce:2000ve,bidi-tutorial,DBLP:conf/icfp/DunfieldK13,Chlipala:2005da}. The type synthesis
judgement~$\hsyn{\hGamma}{\hexp}{\htau}$ synthesizes a type~$\htau$
for external expression~$\hexp$ under typing context $\hGamma$, which tracks typing
assumptions of the form $x : \htau$ in the usual
manner \cite{pfpl,tapl}.
%
The type analysis judgement~$\hana{\hGamma}{\hexp}{\htau}$ checks
expression~$\hexp$ against a given type~$\htau$.
%
Algorithmically, analysis accepts a type as input, and synthesis gives
a type as output.
%
We start with synthesis for the programmer's ``top level'' external
expression.

% Algorithmically, the type is an output of type synthesis but an input of type analysis.

The primary benefit of specifying the \HazelnutLive external language 
bidirectionally is that the programmer need not annotate each hole with a type. 
%
An empty hole is
written simply $\hehole{u}$, where $u$ is the hole name, which we tacitly assume is unique 
(holes in \Hazelnut were not named).
Rule \rulename{SEHole} specifies that an empty hole synthesizes hole type, written $\tehole$.
%
If an empty hole appears where an expression of some other type is
expected, e.g. under an explicit ascription (governed by Rule \rulename{SAsc})
or in the argument position of a function application (governed by
Rule \rulename{SAp}, discussed below), we apply the \emph{subsumption rule},
Rule \rulename{ASubsume}, which specifies that if an expression $e$ synthesizes
type $\htau$, then it may be checked against any \emph{consistent}
type, $\htau'$.

% !TEX root = hazelnut-dynamics.tex

\begin{figure}[t]
\judgbox{\tconsistent{\htau_1}{\htau_2}}{$\htau_1$ is consistent with $\htau_2$}
\begin{mathpar}
\inferrule[TCHole1]{ }{
  \tconsistent{\tehole}{\htau}
}

\inferrule[TCHole2]{ }{
  \tconsistent{\htau}{\tehole}
}

\inferrule[TCRefl]{ }{
  \tconsistent{\htau}{\htau}
}

\inferrule[TCArr]{
  \tconsistent{\htau_1}{\htau_1'}\\
  \tconsistent{\htau_2}{\htau_2'}
}{
  \tconsistent{\tarr{\htau_1}{\htau_2}}{\tarr{\htau_1'}{\htau_2'}}
}
%
% \inferrule{
%   \tconsistent{\htau_1}{\htau_1'}\\
%   \tconsistent{\htau_2}{\htau_2'}
% }{
%   \tconsistent{\tprod{\htau_1}{\htau_2}}{\tprod{\htau_1'}{\htau_2'}}
% }
%
% \inferrule{
%   \tconsistent{\htau_1}{\htau_1'}\\
%   \tconsistent{\htau_2}{\htau_2'}
% }{
%   \tconsistent{\tsum{\htau_1}{\htau_2}}{\tsum{\htau_1'}{\htau_2'}}
% }
\end{mathpar}

% \vsepRule

% \judgbox{\tinconsistent{\htau_1}{\htau_2}}{$\htau_1$ is inconsistent with $\htau_2$}
% \begin{mathpar}
%     \inferrule[ICBaseArr1]{ }{
%       \tinconsistent{\tb}{\tarr{\htau_1}{\htau_2}}
%     }

%     \inferrule[ICBaseArr2]{ }{
%       \tinconsistent{\tarr{\htau_1}{\htau_2}}{\tb}
%     }

%     \inferrule[ICArr1]{
%       \tinconsistent{\htau_1}{\htau_3}
%     }{
%       \tinconsistent{\tarr{\htau_1}{\htau_2}}{\tarr{\htau_3}{\htau_4}}
%     }

%     \inferrule[ICArr2]{
%       \tinconsistent{\htau_2}{\htau_4}
%     }{
%       \tinconsistent{\tarr{\htau_1}{\htau_2}}{\tarr{\htau_3}{\htau_4}}
%     }
% \end{mathpar}

\vsepRule

\judgbox{\arrmatch{\htau}{\tarr{\htau_1}{\htau_2}}}{$\htau$ has matched arrow type $\tarr{\htau_1}{\htau_2}$}
\begin{mathpar}
\inferrule[MAHole]{ }{
  \arrmatch{\tehole}{\tarr{\tehole}{\tehole}}
}

\inferrule[MAArr]{ }{
  \arrmatch{\tarr{\htau_1}{\htau_2}}{\tarr{\htau_1}{\htau_2}}
}
\end{mathpar}

% \judgbox{\prodmatch{\htau}{\tprod{\htau_1}{\htau_2}}}{$\htau$ has matched product type $\tprod{\htau_1}{\htau_2}$}
% \begin{mathpar}
% \inferrule{ }{
%   \prodmatch{\tehole}{\tprod{\tehole}{\tehole}}
% }

% \inferrule{ }{
%   \prodmatch{\tprod{\htau_1}{\htau_2}}{\tprod{\htau_1}{\htau_2}}
% }
% \end{mathpar}

% \judgbox{\summatch{\htau}{\tsum{\htau_1}{\htau_2}}}{$\htau$ has matched sum type $\tsum{\htau_1}{\htau_2}$}
% \begin{mathpar}
% \inferrule{ }{
%   \summatch{\tehole}{\tsum{\tehole}{\tehole}}
% }

% \inferrule{ }{
%   \summatch{\tsum{\htau_1}{\htau_2}}{\tsum{\htau_1}{\htau_2}}
% }
% \end{mathpar}
\caption{Type Consistency and Matching}
\Label{fig:tconsistent}
\Label{fig:arrmatch}
\end{figure}


Fig.~\ref{fig:tconsistent} specifies the type consistency relation, written $\tconsistent{\htau}{\htau'}$, which specifies that two types are consistent if they differ only up to type holes in corresponding positions.
%
The hole type is consistent with every type, and so, by the subsumption rule, expression holes may appear where an expression of any type is expected. The type consistency relation here coincides with the type consistency relation from gradual type theory by identifying the hole type with the unknown type~\cite{Siek06a}.
%
Type consistency is reflexive and symmetric, but it is \emph{not} transitive.
%
This stands in contrast to subtyping, which is anti-symmetric and transitive; subtyping may be integrated into a gradual type system following \citet{Siek:2007qy}.

Non-empty expression holes, written $\hhole{\hexp}{u}$, behave similarly to empty holes.
%
Rule \rulename{SNEHole} specifies that a non-empty expression hole also synthesizes hole type as long as the expression inside the hole, $\hexp$, synthesizes some (arbitrary) type.
%
Non-empty expression holes therefore internalize the ``red underline'' that many editors display under or around type inconsistencies in a program.

For the familiar forms of the lambda calculus, the rules again follow prior work.
%
For simplicity, the core calculus includes only a single base type~$b$ with a single constant~$c$, governed by Rule \rulename{SConst} (i.e. $b$ is the unit type).
%
%\matt{Extraneous and uninteresting:}
By contrast, \Hazelnut instead defines a number type with a single operation, which we include in Appendix~\ref{sec:extensions} alongside various other simple extensions to the calculus\todo{do this, say which ones}. 
%

Rule \rulename{SVar} synthesizes the corresponding type from $\hGamma$. 
For the sake of exposition, \HazelnutLive includes ``half-annotated'' lambdas, $\halam{x}{\htau}{\hexp}$, in addition to the unannotated lambdas, $\hlam{x}{\hexp}$, from \Hazelnut.
%
Half-annotated lambdas may appear in synthetic position according to Rule \rulename{SLam}, which is standard \cite{Chlipala:2005da}.
%
Unannotated lambdas may only appear where the expected type is known to be either an arrow type or the hole type, which is treated as if it were $\tarr{\tehole}{\tehole}$.
%
To avoid the need for two separate rules, Rule \rulename{ALam} uses the matching relation $\arrmatch{\htau}{\tarr{\htau_1}{\htau_2}}$ defined in \Figref{fig:arrmatch}, which produces the matched arrow type $\tarr{\tehole}{\tehole}$ given the hole type, and operates as the identity on arrow types \cite{DBLP:conf/snapl/SiekVCB15,DBLP:conf/popl/GarciaC15}.\footnote{A system supporting ML-style type reconstruction \cite{damas1982principal} might also include a synthetic rule for unannotated lambdas, e.g. as outlined by \citet{DBLP:conf/icfp/DunfieldK13}, but we stick to this simpler ``Scala-style'' local type inference scheme in this paper \cite{Pierce:2000ve,Odersky:2001lb}.}
%
% \Secref{sec:related} dicusses how \HazelnutLive might be enriched with
% with ML-style type reconstruction~\cite{damas1982principal}, perhaps via
% the approach outlined by~\citet{DBLP:conf/icfp/DunfieldK13}.
%
%%%%%%%%%%%%%%%%%%%%%%%%%%%%%%%%%%%%%%%%%%%%%%%%%%%%%%%%%%%%%%%%%%%%%%%%%%%%%%%%%%%%%%%%%%%%%%%%%%%%%%%%%%%%%%%%%%%%%
% To Cyrus from Matt:
%
% Why say the following here? --- It's discussing a different design that we didn't pursue here.  We should move such discussion to related work.
%
%Note that 
%%

The rule governing function application, Rule \rulename{SAp}, similarly treats an expression of hole type in function position as if it were of type $\tarr{\tehole}{\tehole}$ using the same matched arrow type judgement.

\vspace{-4px}
\subsection{Expansion}
\label{sec:expansion}
\vspace{-1px}

% !TEX root = hazelnut-dynamics.tex

\begin{figure}[p]
\judgbox
  {\expandSyn{\hGamma}{\hexp}{\htau}{\dexp}{\Delta}}
  {$\hexp$ synthesizes type $\htau$ and expands to $\dexp$}
\begin{mathpar}
\inferrule[ESConst]{ }{
  \expandSyn{\hGamma}{c}{b}{c}{\emptyset}
}

\inferrule[ESVar]{
  x : \htau \in \hGamma
}{
  \expandSyn{\hGamma}{x}{\htau}{x}{\emptyset}
}

\inferrule[ESLam]{
  \expandSyn{\hGamma, x : \htau_1}{\hexp}{\htau_2}{\dexp}{\Delta}
}{
  \expandSyn{\hGamma}{\halam{x}{\htau_1}{\hexp}}{\tarr{\htau_1}{\htau_2}}{\halam{x}{\htau_1}{\dexp}}{\Delta}
}

\inferrule[ESAp]{
  \hsyn{\hGamma}{\hexp_1}{\htau_1}\\
  \arrmatch{\htau_1}{\tarr{\htau_2}{\htau}}
  \\\\
  \expandAna{\hGamma}{\hexp_1}{\tarr{\htau_2}{\htau}}{\dexp_1}{\htau_1'}{\Delta_1}\\
  \expandAna{\hGamma}{\hexp_2}{\htau_2}{\dexp_2}{\htau_2'}{\Delta_2}
}{
  \expandSyn
    {\hGamma}
    {\hap{\hexp_1}{\hexp_2}}
    {\htau}
    {\hap{(\dcasttwo{\dexp_1}{\htau_1'}{\tarr{\htau_2}{\htau}})}
         {\dcasttwo{\dexp_2}{\htau_2'}{\htau_2}}}
    {\Dunion{\Delta_1}{\Delta_2}}
}
%
%% \inferrule[ESAp1]{
%%   \hsyn{\hGamma}{\hexp_1}{\tehole}\\
%%   \expandAna{\hGamma}{\hexp_1}{\tarr{\htau_2}{\tehole}}{\dexp_1}{\htau_1}{\Delta_1}\\
%%   \expandAna{\hGamma}{\hexp_2}{\tehole}{\dexp_2}{\htau_2}{\Delta_2}
%% }{
%%   \expandSyn{\hGamma}{\hap{\hexp_1}{\hexp_2}}{\tehole}{\hap{(\dcast{\tarr{\htau_2}{\tehole}}{\dexp_1})}{\dexp_2}}{\Dunion{\Delta_1}{\Delta_2}}
%% }
%% 
%% \inferrule[ESAp2]{
%%   \expandSyn{\hGamma}{\hexp_1}{\tarr{\htau_2}{\htau}}{\dexp_1}{\Delta_1}\\
%%   \expandAna{\hGamma}{\hexp_2}{\htau_2}{\dexp_2}{\htau'_2}{\Delta_2}\\
%%   \htau_2 \neq \htau'_2
%% }{
%%   \expandSyn{\hGamma}{\hap{\hexp_1}{\hexp_2}}{\htau}{\hap{\dexp_1}{\dcast{\htau_2}{\dexp_2}}}{\Dunion{\Delta_1}{\Delta_2}}
%% }
%% 
%% \inferrule[ESAp3]{
%%   \expandSyn{\hGamma}{\hexp_1}{\tarr{\htau_2}{\htau}}{\dexp_1}{\Delta_1}\\
%%   \expandAna{\hGamma}{\hexp_2}{\htau_2}{\dexp_2}{\htau_2}{\Delta_2}
%% }{
%%   \expandSyn{\hGamma}{\hap{\hexp_1}{\hexp_2}}{\htau}{\hap{\dexp_1}{\dexp_2}}{\Dunion{\Delta_1}{\Delta_2}}
%% }\\
%
%
% \inferrule[expand-pair]{
%   \expandSyn{\hGamma}{\hexp_1}{\htau_1}{\dexp_1}{\Delta_1}\\
%   \expandSyn{\hGamma}{\hexp_2}{\htau_2}{\dexp_2}{\Delta_2}
% }{
%   \expandSyn{\hGamma}{\hpair{\hexp_1}{\hexp_2}}{\tprod{\htau_1}{\htau_2}}{\hpair{\dexp_1}{\dexp_2}}{\Dunion{\Delta_1}{\Delta_2}}
% }
%
% \inferrule[expand-prj]{
%   a
% }{
%   b
% }
%
% (inj)
%
%
% \inferrule[expand-plus]{ }{
%   \expandSyn{\hGamma}{\hadd{\hexp_1}{\hexp_2}}{\tnum}{\hadd{\dexp_1}{\dexp_2}}{\Dunion{\Delta_1}{\Delta_2}}
% }

\inferrule[ESEHole]{ }{
  \expandSyn{\hGamma}{\hehole{u}}{\tehole}{\dehole{u}{\idof{\hGamma}}{}}{\Dbinding{u}{\hGamma}{\tehole}}
}

\inferrule[ESNEHole]{
  \expandSyn{\hGamma}{\hexp}{\htau}{\dexp}{\Delta}
}{
  \expandSyn{\hGamma}{\hhole{\hexp}{u}}{\tehole}{\dhole{\dexp}{u}{\idof{\hGamma}}{}}{\Delta, \Dbinding{u}{\hGamma}{\tehole}}
}\\
%

\inferrule[ESAsc]{
  \expandAna{\hGamma}{\hexp}{\htau}{\dexp}{\htau'}{\Delta}
}{
  \expandSyn{\hGamma}{\hexp : \htau}{\htau}{\dcasttwo{\dexp}{\htau'}{\htau}}{\Delta}
}

%% \inferrule[ESAsc1]{
%%   \expandAna{\hGamma}{\hexp}{\htau}{\dexp}{\htau'}{\Delta}\\
%%   \htau \neq \htau'
%% }{
%%   \expandSyn{\hGamma}{\hexp : \htau}{\htau}{\dcast{\htau}{\dexp}}{\Delta}
%% }
%% 
%% \inferrule[ESAsc2]{
%%   \expandAna{\hGamma}{\hexp}{\htau}{\dexp}{\htau}{\Delta}
%% }{
%%   \expandSyn{\hGamma}{\hexp : \htau}{\htau}{\dexp}{\Delta}
%% }
\end{mathpar}

\vsepRule

\judgbox
  {\expandAna{\hGamma}{\hexp}{\htau_1}{\dexp}{\htau_2}{\Delta}}
  {$\hexp$ analyzes against type $\htau_1$ and
   expands to $\dexp$ of consistent type $\htau_2$}
\begin{mathpar}
\inferrule[EALam]{
  \arrmatch{\htau}{\tarr{\htau_1}{\htau_2}}\\
  \expandAna{\hGamma, x : \htau_1}{\hexp}{\htau_2}{\dexp}{\htau'_2}{\Delta}
}{
  \expandAna{\hGamma}{\hlam{x}{\hexp}}{\htau}{\halam{x}{\htau_1}{\dexp}}{\tarr{\htau_1}{\htau_2'}}{\Delta}
}

%% \inferrule[EALam]{
%%   \expandAna{\hGamma, x : \htau_1}{\hexp}{\htau_2}{\dexp}{\htau'_2}{\Delta}
%% }{
%%   \expandAna{\hGamma}{\hlam{x}{\hexp}}{\tarr{\htau_1}{\htau_2}}{\halam{x}{\htau_1}{\dexp}}{\tarr{\htau_1}{\htau_2'}}{\Delta}
%% }
%% 
%% \inferrule[EALamHole]{
%%   \expandAna{\hGamma, x : \tehole}{\hexp}{\tehole}{\dexp}{\htau}{\Delta}
%% }{
%%   \expandAna{\hGamma}{\hlam{x}{\hexp}}{\tehole}{\halam{x}{\tehole}{\dexp}}{\tarr{\tehole}{\htau}}{\Delta}
%% }
%% 
\inferrule[EASubsume]{
  \hexp \neq \hehole{u}\\
  \hexp \neq \hhole{\hexp'}{u}\\\\
  \expandSyn{\hGamma}{\hexp}{\htau'}{\dexp}{\Delta}\\
  \tconsistent{\htau}{\htau'}
}{
  \expandAna{\hGamma}{\hexp}{\htau}{\dexp}{\htau'}{\Delta}
}

\inferrule[EAEHole]{ }{
  \expandAna{\hGamma}{\hehole{u}}{\htau}{\dehole{u}{\idof{\hGamma}}{}}{\htau}{\Dbinding{u}{\hGamma}{\htau}}
}

\inferrule[EANEHole]{
  \expandSyn{\hGamma}{\hexp}{\htau'}{\dexp}{\Delta}\\
}{
  \expandAna{\hGamma}{\hhole{\hexp}{u}}{\htau}{\dhole{\dexp}{u}{\idof{\hGamma}}{}}{\htau}{\Delta, \Dbinding{u}{\hGamma}{\htau}}
}
\end{mathpar}
\caption{Expansion}
\label{fig:expansion}
\label{fig:expandSyn}
\label{fig:expandAna}
\end{figure}

% !TEX root = hazelnut-dynamics.tex

\begin{figure}[p]
\judgbox{\hasType{\Delta}{\hGamma}{\dexp}{\htau}}{$\dexp$ is assigned type $\htau$}
\begin{mathpar}
\inferrule[TAConst]{ }{
  \hasType{\Delta}{\hGamma}{c}{b}
}

\inferrule[TAVar]{
  x : \htau \in \hGamma
}{
	\hasType{\Delta}{\hGamma}{x}{\htau}
}

\inferrule[TALam]{
  \hasType{\Delta}{\hGamma, x : \htau_1}{\dexp}{\htau_2}
}{
  \hasType{\Delta}{\hGamma}{\halam{x}{\htau_1}{\dexp}}{\tarr{\htau_1}{\htau_2}}
}

\inferrule[TAAp]{
  \hasType{\Delta}{\hGamma}{\dexp_1}{\tarr{\htau_2}{\htau}}\\
  \hasType{\Delta}{\hGamma}{\dexp_2}{\htau_2}
}{
  \hasType{\Delta}{\hGamma}{\hap{\dexp_1}{\dexp_2}}{\htau}
}

\inferrule[TAEHole]{
  \Dbinding{u}{\hGamma'}{\htau} \in \Delta\\
  \hasType{\Delta}{\hGamma}{\sigma}{\hGamma'}
}{
  \hasType{\Delta}{\hGamma}{\dehole{u}{\sigma}{}}{\htau}
}

\inferrule[TANEHole]{
  \hasType{\Delta}{\hGamma}{\dexp}{\htau'}\\\\
  \Dbinding{u}{\hGamma'}{\htau} \in \Delta\\
  \hasType{\Delta}{\hGamma}{\sigma}{\hGamma'}
}{
  \hasType{\Delta}{\hGamma}{\dhole{\dexp}{u}{\sigma}{}}{\htau}
}

\inferrule[TACast]{
  \hasType{\Delta}{\Gamma}{\dexp}{\htau_1}\\
  \tconsistent{\htau_1}{\htau_2}
}{
  \hasType{\Delta}{\hGamma}{\dcasttwo{\dexp}{\htau_1}{\htau_2}}{\htau_2}
}

\inferrule[TAFailedCast]{
  \hasType{\Delta}{\Gamma}{\dexp}{\htau_1}\\
  \isGround{\htau_1}\\
  \isGround{\htau_2}\\
  \htau_1\neq\htau_2
}{
  \hasType{\Delta}{\hGamma}{\dcastfail{\dexp}{\htau_1}{\htau_2}}{\htau_2}
}
\end{mathpar}
\caption{Type Assignment for Internal Expressions}
\Label{fig:hasType}
\end{figure}


Each well-typed external expression~$e$ expands to a well-typed internal expression~$d$, for evaluation.
%
\Figref{fig:expansion} specifies expansion, and \Figref{fig:hasType} specifies type assignment for internal expressions.
%
% \Secref{sec:evaluation} discusses internal expression evaluation.

As with the type system for the external language (above), 
we specify expansion bidirectionally \cite{DBLP:conf/ppdp/FerreiraP14}.
%
The synthetic expansion judgement~$\expandSyn{\hGamma}{\hexp}{\htau}{\dexp}{\Delta}$  
%
produces an expansion~$d$ and a hole context~$\hDelta$ when synthesizing type $\htau$ for $\hexp$. 
%
%We say more about hole contexts, which are used in the type assignment judgement, $\hasType{\Delta}{\hGamma}{d}{\htau}$, below.
We describe hole contexts, which serve as ``inputs'' to the type assignment judgement~$\hasType{\Delta}{\hGamma}{d}{\htau}$, further below. 
%
The analytic expansion judgement~$\expandAna{\hGamma}{\hexp}{\htau}{\dexp}{\htau'}{\Delta}$, produces an expansion~$d$ of type~$\htau'$, and a hole context~$\hDelta$, when checking~$\hexp$ against~$\htau$.
%
The following theorem establishes that expansions are well-typed and in the analytic case that the assigned type, $\htau'$, is consistent with provided type, $\htau$.
%
\begin{thm}[Typed Expansion]\label{thm:typed-expansion} ~
  \begin{enumerate}[nolistsep]
    \item
      If $\expandSyn{\hGamma}{\hexp}{\htau}{\dexp}{\Delta}$
      then $\hasType{\Delta}{\hGamma}{\dexp}{\htau}$.
    \item
      If $\expandAna{\hGamma}{\hexp}{\htau}{\dexp}{\htau'}{\Delta}$
      then $\tconsistent{\htau}{\htau'}$ and $\hasType{\Delta}{\hGamma}{\dexp}{\htau'}$.
  \end{enumerate}
\end{thm}
\noindent
%
%The reason analytic expansion produces an expansion of consistent
%type is because the subsumption rule, as previously discussed, allows
%us to check an external expression against any type consistent with
%the type the expression actually synthesizes, whereas every internal
%expression can be assigned at most one type, i.e. the following
%standard unicity property holds of the type assignment system.
%
The reason that $\htau'$ is only consistent with the provided type $\htau$ is because 
%
 the subsumption rule permits us
to check an external expression against any type consistent with the
type that the expression \emph{actually} synthesizes, whereas every internal
expression can be assigned at most one type, i.e. the following
standard unicity property holds of the type assignment system.
%
\begin{thm}[Type Assignment Unicity]
  If $\hasType{\Delta}{\hGamma}{\dexp}{\htau}$
  and $\hasType{\Delta}{\hGamma}{\dexp}{\htau'}$
  then $\htau=\htau'$.
\end{thm}
\noindent
Consequently, analytic expansion reports the type actually assigned to the expansion it produces.
%
For example, we can derive that $\expandAna{\hGamma}{c}{\tehole}{c}{b}{\emptyset}$.% where $\emptyset$ is the empty hole context.

Before describing the rules in detail, let us state two other guiding theorems. 
The following theorem establishes that an expansion exists for every well-typed external expression. 
The mechanization also establishes when an expansion exists, it is unique (not shown).
 \begin{thm}[Expandability] \label{thm:expandability}~
  \begin{enumerate}[nolistsep]
    \item
      If $\hsyn{\hGamma}{\hexp}{\htau}$
      then $\expandSyn{\hGamma}{\hexp}{\htau}{\dexp}{\Delta}$
      for some $\dexp$ and $\Delta$.
    \item
      If $\hana{\hGamma}{\hexp}{\htau}$
      then $\expandAna{\hGamma}{\hexp}{\htau}{\dexp}{\htau'}{\Delta}$
      for some $\dexp$ and $\htau'$ and $\Delta$.
  \end{enumerate}
\end{thm}
% \noindent
% The following theorem establishes that when an expansion exists, it is unique.
% \begin{thm}[Expansion Unicity] \label{thm:expansion-unicity}~
%   % \begin{enumerate}[nolistsep]
%   %   \item
%       If $\expandSyn{\hGamma}{\hexp}{\htau}{\dexp}{\Delta}$
%       and $\expandSyn{\hGamma}{\hexp}{\htau'}{\dexp'}{\Delta'}$
%       then $\htau=\htau'$ and $\dexp=\dexp'$ and $\Delta=\Delta'$.
%   %   \item
%   %     If $\expandAna{\hGamma}{\hexp}{\htau_1}{\dexp}{\htau_2}{\Delta}$
%   %     and $\expandAna{\hGamma}{\hexp}{\htau_1}{\dexp'}{\htau_2'}{\Delta'}$
%   %     then $\dexp=\dexp'$ and $\htau_2=\htau_2'$ and $\Delta=\Delta'$.
%   % \end{enumerate}
% \end{thm}
% \noindent
The following theorem establishes that expansion generalizes external typing.
\begin{thm}[Expansion Generality] \label{thm:expansion-generality}~
  \begin{enumerate}[nolistsep]
    \item
      If $\expandSyn{\hGamma}{\hexp}{\htau}{\dexp}{\Delta}$
      then $\hsyn{\hGamma}{\hexp}{\htau}$.
    \item
      If $\expandAna{\hGamma}{\hexp}{\htau}{\dexp}{\htau'}{\Delta}$
      then $\hana{\hGamma}{\hexp}{\htau}$.
  \end{enumerate}
\end{thm}

The rules governing expansion of constants, variables and lambda expressions
--- Rules \rulename{ESConst}, \rulename{ESVar}, \rulename{ESLam} and \rulename{EALam}
--- mirror the corresponding type assignment rules
--- Rules \rulename{TAConst}, \rulename{TAVar} and \rulename{TALam}
--- mirror the bidirectional typing rules from Fig.~\ref{fig:bidirectional-typing}.
%
% Consequently, the corresponding cases of
% Theorem~\ref{thm:typed-expansion}, Theorem~\ref{thm:expandability} and
% Theorem~\ref{thm:expansion-generality} are straightforward.
%
To support type assignment, all lambdas in the internal language are
half-annotated---Rule \rulename{EALam} inserts the annotation when expanding an
unannotated external lambda based on the given type.
%
The rules governing hole expansion, and the rules that perform \emph{cast insertion}---those governing
function application and type ascription---are more interesting. We consider each of these two groups of rules, in turn, below.

\subsubsection{Hole Expansion}\label{sec:hole-expansion} 
Rules \rulename{ESEHole}, \rulename{ESNEHole}, \rulename{EAEHole} and \rulename{EANEHole} govern the expansion of empty and non-empty expression holes to empty and non-empty \emph{hole closures}, $\dehole{u}{\sigma}{}$ and $\dhole{\dexp}{u}{\sigma}{}$. 
%
The hole name~$u$ on a hole closure identifies the external hole to which the hole closure corresponds.
%
While we assume each hole name to be unique in the external language, once evaluation begins, there may be multiple hole closures with the same name due to substitution.
%
For example, the result from Fig.~\ref{fig:grades-example} shows four closures for the hole named 1\todo{update with actual number (four?) and hole name later}{}.
%
There, we number each hole closure for a given hole sequentially, \li{1:1}, \li{1:2} and so on, but this is strictly for the sake of presentation, so we omit hole closure numbers from the core calculus.

%
For each hole, $u$, in an external expression, the hole context generated by expansion, $\Delta$, contains a hypothesis of the form~$\Dbinding{u}{\hGamma}{\htau}$, which records the hole's type, $\tau$, and the typing context, $\Gamma$, from where it appears in the original expression.\footnote{
We use a hole context, rather than recording the typing context and type directly on each hole closure, to ensure that all closures for a hole name have the same typing context and type.}
%
We borrow this hole context notation from contextual modal type theory (CMTT) \cite{Nanevski2008}, identifying hole names with metavariables and hole contexts with modal contexts (we say more about the connection with CMTT below).
%
% Each hole expansion rule records the ``current'' typing context under which the hole is expanded.
%
In the synthetic hole expansion rules~\rulename{ESEHole} and~\rulename{ESNEHole}, the generated hole context assigns the hole type~$\tehole$ to hole name~$u$, as in the external typing rules.
%
However, the first two premises of the expansion subsumption rule~\rulename{EASubsume} disallow the use of subsumption for holes in analytic position.
%
Instead, we employ separate analytic rules~\rulename{EAEHole} and~\rulename{EANEHole}, which each record the checked type~$\tau$ in the hole context.
%
Consequently, we can use type assignment for the internal language --- the type assignment rules \rulename{TAEHole} and \rulename{TANEHole} in Fig.~\ref{fig:hasType} assign a hole closure for hole name~$u$ the corresponding type from the hole context.

Each hole closure also has an associated environment~$\sigma$ which consists of a finite substitution of the form $[d_1/x_1, ~\cdots, d_n/x_n]$ for $n \geq 0$.
%
The closure environment keeps a record of the substitutions that occur around the hole as evaluation occurs.
%
Initially, when no evaluation has yet occurred, the hole expansion rules generate the identity substitution for the typing context associated with hole name~$u$ in hole context~$\Delta$, which we notate $\idof{\hGamma}$, and define as follows.
%
\begin{defn}[Identity Substitution] $\idof{x_1 : \tau_1, ~\cdots, x_n : \tau_n} = [x_1/x_1, ~\cdots, x_n/x_n]$
\end{defn}
\noindent
The type assignment rules for hole closures~{TAEHole} and {TANEHole} each require that the hole closure environment~$\sigma$ be consistent with the corresponding typing context, written as $\hasType{\Delta}{\hGamma}{\sigma}{\hGamma'}$.
%
Formally, we define this relation in terms of type assignment as follows:
\begin{defn}[Substitution Typing]
$\hasType{\Delta}{\hGamma}{\sigma}{\hGamma'}$ iff $\domof{\sigma} = \domof{\hGamma'}$ and for each $x : \htau \in \hGamma'$ we have that $d/x \in \sigma$ and $\hasType{\Delta}{\hGamma}{d}{\tau}$.
\end{defn}
\noindent
It is easy to verify that the identity substitution satisfies this requirement, i.e. that $\hasType{\Delta}{\hGamma}{\idof{\hGamma}}{\hGamma}$. 

Empty hole closures, $\dehole{u}{\sigma}{}$,  correspond to the metavariable closures (a.k.a. deferred substitutions) from CMTT, $\cmttclo{u}{\sigma}$.
%
\Secref{sec:evaluation} defines how these closure environments evolve during evaluation.
%
Non-empty hole closures~$\dhole{d}{u}{\sigma}{}$ have no direct correspondence with a notion from CMTT (see Sec.~\ref{sec:resumption}).

\subsubsection{Cast Insertion}\label{sec:cast-insertion}
%
% USE A TOPIC SENTENCE SO THAT THE READER IS GUIDED A LITTLE MORE, e.g.,
%
Holes in types require us to defer certain structural checks to run time. 
%-----------------------------------------------------------------------------------------------
%
To see why this is necessary, consider the following
example: $\hap{(\halam{x}{\tehole}{\hap{x}{c}})}{c}$.
%
Viewed as an external expression, this example synthesizes type
$\tehole$, since the hole type annotation on variable~$x$ permits
applying~$x$ as a function of type~$\tarr{\tehole}{\tehole}$, and base
constant~$c$ may be checked against type~$\tehole$, by subsumption.
%
However, viewed as an internal expression, this example is not
well-typed---the type assignment system defined in
\Figref{fig:hasType} lacks subsumption.
%
Indeed, it would violate type safety if we could assign a type to this
example in the internal language, because beta reduction of this
example viewed as an internal expression would result in $c(c)$, which
is clearly not well-typed.
%
The difficulty arises because leaving the argument type unknown also leaves unknown how
the argument is being used (in this case, as a function).\footnote{In a system where type reconstruction is first used
to try to fill in type holes, we could express a similar example by
using $x$ at two or more different types, thereby causing type
reconstruction to fail.
%
% On the other hand, if it is acceptable to arbitrarily choose one of
%the possible types, and type reconstruction is complete, then type
%holes will never appear in the internal language and the cast
%insertion machinery described in this section can be omitted
%entirely, leaving only the hole closure machinery described
%previously.
}
By our interpretation of hole types as unknown types from gradual type
theory, we can address the problem by performing cast insertion.
%

The cast form in Hazelnut Live is $\dcasttwo{\dexp}{\htau_1}{\htau_2}$.
%
This form serves to ``box'' an expression of type $\htau_1$ for
treatment as an expression of a consistent type $\htau_2$
(Rule TACast in \Figref{fig:hasType}).%
\footnote{
In the earliest work on gradual type theory, the cast form only gave
the target type~$\htau_2$ \cite{Siek06a}, but it simplifies the dynamic semantics substantially
to include the assigned type~$\htau_1$ in the syntax \cite{DBLP:conf/snapl/SiekVCB15}.
}

Expansion inserts casts at function applications and ascriptions.
%
The latter is more straightforward: Rule~\rulename{ESAsc}
in \Figref{fig:expandSyn} inserts a cast from the assigned type to the
ascribed type.
%
Theorem~\ref{thm:typed-expansion} inductively ensures that the two
types are consistent.
%
We include ascription for exposition purposes---this form is derivable
by using application together with the half-annotated identity, $e
: \tau = \hap{(\halam{x}{\htau}{x})}{e}$; as such, application
expansion, discussed below, is more general.

Rule~\rulename{ESAp} expands function applications.
%
To understand the rule, consider the expansion of external
expression~$\hap{(\halam{x}{\tehole}{\hap{x}{c}})}{c}$, the example
discussed above:
\[
        \hap{\dcasttwo{
        (\halam{x}{\tehole}{\underbrace{
                \hap{\dcasttwo{x}{\tehole}{\tarr{\tehole}{\tehole}}}
                {\dcasttwo{c}{b}{\tehole}}
                }_{\textrm{expansion of function body}~x(c)}
        }
        )}{\tarr{\tehole}{\tehole}}{
           \tarr{\tehole}{\tehole}}
           }
           {\dcasttwo{c}{b}{\tehole}}
\]
Consider the (indicated) function body,
%
where expansion inserts a cast on both the function expression~$x$ and its argument~$c$.
%
Together, these casts for~$x$ and~$c$ permit assigning a type to the
function body according to the rules in \Figref{fig:hasType}, where we
could not do so under the same context without casts.
%
We separately consider the expansions of~$x$ and of~$c$.

First, consider the function position of this application, here variable~$x$.
%
Without any cast, the type of variable~$x$ is hole type~$\tehole$;
however, the inserted cast on~$x$ permits treating it as though it has
arrow type $\tarr{\tehole}{\tehole}$, as described
in \Secref{sec:external-statics}.
%
The first three premises of Rule~\rulename{ESAp} accomplish this
%
by first synthesizing a type for the function expression, here
$\tehole$, then
%
by determining the matched arrow type~$\tarr{\tehole}{\tehole}$, and
finally,
%
by performing analytic expansion on the function expression with this
matched arrow type.
%
The resulting expansion has some type~$\tau_1'$ consistent with the matched arrow type.
%
In this case, because the subexpression~$x$ is a variable, analytic
expansion goes through subsumption so that type~$\tau_1'$ is
simply~$\tehole$.
%
The conclusion of the rule inserts the corresponding cast.
%
We go through type synthesis, \emph{then} analytic expansion, so that the hole
context records the matched arrow type for holes in function position,
rather than the type~$\tehole$ for all such holes, as would be the case
in a variant of this rule using synthetic expansion for the function
expression.

Next, consider the application's argument, here constant~$c$.
%
The conclusion of Rule~\rulename{ESAp} inserts the cast on the argument's
expansion, from the type it is assigned by the final premise of the
rule~(type~$b$ here), to the argument type of the matched arrow type
of the function expression~(type~$\tehole$ here).

The example's second, outermost application goes through the same
application expansion rule.
%
In this case, the cast on the function is the identity cast for
$\tarr{\tehole}{\tehole}$.
%
For simplicity, we do not attempt to avoid the insertion of identity
casts in the core calculus; these will simply never fail during
evaluation.
%
However, it is safe in practice to eliminate such identity casts during expansion,
and some formal accounts of gradual typing do so by defining three
application expansion rules, including the original account of \citet{Siek06a}.

\subsection{Dynamic Semantics}
\label{sec:evaluation}

To recap, the result of expansion is a well-typed internal expression with appropriately initialized hole closures and casts. 
We can now specify, in Figures~\ref{fig:isGround}-\ref{fig:step}, the dynamic semantics of \HazelnutLive as a ``small-step'' transition system equipped with a meaningful notion of type safety even for incomplete programs, i.e. expressions typed under a non-empty hole context, $\Delta$. 
We establish that evaluation does not stop immediately when it encounters a hole, nor when a cast fails, by precisely characterizing when evaluation \emph{does} stop. 
We also establish that an essentially standard notion of type safety holds when running complete programs. 

It is perhaps worth stating at the outset that a dynamic semantics equipped with these properties does not simply ``fall out'' from the observations made above that (1) empty hole closures correspond to metavariable closures from CMTT \cite{Nanevski2008} and (2) casts also arise in gradual type theory \cite{DBLP:conf/snapl/SiekVCB15}. 
%
We say more after discussing type safety below.\todo{do this, or change to related work}{}
%

%  specify the dynamic semantics of \HazelnutLive. 
% %
% The dynamic semantics is capable of running incomplete programs, i.e. those with hole closures, without aborting at holes, or when a cast fails.
% %







The dynamic semantics consists of ,
%
where cast-related machinery closely follows the cast calculus from
the ``refined'' account of the gradually typed lambda calculus
by \citet{DBLP:conf/snapl/SiekVCB15}, which is known to be
theoretically well-behaved.
%
In particular, \Figref{fig:isGround} defines the judgement
$\isGround{\htau}$, which distinguishes the base type~$b$ and the
least specific arrow type~$\tarr{\tehole}{\tehole}$, as \emph{ground
types}; this judgement helps simplify the treatment of function casts.
%
\Figref{fig:isFinal} defines the judgement $\isFinal{d}$, which
distinguishes the final forms of the transition system.
%
There are two classes of final forms: (possibly-)boxed values and
indeterminate forms.%
%
\footnote{
        In most accounts of the cast calculus, values and ground types
        are distinguished with separate grammars together with an
        implicit identification convention.
        %
        Our judgmental formulation is more faithful to the mechanization and
        cleaner for our purposes, because we are distinguishing several
        classes of final forms.
}


The judgement $\isBoxedValue{d}$ defines (possibly-)boxed values as
either ordinary values~(Rule~\rulename{BVVal}), or one of two cast forms: casts
between function types (except identity casts) and casts from a ground
type to the hole type. In each case, the cast operates on a boxed
value.
%
%, which correspond to the values from the cast calculus and include
%the classic values from the lambda calculus, distingui%shed by
%$\isValue{d}$,
%
The judgement $\isIndet{d}$ defines \emph{indeterminate} forms, so
named because they arise from the presence of expression holes and
failed casts in a program, and so, conceptually, their ultimate value
awaits programmer action (see Sec.~\ref{sec:resumption}).
%
The first two rules specify that \emph{empty} hole closures are always
indeterminate, and that \emph{non}-empty hole closures are indeterminate when
they consist of a \emph{final} inner expression.
%
Below, we discuss the the remaining indeterminate forms simultaneously
with the corresponding transition rules, also discussed
below.\todo{relationship to weak head normal forms}

The transition rules are defined in Fig.~\ref{fig:instruction-transitions}-\ref{fig:step}.
%
Top-level transitions are \emph{steps}~$\stepsToD{}{d}{d'}$, governed by Rule~\rulename{Step} in \Figref{fig:step}, which
%
(1) decomposes $d$ into an evaluation context, $\evalctx$, and a selected sub-term, $d_0$;
%
(2) takes an \emph{instruction transition}, $\reducesE{}{d_0}{d_0'}$, as specified in \Figref{fig:instruction-transitions};
%
and (3) places $d_0'$ back in the selected position, indicated in
the evaluation context by the \emph{mark}~$\evalhole$, to obtain $d'$.%
\footnote{
        We say ``mark'' here to avoid confusion with the (orthogonal) ``holes'' of \HazelnutLive.
        %
        %the form $\evalhole$ in the grammar of
        %evaluation contexts is referred to as the \emph{hole}, but
        %this hole is a technical device entirely orthogonal to the
        %holes of this paper, so we use the term ``mark'' instead.
        }
%

%% \matt{Next paragraph can be dropped entirely for space}
%% %
%% This approach was originally developed in the reduction semantics of \citet{DBLP:journals/tcs/FelleisenH92} and is the predominant style of operational semantics in the literature on gradual typing. 
%% Because we distinguish final forms judgementally, rather than syntactically, we use a judgemental formulation of this approach called a \emph{contextual dynamics} by \citet{pfpl}. 
%% It would be straightforward to derive an equivalent structural operational semantics \cite{DBLP:journals/jlp/Plotkin04a} by using search rules instead of evaluation contexts (\citet{pfpl} relates the two approaches).
%% \todo{Put the search rules in the appendix?}{}

%% \begin{figure}[t]
\judgbox{\stepsToD{\Delta}{\dexp_1}{\dexp_2}}{$\dexp_1$
steps to $\dexp_2$}
\begin{mathpar}
\inferrule[STEHoleEvaled]
{ }
{\stepsToD{\Delta}{\dehole{\mvar}{\subst}{\unevaled}}{\dehole{\mvar}{\subst}{\evaled}}}

\inferrule[STNEHoleStep]
{\stepsToD{\Delta}{\dexp_1}{\dexp_2} }
{\stepsToD{\Delta}{\dhole{\dexp_1}{\mvar}{\subst}{\unevaled}}{\dhole{\dexp_2}{\mvar}{\subst}{\unevaled}}}

\inferrule[STNEHoleEvaled]
{\isFinal{\dexp}}
{\stepsToD{\Delta}{\dhole{\dexp}{\mvar}{\subst}{\unevaled}}{\dhole{\dexp}{\mvar}{\subst}{\evaled}}}

\inferrule[STCast]
{
\isValue{\dexp}\\
\hasType{\Delta}{\emptyset}{\dexp}{\htau_2} \\
\tconsistent{\tau_1}{\tau_2}}
{\stepsToD{\Delta}{\dcast{\htau_1}{\dexp}}{\dexp}}

\inferrule[STApStep1]
{\stepsToD{\Delta}{\dexp_1}{\dexp_1'}}
{\stepsToD{\Delta}{\dap{\dexp_1}{\dexp_2}}{\dap{\dexp_1'}{\dexp_2}}}

\inferrule[STApStep2]
{ \isFinal{\dexp_1} \\ \stepsToD{\Delta}{\dexp_2}{\dexp_2'}}
{\stepsToD{\Delta}{\dap{\dexp_1}{\dexp_2}}{\dap{\dexp_1}{\dexp_2'}}}

\inferrule[STApBeta]
{ \isFinal{\dexp_2} }
{\stepsToD{\Delta}{\dapP{\dlam{x}{\htau}{\dexp_1}}{\dexp_2}}{ [\dexp_2/x]\dexp_1 }}
\end{mathpar}
\caption{Structural Dynamics}
\label{fig:stepsTo}
\end{figure}


\subsubsection{Application and Substitution} 
%
Rule \rulename{ITBeta} in Fig.~\ref{fig:instruction-transitions} specifies the
standard beta reduction transition.
%
The bracketed premises of the form $\maybePremise{\isFinal{\dexp}}$ in
Fig.~\ref{fig:instruction-transitions}-\ref{fig:step} may be \emph{included}
to specify an eager, left-to-right evaluation strategy, or \emph{excluded} to
leave the evaluation order unspecified.
%
In our metatheory, we use the latter version of the rules, both for
the sake of generality, and to support fill-and-resume~(\Secref{sec:resumption}).

A substitution~$[d/x]d'$ operates in the standard capture-avoiding manner~\cite{pfpl}. 
%
The only cases of special interest arise when substitution reaches a hole closure:
\[
\begin{array}{rcl}
  [d/x]\dehole{u}{\sigma}{} & = & \dehole{u}{[d/x]\sigma}{} \\%
  \substitute{d}{x}{\dhole{d'}{u}{\sigma}{}} & = & \dhole{[d/x]d'}{u}{[d/x]\sigma}{}
\end{array}
\]
In both cases, we write~$[d/x]\sigma$ to perform substitution on each expression in the hole environment~$\sigma$. 
%
For example, $\stepsToD{}
    {\hap{(\halam{x}{b}{\halam{y}{b}{\dehole{u}{[x/x, y/y]}{}}})}{c}}
    {\halam{y}{b}{\dehole{u}{[c/x, y/y]}{}}}$. 
%
Reduction may duplicate hole closures, just as with any other substituted term;
%
consequently, the environments of different closures for the same hole name may differ, 
e.g., when an reduction applies a function with a hole closure body multiple times.%
\todo{show example here or refer back to sec 2?}{} 
%
Hole closures may also appear within the environments of other hole
closures, giving rise to the closure paths described in
Sec.~\ref{sec:paths}\todo{where?}{}.

% !TEX root = hazelnut-dynamics.tex

\begin{figure}
\begin{subfigure}[t]{0.5\textwidth}
\judgbox{\isGround{\htau}}{$\htau$ is a ground type}
\begin{mathpar}
\inferrule[GBase]{ }{
  \isGround{b}
}

\inferrule[GHole]{ }{
  \isGround{\tarr{\tehole}{\tehole}}
}
\end{mathpar}
\end{subfigure}
\hfill
\begin{subfigure}[t]{0.46\textwidth}
\judgbox{\groundmatch{\htau}{\htau'}}{$\htau$ has matched ground type $\htau'$}
\begin{mathpar}
\inferrule[MGArr]{
  \tarr{\htau_1}{\htau_2}\neq\tarr{\tehole}{\tehole}
}{
  \groundmatch{\tarr{\htau_1}{\htau_2}}{\tarr{\tehole}{\tehole}}
}
\end{mathpar}
\end{subfigure}
\CaptionLabel{Ground Types}{fig:isGround}
\label{fig:groundmatch}
\end{figure}

% !TEX root = hazelnut-dynamics.tex
\begin{figure}[t]

\begin{tabular}[t]{cc}

\begin{minipage}{0.5\textwidth}
\judgbox{\isFinal{\dexp}}{$\dexp$ is final}
\begin{mathpar}
%% \inferrule[FVal]
%% {\isValue{\dexp}}{\isFinal{\dexp}}
\inferrule[FBoxedVal]
{\isBoxedValue{\dexp}}{\isFinal{\dexp}}
\and
\inferrule[FIndet]
{\isIndet{\dexp}}{\isFinal{\dexp}}
\end{mathpar}
\end{minipage}

& 

\begin{minipage}{0.5\textwidth}
    
\judgbox{\isValue{\dexp}}{$\dexp$ is a value}
\begin{mathpar}
\inferrule[VConst]{ }{
  \isValue{c}
}

\inferrule[VLam]{ }{
  \isValue{\halam{x}{\htau}{\dexp}}
}
\end{mathpar}
\end{minipage}

\end{tabular}

\vsepRule

\judgbox{\isBoxedValue{\dexp}}{$\dexp$ is a boxed value}
\begin{mathpar}
\inferrule[BVVal\rkc{name?}]{
  \isValue{\dexp}
}{
  \isBoxedValue{\dexp}
}

\inferrule[BVCastArr\rkc{name?}]{
  \tarr{\htau_1}{\htau_2} \neq \tarr{\htau_3}{\htau_4}\\
  \isBoxedValue{\dexp}
}{
  \isBoxedValue{\dcasttwo{\dexp}{\tarr{\htau_1}{\htau_2}}{\tarr{\htau_3}{\htau_4}}}
}

\inferrule[BVHoleCast\rkc{name?}]{
  \isBoxedValue{\dexp}\\
  \isGround{\htau}
}{
  \isBoxedValue{\dcasttwo{\dexp}{\htau}{\tehole}}
}
\end{mathpar}

\vsepRule

\judgbox{\isIndet{\dexp}}{$\dexp$ is indeterminate}
\begin{mathpar}
\inferrule[IEHole]
{ }
{\isIndet{\dehole{\mvar}{\subst}{}}}

\inferrule[INEHole]
{\isFinal{\dexp}}
{\isIndet{\dhole{\dexp}{\mvar}{\subst}{}}}

\inferrule[IAp]
{\dexp_1\neq
   \dcasttwo{\dexp_1'}
            {\tarr{\htau_1}{\htau_2}}
            {\tarr{\htau_3}{\htau_4}}\\
 \isIndet{\dexp_1}\\
% \isFinal{\dexp_2}~\text{\cy{??}}}
 \isFinal{\dexp_2}}
{\isIndet{\dap{\dexp_1}{\dexp_2}}}

\inferrule[ICastGroundHole] {
  \isIndet{\dexp}\\
  \isGround{\htau}
}{
  \isIndet{\dcasttwo{\dexp}{\htau}{\tehole}}
}

\inferrule[ICastHoleGround] {
  \dexp\neq\dcasttwo{\dexp'}{\htau'}{\tehole}\\
  \isIndet{\dexp}\\
  \isGround{\htau}
}{
  \isIndet{\dcasttwo{\dexp}{\tehole}{\htau}}
}

\inferrule[ICastArr]{
  \tarr{\htau_1}{\htau_2} \neq \tarr{\htau_3}{\htau_4}\\
  \isIndet{\dexp}
}{
  \isIndet{\dcasttwo{\dexp}{\tarr{\htau_1}{\htau_2}}{\tarr{\htau_3}{\htau_4}}}
}

\inferrule[IFailedCast] {
  \isFinal{\dexp}\\
  \isGround{\htau_1}\\
  \isGround{\htau_2}\\
  \htau_1\neq\htau_2
}{
  \isIndet{\dcastfail{\dexp}{\htau_1}{\htau_2}}
}

%% \inferrule[ICast]
%% {\isIndet{\dexp}}
%% {\isIndet{\dcast{\htau}{\dexp}}}

\end{mathpar}

%\vsepRule

\CaptionLabel{Final Forms}{fig:isFinal}
\label{fig:isValue}
\label{fig:isIndet}
\end{figure}




The ITBeta rule is not the only rule we need to handle function
application, because lambdas merely one (final) form of arrow type.
%
Two other situations may also arise.

First, the expression in function position might be a cast between
arrow types, in which case we apply the arrow cast conversion rule,
Rule \rulename{ITApCast}, to rewrite the application form, obtaining an
equivalent application where the expression~$d_1$ under the function
cast is exposed.
%
We know from inverting the typing rules that~$d_1$ has type
$\tarr{\htau_1}{\htau_2}$, and that~$d_2$ has type~$\htau_1'$, where
$\tconsistent{\htau_1}{\htau_1'}$; consequently, we maintain type
safety by placing a cast on~$d_2$ from~$\htau_1'$ to~$\htau_1$.
%
The result of this application has type $\htau_2$, but the
original cast promised that the result would have consistent type
$\htau_2'$, so we also need a cast on the result from $\htau_2$ to
$\htau_2'$.

Second, the expression in function position may be indeterminate,
where arrow cast conversion is not applicable,
e.g. $\hap{(\dehole{u}{\sigma}{})}{c}$.
%
In this case, the application is indeterminate (Rule~\rulename{IAp}
in \Figref{fig:isFinal}), and the application's reduction reduces no
further.

% !TEX root = hazelnut-dynamics.tex
\begin{comment}
\begin{figure}[t]

\begin{comment}
\vsepRule

\judgbox{\isevalctx{\evalctx}}{$\evalctx$ is an evaluation context}
\begin{mathpar}
\inferrule[ECDot]{ }{
  \isevalctx{\evalhole}
}

%% \inferrule[ECLam]{
%%   \isevalctx{\evalctx}
%% }{
%%   \isevalctx{\halam{x}{\htau}{\evalctx}}
%% }

\inferrule[ECAp1]{
  \isevalctx{\evalctx}
}{
  \isevalctx{\hap{\evalctx}{\dexp}}
}

\inferrule[ECAp2]{
  \maybePremise{\isFinal{\dexp}}\\
  \isevalctx{\evalctx}
}{
  \isevalctx{\hap{\dexp}{\evalctx}}
}

\inferrule[ECNEHole]{
  \isevalctx{\evalctx}
}{
  \isevalctx{\dhole{\evalctx}{\mvar}{\subst}{}}
}

\inferrule[ECCast]{
  \isevalctx{\evalctx}
}{
  \isevalctx{\dcasttwo{\evalctx}{\htau_1}{\htau_2}}
}

\inferrule[ECFailedCast]{
  \isevalctx{\evalctx}
}{
  \isevalctx{\dcastfail{\evalctx}{\htau_1}{\htau_2}}
}
\end{mathpar}
% \end{comment}
\vsepRule


\caption{Evaluation Contexts}
\label{fig:eval-contexts}
\end{figure}
\end{comment}

%% \vsepRule

\begin{figure}
\judgbox{\reducesE{}{\dexp}{\dexp'}}{$\dexp$ takes an instruction transition to $\dexp'$}
\begin{mathpar}
\inferrule[ITBeta]{
  \maybePremise{\isFinal{\dexp_2}}
}{
  \reducesE{}{\hap{(\halam{x}{\htau}{\dexp_1})}{\dexp_2}}{[\dexp_2/x]\dexp_1}
}

\inferrule[ITApCast]{
  \maybePremise{\isFinal{\dexp_1}}\\
  \maybePremise{\isFinal{\dexp_2}}\\
  \tarr{\htau_1}{\htau_2} \neq \tarr{\htau_1'}{\htau_2'}
}{
  \reducesE{}
    {\hap{\dcasttwo{\dexp_1}{\tarr{\htau_1}{\htau_2}}{\tarr{\htau_1'}{\htau_2'}}}{\dexp_2}}
    {\dcasttwo{(\hap{\dexp_1}{\dcasttwo{\dexp_2}{\htau_1'}{\htau_1}})}{\htau_2}{\htau_2'}}
}

\inferrule[ITCastId]{
  \maybePremise{\isFinal{\dexp}}
}{
  \reducesE{}{\dcasttwo{\dexp}{\htau}{\htau}}{\dexp}
}

\inferrule[ITCastSucceed]{
  \maybePremise{\isFinal{\dexp}}\\
  \isGround{\htau}
}{
  \reducesE{}{\dcastthree{\dexp}{\htau}{\tehole}{\htau}}{\dexp}
}

\inferrule[ITCastFail]{
  \maybePremise{\isFinal{\dexp}}\\
  \htau_1\neq\htau_2\\\\
  \isGround{\htau_1}\\
  \isGround{\htau_2}
}{
  \reducesE{}
    {\dcastthree{\dexp}{\htau_1}{\tehole}{\htau_2}}
    {\dcastfail{\dexp}{\htau_1}{\htau_2}}
}

\inferrule[ITGround]{
  \maybePremise{\isFinal{\dexp}}\\
  \groundmatch{\htau}{\htau'}
}{
  \reducesE{}
    {\dcasttwo{\dexp}{\htau}{\tehole}}
    {\dcastthree{\dexp}{\htau}{\htau'}{\tehole}}
}

\inferrule[ITExpand]{
  \maybePremise{\isFinal{\dexp}}\\
  \groundmatch{\htau}{\htau'}
}{
  \reducesE{}
    {\dcasttwo{\dexp}{\tehole}{\htau}}
    {\dcastthree{\dexp}{\tehole}{\htau'}{\htau}}
}

%% \inferrule[ITCast]{
%%   \isFinal{d}\\
%%   \hasType{\Delta}{\emptyset}{d}{\tau_2}\\
%%   \tconsistent{\tau_1}{\tau_2}
%% }{
%%   \reducesE{\Delta}{\dcast{\htau_1}{d}}{d}
%% }
%% 
%% \inferrule[ITEHole]{ }{
%%   \reducesE{\Delta}{\dehole{\mvar}{\subst}{\unevaled}}{\dehole{\mvar}{\subst}{\evaled}}
%% }
%% 
%% \inferrule[ITNEHole]{
%%   \isFinal{d}
%% }{
%%   \reducesE{\Delta}{\dhole{d}{\mvar}{\subst}{\unevaled}}{\dhole{d}{\mvar}{\subst}{\evaled}}
%% }
\end{mathpar}
\CaptionLabel{Instruction Transitions}{fig:instruction-transitions}
\end{figure}

\begin{figure}
$\arraycolsep=4pt\begin{array}{rllllll}
\mathsf{EvalCtx} & \evalctx & ::= &
  \evalhole ~\vert~
  \hap{\evalctx}{\dexp} ~\vert~
  \hap{\dexp}{\evalctx} ~\vert~
  \dhole{\evalctx}{\mvar}{\subst}{} ~\vert~
  \dcasttwo{\evalctx}{\htau}{\htau} ~\vert~
  \dcastfail{\evalctx}{\htau}{\htau}
\end{array}$

\vsepRule

\judgbox{\selectEvalCtx{\dexp}{\evalctx}{\dexp'}}{$\dexp$ is obtained by placing $\dexp'$ at the mark in $\evalctx$}
\begin{mathpar}
\inferrule[FHOuter]{ }{
  \selectEvalCtx{\dexp}{\evalhole}{\dexp}
}

%% \inferrule[FLam]{
%%   \selectEvalCtx{d}{\evalctx}{d'}
%% }{
%%   \selectEvalCtx{\halam{x}{\htau}{d}}{\halam{x}{\htau}{\evalctx}}{d'}
%% }

\inferrule[FHAp1]{
  \selectEvalCtx{\dexp_1}{\evalctx}{\dexp_1'}
}{
  \selectEvalCtx{\hap{\dexp_1}{\dexp_2}}{\hap{\evalctx}{\dexp_2}}{\dexp_1'}
}

\inferrule[FHAp2]{
  \maybePremise{\isFinal{\dexp_1}}\\
  \selectEvalCtx{\dexp_2}{\evalctx}{\dexp_2'}
}{
  \selectEvalCtx{\hap{\dexp_1}{\dexp_2}}{\hap{\dexp_1}{\evalctx}}{\dexp_2'}
}

%% \inferrule[FHEHole]{ }{
%%   \selectEvalCtx{\dehole{\mvar}{\subst}{}}{\evalhole}{\dehole{\mvar}{\subst}{}}
%% }
%% 
%% \inferrule[FHNEHoleEvaled]{ }{
%%   \selectEvalCtx{\dhole{d}{\mvar}{\subst}{\evaled}}{\evalhole}{\dhole{d}{\mvar}{\subst}{\evaled}}
%% }

\inferrule[FHNEHoleInside]{
  \selectEvalCtx{\dexp}{\evalctx}{\dexp'}
}{
  \selectEvalCtx{\dhole{\dexp}{\mvar}{\subst}{}}{\dhole{\evalctx}{\mvar}{\subst}{}}{\dexp'}
}

%% \inferrule[FHNEHoleFinal]{
%%   \isFinal{d}
%% }{
%%   \selectEvalCtx{\dhole{d}{\mvar}{\subst}{\unevaled}}{\evalhole}{\dhole{d}{\mvar}{\subst}{\unevaled}}
%% }

\inferrule[FHCastInside]{
  \selectEvalCtx{\dexp}{\evalctx}{\dexp'}
}{
  \selectEvalCtx{\dcasttwo{\dexp}{\htau_1}{\htau_2}}
                {\dcasttwo{\evalctx}{\htau_1}{\htau_2}}
                {\dexp'}
}

\inferrule[FHFailedCast]{
  \selectEvalCtx{\dexp}{\evalctx}{\dexp'}
}{
  \selectEvalCtx{\dcastfail{\dexp}{\htau_1}{\htau_2}}
                {\dcastfail{\evalctx}{\htau_1}{\htau_2}}
                {\dexp'}
}

%% \inferrule[FHCastFinal]{
%%   \isFinal{d}
%% }{
%%   \selectEvalCtx{\dcast{\htau}{d}}{\evalhole}{\dcast{\htau}{d}}
%% }
\end{mathpar}

\vsepRule

\judgbox{\stepsToD{}{\dexp}{\dexp'}}{$\dexp$ steps to $\dexp'$}
\vspace{-10px}
\begin{mathpar}
\inferrule[Step]{
  \selectEvalCtx{d}{\evalctx}{\dexp_0}\\
  \reducesE{}{\dexp_0}{\dexp_0'}\\
  \selectEvalCtx{\dexp'}{\evalctx}{\dexp_0'}
}{
  \stepsToD{}{\dexp}{\dexp'}
}
\end{mathpar}
\CaptionLabel{Evaluation Contexts and Steps}{fig:step}
\end{figure}

The rules take care to maintain the following property, which establishes that expressions that step are not final, and that final expressions do not step.%
\todo{prove this as stated from Ian's disjointness proofs}{}%
\todo{do we need a typing premise?}{}
\begin{thm}[Finality] There does not exist $d$ such that both $\isFinal{d}$ and $\stepsToD{}{d}{d'}$ for some $d'$.
\end{thm}

\subsubsection{Casts}
Rule \rulename{ITCastId} drops identity casts. The remaining instruction
transition rules assign meaning to non-trivial casts.
%
As discussed in Sec.~\ref{sec:cast-insertion}, the structure of a term
cast to hole type is statically obscure,
%
so we must wait for \emph{use} of the term at some other type, via a
cast away from hole type, to detect dynamic type errors.
%
Rules {ITCastSucceed} and {ITCastFail} handle this situation when the
two types involved are ground types (Fig.~\ref{fig:isGround}).
%
If the two ground types are equal, then the cast succeeds and the cast
may be dropped.
%
If they are not equal, then the cast fails and the failed cast form,
$\dcastfail{\dexp}{\htau_1}{\htau_2}$, arises.
%
Rule \rulename{TAFailedCast} specifies that a failed cast is well-typed exactly
when $d$ has ground type $\tau_1$ and $\tau_2$ is a ground type
disequal to $\tau_1$.
%
Rule \rulename{IFailedCast} specifies that a failed cast operates as an
indeterminate form once $d$ is final.
%
This rule allows evaluation to proceed after cast failure, much as
with hole closures.

Rules {TAFailedCast} and {IFailedCast} each operate at ground type. 
%
The two remaining instruction transition rules, Rule~\rulename{ITGround}
and~{ITExpand}, insert intermediate casts from non-ground type to a
consistent ground type, and \emph{vice versa}.
%
These rules serve as technical devices, permitting us to restrict our
interest exclusively to ground types.
%
Here, the only non-ground types are the arrow types, so the grounding
judgement~$\groundmatch{\tau_1}{\htau_2}$ (\Figref{fig:groundmatch}),
produces the ground arrow type~$\tarr{\tehole}{\tehole}$.
%
More generally, this judgement is governed by the following invariant:
\begin{lem}[Grounding] 
  If $\groundmatch{\htau}{\htau'}$
  then $\isGround{\htau'}$
  and $\tconsistent{\htau}{\htau'}$
  and $\htau\neq\htau'$.
\end{lem}

In all other cases, casts either consist of boxed values or
indeterminate terms according to the remaining rules
in \Figref{fig:isFinal}.
%
In particular, Rule \rulename{ICastHoleGround} handles casts that are not of
the form $\dcastthree{\dexp}{\htau_1}{\tehole}{\htau_2}$, from hole to
ground type.
%
In the cast calculus, since the only canonical form of type $\tehole$
is $\dcasttwo{\dexp}{\htau_1}{\tehole}$, there exist no such
irreducible terms.

\subsubsection{Type Safety} 
%
Type safety establishes that the static and dynamic semantics of a
language cohere.
%
We follow the approach developed by \citet{wright94:_type_soundness},
now standard \cite{pfpl}, which distinguishes two type safety
properties, preservation and progress.
%
To permit the evaluation of incomplete programs, we establish these
properties for terms typed under arbitrary hole context $\Delta$.
%
We assume an empty context for~$\hGamma$; to run open programs, the
system may treat free variables as empty holes with a corresponding
name.

The preservation theorem establishes that transitions preserve type
assignment, i.e. that the type of an expression accurately predicts
the type of the result of reducing that expression.

\begin{thm}[Preservation]
  If $\hasType{\Delta}{\emptyset}{\dexp}{\htau}$ and
  $\stepsToD{\Delta}{\dexp}{\dexp'}$ then
  $\hasType{\Delta}{\emptyset}{\dexp'}{\htau}$.
\end{thm}
\noindent
%
The proof relies on an analogous preservation lemma for instruction
transitions and a standard substitution lemma (stated in
Appendix~\ref{sec:additional-defns}).
%
Hole closures can disappear during evaluation, so we rely on weakening
of $\Delta$.

The progress theorem establishes that the dynamic semantics accounts
for every well-typed term, i.e. that we have not forgotten some
necessary rules or premises.
%
\begin{thm}[Progress]
  If $\hasType{\Delta}{\emptyset}{\dexp}{\htau}$ then either
  (a) $\stepsToD{}{\dexp}{\dexp'}$ or
  (b) $\isBoxedValue{\dexp}$ or 
  (c) $\isIndet{\dexp}$.
\end{thm}
\noindent
The key to establishing the progress theorem under a non-empty hole
context consists of explicitly accounting for indeterminate forms,
i.e. those rooted at either a hole closure or a failed cast.
%
The proof relies on canonical forms lemmas stated in
Appendix~\ref{sec:additional-defns}.

\subsubsection{Complete Programs} 
%
Although this paper focuses running \emph{incomplete} programs, it helps
to know that its machinery does not interfere
with reasoning about the behavior of \emph{complete} programs, i.e. those
with no holes.
%
Appendix~\ref{sec:additional-defns} defines complete programs via judgements~$\isComplete{e}$ and~$\isComplete{d}$. 
%
Failed casts cannot appear in complete internal expressions. 
%
The following theorem establishes that expansion preserves program completeness.
\begin{thm}[Complete Expansion] ~
  \begin{enumerate}[nolistsep]
    \item
      If $\isComplete{\hexp}$
      and $\expandSyn{\hGamma}{\hexp}{\htau}{\dexp}{\Delta}$
      then $\isComplete{\dexp}$.
    \item
      If $\isComplete{\hexp}$
      and $\expandAna{\hGamma}{\hexp}{\htau}{\dexp}{\htau'}{\Delta}$
      then $\isComplete{\dexp}$.
  \end{enumerate}
\end{thm}

The following preservation theorem establishes that stepping preserves program completeness.\todo{can we do complete preservation without the typing assumption?}
\begin{thm}[Complete Preservation]
  If $\hasType{\hDelta}{\emptyset}{\dexp}{\htau}$
  and $\isComplete{\dexp}$
  and $\stepsToD{}{\dexp}{\dexp'}$
  then $\hasType{\hDelta}{\emptyset}{\dexp'}{\htau}$
  and $\isComplete{\dexp'}$.
\end{thm}

The following progress theorem establishes that evaluating a complete
program always results in classic values, not boxed values nor
indeterminate forms.
%
\begin{thm}[Complete Progress]
  If $\hasType{\hDelta}{\emptyset}{\dexp}{\htau}$
  and $\isComplete{\dexp}$
  then either $\stepsToD{}{\dexp}{\dexp'}$
  or $\isValue{\dexp}$.
\end{thm}

%% \begin{figure}[!ht]
%%   \begin{definition}
%%     $\hasType{\Delta}{\hGamma}{\sigma}{\hGamma'}$ iff for each $\dexp/x \in \sigma$, we have $x : \htau \in \hGamma'$ and $\hasType{\Delta}{\hGamma}{\dexp}{\htau}$.
%%   \end{definition}
%%   \caption{substitution type assignment}
%%   \label{fig:subassign}
%% \end{figure}


%% \begin{figure}[!ht]
%%   \caption{substitution type assignment}
%% \end{figure}


\subsection{Agda Mechanization}
\label{sec:agda-mechanization}
\vspace{-3px}

The supplemental material includes our Agda-based
mechanization  \cite{norell2009dependently,norell:thesis} 
of the semantics and metatheory of \HazelnutLive, 
%including proofs of all of the 
including the theorems stated above, and all necessary lemmas\todo{make this so ian!}{}. 
%
%We choose the 
%(as did the mechanization of \Hazelnut by \citet{popl-paper}, though only a few definitions are common). 
%
%Agda is a good choice because it is designed to explicitly communicate a proof's structure, as is our goal, rather than relying on proof automation. 
%
%Agda itself was also an inspiration for this work because it supports holes, albeit in a more limited form than described here (cf. Sec.~\ref{sec:intro}). 
%
%
Our approach is standard: We model judgements as 
inductive datatypes, and rules as dependently typed constructors of these judgements\todo{cite something?}{}. 
%
We adopt Barendregt's convention for bound variables \cite{urban,barendregt84:_lambda_calculus} and encode typing 
contexts, hole contexts and hole environments using metafunctions.\todo{say something about axioms?}{}

\subsection{Implementation}\label{sec:implementation}

The supplemental material includes a browser-based implementation
of \HazelnutLive, which consists of a functional reactive program
%
% Functional Reactive Animation (Hudak 1997) is the canonical FRP cite, not Elm, if you want one; we don't need one, I think. --Matt
%
%\cite{DBLP:conf/pldi/CzaplickiC13} 
%
written with the Reason toolchain for OCaml \cite{reason-what,leroy03:_ocaml} 
together with the OCaml \lismall{React} library \cite{OcamlReact} 
and the \lismall{js_of_ocaml} compiler and its associated libraries \cite{vouillon2014bytecode}. 
%
Appendix~\ref{sec:ref-impl-screenshots}\todo{do this}{} provides a screenshot of the user interface, 
which implements the core calculus of this section, 
along with the corresponding live programming features from \Secref{sec:examples}.

To facilitate interactively editing terms, the implementation employs
the editor semantics of~\Hazelnut, exposing a language of structured
edit actions (summarized in the sidebar on the left of
Fig.~\ref{fig:screenshot})
%
As with prior work, this editor semantics enjoys
the \Property{Sensibility} property by automatically inserting holes
to guarantee that every edit state has some (possibly incomplete) type
.\todo{put statement in the appendix}{}

By composing this meta-theoretic property with
the \Property{Expandability}, \Property{Typed Expansion}
and \Property{Progress} properties from this paper, we establish a
uniquely powerful invariant: \emph{Every} edit state has a \emph{static
meaning} (a la~\Hazelnut) and a \emph{dynamic meaning} (a la~\HazelnutLive).
%
In other words, live feedback never ``flickers in and out'' nor reflects ``stale'' edit states.
%
The current implementation serves as a proof-of-concept of this invariant, 
by providing a working prototype for this section's definitions.

%, and provides to be of use to researchers studying the calculus as presented in this section.
% Consistent with this goal, it closely follows the theoretical account in this section, rather than including advanced language features.

%%%%%%%%%%%%%%%%%%%%%%%%%%%%%%%%%%%%%%%%%%%%%%%%%%%%%%%%%%%%%%%%%%%%%%%%%%%%%%%%%%%%%%%%%%%%%%%%%%%

%% \matt{I'd drop everything else in this sub-section; 
%% it leaves us open to attack from a hostile reviewer 
%% who is cranky that the implementation is not ``complete'' yet; also, we'd save the space}

%%   \halam{x}{\htau}{\evalctx} ~\vert~

%%%%%%%%%%%%%%%%%%%%%%%%%%%%%%%%%%%%%%%%%%%%%%%%%%%%%%%%%%%%%%%%%%%%%%%%%%%

\newcommand{\commutativitySec}{A Contextual Modal Interpretation of Fill-and-Resume}
\section{\protect\commutativitySec}
\label{sec:resumption}


%The result of evaluation is a final internal expression with hole closures, each with an associated hole environment, $\sigma$. These hole environments can be reported directly to the programmer, e.g. via the sidebar shown in Fig.~X\todo{fig}, to help them as they think about how to fill in the corresponding hole in the external expression. Hole environments might also be useful indirectly, e.g. by informing an edit action synthesis and suggestion system. In any case, 
% When the programmer performs one or more edit actions to fill in a hole in the program, a new result must be computed. Na\"ively, the system would need to compute the result ``from scratch'' on each such edit. A more efficient approach supported by the structure of \HazelnutLive is to resume evaluation from where it left off after an edit that amounts to hole filling. We call this feature \emph{fill-and-resume}, by analogy to the ``edit-and-resume'' features available in some systems, e.g. Visual Studio (see below for a comparison)\todo{cite and compare}{}.

\todo{cite tanimoto paper, mention Level 4 liveness property}When the programmer interactively fills a program hole, that program's
reduction and result are generally each affected, and the system
should respond by (quickly) updating the program's output.
%
Na\"ively, this response requires recomputing the program output
``from scratch'' after each such edit.
%
However, by virtue of its design principles (based on
CMTT), \HazelnutLive offers a more efficient approach: To \emph{resume
evaluation} of previously indeterminate expressions, which are
(generally) more determined after the programmer fills a hole than
beforehand.
%
This \emph{fill-and-resume} behavior formalizes the (ad hoc)
``edit-and-resume'' behavior permitted by some current systems,
e.g.,~Microsoft Visual Studio.\todo{cite}

%(see below for a comparison)\todo{cite and compare}{}. ---NO TIME/SPACE for this.

%%%%%%%%% New par
Formally, 
%
we interpret hole environments as \emph{delayed substitutions},
the same interpretation suggested for closures in contextual modal
type theory (CMTT) by \citet{Nanevski2008}.
%
%In practice, it may be useful to cache results from several previous
%expansions, e.g., by employing off-the-shelf programming language
%abstractions for incremental computation~\cite{Hammer14,Hammer15}.
%
We define hole filling in terms of the expanded (internal) program,
and define filling a hole in an external language term as filling the
corresponding hole in its expansion.

%%%%%%%%% New par
\Figref{fig:substitution} defines the hole filling operation~$\instantiate{d}{u}{d'}$,
mirroring the contextual substitution operation of~CMTT.
%
Unlike usual notions of capture-avoiding substitution, 
hole filling imposes no condition on the binder when passing into the
body of a lambda expression---the expression that fills a hole can, of
course, refer to variables in scope where the hole appears.
%
When hole filling encounters an empty closure for the hole being
instantiated, $\instantiate{d}{u}{\dehole{u}{\sigma}{}}$, the result
is $[\instantiate{d}{u}{\sigma}]d$.
%
That is, we apply the delayed substitution to the fill expression~$d$
after first recursively filling any instances of hole~$u$ in~$\sigma$.
%
Hole filling for non-empty closures is analogous, where it discards
the previously-enveloped expression.
%
%
% \matt{Will any reader actually wonder this? Does this thought connect to any other statement elsewhere in the paper?}
%
%This case shows why we cannot interpret a non-empty hole as an empty
%hole of arrow type applied to the enveloped expression---the hole
%filling operation would not operate as expected under this
%interpretation.

\begin{figure}[t]
\judgbox
  {[\dexp_1 / x] \dexp_2 = \dexp_3}
  {$\dexp_3$ is the result of substituting $\dexp_1$ for $x$ in $\dexp_2$}
\[
\begin{array}{lcll}
[\dexp_1 / x] c
&=&
c\\
{[\dexp_1 / x] x}
&=&
\dexp_1\\
{[\dexp_1 / x] y}
&=&
y & (y \neq x)\\
{[\dexp_1 / x] \dlam{y}{\htau}{\dexp_2}}
&=&
{\dlam{y}{\htau}{[\dexp_1 / x] \dexp_2}}
& (y \neq x)
\\
{[\dexp_1 / x] \dap{d_2}{d_3}}
&=&
{\dapP{[\dexp_1 / x] d_2}{[\dexp_1 / x] d_3}}
\\
% {[\dexp_1 / x] \dinj{i}{e_2}}
% &=&
% {\dinj{i}{[\dexp_1 / x] e_2}}
% \\
% {[\dexp_1 / x] \dcase{e}{x}{e_x}{y}{e_y}}
% &=&
% {\dcase{[\dexp_1 / x]e}{x}{[\dexp_1 / x]e_x}{y}{[\dexp_1 / x]e_y}}
% \\
{[\dexp_1 / x]} \dehole{\mvar}{\subst}{}
&=&
\dehole{\mvar}{[\dexp_1 / x] \subst}{}
\\
{[\dexp_1 / x] \dhole{\dexp_2}{\mvar}{\subst}{}}
&=&
\dhole{ [ \dexp_1 / x ] \dexp_2 }{\mvar}{[\dexp_1 / x] \subst}{}
\\
{[\dexp_1 / x] \dcast{\htau}{\dexp}}
&=&
\dcast{\htau}{[\dexp_1 / x] \dexp}
\end{array}
\]
\caption{Substitution}
\end{figure}


%%%%%%%%% New par
The following theorem characterizes the static behavior of hole filling.
\begin{thm}[Filling]
  If $\hasType{\hDelta, \Dbinding{u}{\hGamma'}{\htau'}}{\hGamma}{\dexp}{\tau}$
  and $\hasType{\hDelta}{\hGamma'}{\dexp'}{\htau'}$
  then $\hasType{\hDelta}{\hGamma}{\instantiate{\dexp'}{u}{\dexp}}{\tau}$.
\begin{proof}
We prove a stronger version of the theorem which accounts for both
terms and finite substitutions (see \cite{Nanevski2008}).
%
Each case proceeds by rule induction on the first assumption, applying
the finite substitution lemma.
\end{proof}
\end{thm}

Dynamically, the fill-and-resume enjoys an
important \emph{commutativity} property: If there is some sequence of
steps that go from $d_1$ to $d_2$, then one can fill a hole in these
terms at \emph{either} the beginning or at the end of that step
sequence.
%
We write $\multiStepsTo{\dexp_1}{\dexp_2}$ for the reflexive,
transitive closure of stepping (see \Figref{fig:multi-step} in
Appendix~\ref{sec:additional-defns}).
%
\begin{thm}[Commutativity]
  If $\hasType{\hDelta, \Dbinding{u}{\hGamma'}{\htau'}}{\emptyset}{\dexp_1}{\tau}$
  and $\hasType{\hDelta}{\hGamma'}{\dexp'}{\htau'}$ and $\multiStepsTo{\dexp_1}{\dexp_2}$
  then $\multiStepsTo{\instantiate{\dexp'}{u}{\dexp_1}}
                     {\instantiate{\dexp'}{u}{\dexp_2}}$.
\end{thm}
%
Critically, this property relies on the more permissive version of
reduction from \Secref{sec:calculus}, where evaluation order is
unspecified (i.e. formally, we omit the bracketed premises).
%
In general, resuming from $\instantiate{\dexp'}{u}{d_2}$ will not
reduce sub-expressions in the same order as a ``fresh'' left-to-right
reduction sequence starting from $\instantiate{\dexp'}{u}{d_1}$.
%
In other words, this notion of fill-and-resume only works for
languages where evaluation order always results in \emph{confluent}
reductions; viz., those that \emph{eventually} yield the same
resulting term.

%% Everyone knows what confluence means, or they should.  We don't need to cite Church :)
%%
%There are various ways to encode this intuition. One way to do so is
%by establishing a confluence property (which is closely related to
%the Church-Rosser property \cite{church1936some}).

The most general confluence property does not hold for the dynamic
semantics in Sec.~\ref{sec:calculus} for the usual reason: We do not
reduce under binders (\citet{DBLP:conf/birthday/BlancLM05} discuss the
standard counterexample).
%
We could recover confluence by specifying reduction under binders,
either generally or in a more restricted form where only closed
sub-expressions are
reduced \cite{DBLP:journals/tcs/CagmanH98,DBLP:conf/birthday/BlancLM05,levy1999explicit}.
%
However, reduction under binders conflicts with the standard implementation approaches
for most programming languages \cite{DBLP:conf/birthday/BlancLM05}.
%
A more satisfying approach considers confluence modulo equality \cite{Huet:1980ng}. 
%
The simplest such approach restricts our interest to top-level expressions
of base type that result in values, in which case the following
special case of confluence does hold (trivially when the only base
type has a single value, but also more generally for other base
types).
\begin{lem}[Base Confluence]
  If $\hasType{\Delta}{\emptyset}{\dexp}{b}$ and 
  $\multiStepsTo{\dexp}{\dexp_1}$
  and $\isValue{\dexp_1}$
  and $\multiStepsTo{\dexp}{\dexp_2}$
  then $\multiStepsTo{\dexp_2}{\dexp_1}$.
\end{lem}
We can then prove the following property, which establishes that fill-and-resume is sound.
\begin{thm}[Resumption]
  If $\hasType{\hDelta, \Dbinding{u}{\hGamma'}{\htau'}}{\emptyset}{\dexp}{b}$
  and $\hasType{\hDelta}{\hGamma'}{\dexp'}{\htau'}$ 
  and $\multiStepsTo{\dexp}{\dexp_1}$
  and $\multiStepsTo{\instantiate{\dexp'}{u}{\dexp}}{\dexp_2}$
  and $\isValue{\dexp_2}$
  then $\multiStepsTo{\instantiate{\dexp'}{u}{\dexp_1}}{\dexp_2}$.
  \begin{proof}
    By Commutativity,
    $\multiStepsTo{\instantiate{\dexp'}{u}{\dexp}}
                  {\instantiate{\dexp'}{u}{\dexp_1}}$.
    By Base Confluence, we can conclude.
  \end{proof}
\end{thm}

We outline the proofs in Appendix~\ref{sec:hole-filling}, along with a
number of other necessary lemmas and definitions.
%
In particular, additional machinery handles the situation wherein a
now-filled non-empty hole had taken a step in the original evaluation.
\matt{Cyrus --- I tried to simplify and shorten the earlier version of the prior sentence, but I \emph{may} have unintentionally goofed up what it was trying to say; please double check the prior sentence}

\matt{We could (should?) move the next paragraph to the appendix, 
before/after the proofs mentioned above; 
its not really adding much to the discussion above, 
but may be interesting to the very studious reader/prover; also, leaving it here also gives a lazy/cranky reviewer a nice weapon to hit us with}

We exclude these proofs and definitions from the Agda mechanization
for two reasons.
%
First, fill-and-resume is merely an optimization, and unlike the meta
theory of \Secref{sec:calculus}, these properties are generally not
conserved by certain reasonable extensions of the core
calculus~(e.g., \matt{such as what?}).
%
Second, to properly encode the hole filling operation, such a
mechanization requires a significantly more complex representation of
hole environments; unfortunately, Agda cannot be easily convinced that
the corresponding definition is well-founded (\citet{Nanevski2008}
establish that it is in fact well-founded).
%
By contrast, the developments in \Secref{sec:calculus} do not require
these more complex (and somewhat problematic) representations.

\begin{comment}

\begin{theorem}[Maximum Informativity]
If the expansion produces $t1$, and there exists another possible type choice
$t2$, then $t1 \sim t2$ and $t1 JOIN t2 = t1$
\end{theorem}\footnote{idea is that special casing the holes in EANEHole gives you ``the
most descriptive hole types'' for some sense of what that means -- they'd
all just be hole other wise. from Matt:
\begin{quote}
It sounds like we need a something akin to an abstract domain (a lattice),
where hole has the least information, and a fully-defined type (without
holes) has the most information.  You can imagine that this lattice really
expands the existing definition we have of type consistency, which is
merely the predicate that says whether two types are comparable
(“join-able”) in this lattice.  lattice join is the operation that goes
through the structure of two (consistent) types, and chooses the structure
that is more defined (i.e., non-hole, if given the choice between hole and
non-hole).

The rule choosenonhole below is the expansion of this consistency rule that
we already have (hole consistent with everything)
\end{quote}}

\begin{verbatim}
t not hole
-------------------- :: choose-non-hole
hole JOIN t  = t
\end{verbatim}
\begin{verbatim}
------------ :: hole-consistent-with-everything
hole ~ t
\end{verbatim}

\end{comment}
