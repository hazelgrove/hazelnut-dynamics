% !TEX root = hazelnut-dynamics.tex

\clearpage
\newcommand{\calculusSec}{Hazelnut Live, Formally}
\section{\protect\calculusSec}
\label{sec:calculus}
\cy{exposition}

\subsection{Calculus}

\subsubsection{Syntax}

\begin{figure}[t]
$\arraycolsep=4pt\begin{array}{rllllll}
\mathsf{HTyp} & \htau & ::= &
  b ~\vert~
  \tarr{\htau}{\htau} ~\vert~
  % \tprod{\htau}{\htau} ~\vert~
  % \tsum{\htau}{\htau} ~\vert~
  \tehole\\
\mathsf{HExp} & \hexp & ::= &
  c ~\vert~
  x ~\vert~
  \halam{x}{\htau}{\hexp} ~\vert~
  \hap{\hexp}{\hexp} ~\vert~
  % \hpair{\hexp}{\hexp} ~\vert~
  % \hprj{i}{\hexp} ~\vert~
  % \hinj{i}{\hexp} ~\vert~
  % \hcase{\hexp}{x}{\hexp}{x}{\hexp} ~\vert~
  % \hadd{\hexp}{\hexp} ~\vert~
  \hehole{u} ~\vert~
  \hhole{\hexp}{u} ~\vert~
  {\hlam{x}{\hexp}} ~\vert~
  \hexp : \htau\\
% \mathsf{Mark} & \markname{} & ::= &
%   \evaled{} ~\vert~  \unevaled{}\\
 \mathsf{DHExp} & \dexp  & ::= &
  c ~\vert~
  x ~\vert~
  {\halam{x}{\htau}{\dexp}} ~\vert~
  \hap{\dexp}{\dexp} ~\vert~
  % \hpair{\dexp}{\dexp} ~\vert~
  % \hprj{i}{\dexp} ~\vert~
  % \hinj{i}{\dexp} ~\vert~
  % \hcase{\dexp}{x}{\dexp}{x}{\dexp} ~\vert~
  % \hadd{\dexp}{\dexp} ~\vert~
  \dehole{\mvar}{\subst}{} ~\vert~
  \dhole{\dexp}{\mvar}{\subst}{} ~\vert~
  \dcasttwo{\dexp}{\htau}{\htau} ~\vert~
  \dcastfail{\dexp}{\htau}{\htau}\\
\end{array}$
\caption{Syntax of H-types, H-expressions and dynamic H-expressions.
We write $x$ to range over variables,
$u$ over hole names, and
$\sigma$ over finite substitutions (i.e., environments) from
variables to dynamic H-expressions.}
\label{fig:HTyp}
\label{fig:HExp}
\end{figure}


{\color{gray}\blindtext}

Shorthand:

$$
\dcastthree{\dexp}{\htau_1}{\htau_2}{\htau_3} \defeq
  \dcasttwo{\dcasttwo{\dexp}{\htau_1}{\htau_2}}{\htau_2}{\htau_3}
$$

Explain that $c$ is a placeholder constant (i.e., unit), $b$ is a placeholder
type (i.e., unit).



\subsubsection{Statics}

% !TEX root = hazelnut-dynamics.tex

\begin{figure}[t]
\judgbox{\tconsistent{\htau_1}{\htau_2}}{$\htau_1$ is consistent with $\htau_2$}
\begin{mathpar}
\inferrule[TCHole1]{ }{
  \tconsistent{\tehole}{\htau}
}

\inferrule[TCHole2]{ }{
  \tconsistent{\htau}{\tehole}
}

\inferrule[TCRefl]{ }{
  \tconsistent{\htau}{\htau}
}

\inferrule[TCArr]{
  \tconsistent{\htau_1}{\htau_1'}\\
  \tconsistent{\htau_2}{\htau_2'}
}{
  \tconsistent{\tarr{\htau_1}{\htau_2}}{\tarr{\htau_1'}{\htau_2'}}
}
%
% \inferrule{
%   \tconsistent{\htau_1}{\htau_1'}\\
%   \tconsistent{\htau_2}{\htau_2'}
% }{
%   \tconsistent{\tprod{\htau_1}{\htau_2}}{\tprod{\htau_1'}{\htau_2'}}
% }
%
% \inferrule{
%   \tconsistent{\htau_1}{\htau_1'}\\
%   \tconsistent{\htau_2}{\htau_2'}
% }{
%   \tconsistent{\tsum{\htau_1}{\htau_2}}{\tsum{\htau_1'}{\htau_2'}}
% }
\end{mathpar}

% \vsepRule

% \judgbox{\tinconsistent{\htau_1}{\htau_2}}{$\htau_1$ is inconsistent with $\htau_2$}
% \begin{mathpar}
%     \inferrule[ICBaseArr1]{ }{
%       \tinconsistent{\tb}{\tarr{\htau_1}{\htau_2}}
%     }

%     \inferrule[ICBaseArr2]{ }{
%       \tinconsistent{\tarr{\htau_1}{\htau_2}}{\tb}
%     }

%     \inferrule[ICArr1]{
%       \tinconsistent{\htau_1}{\htau_3}
%     }{
%       \tinconsistent{\tarr{\htau_1}{\htau_2}}{\tarr{\htau_3}{\htau_4}}
%     }

%     \inferrule[ICArr2]{
%       \tinconsistent{\htau_2}{\htau_4}
%     }{
%       \tinconsistent{\tarr{\htau_1}{\htau_2}}{\tarr{\htau_3}{\htau_4}}
%     }
% \end{mathpar}

\vsepRule

\judgbox{\arrmatch{\htau}{\tarr{\htau_1}{\htau_2}}}{$\htau$ has matched arrow type $\tarr{\htau_1}{\htau_2}$}
\begin{mathpar}
\inferrule[MAHole]{ }{
  \arrmatch{\tehole}{\tarr{\tehole}{\tehole}}
}

\inferrule[MAArr]{ }{
  \arrmatch{\tarr{\htau_1}{\htau_2}}{\tarr{\htau_1}{\htau_2}}
}
\end{mathpar}

% \judgbox{\prodmatch{\htau}{\tprod{\htau_1}{\htau_2}}}{$\htau$ has matched product type $\tprod{\htau_1}{\htau_2}$}
% \begin{mathpar}
% \inferrule{ }{
%   \prodmatch{\tehole}{\tprod{\tehole}{\tehole}}
% }

% \inferrule{ }{
%   \prodmatch{\tprod{\htau_1}{\htau_2}}{\tprod{\htau_1}{\htau_2}}
% }
% \end{mathpar}

% \judgbox{\summatch{\htau}{\tsum{\htau_1}{\htau_2}}}{$\htau$ has matched sum type $\tsum{\htau_1}{\htau_2}$}
% \begin{mathpar}
% \inferrule{ }{
%   \summatch{\tehole}{\tsum{\tehole}{\tehole}}
% }

% \inferrule{ }{
%   \summatch{\tsum{\htau_1}{\htau_2}}{\tsum{\htau_1}{\htau_2}}
% }
% \end{mathpar}
\caption{Type Consistency and Matching}
\Label{fig:tconsistent}
\Label{fig:arrmatch}
\end{figure}

% !TEX root = hazelnut-dynamics.tex

\begin{figure}[t]
\judgbox{\hsyn{\hGamma}{\hexp}{\htau}}{$\hexp$ synthesizes type $\htau$}
\begin{mathpar}
\inferrule[SConst]{ }{
  \hsyn{\hGamma}{c}{b}
}

\inferrule[SVar]{
  x : \htau \in \hGamma
}{
  \hsyn{\hGamma}{x}{\htau}
}

\inferrule[SLam]{
  \hsyn{\hGamma, x : \htau_1}{\hexp}{\htau_2}
}{
  \hsyn{\hGamma}{\halam{x}{\htau_1}{\hexp}}{\tarr{\htau_1}{\htau_2}}
}

\inferrule[SAp]{
  \arrayenvbr{
    \hsyn{\hGamma}{\hexp_1}{\htau_1}    
    \\
    \hana{\hGamma}{\hexp_2}{\htau_2}
  }
  \\
    \arrmatch{\htau_1}{\tarr{\htau_2}{\htau}}
}{
  \hsyn{\hGamma}{\hap{\hexp_1}{\hexp_2}}{\htau}
}

\inferrule[SEHole]{ }{
  \hsyn{\hGamma}{\hehole{u}}{\tehole}
}

\inferrule[SNEHole]{
  \hsyn{\hGamma}{\hexp}{\htau}
}{
  \hsyn{\hGamma}{\hhole{\hexp}{u}}{\tehole}
}

\inferrule[SAsc]{
  \hana{\hGamma}{\hexp}{\htau}
}{
  \hsyn{\hGamma}{\hexp : \htau}{\htau}
}
\end{mathpar}

\vsepRule

\judgbox{\hana{\hGamma}{\hexp}{\htau}}{$\hexp$ analyzes against type $\htau$}
\begin{mathpar}
\inferrule[ALam]{
  \arrmatch{\htau}{\tarr{\htau_1}{\htau_2}}\\
  \hana{\hGamma, x : \htau_1}{\hexp}{\htau_2}
}{
  \hana{\hGamma}{\hlam{x}{\hexp}}{\htau}
}

\inferrule[ASubsume]{
  \hsyn{\hGamma}{\hexp}{\htau}\\
  \tconsistent{\htau}{\htau'}
}{
  \hana{\hGamma}{\hexp}{\htau'}
}
\end{mathpar}
\CaptionLabel{Bidirectional Typing of External Expressions}{fig:bidirectional-typing}
\end{figure}


{\color{gray}\blindtext}
{\color{gray}\blindtext}
{\color{gray}\blindtext}

A typing context, $\hGamma$, is a list of bindings $x : \htau$.

\subsubsection{Expansion}

% !TEX root = hazelnut-dynamics.tex

\begin{figure}[p]
\judgbox
  {\expandSyn{\hGamma}{\hexp}{\htau}{\dexp}{\Delta}}
  {$\hexp$ synthesizes type $\htau$ and expands to $\dexp$}
\begin{mathpar}
\inferrule[ESConst]{ }{
  \expandSyn{\hGamma}{c}{b}{c}{\emptyset}
}

\inferrule[ESVar]{
  x : \htau \in \hGamma
}{
  \expandSyn{\hGamma}{x}{\htau}{x}{\emptyset}
}

\inferrule[ESLam]{
  \expandSyn{\hGamma, x : \htau_1}{\hexp}{\htau_2}{\dexp}{\Delta}
}{
  \expandSyn{\hGamma}{\halam{x}{\htau_1}{\hexp}}{\tarr{\htau_1}{\htau_2}}{\halam{x}{\htau_1}{\dexp}}{\Delta}
}

\inferrule[ESAp]{
  \hsyn{\hGamma}{\hexp_1}{\htau_1}\\
  \arrmatch{\htau_1}{\tarr{\htau_2}{\htau}}
  \\\\
  \expandAna{\hGamma}{\hexp_1}{\tarr{\htau_2}{\htau}}{\dexp_1}{\htau_1'}{\Delta_1}\\
  \expandAna{\hGamma}{\hexp_2}{\htau_2}{\dexp_2}{\htau_2'}{\Delta_2}
}{
  \expandSyn
    {\hGamma}
    {\hap{\hexp_1}{\hexp_2}}
    {\htau}
    {\hap{(\dcasttwo{\dexp_1}{\htau_1'}{\tarr{\htau_2}{\htau}})}
         {\dcasttwo{\dexp_2}{\htau_2'}{\htau_2}}}
    {\Dunion{\Delta_1}{\Delta_2}}
}
%
%% \inferrule[ESAp1]{
%%   \hsyn{\hGamma}{\hexp_1}{\tehole}\\
%%   \expandAna{\hGamma}{\hexp_1}{\tarr{\htau_2}{\tehole}}{\dexp_1}{\htau_1}{\Delta_1}\\
%%   \expandAna{\hGamma}{\hexp_2}{\tehole}{\dexp_2}{\htau_2}{\Delta_2}
%% }{
%%   \expandSyn{\hGamma}{\hap{\hexp_1}{\hexp_2}}{\tehole}{\hap{(\dcast{\tarr{\htau_2}{\tehole}}{\dexp_1})}{\dexp_2}}{\Dunion{\Delta_1}{\Delta_2}}
%% }
%% 
%% \inferrule[ESAp2]{
%%   \expandSyn{\hGamma}{\hexp_1}{\tarr{\htau_2}{\htau}}{\dexp_1}{\Delta_1}\\
%%   \expandAna{\hGamma}{\hexp_2}{\htau_2}{\dexp_2}{\htau'_2}{\Delta_2}\\
%%   \htau_2 \neq \htau'_2
%% }{
%%   \expandSyn{\hGamma}{\hap{\hexp_1}{\hexp_2}}{\htau}{\hap{\dexp_1}{\dcast{\htau_2}{\dexp_2}}}{\Dunion{\Delta_1}{\Delta_2}}
%% }
%% 
%% \inferrule[ESAp3]{
%%   \expandSyn{\hGamma}{\hexp_1}{\tarr{\htau_2}{\htau}}{\dexp_1}{\Delta_1}\\
%%   \expandAna{\hGamma}{\hexp_2}{\htau_2}{\dexp_2}{\htau_2}{\Delta_2}
%% }{
%%   \expandSyn{\hGamma}{\hap{\hexp_1}{\hexp_2}}{\htau}{\hap{\dexp_1}{\dexp_2}}{\Dunion{\Delta_1}{\Delta_2}}
%% }\\
%
%
% \inferrule[expand-pair]{
%   \expandSyn{\hGamma}{\hexp_1}{\htau_1}{\dexp_1}{\Delta_1}\\
%   \expandSyn{\hGamma}{\hexp_2}{\htau_2}{\dexp_2}{\Delta_2}
% }{
%   \expandSyn{\hGamma}{\hpair{\hexp_1}{\hexp_2}}{\tprod{\htau_1}{\htau_2}}{\hpair{\dexp_1}{\dexp_2}}{\Dunion{\Delta_1}{\Delta_2}}
% }
%
% \inferrule[expand-prj]{
%   a
% }{
%   b
% }
%
% (inj)
%
%
% \inferrule[expand-plus]{ }{
%   \expandSyn{\hGamma}{\hadd{\hexp_1}{\hexp_2}}{\tnum}{\hadd{\dexp_1}{\dexp_2}}{\Dunion{\Delta_1}{\Delta_2}}
% }

\inferrule[ESEHole]{ }{
  \expandSyn{\hGamma}{\hehole{u}}{\tehole}{\dehole{u}{\idof{\hGamma}}{}}{\Dbinding{u}{\hGamma}{\tehole}}
}

\inferrule[ESNEHole]{
  \expandSyn{\hGamma}{\hexp}{\htau}{\dexp}{\Delta}
}{
  \expandSyn{\hGamma}{\hhole{\hexp}{u}}{\tehole}{\dhole{\dexp}{u}{\idof{\hGamma}}{}}{\Delta, \Dbinding{u}{\hGamma}{\tehole}}
}\\
%

\inferrule[ESAsc]{
  \expandAna{\hGamma}{\hexp}{\htau}{\dexp}{\htau'}{\Delta}
}{
  \expandSyn{\hGamma}{\hexp : \htau}{\htau}{\dcasttwo{\dexp}{\htau'}{\htau}}{\Delta}
}

%% \inferrule[ESAsc1]{
%%   \expandAna{\hGamma}{\hexp}{\htau}{\dexp}{\htau'}{\Delta}\\
%%   \htau \neq \htau'
%% }{
%%   \expandSyn{\hGamma}{\hexp : \htau}{\htau}{\dcast{\htau}{\dexp}}{\Delta}
%% }
%% 
%% \inferrule[ESAsc2]{
%%   \expandAna{\hGamma}{\hexp}{\htau}{\dexp}{\htau}{\Delta}
%% }{
%%   \expandSyn{\hGamma}{\hexp : \htau}{\htau}{\dexp}{\Delta}
%% }
\end{mathpar}

\vsepRule

\judgbox
  {\expandAna{\hGamma}{\hexp}{\htau_1}{\dexp}{\htau_2}{\Delta}}
  {$\hexp$ analyzes against type $\htau_1$ and
   expands to $\dexp$ of consistent type $\htau_2$}
\begin{mathpar}
\inferrule[EALam]{
  \arrmatch{\htau}{\tarr{\htau_1}{\htau_2}}\\
  \expandAna{\hGamma, x : \htau_1}{\hexp}{\htau_2}{\dexp}{\htau'_2}{\Delta}
}{
  \expandAna{\hGamma}{\hlam{x}{\hexp}}{\htau}{\halam{x}{\htau_1}{\dexp}}{\tarr{\htau_1}{\htau_2'}}{\Delta}
}

%% \inferrule[EALam]{
%%   \expandAna{\hGamma, x : \htau_1}{\hexp}{\htau_2}{\dexp}{\htau'_2}{\Delta}
%% }{
%%   \expandAna{\hGamma}{\hlam{x}{\hexp}}{\tarr{\htau_1}{\htau_2}}{\halam{x}{\htau_1}{\dexp}}{\tarr{\htau_1}{\htau_2'}}{\Delta}
%% }
%% 
%% \inferrule[EALamHole]{
%%   \expandAna{\hGamma, x : \tehole}{\hexp}{\tehole}{\dexp}{\htau}{\Delta}
%% }{
%%   \expandAna{\hGamma}{\hlam{x}{\hexp}}{\tehole}{\halam{x}{\tehole}{\dexp}}{\tarr{\tehole}{\htau}}{\Delta}
%% }
%% 
\inferrule[EASubsume]{
  \hexp \neq \hehole{u}\\
  \hexp \neq \hhole{\hexp'}{u}\\\\
  \expandSyn{\hGamma}{\hexp}{\htau'}{\dexp}{\Delta}\\
  \tconsistent{\htau}{\htau'}
}{
  \expandAna{\hGamma}{\hexp}{\htau}{\dexp}{\htau'}{\Delta}
}

\inferrule[EAEHole]{ }{
  \expandAna{\hGamma}{\hehole{u}}{\htau}{\dehole{u}{\idof{\hGamma}}{}}{\htau}{\Dbinding{u}{\hGamma}{\htau}}
}

\inferrule[EANEHole]{
  \expandSyn{\hGamma}{\hexp}{\htau'}{\dexp}{\Delta}\\
}{
  \expandAna{\hGamma}{\hhole{\hexp}{u}}{\htau}{\dhole{\dexp}{u}{\idof{\hGamma}}{}}{\htau}{\Delta, \Dbinding{u}{\hGamma}{\htau}}
}
\end{mathpar}
\caption{Expansion}
\label{fig:expansion}
\label{fig:expandSyn}
\label{fig:expandAna}
\end{figure}

% !TEX root = hazelnut-dynamics.tex

\begin{figure}[p]
\judgbox{\hasType{\Delta}{\hGamma}{\dexp}{\htau}}{$\dexp$ is assigned type $\htau$}
\begin{mathpar}
\inferrule[TAConst]{ }{
  \hasType{\Delta}{\hGamma}{c}{b}
}

\inferrule[TAVar]{
  x : \htau \in \hGamma
}{
	\hasType{\Delta}{\hGamma}{x}{\htau}
}

\inferrule[TALam]{
  \hasType{\Delta}{\hGamma, x : \htau_1}{\dexp}{\htau_2}
}{
  \hasType{\Delta}{\hGamma}{\halam{x}{\htau_1}{\dexp}}{\tarr{\htau_1}{\htau_2}}
}

\inferrule[TAAp]{
  \hasType{\Delta}{\hGamma}{\dexp_1}{\tarr{\htau_2}{\htau}}\\
  \hasType{\Delta}{\hGamma}{\dexp_2}{\htau_2}
}{
  \hasType{\Delta}{\hGamma}{\hap{\dexp_1}{\dexp_2}}{\htau}
}

\inferrule[TAEHole]{
  \Dbinding{u}{\hGamma'}{\htau} \in \Delta\\
  \hasType{\Delta}{\hGamma}{\sigma}{\hGamma'}
}{
  \hasType{\Delta}{\hGamma}{\dehole{u}{\sigma}{}}{\htau}
}

\inferrule[TANEHole]{
  \hasType{\Delta}{\hGamma}{\dexp}{\htau'}\\\\
  \Dbinding{u}{\hGamma'}{\htau} \in \Delta\\
  \hasType{\Delta}{\hGamma}{\sigma}{\hGamma'}
}{
  \hasType{\Delta}{\hGamma}{\dhole{\dexp}{u}{\sigma}{}}{\htau}
}

\inferrule[TACast]{
  \hasType{\Delta}{\Gamma}{\dexp}{\htau_1}\\
  \tconsistent{\htau_1}{\htau_2}
}{
  \hasType{\Delta}{\hGamma}{\dcasttwo{\dexp}{\htau_1}{\htau_2}}{\htau_2}
}

\inferrule[TAFailedCast]{
  \hasType{\Delta}{\Gamma}{\dexp}{\htau_1}\\
  \isGround{\htau_1}\\
  \isGround{\htau_2}\\
  \htau_1\neq\htau_2
}{
  \hasType{\Delta}{\hGamma}{\dcastfail{\dexp}{\htau_1}{\htau_2}}{\htau_2}
}
\end{mathpar}
\caption{Type Assignment for Internal Expressions}
\Label{fig:hasType}
\end{figure}


{\color{gray}\blindtext}
{\color{gray}\blindtext}
{\color{gray}\blindtext}

A hole context, $\hDelta$, is a list of bindings $\Dbinding{u}{\hGamma}{\htau}$.

\subsubsection{Dynamics}

% !TEX root = hazelnut-dynamics.tex
\begin{figure}[t]

\begin{tabular}[t]{cc}

\begin{minipage}{0.5\textwidth}
\judgbox{\isFinal{\dexp}}{$\dexp$ is final}
\begin{mathpar}
%% \inferrule[FVal]
%% {\isValue{\dexp}}{\isFinal{\dexp}}
\inferrule[FBoxedVal]
{\isBoxedValue{\dexp}}{\isFinal{\dexp}}
\and
\inferrule[FIndet]
{\isIndet{\dexp}}{\isFinal{\dexp}}
\end{mathpar}
\end{minipage}

& 

\begin{minipage}{0.5\textwidth}
    
\judgbox{\isValue{\dexp}}{$\dexp$ is a value}
\begin{mathpar}
\inferrule[VConst]{ }{
  \isValue{c}
}

\inferrule[VLam]{ }{
  \isValue{\halam{x}{\htau}{\dexp}}
}
\end{mathpar}
\end{minipage}

\end{tabular}

\vsepRule

\judgbox{\isBoxedValue{\dexp}}{$\dexp$ is a boxed value}
\begin{mathpar}
\inferrule[BVVal\rkc{name?}]{
  \isValue{\dexp}
}{
  \isBoxedValue{\dexp}
}

\inferrule[BVCastArr\rkc{name?}]{
  \tarr{\htau_1}{\htau_2} \neq \tarr{\htau_3}{\htau_4}\\
  \isBoxedValue{\dexp}
}{
  \isBoxedValue{\dcasttwo{\dexp}{\tarr{\htau_1}{\htau_2}}{\tarr{\htau_3}{\htau_4}}}
}

\inferrule[BVHoleCast\rkc{name?}]{
  \isBoxedValue{\dexp}\\
  \isGround{\htau}
}{
  \isBoxedValue{\dcasttwo{\dexp}{\htau}{\tehole}}
}
\end{mathpar}

\vsepRule

\judgbox{\isIndet{\dexp}}{$\dexp$ is indeterminate}
\begin{mathpar}
\inferrule[IEHole]
{ }
{\isIndet{\dehole{\mvar}{\subst}{}}}

\inferrule[INEHole]
{\isFinal{\dexp}}
{\isIndet{\dhole{\dexp}{\mvar}{\subst}{}}}

\inferrule[IAp]
{\dexp_1\neq
   \dcasttwo{\dexp_1'}
            {\tarr{\htau_1}{\htau_2}}
            {\tarr{\htau_3}{\htau_4}}\\
 \isIndet{\dexp_1}\\
% \isFinal{\dexp_2}~\text{\cy{??}}}
 \isFinal{\dexp_2}}
{\isIndet{\dap{\dexp_1}{\dexp_2}}}

\inferrule[ICastGroundHole] {
  \isIndet{\dexp}\\
  \isGround{\htau}
}{
  \isIndet{\dcasttwo{\dexp}{\htau}{\tehole}}
}

\inferrule[ICastHoleGround] {
  \dexp\neq\dcasttwo{\dexp'}{\htau'}{\tehole}\\
  \isIndet{\dexp}\\
  \isGround{\htau}
}{
  \isIndet{\dcasttwo{\dexp}{\tehole}{\htau}}
}

\inferrule[ICastArr]{
  \tarr{\htau_1}{\htau_2} \neq \tarr{\htau_3}{\htau_4}\\
  \isIndet{\dexp}
}{
  \isIndet{\dcasttwo{\dexp}{\tarr{\htau_1}{\htau_2}}{\tarr{\htau_3}{\htau_4}}}
}

\inferrule[IFailedCast] {
  \isFinal{\dexp}\\
  \isGround{\htau_1}\\
  \isGround{\htau_2}\\
  \htau_1\neq\htau_2
}{
  \isIndet{\dcastfail{\dexp}{\htau_1}{\htau_2}}
}

%% \inferrule[ICast]
%% {\isIndet{\dexp}}
%% {\isIndet{\dcast{\htau}{\dexp}}}

\end{mathpar}

%\vsepRule

\CaptionLabel{Final Forms}{fig:isFinal}
\label{fig:isValue}
\label{fig:isIndet}
\end{figure}

%% \begin{figure}[t]
\judgbox{\stepsToD{\Delta}{\dexp_1}{\dexp_2}}{$\dexp_1$
steps to $\dexp_2$}
\begin{mathpar}
\inferrule[STEHoleEvaled]
{ }
{\stepsToD{\Delta}{\dehole{\mvar}{\subst}{\unevaled}}{\dehole{\mvar}{\subst}{\evaled}}}

\inferrule[STNEHoleStep]
{\stepsToD{\Delta}{\dexp_1}{\dexp_2} }
{\stepsToD{\Delta}{\dhole{\dexp_1}{\mvar}{\subst}{\unevaled}}{\dhole{\dexp_2}{\mvar}{\subst}{\unevaled}}}

\inferrule[STNEHoleEvaled]
{\isFinal{\dexp}}
{\stepsToD{\Delta}{\dhole{\dexp}{\mvar}{\subst}{\unevaled}}{\dhole{\dexp}{\mvar}{\subst}{\evaled}}}

\inferrule[STCast]
{
\isValue{\dexp}\\
\hasType{\Delta}{\emptyset}{\dexp}{\htau_2} \\
\tconsistent{\tau_1}{\tau_2}}
{\stepsToD{\Delta}{\dcast{\htau_1}{\dexp}}{\dexp}}

\inferrule[STApStep1]
{\stepsToD{\Delta}{\dexp_1}{\dexp_1'}}
{\stepsToD{\Delta}{\dap{\dexp_1}{\dexp_2}}{\dap{\dexp_1'}{\dexp_2}}}

\inferrule[STApStep2]
{ \isFinal{\dexp_1} \\ \stepsToD{\Delta}{\dexp_2}{\dexp_2'}}
{\stepsToD{\Delta}{\dap{\dexp_1}{\dexp_2}}{\dap{\dexp_1}{\dexp_2'}}}

\inferrule[STApBeta]
{ \isFinal{\dexp_2} }
{\stepsToD{\Delta}{\dapP{\dlam{x}{\htau}{\dexp_1}}{\dexp_2}}{ [\dexp_2/x]\dexp_1 }}
\end{mathpar}
\caption{Structural Dynamics}
\label{fig:stepsTo}
\end{figure}

% !TEX root = hazelnut-dynamics.tex
\begin{comment}
\begin{figure}[t]

\begin{comment}
\vsepRule

\judgbox{\isevalctx{\evalctx}}{$\evalctx$ is an evaluation context}
\begin{mathpar}
\inferrule[ECDot]{ }{
  \isevalctx{\evalhole}
}

%% \inferrule[ECLam]{
%%   \isevalctx{\evalctx}
%% }{
%%   \isevalctx{\halam{x}{\htau}{\evalctx}}
%% }

\inferrule[ECAp1]{
  \isevalctx{\evalctx}
}{
  \isevalctx{\hap{\evalctx}{\dexp}}
}

\inferrule[ECAp2]{
  \maybePremise{\isFinal{\dexp}}\\
  \isevalctx{\evalctx}
}{
  \isevalctx{\hap{\dexp}{\evalctx}}
}

\inferrule[ECNEHole]{
  \isevalctx{\evalctx}
}{
  \isevalctx{\dhole{\evalctx}{\mvar}{\subst}{}}
}

\inferrule[ECCast]{
  \isevalctx{\evalctx}
}{
  \isevalctx{\dcasttwo{\evalctx}{\htau_1}{\htau_2}}
}

\inferrule[ECFailedCast]{
  \isevalctx{\evalctx}
}{
  \isevalctx{\dcastfail{\evalctx}{\htau_1}{\htau_2}}
}
\end{mathpar}
% \end{comment}
\vsepRule


\caption{Evaluation Contexts}
\label{fig:eval-contexts}
\end{figure}
\end{comment}

%% \vsepRule

\begin{figure}
\judgbox{\reducesE{}{\dexp}{\dexp'}}{$\dexp$ takes an instruction transition to $\dexp'$}
\begin{mathpar}
\inferrule[ITBeta]{
  \maybePremise{\isFinal{\dexp_2}}
}{
  \reducesE{}{\hap{(\halam{x}{\htau}{\dexp_1})}{\dexp_2}}{[\dexp_2/x]\dexp_1}
}

\inferrule[ITApCast]{
  \maybePremise{\isFinal{\dexp_1}}\\
  \maybePremise{\isFinal{\dexp_2}}\\
  \tarr{\htau_1}{\htau_2} \neq \tarr{\htau_1'}{\htau_2'}
}{
  \reducesE{}
    {\hap{\dcasttwo{\dexp_1}{\tarr{\htau_1}{\htau_2}}{\tarr{\htau_1'}{\htau_2'}}}{\dexp_2}}
    {\dcasttwo{(\hap{\dexp_1}{\dcasttwo{\dexp_2}{\htau_1'}{\htau_1}})}{\htau_2}{\htau_2'}}
}

\inferrule[ITCastId]{
  \maybePremise{\isFinal{\dexp}}
}{
  \reducesE{}{\dcasttwo{\dexp}{\htau}{\htau}}{\dexp}
}

\inferrule[ITCastSucceed]{
  \maybePremise{\isFinal{\dexp}}\\
  \isGround{\htau}
}{
  \reducesE{}{\dcastthree{\dexp}{\htau}{\tehole}{\htau}}{\dexp}
}

\inferrule[ITCastFail]{
  \maybePremise{\isFinal{\dexp}}\\
  \htau_1\neq\htau_2\\\\
  \isGround{\htau_1}\\
  \isGround{\htau_2}
}{
  \reducesE{}
    {\dcastthree{\dexp}{\htau_1}{\tehole}{\htau_2}}
    {\dcastfail{\dexp}{\htau_1}{\htau_2}}
}

\inferrule[ITGround]{
  \maybePremise{\isFinal{\dexp}}\\
  \groundmatch{\htau}{\htau'}
}{
  \reducesE{}
    {\dcasttwo{\dexp}{\htau}{\tehole}}
    {\dcastthree{\dexp}{\htau}{\htau'}{\tehole}}
}

\inferrule[ITExpand]{
  \maybePremise{\isFinal{\dexp}}\\
  \groundmatch{\htau}{\htau'}
}{
  \reducesE{}
    {\dcasttwo{\dexp}{\tehole}{\htau}}
    {\dcastthree{\dexp}{\tehole}{\htau'}{\htau}}
}

%% \inferrule[ITCast]{
%%   \isFinal{d}\\
%%   \hasType{\Delta}{\emptyset}{d}{\tau_2}\\
%%   \tconsistent{\tau_1}{\tau_2}
%% }{
%%   \reducesE{\Delta}{\dcast{\htau_1}{d}}{d}
%% }
%% 
%% \inferrule[ITEHole]{ }{
%%   \reducesE{\Delta}{\dehole{\mvar}{\subst}{\unevaled}}{\dehole{\mvar}{\subst}{\evaled}}
%% }
%% 
%% \inferrule[ITNEHole]{
%%   \isFinal{d}
%% }{
%%   \reducesE{\Delta}{\dhole{d}{\mvar}{\subst}{\unevaled}}{\dhole{d}{\mvar}{\subst}{\evaled}}
%% }
\end{mathpar}
\CaptionLabel{Instruction Transitions}{fig:instruction-transitions}
\end{figure}

\begin{figure}
$\arraycolsep=4pt\begin{array}{rllllll}
\mathsf{EvalCtx} & \evalctx & ::= &
  \evalhole ~\vert~
  \hap{\evalctx}{\dexp} ~\vert~
  \hap{\dexp}{\evalctx} ~\vert~
  \dhole{\evalctx}{\mvar}{\subst}{} ~\vert~
  \dcasttwo{\evalctx}{\htau}{\htau} ~\vert~
  \dcastfail{\evalctx}{\htau}{\htau}
\end{array}$

\vsepRule

\judgbox{\selectEvalCtx{\dexp}{\evalctx}{\dexp'}}{$\dexp$ is obtained by placing $\dexp'$ at the mark in $\evalctx$}
\begin{mathpar}
\inferrule[FHOuter]{ }{
  \selectEvalCtx{\dexp}{\evalhole}{\dexp}
}

%% \inferrule[FLam]{
%%   \selectEvalCtx{d}{\evalctx}{d'}
%% }{
%%   \selectEvalCtx{\halam{x}{\htau}{d}}{\halam{x}{\htau}{\evalctx}}{d'}
%% }

\inferrule[FHAp1]{
  \selectEvalCtx{\dexp_1}{\evalctx}{\dexp_1'}
}{
  \selectEvalCtx{\hap{\dexp_1}{\dexp_2}}{\hap{\evalctx}{\dexp_2}}{\dexp_1'}
}

\inferrule[FHAp2]{
  \maybePremise{\isFinal{\dexp_1}}\\
  \selectEvalCtx{\dexp_2}{\evalctx}{\dexp_2'}
}{
  \selectEvalCtx{\hap{\dexp_1}{\dexp_2}}{\hap{\dexp_1}{\evalctx}}{\dexp_2'}
}

%% \inferrule[FHEHole]{ }{
%%   \selectEvalCtx{\dehole{\mvar}{\subst}{}}{\evalhole}{\dehole{\mvar}{\subst}{}}
%% }
%% 
%% \inferrule[FHNEHoleEvaled]{ }{
%%   \selectEvalCtx{\dhole{d}{\mvar}{\subst}{\evaled}}{\evalhole}{\dhole{d}{\mvar}{\subst}{\evaled}}
%% }

\inferrule[FHNEHoleInside]{
  \selectEvalCtx{\dexp}{\evalctx}{\dexp'}
}{
  \selectEvalCtx{\dhole{\dexp}{\mvar}{\subst}{}}{\dhole{\evalctx}{\mvar}{\subst}{}}{\dexp'}
}

%% \inferrule[FHNEHoleFinal]{
%%   \isFinal{d}
%% }{
%%   \selectEvalCtx{\dhole{d}{\mvar}{\subst}{\unevaled}}{\evalhole}{\dhole{d}{\mvar}{\subst}{\unevaled}}
%% }

\inferrule[FHCastInside]{
  \selectEvalCtx{\dexp}{\evalctx}{\dexp'}
}{
  \selectEvalCtx{\dcasttwo{\dexp}{\htau_1}{\htau_2}}
                {\dcasttwo{\evalctx}{\htau_1}{\htau_2}}
                {\dexp'}
}

\inferrule[FHFailedCast]{
  \selectEvalCtx{\dexp}{\evalctx}{\dexp'}
}{
  \selectEvalCtx{\dcastfail{\dexp}{\htau_1}{\htau_2}}
                {\dcastfail{\evalctx}{\htau_1}{\htau_2}}
                {\dexp'}
}

%% \inferrule[FHCastFinal]{
%%   \isFinal{d}
%% }{
%%   \selectEvalCtx{\dcast{\htau}{d}}{\evalhole}{\dcast{\htau}{d}}
%% }
\end{mathpar}

\vsepRule

\judgbox{\stepsToD{}{\dexp}{\dexp'}}{$\dexp$ steps to $\dexp'$}
\vspace{-10px}
\begin{mathpar}
\inferrule[Step]{
  \selectEvalCtx{d}{\evalctx}{\dexp_0}\\
  \reducesE{}{\dexp_0}{\dexp_0'}\\
  \selectEvalCtx{\dexp'}{\evalctx}{\dexp_0'}
}{
  \stepsToD{}{\dexp}{\dexp'}
}
\end{mathpar}
\CaptionLabel{Evaluation Contexts and Steps}{fig:step}
\end{figure}

{\color{gray}\blindtext}
{\color{gray}\blindtext}
{\color{gray}\blindtext}
{\color{gray}\blindtext}
{\color{gray}\blindtext}

\begin{lemma}[Focus Formation] ~
  If $\selectEvalCtx{\dexp}{\evalctx}{\dexp'}$
  then $\isevalctx{\evalctx}$.
\end{lemma}

\begin{lemma}[Matched Ground Type Invariant] ~
  If $\groundmatch{\htau}{\htau'}$
  then $\isGround{\htau'}$
  and $\tconsistent{\htau}{\htau'}$
  and $\htau\neq\htau'$.
\end{lemma}

\subsubsection{Examples}




\cy{definition of complete types and terms? do we care about completeness
  of $\hexp$s or $\dexp$s?}


%% \begin{figure}[!ht]
%%   \begin{definition}
%%     $\hasType{\Delta}{\hGamma}{\sigma}{\hGamma'}$ iff for each $\dexp/x \in \sigma$, we have $x : \htau \in \hGamma'$ and $\hasType{\Delta}{\hGamma}{\dexp}{\htau}$.
%%   \end{definition}
%%   \caption{substitution type assignment}
%%   \label{fig:subassign}
%% \end{figure}


%% \begin{figure}[!ht]
  \begin{definition}
    $\hasType{\Delta}{\hGamma}{\sigma}{\hGamma'}$ iff $\domof{\sigma} = \domof{\Gamma'}$ and for each $\dexp/x \in \sigma$, we have $x : \htau \in \hGamma'$ and $\hasType{\Delta}{\hGamma}{\dexp}{\htau}$.
  \end{definition}
%%   \caption{substitution type assignment}
%% \end{figure}





%%   \halam{x}{\htau}{\evalctx} ~\vert~



\clearpage
%% \subsection{Metatheory}
\subsection{Metatheory: Type Soundness}

{\color{gray}\blindtext}

\begin{theorem}[Expandability] ~
  \begin{enumerate}
    \item
      If $\hsyn{\hGamma}{\hexp}{\htau}$
      then $\expandSyn{\hGamma}{\hexp}{\htau}{\dexp}{\Delta}$
      for some $\dexp$ and $\Delta$.
    \item
      If $\hana{\hGamma}{\hexp}{\htau}$
      then $\expandAna{\hGamma}{\hexp}{\htau}{\dexp}{\htau'}{\Delta}$
      for some $\dexp$ and $\htau'$ and $\Delta$.
  \end{enumerate}
\end{theorem}

\begin{theorem}[Correspondence] ~
  \begin{enumerate}
    \item
      If $\expandSyn{\hGamma}{\hexp}{\htau}{\dexp}{\Delta}$
      then $\hsyn{\hGamma}{\hexp}{\htau}$.
    \item
      If $\expandAna{\hGamma}{\hexp}{\htau}{\dexp}{\htau'}{\Delta}$
      then $\hana{\hGamma}{\hexp}{\htau}$.
  \end{enumerate}
\end{theorem}

\begin{theorem}[Typed Expansion] ~
  \begin{enumerate}
    \item
      If $\expandSyn{\hGamma}{\hexp}{\htau}{\dexp}{\Delta}$
      then $\hasType{\Delta}{\hGamma}{\dexp}{\htau}$.
    \item
      If $\expandAna{\hGamma}{\hexp}{\htau}{\dexp}{\htau'}{\Delta}$
      then $\tconsistent{\htau}{\htau'}$ and $\hasType{\Delta}{\hGamma}{\dexp}{\htau'}$.
  \end{enumerate}
\end{theorem}

\begin{theorem}[Expansion Unicity] ~
  \begin{enumerate}
    \item
      If $\expandSyn{\hGamma}{\hexp}{\htau}{\dexp}{\Delta}$
      and $\expandSyn{\hGamma}{\hexp}{\htau'}{\dexp'}{\Delta'}$
      then $\htau=\htau'$ and $\dexp=\dexp'$ and $\Delta=\Delta'$.
    \item
      If $\expandAna{\hGamma}{\hexp}{\htau_1}{\dexp}{\htau_2}{\Delta}$
      and $\expandAna{\hGamma}{\hexp}{\htau_1}{\dexp'}{\htau_2'}{\Delta'}$
      then $\dexp=\dexp'$ and $\htau_2=\htau_2'$ and $\Delta=\Delta'$.
  \end{enumerate}
\end{theorem}

\begin{theorem}[Type Assignment Unicity]
  If $\hasType{\Delta}{\hGamma}{\dexp}{\htau}$
  and $\hasType{\Delta}{\hGamma}{\dexp}{\htau'}$
  then $\htau=\htau'$.
\end{theorem}

\begin{theorem}[Preservation]
  If $\hasType{\Delta}{\emptyset}{\dexp}{\htau}$ and
  $\stepsToD{\Delta}{\dexp}{\dexp'}$ then
  $\hasType{\Delta}{\emptyset}{\dexp'}{\htau}$.
\end{theorem}

\begin{theorem}[Progress]
  If $\hasType{\Delta}{\emptyset}{\dexp}{\htau}$ then either
  (i) $\stepsToD{}{\dexp}{\dexp'}$ or
  (ii) $\isIndet{\dexp}$ or
  (iii) $\isBoxedValue{\dexp}$.
\end{theorem}

\begin{theorem}[Canonical Value Forms]
  If $\hasType{\hDelta}{\emptyset}{\dexp}{\htau}$ and $\isValue{\dexp}$
  then $\htau\neq\tehole$ and
  \begin{enumerate}[label=(\alph*)]
    \item If $\htau=b$ then $\dexp=c$.
    \item If $\htau=\tarr{\htau_1}{\htau_2}$
          then $\dexp=\halam{x}{\htau_1}{\dexp'}$
          where $\hasType{\hDelta}{x : \htau_1}{\dexp'}{\htau_2}$.
  \end{enumerate}
\end{theorem}

\begin{theorem}[Canonical Boxed Forms]
  If $\hasType{\hDelta}{\emptyset}{\dexp}{\htau}$ and $\isBoxedValue{\dexp}$
  then
  \begin{enumerate}[label=(\alph*)]
    \item If $\htau=b$ then $\dexp=c$.
    \item If $\htau=\tarr{\htau_1}{\htau_2}$ then either
      \begin{enumerate}
        \item[i.]
          $\dexp=\halam{x}{\htau_1}{\dexp'}$
          where $\hasType{\hDelta}{x : \htau_1}{\dexp'}{\htau_2}$, or
        \item[ii.]
          $\dexp=\dcasttwo{\dexp'}{\tarr{\htau_1'}{\htau_2'}}{\tarr{\htau_1}{\htau_2}}$
          where $\tarr{\htau_1'}{\htau_2'}\neq\tarr{\htau_1}{\htau_2}$
          and $\hasType{\hDelta}{\emptyset}{\dexp'}{\tarr{\htau_1'}{\htau_2'}}$.
      \end{enumerate}
    \item If $\htau=\tehole$
          then $\dexp=\dcasttwo{\dexp'}{\htau'}{\tehole}$
          where $\isGround{\htau'}$
          and $\hasType{\hDelta}{\emptyset}{\dexp'}{\htau'}$.
  \end{enumerate}
\end{theorem}

\begin{theorem}[Canonical Indeterminate Forms]
  \rkc{TODO}

  \rkc{TODO}

  \rkc{TODO}

  \rkc{TODO}

  \rkc{TODO}

  \rkc{TODO}

  \rkc{TODO}

  \rkc{TODO}
\end{theorem}

% !TEX root = hazelnut-dynamics.tex
\begin{figure}[h]
\judgbox{\isComplete{\htau}}{$\htau$ is complete}
\begin{mathpar}
\inferrule[BComplete]{ }{
  \isComplete{b}
}

\inferrule[ArrComplete]{
  \isComplete{\htau_1}\\
  \isComplete{\htau_2}
}{
  \isComplete{\tarr{\htau_1}{\htau_2}}
}
\end{mathpar}

\vsepRule

\judgbox{\isComplete{\hexp}}{$\hexp$ is complete}
\begin{mathpar}
\inferrule[EVarComplete]{ }{
  \isComplete{x}
}

\inferrule[EConstComplete]{ }{
  \isComplete{c}
}

\inferrule[ESynLamComplete]{
  \isComplete{\htau}\\
  \isComplete{\hexp}
}{
  \isComplete{\halam{x}{\htau}{\hexp}}
}

\inferrule[EAnaLamComplete]{
  \isComplete{\hexp}
}{
  \isComplete{\hlam{x}{\hexp}}
}

\inferrule[EApComplete]{
  \isComplete{\hexp_1}\\
  \isComplete{\hexp_2}
}{
  \isComplete{\hap{\hexp_1}{\hexp_2}}
}

\inferrule[EAscComplete]{
  \isComplete{\hexp}\\
  \isComplete{\htau}
}{
  \isComplete{\hexp : \htau}
}
\end{mathpar}

\vsepRule


\judgbox{\isComplete{\dexp}}{$\dexp$ is complete}
\begin{mathpar}
\inferrule[DVarComplete]{ }{
  \isComplete{x}
}

\inferrule[DConstComplete]{ }{
  \isComplete{c}
}

\inferrule[DLamComplete]{
  \isComplete{\htau}\\
  \isComplete{\dexp}
}{
  \isComplete{\dlam{x}{\htau}{\dexp}}
}

\inferrule[DApComplete]{
  \isComplete{\dexp_1}\\
  \isComplete{\dexp_2}
}{
  \isComplete{\dap{\dexp_1}{\dexp_2}}
}

\inferrule[DCastComplete]{
  \isComplete{\dexp}\\
  \isComplete{\htau_1}\\
  \isComplete{\htau_2}
}{
  \isComplete{\dcasttwo{\dexp}{\htau_1}{\htau_2}}
}
\end{mathpar}

\caption{Complete types, external expressions, and internal expressions}
\label{fig:complete}
\end{figure}

When two types are complete and consistent, they are equal.

\begin{lem}[Complete Consistency] If $\tconsistent{\htau_1}{\htau_2}$ an $\isComplete{\htau_1}$ and $\isComplete{\htau_2}$ then $\htau_1 = \htau_2$. 
\end{lem}
\begin{proof} By straightforward rule induction. \end{proof}

This implies that in a well-typed internal expression, every cast is an identity cast.


\autoref{fig:complete} defines complete types and expressions...

\begin{theorem}[Complete Progress \cy{??}]
If $\hcomplete{\hexp}$ and
$\expandSyn{\emptyset}{\hexp}{\htau}{\dexp}{\Delta}$ then either
$\isValue{\dexp}$ or $\stepsToD{\Delta}{\dexp}{\dexp'}$ for some $\dexp'$.

\end{theorem}

\begin{theorem}[Complete Preservation \cy{??}]
If $\hcomplete{\hexp}$ and
$\expandSyn{\emptyset}{\hexp}{\htau}{\dexp}{\Delta}$ and
$\stepsToD{\Delta}{\dexp}{\dexp'}$ then
$\hasType{\Delta}{\emptyset}{\dexp'}{\htau}$
\end{theorem}


\subsection{Metatheory: Correctness of Edit-and-Resume}

\begin{figure}[t]
\judgbox
  {[\dexp_1 / x] \dexp_2 = \dexp_3}
  {$\dexp_3$ is the result of substituting $\dexp_1$ for $x$ in $\dexp_2$}
\[
\begin{array}{lcll}
[\dexp_1 / x] c
&=&
c\\
{[\dexp_1 / x] x}
&=&
\dexp_1\\
{[\dexp_1 / x] y}
&=&
y & (y \neq x)\\
{[\dexp_1 / x] \dlam{y}{\htau}{\dexp_2}}
&=&
{\dlam{y}{\htau}{[\dexp_1 / x] \dexp_2}}
& (y \neq x)
\\
{[\dexp_1 / x] \dap{d_2}{d_3}}
&=&
{\dapP{[\dexp_1 / x] d_2}{[\dexp_1 / x] d_3}}
\\
% {[\dexp_1 / x] \dinj{i}{e_2}}
% &=&
% {\dinj{i}{[\dexp_1 / x] e_2}}
% \\
% {[\dexp_1 / x] \dcase{e}{x}{e_x}{y}{e_y}}
% &=&
% {\dcase{[\dexp_1 / x]e}{x}{[\dexp_1 / x]e_x}{y}{[\dexp_1 / x]e_y}}
% \\
{[\dexp_1 / x]} \dehole{\mvar}{\subst}{}
&=&
\dehole{\mvar}{[\dexp_1 / x] \subst}{}
\\
{[\dexp_1 / x] \dhole{\dexp_2}{\mvar}{\subst}{}}
&=&
\dhole{ [ \dexp_1 / x ] \dexp_2 }{\mvar}{[\dexp_1 / x] \subst}{}
\\
{[\dexp_1 / x] \dcast{\htau}{\dexp}}
&=&
\dcast{\htau}{[\dexp_1 / x] \dexp}
\end{array}
\]
\caption{Substitution}
\end{figure}


{\color{gray}\blindtext}

\begin{theorem}[Commutativity \cy{state this}]
\end{theorem}


\begin{theorem}[Maximum Informativity]
If the expansion produces $t1$, and there exists another possible type choice
$t2$, then $t1 \sim t2$ and $t1 JOIN t2 = t1$
\end{theorem}\footnote{idea is that special casing the holes in EANEHole gives you ``the
most descriptive hole types'' for some sense of what that means -- they'd
all just be hole other wise. from Matt:
\begin{quote}
It sounds like we need a something akin to an abstract domain (a lattice),
where hole has the least information, and a fully-defined type (without
holes) has the most information.  You can imagine that this lattice really
expands the existing definition we have of type consistency, which is
merely the predicate that says whether two types are comparable
(“join-able”) in this lattice.  lattice join is the operation that goes
through the structure of two (consistent) types, and chooses the structure
that is more defined (i.e., non-hole, if given the choice between hole and
non-hole).

The rule choosenonhole below is the expansion of this consistency rule that
we already have (hole consistent with everything)
\end{quote}}

\begin{verbatim}
t not hole
-------------------- :: choose-non-hole
hole JOIN t  = t
\end{verbatim}
\begin{verbatim}
------------ :: hole-consistent-with-everything
hole ~ t
\end{verbatim}

\cy{canonical forms}

\cy{indeterminate forms}
