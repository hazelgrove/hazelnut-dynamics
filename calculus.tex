% !TEX root = hazelnut-dynamics.tex

\clearpage
\newcommand{\calculusSec}{Hazelnut Live, Formally}
\section{\protect\calculusSec}
\label{sec:calculus}

We will now make the intuitions developed in the previous section formally precise by specifying the \HazelnutLive core calculus and  its accompanying metatheory. We have mechanized these formal developments using the Agda proof assistant \cite{norell:thesis,norell2009dependently} (see Sec.~\ref{sec:agda-mechanization} for some additional details, and the supplemental material for the full mechanization). % The proofs of the metatheorems are established by this mechanization; the proofs are only very briefly outlined in this section.

The syntax of the core calculus, specified in Fig.~\ref{fig:hazelnut-live-syntax}, consists of types and expressions with holes. We distinguish between \emph{external} expressions, $e$, and \emph{internal} expressions, $d$. External expressions correspond to programs as entered by the programmer (see Sec.~\ref{sec:implementation}\todo{or maybe Sec. 2?}{} for discussion of manual, semi-automated and fully automated hole entry methods). Each well-typed external expression (as specified in Sec.~\ref{sec:external-statics} below) expands to a well-typed internal expression (see Sec.~\ref{sec:expansion}) before it is evaluated (see Sec.~\ref{sec:evaluation}). We distinguish the external and internal languages because (1) the external language supports type inference and explicit type ascriptions, $\hexp : \htau$, but it is  formally simpler to eliminate ascriptions and specify a type assignment system when defining the dynamic semantics\todo{cite frank's notes}; and (2) we need additional syntactic machinery during evaluation for tracking hole closures and dynamic type casts. This machinery is inserted by the expansion step, rather than entered explicitly by the programmer. In this regard, the internal language is analagous to the cast calculus in the gradually typed lambda calculus \cite{DBLP:conf/snapl/SiekVCB15,Siek06a}, though as we will see the \HazelnutLive internal language goes beyond the cast calculus in several respects.

\begin{figure}[t]
$\arraycolsep=4pt\begin{array}{rllllll}
\mathsf{HTyp} & \htau & ::= &
  b ~\vert~
  \tarr{\htau}{\htau} ~\vert~
  % \tprod{\htau}{\htau} ~\vert~
  % \tsum{\htau}{\htau} ~\vert~
  \tehole\\
\mathsf{HExp} & \hexp & ::= &
  c ~\vert~
  x ~\vert~
  \halam{x}{\htau}{\hexp} ~\vert~
  \hap{\hexp}{\hexp} ~\vert~
  % \hpair{\hexp}{\hexp} ~\vert~
  % \hprj{i}{\hexp} ~\vert~
  % \hinj{i}{\hexp} ~\vert~
  % \hcase{\hexp}{x}{\hexp}{x}{\hexp} ~\vert~
  % \hadd{\hexp}{\hexp} ~\vert~
  \hehole{u} ~\vert~
  \hhole{\hexp}{u} ~\vert~
  {\hlam{x}{\hexp}} ~\vert~
  \hexp : \htau\\
% \mathsf{Mark} & \markname{} & ::= &
%   \evaled{} ~\vert~  \unevaled{}\\
 \mathsf{DHExp} & \dexp  & ::= &
  c ~\vert~
  x ~\vert~
  {\halam{x}{\htau}{\dexp}} ~\vert~
  \hap{\dexp}{\dexp} ~\vert~
  % \hpair{\dexp}{\dexp} ~\vert~
  % \hprj{i}{\dexp} ~\vert~
  % \hinj{i}{\dexp} ~\vert~
  % \hcase{\dexp}{x}{\dexp}{x}{\dexp} ~\vert~
  % \hadd{\dexp}{\dexp} ~\vert~
  \dehole{\mvar}{\subst}{} ~\vert~
  \dhole{\dexp}{\mvar}{\subst}{} ~\vert~
  \dcasttwo{\dexp}{\htau}{\htau} ~\vert~
  \dcastfail{\dexp}{\htau}{\htau}\\
\end{array}$
\caption{Syntax of H-types, H-expressions and dynamic H-expressions.
We write $x$ to range over variables,
$u$ over hole names, and
$\sigma$ over finite substitutions (i.e., environments) from
variables to dynamic H-expressions.}
\label{fig:HTyp}
\label{fig:HExp}
\end{figure}


% \rkc{this syntactic sugar is used in four places: ITCastSucceed, ITCastFail,
% ITGround, and ITExpand. that's not many, and those rules don't look much more
% cluttered without the sugar, so consider eliminating it. if so, just toggle the
% definition of the dcastthree macro to the unsugared option.}

\subsection{Static Semantics of the External Language}
\label{sec:external-statics}

%%%%%%%%%%%%%%%%%%%%%%%%%%%%%%%%%%%%%%%%%%%%%%%%%%%%%%%%%%%%%%%%%%%%%%%%%%%%%%%%%%%%%%%%%%%%%%%%%%%%%%%%%%
%% Matt's high-level Writing comments for Cyrus (for Section 3 mostly):

% Please, please please: Use line breaks.  Git does't work well without them.
% Change your editor or editor settings please.

% Cyrus, your writing is generally very good, but is often harder for
% me to digest than it should be, as a reader.  I think it'd be easier
% to read on a first pass if we followed my preferences below, which I
% humbly offer as gentle suggestions:

% I think that some these suggestions are especially important for reading
% really wide columns of text, as in this new PACMPL format. :)

% My nits boil down to these preferences:

% 0. NO PASSIVE VOICE --- EVER.  Instead, use ``We'' as the subject,
% and choose a more interesting verb than 'is' for the sentence.
% (Emacs says that there are 100 occurences of the word 'is' in the
% text below.  That's way, way too much passive voice.)

% 1. Shorter sentences.

% 2. Fewer parenthenticals: just use more sentences instead.

% 3. Shorter paragraphs, with more defined 'paragraph structure'. (E.g., topic sentence)

% 4. Shorter sentence subjects (with subject closer to verb).

% 5. Use macros for fig and section references.

% 6. If a sentence introduces a figure, that sentence should be the
% first sentence of a paragraph, and the figure reference should be
% the first word of that sentence, ideally.  As a reader, then I look
% at figures first, and find the paragraphs for them easily by
% scanning the text.

% 7. For each greek letter, prefix with an english word 'classifier',
% e.g., 'external expression~$e$', not just '$e$'.  Use way, way fewer
% commas please.

% 8. Use singlar over plural sentence subjects, whenever possible;
% e.g., an ``empty hole is written..'' not ``empty holes are
% written''; or ``an empty hole synthesizes'' not ``empty holes
% synthesize''

% 9. Fewer commas. Each 'extra' comma is a burden to the reader (it's
% asking the reader to allocate a stack frame in their brain!)

% 10. Citations in this Names+Year form break sentence and paragraph
% flow for the reader, sometimes quite severly.  Put long citation
% chains at the ends of sentences (or better yet, in their own little
% paragraph, as I did below).

% *** I'm transforming the next paragraph below as a demonstration,
% creating smaller, more cohesive paragraphs with shorter sentences.
% See what you think.
%
% I'm not doing all of Section 3, tho, since
%
% (1) you may hate what I've done, and
%
% (2) I don't want to do a lot of work that will just be
% reverted/resented :)

% BTW: My version is _shorter_, not longer than the original :)
%
% Also, In my humble opinion, it is way, way easier to read. ;)

%%%%%%%%%%%%%%%%%%%%%%%%%%%%%%%%%%%%%%%%%%%%%%%%%%%%%%%%%%%%%%%%%%%%%%%%%%%%%%%%%%%%%%%%%%%%%%%%%%%%%%%%%%

% !TEX root = hazelnut-dynamics.tex

\begin{figure}[t]
\judgbox{\hsyn{\hGamma}{\hexp}{\htau}}{$\hexp$ synthesizes type $\htau$}
\begin{mathpar}
\inferrule[SConst]{ }{
  \hsyn{\hGamma}{c}{b}
}

\inferrule[SVar]{
  x : \htau \in \hGamma
}{
  \hsyn{\hGamma}{x}{\htau}
}

\inferrule[SLam]{
  \hsyn{\hGamma, x : \htau_1}{\hexp}{\htau_2}
}{
  \hsyn{\hGamma}{\halam{x}{\htau_1}{\hexp}}{\tarr{\htau_1}{\htau_2}}
}

\inferrule[SAp]{
  \arrayenvbr{
    \hsyn{\hGamma}{\hexp_1}{\htau_1}    
    \\
    \hana{\hGamma}{\hexp_2}{\htau_2}
  }
  \\
    \arrmatch{\htau_1}{\tarr{\htau_2}{\htau}}
}{
  \hsyn{\hGamma}{\hap{\hexp_1}{\hexp_2}}{\htau}
}

\inferrule[SEHole]{ }{
  \hsyn{\hGamma}{\hehole{u}}{\tehole}
}

\inferrule[SNEHole]{
  \hsyn{\hGamma}{\hexp}{\htau}
}{
  \hsyn{\hGamma}{\hhole{\hexp}{u}}{\tehole}
}

\inferrule[SAsc]{
  \hana{\hGamma}{\hexp}{\htau}
}{
  \hsyn{\hGamma}{\hexp : \htau}{\htau}
}
\end{mathpar}

\vsepRule

\judgbox{\hana{\hGamma}{\hexp}{\htau}}{$\hexp$ analyzes against type $\htau$}
\begin{mathpar}
\inferrule[ALam]{
  \arrmatch{\htau}{\tarr{\htau_1}{\htau_2}}\\
  \hana{\hGamma, x : \htau_1}{\hexp}{\htau_2}
}{
  \hana{\hGamma}{\hlam{x}{\hexp}}{\htau}
}

\inferrule[ASubsume]{
  \hsyn{\hGamma}{\hexp}{\htau}\\
  \tconsistent{\htau}{\htau'}
}{
  \hana{\hGamma}{\hexp}{\htau'}
}
\end{mathpar}
\CaptionLabel{Bidirectional Typing of External Expressions}{fig:bidirectional-typing}
\end{figure}


We give an overview of the static semantics of the \HazelnutLive
external language.

\Figref{fig:bidirectional-typing} defines \emph{bidirectional} typing
%
as two mutually defined judgements: The type synthesis
judgement~$\hsyn{\hGamma}{\hexp}{\htau}$ synthesizes a type~$\htau$
for external expression~$\hexp$ under typing context $\hGamma$, which tracks typing
assumptions of the form $x : \htau$ in the usual
manner \cite{pfpl,tapl}.
%
The type analysis judgement~$\hana{\hGamma}{\hexp}{\htau}$ checks
expression~$\hexp$ against a given type~$\htau$.
%
Algorithmically, analysis accepts a type as input, and synthesis gives
a type as output.
%
We use synthesis for each of the programmer's ``top level'' external
exressions~$e$, giving the corresponding type as feedback.

% Algorithmically, the type is an output of type synthesis but an input of type analysis.

The benefit of specifying the \HazelnutLive external language
bidirectionally is that the programmer does not need to annotate each
hole with a type
%
\cite{Pierce:2000ve,bidi-tutorial,DBLP:conf/icfp/DunfieldK13,Chlipala:2005da}.
%
These typing rules closely follow the \Hazelnut type
system \cite{popl-paper}; we discuss minor differences below.

We supplement each expression hole from \Hazelnut with a
hole \emph{name}, which are new to \HazelnutLive; an empty hole is
written $\hehole{u}$, where $u$ is a unique name for the hole.
%
Rule {SEHole}\todo{rule name macros}{} specifies that an empty hole synthesizes hole type, written $\tehole$.
%
If an empty hole appears where an expression of some other type is
expected, e.g. under an explicit ascription (governed by Rule {SAsc})
or in the argument position of a function application (governed by
Rule {SAp}, discussed below), we apply the \emph{subsumption rule},
Rule {ASubsume}, which specifies that if an expression $e$ synthesizes
type $\htau$, then it can be checked against any \emph{consistent}
type, $\htau'$.

% !TEX root = hazelnut-dynamics.tex

\begin{figure}[t]
\judgbox{\tconsistent{\htau_1}{\htau_2}}{$\htau_1$ is consistent with $\htau_2$}
\begin{mathpar}
\inferrule[TCHole1]{ }{
  \tconsistent{\tehole}{\htau}
}

\inferrule[TCHole2]{ }{
  \tconsistent{\htau}{\tehole}
}

\inferrule[TCRefl]{ }{
  \tconsistent{\htau}{\htau}
}

\inferrule[TCArr]{
  \tconsistent{\htau_1}{\htau_1'}\\
  \tconsistent{\htau_2}{\htau_2'}
}{
  \tconsistent{\tarr{\htau_1}{\htau_2}}{\tarr{\htau_1'}{\htau_2'}}
}
%
% \inferrule{
%   \tconsistent{\htau_1}{\htau_1'}\\
%   \tconsistent{\htau_2}{\htau_2'}
% }{
%   \tconsistent{\tprod{\htau_1}{\htau_2}}{\tprod{\htau_1'}{\htau_2'}}
% }
%
% \inferrule{
%   \tconsistent{\htau_1}{\htau_1'}\\
%   \tconsistent{\htau_2}{\htau_2'}
% }{
%   \tconsistent{\tsum{\htau_1}{\htau_2}}{\tsum{\htau_1'}{\htau_2'}}
% }
\end{mathpar}

% \vsepRule

% \judgbox{\tinconsistent{\htau_1}{\htau_2}}{$\htau_1$ is inconsistent with $\htau_2$}
% \begin{mathpar}
%     \inferrule[ICBaseArr1]{ }{
%       \tinconsistent{\tb}{\tarr{\htau_1}{\htau_2}}
%     }

%     \inferrule[ICBaseArr2]{ }{
%       \tinconsistent{\tarr{\htau_1}{\htau_2}}{\tb}
%     }

%     \inferrule[ICArr1]{
%       \tinconsistent{\htau_1}{\htau_3}
%     }{
%       \tinconsistent{\tarr{\htau_1}{\htau_2}}{\tarr{\htau_3}{\htau_4}}
%     }

%     \inferrule[ICArr2]{
%       \tinconsistent{\htau_2}{\htau_4}
%     }{
%       \tinconsistent{\tarr{\htau_1}{\htau_2}}{\tarr{\htau_3}{\htau_4}}
%     }
% \end{mathpar}

\vsepRule

\judgbox{\arrmatch{\htau}{\tarr{\htau_1}{\htau_2}}}{$\htau$ has matched arrow type $\tarr{\htau_1}{\htau_2}$}
\begin{mathpar}
\inferrule[MAHole]{ }{
  \arrmatch{\tehole}{\tarr{\tehole}{\tehole}}
}

\inferrule[MAArr]{ }{
  \arrmatch{\tarr{\htau_1}{\htau_2}}{\tarr{\htau_1}{\htau_2}}
}
\end{mathpar}

% \judgbox{\prodmatch{\htau}{\tprod{\htau_1}{\htau_2}}}{$\htau$ has matched product type $\tprod{\htau_1}{\htau_2}$}
% \begin{mathpar}
% \inferrule{ }{
%   \prodmatch{\tehole}{\tprod{\tehole}{\tehole}}
% }

% \inferrule{ }{
%   \prodmatch{\tprod{\htau_1}{\htau_2}}{\tprod{\htau_1}{\htau_2}}
% }
% \end{mathpar}

% \judgbox{\summatch{\htau}{\tsum{\htau_1}{\htau_2}}}{$\htau$ has matched sum type $\tsum{\htau_1}{\htau_2}$}
% \begin{mathpar}
% \inferrule{ }{
%   \summatch{\tehole}{\tsum{\tehole}{\tehole}}
% }

% \inferrule{ }{
%   \summatch{\tsum{\htau_1}{\htau_2}}{\tsum{\htau_1}{\htau_2}}
% }
% \end{mathpar}
\caption{Type Consistency and Matching}
\Label{fig:tconsistent}
\Label{fig:arrmatch}
\end{figure}


%%%%%%%%%%%%%%%%%%%% MATT stopped editing Cyrus's text here.

Fig.~\ref{fig:tconsistent} specifies the type consistency relation, written $\tconsistent{\htau}{\htau'}$, which specifies that two types are consistent if they differ only up to type holes in corresponding positions. The hole type is consistent with every type, and so, by the subsumption rule, expression holes can appear where an expression of any type is expected. The type consistency relation here coincides with the type consistency relation from gradual type theory by identifying the hole type with the unknown type~\cite{Siek06a}. Note that type consistency is reflexive and symmetric but it is not transitive (unlike subtyping, which is anti-symmetric and transitive; subtyping can be integrated into a gradual type system following \citet{Siek:2007qy}). 

Non-empty expression holes, written $\hhole{\hexp}{u}$, behave similarly. Rule {SNEHole} specifies that non-empty expression holes also synthesize hole type as long as the expression inside the hole, $\hexp$, synthesizes some (arbitrary) type. Non-empty expression holes therefore internalize the ``red squiggles'' that many editors display under or around type inconsistencies in a program.\todo{example?}\todo{briefly say something about binding inconsistencies?}

For the familiar forms of the lambda calculus, the rules again follow prior work. For simplicity, the core calculus includes only a single base type, $b$, with a single constant, $c$, governed by Rule {SConst} (i.e. $b$ is the unit type). \Hazelnut instead defined a number type with a single operation, which we include in Appendix~\ref{sec:extensions} alongside various other standard extensions to the core calculus\todo{do this, say more?}. 
%
\rkc{perhaps too much info / unnecessary to say Hazelnut had numbers instead of unit?} 
%
Rule {SVar} specifies that variables synthesize the corresponding type from $\hGamma$. 

For the sake of exposition, \HazelnutLive includes ``half-annotated'' lambdas, $\halam{x}{\htau}{\hexp}$, in addition to the unannotated lambdas, $\hlam{x}{\hexp}$, from \Hazelnut.  Half-annotated lambdas can appear in synthetic position  according to Rule {SLam}, which is standard \cite{Chlipala:2005da}. Unannotated lambdas can only appear where the expected type is known to be either an arrow type or the hole type, which is treated as if it were $\tarr{\tehole}{\tehole}$. To avoid the need for two separate rules, Rule {ALam} uses the auxiliary relation $\arrmatch{\htau}{\tarr{\htau_1}{\htau_2}}$ in Fig.~\ref{fig:arrmatch}, which produces the matched arrow type $\tarr{\tehole}{\tehole}$ given the hole type, and operates as the identity on arrow types \cite{DBLP:conf/snapl/SiekVCB15,DBLP:conf/popl/GarciaC15}. Note that a system supporting ML-style type reconstruction \cite{damas1982principal} might include a synthetic rule for unannotated lambdas, e.g. as outlined by \citet{DBLP:conf/icfp/DunfieldK13}. 

The rule governing function application, Rule {SAp}, similarly treats expressions of hole type in function position as if they were of type $\tarr{\tehole}{\tehole}$ using the same matched arrow type judgement.

\subsection{Expansion}
\label{sec:expansion}

% !TEX root = hazelnut-dynamics.tex

\begin{figure}[p]
\judgbox
  {\expandSyn{\hGamma}{\hexp}{\htau}{\dexp}{\Delta}}
  {$\hexp$ synthesizes type $\htau$ and expands to $\dexp$}
\begin{mathpar}
\inferrule[ESConst]{ }{
  \expandSyn{\hGamma}{c}{b}{c}{\emptyset}
}

\inferrule[ESVar]{
  x : \htau \in \hGamma
}{
  \expandSyn{\hGamma}{x}{\htau}{x}{\emptyset}
}

\inferrule[ESLam]{
  \expandSyn{\hGamma, x : \htau_1}{\hexp}{\htau_2}{\dexp}{\Delta}
}{
  \expandSyn{\hGamma}{\halam{x}{\htau_1}{\hexp}}{\tarr{\htau_1}{\htau_2}}{\halam{x}{\htau_1}{\dexp}}{\Delta}
}

\inferrule[ESAp]{
  \hsyn{\hGamma}{\hexp_1}{\htau_1}\\
  \arrmatch{\htau_1}{\tarr{\htau_2}{\htau}}
  \\\\
  \expandAna{\hGamma}{\hexp_1}{\tarr{\htau_2}{\htau}}{\dexp_1}{\htau_1'}{\Delta_1}\\
  \expandAna{\hGamma}{\hexp_2}{\htau_2}{\dexp_2}{\htau_2'}{\Delta_2}
}{
  \expandSyn
    {\hGamma}
    {\hap{\hexp_1}{\hexp_2}}
    {\htau}
    {\hap{(\dcasttwo{\dexp_1}{\htau_1'}{\tarr{\htau_2}{\htau}})}
         {\dcasttwo{\dexp_2}{\htau_2'}{\htau_2}}}
    {\Dunion{\Delta_1}{\Delta_2}}
}
%
%% \inferrule[ESAp1]{
%%   \hsyn{\hGamma}{\hexp_1}{\tehole}\\
%%   \expandAna{\hGamma}{\hexp_1}{\tarr{\htau_2}{\tehole}}{\dexp_1}{\htau_1}{\Delta_1}\\
%%   \expandAna{\hGamma}{\hexp_2}{\tehole}{\dexp_2}{\htau_2}{\Delta_2}
%% }{
%%   \expandSyn{\hGamma}{\hap{\hexp_1}{\hexp_2}}{\tehole}{\hap{(\dcast{\tarr{\htau_2}{\tehole}}{\dexp_1})}{\dexp_2}}{\Dunion{\Delta_1}{\Delta_2}}
%% }
%% 
%% \inferrule[ESAp2]{
%%   \expandSyn{\hGamma}{\hexp_1}{\tarr{\htau_2}{\htau}}{\dexp_1}{\Delta_1}\\
%%   \expandAna{\hGamma}{\hexp_2}{\htau_2}{\dexp_2}{\htau'_2}{\Delta_2}\\
%%   \htau_2 \neq \htau'_2
%% }{
%%   \expandSyn{\hGamma}{\hap{\hexp_1}{\hexp_2}}{\htau}{\hap{\dexp_1}{\dcast{\htau_2}{\dexp_2}}}{\Dunion{\Delta_1}{\Delta_2}}
%% }
%% 
%% \inferrule[ESAp3]{
%%   \expandSyn{\hGamma}{\hexp_1}{\tarr{\htau_2}{\htau}}{\dexp_1}{\Delta_1}\\
%%   \expandAna{\hGamma}{\hexp_2}{\htau_2}{\dexp_2}{\htau_2}{\Delta_2}
%% }{
%%   \expandSyn{\hGamma}{\hap{\hexp_1}{\hexp_2}}{\htau}{\hap{\dexp_1}{\dexp_2}}{\Dunion{\Delta_1}{\Delta_2}}
%% }\\
%
%
% \inferrule[expand-pair]{
%   \expandSyn{\hGamma}{\hexp_1}{\htau_1}{\dexp_1}{\Delta_1}\\
%   \expandSyn{\hGamma}{\hexp_2}{\htau_2}{\dexp_2}{\Delta_2}
% }{
%   \expandSyn{\hGamma}{\hpair{\hexp_1}{\hexp_2}}{\tprod{\htau_1}{\htau_2}}{\hpair{\dexp_1}{\dexp_2}}{\Dunion{\Delta_1}{\Delta_2}}
% }
%
% \inferrule[expand-prj]{
%   a
% }{
%   b
% }
%
% (inj)
%
%
% \inferrule[expand-plus]{ }{
%   \expandSyn{\hGamma}{\hadd{\hexp_1}{\hexp_2}}{\tnum}{\hadd{\dexp_1}{\dexp_2}}{\Dunion{\Delta_1}{\Delta_2}}
% }

\inferrule[ESEHole]{ }{
  \expandSyn{\hGamma}{\hehole{u}}{\tehole}{\dehole{u}{\idof{\hGamma}}{}}{\Dbinding{u}{\hGamma}{\tehole}}
}

\inferrule[ESNEHole]{
  \expandSyn{\hGamma}{\hexp}{\htau}{\dexp}{\Delta}
}{
  \expandSyn{\hGamma}{\hhole{\hexp}{u}}{\tehole}{\dhole{\dexp}{u}{\idof{\hGamma}}{}}{\Delta, \Dbinding{u}{\hGamma}{\tehole}}
}\\
%

\inferrule[ESAsc]{
  \expandAna{\hGamma}{\hexp}{\htau}{\dexp}{\htau'}{\Delta}
}{
  \expandSyn{\hGamma}{\hexp : \htau}{\htau}{\dcasttwo{\dexp}{\htau'}{\htau}}{\Delta}
}

%% \inferrule[ESAsc1]{
%%   \expandAna{\hGamma}{\hexp}{\htau}{\dexp}{\htau'}{\Delta}\\
%%   \htau \neq \htau'
%% }{
%%   \expandSyn{\hGamma}{\hexp : \htau}{\htau}{\dcast{\htau}{\dexp}}{\Delta}
%% }
%% 
%% \inferrule[ESAsc2]{
%%   \expandAna{\hGamma}{\hexp}{\htau}{\dexp}{\htau}{\Delta}
%% }{
%%   \expandSyn{\hGamma}{\hexp : \htau}{\htau}{\dexp}{\Delta}
%% }
\end{mathpar}

\vsepRule

\judgbox
  {\expandAna{\hGamma}{\hexp}{\htau_1}{\dexp}{\htau_2}{\Delta}}
  {$\hexp$ analyzes against type $\htau_1$ and
   expands to $\dexp$ of consistent type $\htau_2$}
\begin{mathpar}
\inferrule[EALam]{
  \arrmatch{\htau}{\tarr{\htau_1}{\htau_2}}\\
  \expandAna{\hGamma, x : \htau_1}{\hexp}{\htau_2}{\dexp}{\htau'_2}{\Delta}
}{
  \expandAna{\hGamma}{\hlam{x}{\hexp}}{\htau}{\halam{x}{\htau_1}{\dexp}}{\tarr{\htau_1}{\htau_2'}}{\Delta}
}

%% \inferrule[EALam]{
%%   \expandAna{\hGamma, x : \htau_1}{\hexp}{\htau_2}{\dexp}{\htau'_2}{\Delta}
%% }{
%%   \expandAna{\hGamma}{\hlam{x}{\hexp}}{\tarr{\htau_1}{\htau_2}}{\halam{x}{\htau_1}{\dexp}}{\tarr{\htau_1}{\htau_2'}}{\Delta}
%% }
%% 
%% \inferrule[EALamHole]{
%%   \expandAna{\hGamma, x : \tehole}{\hexp}{\tehole}{\dexp}{\htau}{\Delta}
%% }{
%%   \expandAna{\hGamma}{\hlam{x}{\hexp}}{\tehole}{\halam{x}{\tehole}{\dexp}}{\tarr{\tehole}{\htau}}{\Delta}
%% }
%% 
\inferrule[EASubsume]{
  \hexp \neq \hehole{u}\\
  \hexp \neq \hhole{\hexp'}{u}\\\\
  \expandSyn{\hGamma}{\hexp}{\htau'}{\dexp}{\Delta}\\
  \tconsistent{\htau}{\htau'}
}{
  \expandAna{\hGamma}{\hexp}{\htau}{\dexp}{\htau'}{\Delta}
}

\inferrule[EAEHole]{ }{
  \expandAna{\hGamma}{\hehole{u}}{\htau}{\dehole{u}{\idof{\hGamma}}{}}{\htau}{\Dbinding{u}{\hGamma}{\htau}}
}

\inferrule[EANEHole]{
  \expandSyn{\hGamma}{\hexp}{\htau'}{\dexp}{\Delta}\\
}{
  \expandAna{\hGamma}{\hhole{\hexp}{u}}{\htau}{\dhole{\dexp}{u}{\idof{\hGamma}}{}}{\htau}{\Delta, \Dbinding{u}{\hGamma}{\htau}}
}
\end{mathpar}
\caption{Expansion}
\label{fig:expansion}
\label{fig:expandSyn}
\label{fig:expandAna}
\end{figure}

% !TEX root = hazelnut-dynamics.tex

\begin{figure}[p]
\judgbox{\hasType{\Delta}{\hGamma}{\dexp}{\htau}}{$\dexp$ is assigned type $\htau$}
\begin{mathpar}
\inferrule[TAConst]{ }{
  \hasType{\Delta}{\hGamma}{c}{b}
}

\inferrule[TAVar]{
  x : \htau \in \hGamma
}{
	\hasType{\Delta}{\hGamma}{x}{\htau}
}

\inferrule[TALam]{
  \hasType{\Delta}{\hGamma, x : \htau_1}{\dexp}{\htau_2}
}{
  \hasType{\Delta}{\hGamma}{\halam{x}{\htau_1}{\dexp}}{\tarr{\htau_1}{\htau_2}}
}

\inferrule[TAAp]{
  \hasType{\Delta}{\hGamma}{\dexp_1}{\tarr{\htau_2}{\htau}}\\
  \hasType{\Delta}{\hGamma}{\dexp_2}{\htau_2}
}{
  \hasType{\Delta}{\hGamma}{\hap{\dexp_1}{\dexp_2}}{\htau}
}

\inferrule[TAEHole]{
  \Dbinding{u}{\hGamma'}{\htau} \in \Delta\\
  \hasType{\Delta}{\hGamma}{\sigma}{\hGamma'}
}{
  \hasType{\Delta}{\hGamma}{\dehole{u}{\sigma}{}}{\htau}
}

\inferrule[TANEHole]{
  \hasType{\Delta}{\hGamma}{\dexp}{\htau'}\\\\
  \Dbinding{u}{\hGamma'}{\htau} \in \Delta\\
  \hasType{\Delta}{\hGamma}{\sigma}{\hGamma'}
}{
  \hasType{\Delta}{\hGamma}{\dhole{\dexp}{u}{\sigma}{}}{\htau}
}

\inferrule[TACast]{
  \hasType{\Delta}{\Gamma}{\dexp}{\htau_1}\\
  \tconsistent{\htau_1}{\htau_2}
}{
  \hasType{\Delta}{\hGamma}{\dcasttwo{\dexp}{\htau_1}{\htau_2}}{\htau_2}
}

\inferrule[TAFailedCast]{
  \hasType{\Delta}{\Gamma}{\dexp}{\htau_1}\\
  \isGround{\htau_1}\\
  \isGround{\htau_2}\\
  \htau_1\neq\htau_2
}{
  \hasType{\Delta}{\hGamma}{\dcastfail{\dexp}{\htau_1}{\htau_2}}{\htau_2}
}
\end{mathpar}
\caption{Type Assignment for Internal Expressions}
\Label{fig:hasType}
\end{figure}


Well-typed external expressions expand to well-typed internal expressions, $d$, for evaluation. The rules governing expansion are given in Fig.~\ref{fig:expansion} and the rules governing type assignment for internal expressions are given in Fig.~\ref{fig:hasType}. Evaluation of internal expressions will be discussed in Sec.~\ref{sec:evaluation}.

Expansion is specified, much like the type system for the external language, in the bidirectional style by two mutually defined judgements. The synthetic expansion judgement, $\expandSyn{\hGamma}{\hexp}{\htau}{\dexp}{\Delta}$, synthesizes a type, $\htau$, from $\hexp$, and produces an expansion, $d$, and a hole context, $\hDelta$. We say more about hole contexts below. The analytic expansion judgement, $\expandAna{\hGamma}{\hexp}{\htau}{\dexp}{\htau'}{\Delta}$, checks $\hexp$ against $\htau$ and produces an expansion, $d$, of type $\htau'$, and a hole context, $\hDelta$. The governing theorem below establishes that the type $\htau'$ is necessarily consistent with $\htau$.
%
\rkc{minor, but the syntax of type assignment isn't mentioned before use. it's not
entirely standard, because of the hole context, so perhaps worth mentioning.}
%
\begin{thm}[Typed Expansion]\label{thm:typed-expansion} ~
  \begin{enumerate}[nolistsep]
    \item
      If $\expandSyn{\hGamma}{\hexp}{\htau}{\dexp}{\Delta}$
      then $\hasType{\Delta}{\hGamma}{\dexp}{\htau}$.
    \item
      If $\expandAna{\hGamma}{\hexp}{\htau}{\dexp}{\htau'}{\Delta}$
      then $\tconsistent{\htau}{\htau'}$ and $\hasType{\Delta}{\hGamma}{\dexp}{\htau'}$.
  \end{enumerate}
\end{thm}
\noindent
The reason analytic expansion produces an expansion of consistent type is because the subsumption rule, as previously discussed, allows us to check an external expression against any type consistent with the type the expression actually synthesizes, whereas every internal expression can be assigned at most one type, i.e. the following standard unicity property holds of the type assignment system: 
\begin{thm}[Type Assignment Unicity]
  If $\hasType{\Delta}{\hGamma}{\dexp}{\htau}$
  and $\hasType{\Delta}{\hGamma}{\dexp}{\htau'}$
  then $\htau=\htau'$.
\end{thm}
\noindent
Consequently, analytic expansion reports the type actually assigned to the expansion it produces. For example, we can derive that $\expandAna{\hGamma}{c}{\tehole}{c}{b}{\emptyset}$.% where $\emptyset$ is the empty hole context.

Before describing the rules in detail, let us state a few other useful theorems. The following theorem establishes that an expansion exists for every well-typed external expression.
 \begin{thm}[Expandability] \label{thm:expandability}~
  \begin{enumerate}[nolistsep]
    \item
      If $\hsyn{\hGamma}{\hexp}{\htau}$
      then $\expandSyn{\hGamma}{\hexp}{\htau}{\dexp}{\Delta}$
      for some $\dexp$ and $\Delta$.
    \item
      If $\hana{\hGamma}{\hexp}{\htau}$
      then $\expandAna{\hGamma}{\hexp}{\htau}{\dexp}{\htau'}{\Delta}$
      for some $\dexp$ and $\htau'$ and $\Delta$.
  \end{enumerate}
\end{thm}
\noindent
The following theorem establishes that when an expansion exists, it is unique.
\begin{thm}[Expansion Unicity] \label{thm:expansion-unicity}~
  \begin{enumerate}[nolistsep]
    \item
      If $\expandSyn{\hGamma}{\hexp}{\htau}{\dexp}{\Delta}$
      and $\expandSyn{\hGamma}{\hexp}{\htau'}{\dexp'}{\Delta'}$
      then $\htau=\htau'$ and $\dexp=\dexp'$ and $\Delta=\Delta'$.
    \item
      If $\expandAna{\hGamma}{\hexp}{\htau_1}{\dexp}{\htau_2}{\Delta}$
      and $\expandAna{\hGamma}{\hexp}{\htau_1}{\dexp'}{\htau_2'}{\Delta'}$
      then $\dexp=\dexp'$ and $\htau_2=\htau_2'$ and $\Delta=\Delta'$.
  \end{enumerate}
\end{thm}
\noindent
The following theorem establishes that expansion generalizes external typing.\todo{rename correspondence to generality}
\begin{thm}[Expansion Generality] \label{thm:expansion-generality}~
  \begin{enumerate}[nolistsep]
    \item
      If $\expandSyn{\hGamma}{\hexp}{\htau}{\dexp}{\Delta}$
      then $\hsyn{\hGamma}{\hexp}{\htau}$.
    \item
      If $\expandAna{\hGamma}{\hexp}{\htau}{\dexp}{\htau'}{\Delta}$
      then $\hana{\hGamma}{\hexp}{\htau}$.
  \end{enumerate}
\end{thm}

The rules governing expansion of constants, variables and lambda expressions --- Rules {ESConst}, {ESVar}, {ESLam} and {EALam} --- and the corresponding type assignment rules --- Rules {TAConst}, {TAVar} and {TALam} --- mirror the typing rules from Fig.~\ref{fig:bidirectional-typing} (so the corresponding cases of Theorem~\ref{thm:typed-expansion}, Theorem~\ref{thm:expandability} and Theorem~\ref{thm:expansion-generality} are straightforward). Note that in the internal language, all lambdas are half-annotated, again to support type assignment---Rule {EALam} inserts the annotation onto unannotated external lambdas based on the given type. The rules governing expansion of holes, function application and ascription are more interesting, so let us consider them in turn.
%
\rkc{this list (holes, func app, and ascription) has three items, but there are
only two subsections. see comment at beginning of Sec 3.2.2.}

\subsubsection{Hole Expansion}\label{sec:hole-expansion} Rules {ESEHole}, {ESNEHole}, {EAEHole} and {EANEHole} govern the expansion of empty and non-empty expression holes to \emph{hole closures}, $\dehole{u}{\sigma}{}$ and $\dhole{\dexp}{u}{\sigma}{}$, respectively. 
%
\rkc{perhaps add ``empty and non-empty'' before hole closures.}

The hole name, $u$, on a hole closure identifies the external hole that the hole closure corresponds to. Note that while we assume each hole name to be unique in the external language, there can be multiple hole closures with the same name during evaluation due to substitution (though in the initial expansion, the uniqueness condition happens to hold because evaluation has yet to occur). For example, the result from Fig.~\ref{fig:grades-example} showed four closures for the hole named 1\todo{update with actual number (four?) and hole name later}. There, we numbered each hole closure for a given hole sequentially, \li{1:1}, \li{1:2} and so on, but this is strictly for the sake of presentation so we omit hole closure numbers from the core calculus.

The hole expansion rules are the only rules that introduce hypotheses, of the form $\Dbinding{u}{\hGamma}{\htau}$, into the hole context, $\Delta$. The purpose of the hole context is to record a type, $\tau$, and a typing context, $\Gamma$,   for each hole name, $u$.\footnote{We use a hole context, rather than recording the typing context and type directly on each hole closure, to ensure that all closures for a hole name have the same typing context and type.} This notation for hole contexts is taken from contextual modal type theory (CMTT) \cite{Nanevski2008}, identifying hole names with metavariables and hole contexts with modal contexts (we say more about the connection with CMTT below). In all four hole expansion rules, the typing context recorded in the hole context is simply the current typing context when the hole is expanded. In the synthetic hole expansion rules, {ESEHole} and {ESNEHole}, the generated hole context assigns the hole type, $\tehole$, to $u$, as in the typing rules. However, the first two premises of the expansion subsumption rule, Rule EASubsume, disallow the use of subsumption for holes in analytic position. Instead, we have separate analytic rules, {EAEHole} and {EANEHole}, which record the type that the hole is being checked against into the hole context. This is again so that we can use type assignment for the internal language --- the type assignment rules TAEHole and TANEHole in Fig.~\ref{fig:hasType} assign a hole closure for $u$ the corresponding type from the hole context.

Each hole closure also has an associated environment, $\sigma$, which is a finite substitution of the form $[d_1/x_1, ~\cdots, d_n/x_n]$ for $n \geq 0$.  The purpose of the closure environment is to keep a record of the substitutions that occur around the hole as evaluation occurs. Initially, no evaluation has occurred, so the initial environment generated by the hole expansion rules is the identity substitution for the typing context associated with $u$ in $\Delta$, which we notate $\idof{\hGamma}$ and define as follows.
\begin{defn}[Identity Substitution] $\idof{x_1 : \tau_1, ~\cdots, x_n : \tau_n} = [x_1/x_1, ~\cdots, x_n/x_n]$
\end{defn}
\noindent
The type assignment rules for hole closures, Rules TAEHole and TANEHole, require that we be able to check the environment of each hole closure against the corresponding typing context, written $\hasType{\Delta}{\hGamma}{\sigma}{\hGamma'}$ and defined as follows:\todo{check that the definition in the Agda corresponds}
\begin{defn}[Substitution Typing]
$\hasType{\Delta}{\hGamma}{\sigma}{\hGamma'}$ iff $\domof{\sigma} = \domof{\hGamma'}$ and for each $x : \htau \in \hGamma'$ we have that $d/x \in \sigma$ and $\hasType{\Delta}{\hGamma}{d}{\tau}$.
\end{defn}
\noindent
It is easy to verify that the identity substitution satisfies this requirement, i.e. that $\hasType{\Delta}{\hGamma}{\idof{\hGamma}}{\hGamma}$. 

Empty hole closures, $\dehole{u}{\sigma}{}$,  correspond to the metavariable closures (a.k.a. deferred substitutions) from CMTT, $\cmttclo{u}{\sigma}$.\todo{change notation from CMTT}{} We will see how closure environments evolve during evaluation in Sec.~\ref{sec:evaluation}. Non-empty hole closures, $\dhole{d}{u}{\sigma}{}$, do not directly correspond to a notion from CMTT\todo{mention reason in Sec 4}.

\subsubsection{Cast Insertion}\label{sec:cast-insertion} 
%
\rkc{consider renaming subsection to ``Function Application and Cast
Intersion''. and adding a transition sentence like ``The remaining development
of expansion deals with the delicate interplay between function application and
cast insertion.''}
%
Consider the following example: $\hap{(\halam{x}{\tehole}{\hap{x}{c}})}{c}$. The type synthesized for this example viewed as an external expression is $\tehole$, because the hole type annotation on $x$ allows us to apply it as a function of type $\tarr{\tehole}{\tehole}$, as previously discussed, and $c$ can be checked against type $\tehole$ by subsumption. However, viewed as an internal expression, this example is not well-typed---we do not have subsumption in the type assignment system defined in Fig.~\ref{fig:hasType}. Indeed, it would violate type safety if we could assign a type to this example in the internal language, because beta reduction of this example viewed as an internal expression would result in $c(c)$, which is clearly not well-typed. The difficulty is that leaving the argument type unknown leaves how the argument is being used (in this case, as a function) also unknown.\footnote{In a system where type reconstruction is first used to try to fill in type holes, we could express a similar example by using $x$ at two or more different types, thereby causing type reconstruction to fail.
%On the other hand, if it is acceptable to arbitrarily choose one of the possible types, and type reconstruction is complete, then type holes will never appear in the internal language and the cast insertion machinery described in this section can be omitted entirely, leaving only the hole closure machinery described previously.
}\todo{talk about elsewhere? maybe do type-hole-free version of calculus in appendix if there is time}{} Recall that we can interpret hole types as unknown types from gradual type theory. This leads us to the solution to this problem: cast insertion. 
%
\rkc{Instead of the last two sentences, maybe something like: ``By analogy
between hole types in our system and unknown types from gradual type theory, we
address the problem with the same mechanism used in gradual type theory: cast
insertion.''}

The cast form in Hazelnut Live is $\dcasttwo{\dexp}{\htau_1}{\htau_2}$, which, as specified by Rule TACast in Fig.~\ref{fig:hasType}, serves to ``box'' an expression of type $\htau_1$ for treatment as an expression of some consistent type $\htau_2$.\footnote{In the earliest work on gradual type theory, the cast form only gave the target type, $\htau_2$ \cite{Siek06a}, but it simplifies matters to include the assigned type, $\htau_1$, in the syntax \cite{DBLP:conf/snapl/SiekVCB15}.}

Casts are inserted during the expansion of function applications and ascriptions. The latter is more straightforward: Rule~{ESAsc} in Fig.~\ref{fig:expandSyn} inserts a cast from the assigned type to the ascribed  type. Theorem~\ref{thm:typed-expansion} inductively ensures that the two types are consistent.  Note that we included ascription only for the sake of exposition: it can be defined using application together with the half-annotated identity function: $e : \tau = \hap{(\halam{x}{\htau}{x})}{e}$.
%
\rkc{move this Note sentence to the same place where half-annotated lambdas were
noted. This note breaks the flow here.}

Cast insertion by the application expansion rule, Rule~{ESAp}, has much the same intuition, but requires a bit more care. To understand the rule, consider the expansion for the example above:
\[\hap{\dcasttwo{(\halam{x}{\tehole}{\underline{\hap{\dcasttwo{x}{\tehole}{\tarr{\tehole}{\tehole}}}{\dcasttwo{c}{b}{\tehole}}}})}{\tarr{\tehole}{\tehole}}{\tarr{\tehole}{\tehole}}}{\dcasttwo{c}{b}{\tehole}}
\]
Let us focus on the expansion of the function body, underlined\todo{shaded?}, first. A cast is inserted on both the function expression, $x$, and the the argument, $c$. 

The cast on $x$ allows us to treat the variable $x$, which is of type $\tehole$, as being of the matched arrow type $\tarr{\tehole}{\tehole}$, as described in Sec.~\ref{sec:external-statics}. The first three premises of Rule~{ESAp} accomplish this by first synthesizing a type for the function expression, here $\tehole$, then determining the matched arrow type, $\tarr{\tehole}{\tehole}$, and then performing analytic expansion on the function expression with this matched arrow type. The resulting expansion will have some type $\tau_1'$ consistent with the matched arrow type. In this case, because $x$ is of variable form, analytic expansion goes through subsumption so $\tau_1'$ is simply $\tehole$. The conclusion of the rule inserts the corresponding cast. Note that we go through type synthesis and analytic expansion so that the hole context records the matched arrow type for holes in function position, rather than the type $\tehole$ for all such holes as would be the case in a variant of this rule using synthetic expansion for the function expression\todo{put this in the appendix?}{}.

The conclusion of Rule~{ESAp} also inserts a cast on the argument's expansion, from the type it is assigned by the final premise of the rule, here $b$ as described at the start of Sec.~\ref{sec:expansion}, to the argument type, $\tehole$, of the matched arrow type of the function expression, here $\tehole$.

Observe that together, these casts allow us assign a type to the function body according to the rules in Fig.~\ref{fig:hasType}, where we could not do so under the same context without casts.

The outer application in the example above goes through the same rule. In this case, the cast on the function is the identity cast for $\tarr{\tehole}{\tehole}$. We do not attempt to avoid the insertion of identity casts in the core calculus for simplicity (these will simply never fail during evaluation), but it is safe to do so (and some formal accounts of gradual typing do so by defining three application rules, including the original account of \cite{Siek06a}).


\subsection{Dynamic Semantics}
\label{sec:evaluation}

To recap, the result of expansion is a well-typed internal expression with hole closures and casts, where the former corresponds to metavariable closures from CMTT \cite{Nanevski2008}, and the latter corresponds to casts from gradual type theory \cite{Siek06a,DBLP:conf/snapl/SiekVCB15}. However, the dynamic semantics for \HazelnutLive does not simply ``fall out'' from these observations. 

The problem is first that \citet{Nanevski2008} defined only the logical reductions for CMTT, viewing it as a proof system for intuitionistic contextual modal logic via the propositions-as-types (Curry-Howard) principle. The paper therefore proved only a subject reduction property (which is closely related to type preservation). This is not a full dynamic semantics, and in particular, there is no notion of \emph{progress}, i.e. that well-typed terms cannot get ``stuck'' in an undefined state \cite{wright94:_type_soundness}. In any case, a conventional dynamic semantics for CMTT would not be immediately relevant to our goal of evaluating incomplete programs because, by our interpretation of hole closures, we would need a dynamic semantics for terms with free metavariables. \citet{Nanevski2008} sketched an interpretation of CMTT into the simply-typed lambda calculus with sums under permutation conversion\footnote{Permutation conversions are necessary to encode the commutative reductions of CMTT, which in turn are necessary to prove a strong normalization property. These issues are not relevant in \HazelnutLive because, as in the gradually typed lambda calculus, type holes admit non-termination: we can express the Y combinator as $(\halam{x}{\tehole}{x(x)}) (\halam{x}{\tehole}{x(x)})$.}, which has been studied by \citet{DBLP:journals/iandc/Groote02}, but under this interpretation an analagous problem arises---metavariables become variables of a function type, so again we cannot rely on any standard notion of progress on closed terms.\todo{citations}{}% We also cannot rely on, for example, weak head normalization because \HazelnutLive admits non-termination (due to casts).\todo{citation} 

Furthermore, we need to integrate casts into the dynamic semantics. Fortunately, the dynamic semantics for the cast calculus from the gradually typed lambda calculus provides most, but not all, of the necessary machinery. The first problem is again with progress: in the cast calculus, the only irreducible terms of hole type are casts, which are accounted for by the progress theorem, but in $\HazelnutLive$, holes induce additional irreducible terms. The second missing piece is that in prior work on casts, evaluation aborts when cast failure occurs. Our goal, as discussed in Sec.~\ref{sec:failed-cast-example}, is for cast failure to instead insert a membrane around the dynamic type error, much like a non-empty hole serves as a membrane around a static type error, allowing in both cases for evaluation to safely and meaningfully continue past the error when desired.

The dynamic semantics for \HazelnutLive specified in Figures~\ref{fig:isGround}-\ref{fig:contextual-dynamics} addresses the difficulties just outlined, resulting in a system capable of running incomplete programs without aborting when a hole is encountered. This semantics is equipped with a meaningful notion of type safety (involving both preservation and progress). We also establish that the standard notion of type safety falls out when running complete terms. % We also establish that the standard type safety theorem holds when running complete terms.

The dynamic semantics is specified as a ``small-step'' transition system. The cast-related machinery is based closely on the cast calculus from the ``refined'' account of the gradually typed lambda calculus by \citet{DBLP:conf/snapl/SiekVCB15}, which is known to be theoretically well-behaved. In particular, Fig.~\ref{fig:isGround} defines the judgement $\isGround{\htau}$, which distinguishes the base type, $b$, and the least specific arrow type, $\tarr{\tehole}{\tehole}$, as \emph{ground types}, which play a role in simplifying the treatment of function casts. Fig.~\ref{fig:isFinal} defines the judgement $\isFinal{d}$, which distinguishes the final forms of the transition system. There are two classes of final forms: (possibly-)boxed values and indeterminate forms.\footnote{In most accounts of the cast calculus, values and ground types are distinguished with separate grammars together with an implicit identification convention. Our judgemental formulation is more faithful to the mechanization and cleaner for our purposes, because we are distinguishing several classes of final forms.}

% !TEX root = hazelnut-dynamics.tex

\begin{figure}
\begin{subfigure}[t]{0.5\textwidth}
\judgbox{\isGround{\htau}}{$\htau$ is a ground type}
\begin{mathpar}
\inferrule[GBase]{ }{
  \isGround{b}
}

\inferrule[GHole]{ }{
  \isGround{\tarr{\tehole}{\tehole}}
}
\end{mathpar}
\end{subfigure}
\hfill
\begin{subfigure}[t]{0.46\textwidth}
\judgbox{\groundmatch{\htau}{\htau'}}{$\htau$ has matched ground type $\htau'$}
\begin{mathpar}
\inferrule[MGArr]{
  \tarr{\htau_1}{\htau_2}\neq\tarr{\tehole}{\tehole}
}{
  \groundmatch{\tarr{\htau_1}{\htau_2}}{\tarr{\tehole}{\tehole}}
}
\end{mathpar}
\end{subfigure}
\CaptionLabel{Ground Types}{fig:isGround}
\label{fig:groundmatch}
\end{figure}

% !TEX root = hazelnut-dynamics.tex
\begin{figure}[t]

\begin{tabular}[t]{cc}

\begin{minipage}{0.5\textwidth}
\judgbox{\isFinal{\dexp}}{$\dexp$ is final}
\begin{mathpar}
%% \inferrule[FVal]
%% {\isValue{\dexp}}{\isFinal{\dexp}}
\inferrule[FBoxedVal]
{\isBoxedValue{\dexp}}{\isFinal{\dexp}}
\and
\inferrule[FIndet]
{\isIndet{\dexp}}{\isFinal{\dexp}}
\end{mathpar}
\end{minipage}

& 

\begin{minipage}{0.5\textwidth}
    
\judgbox{\isValue{\dexp}}{$\dexp$ is a value}
\begin{mathpar}
\inferrule[VConst]{ }{
  \isValue{c}
}

\inferrule[VLam]{ }{
  \isValue{\halam{x}{\htau}{\dexp}}
}
\end{mathpar}
\end{minipage}

\end{tabular}

\vsepRule

\judgbox{\isBoxedValue{\dexp}}{$\dexp$ is a boxed value}
\begin{mathpar}
\inferrule[BVVal\rkc{name?}]{
  \isValue{\dexp}
}{
  \isBoxedValue{\dexp}
}

\inferrule[BVCastArr\rkc{name?}]{
  \tarr{\htau_1}{\htau_2} \neq \tarr{\htau_3}{\htau_4}\\
  \isBoxedValue{\dexp}
}{
  \isBoxedValue{\dcasttwo{\dexp}{\tarr{\htau_1}{\htau_2}}{\tarr{\htau_3}{\htau_4}}}
}

\inferrule[BVHoleCast\rkc{name?}]{
  \isBoxedValue{\dexp}\\
  \isGround{\htau}
}{
  \isBoxedValue{\dcasttwo{\dexp}{\htau}{\tehole}}
}
\end{mathpar}

\vsepRule

\judgbox{\isIndet{\dexp}}{$\dexp$ is indeterminate}
\begin{mathpar}
\inferrule[IEHole]
{ }
{\isIndet{\dehole{\mvar}{\subst}{}}}

\inferrule[INEHole]
{\isFinal{\dexp}}
{\isIndet{\dhole{\dexp}{\mvar}{\subst}{}}}

\inferrule[IAp]
{\dexp_1\neq
   \dcasttwo{\dexp_1'}
            {\tarr{\htau_1}{\htau_2}}
            {\tarr{\htau_3}{\htau_4}}\\
 \isIndet{\dexp_1}\\
% \isFinal{\dexp_2}~\text{\cy{??}}}
 \isFinal{\dexp_2}}
{\isIndet{\dap{\dexp_1}{\dexp_2}}}

\inferrule[ICastGroundHole] {
  \isIndet{\dexp}\\
  \isGround{\htau}
}{
  \isIndet{\dcasttwo{\dexp}{\htau}{\tehole}}
}

\inferrule[ICastHoleGround] {
  \dexp\neq\dcasttwo{\dexp'}{\htau'}{\tehole}\\
  \isIndet{\dexp}\\
  \isGround{\htau}
}{
  \isIndet{\dcasttwo{\dexp}{\tehole}{\htau}}
}

\inferrule[ICastArr]{
  \tarr{\htau_1}{\htau_2} \neq \tarr{\htau_3}{\htau_4}\\
  \isIndet{\dexp}
}{
  \isIndet{\dcasttwo{\dexp}{\tarr{\htau_1}{\htau_2}}{\tarr{\htau_3}{\htau_4}}}
}

\inferrule[IFailedCast] {
  \isFinal{\dexp}\\
  \isGround{\htau_1}\\
  \isGround{\htau_2}\\
  \htau_1\neq\htau_2
}{
  \isIndet{\dcastfail{\dexp}{\htau_1}{\htau_2}}
}

%% \inferrule[ICast]
%% {\isIndet{\dexp}}
%% {\isIndet{\dcast{\htau}{\dexp}}}

\end{mathpar}

%\vsepRule

\CaptionLabel{Final Forms}{fig:isFinal}
\label{fig:isValue}
\label{fig:isIndet}
\end{figure}


The judgement $\isBoxedValue{d}$ distinguishes (possibly-)boxed values, which correspond to the values from the cast calculus and include the classic values from the lambda calculus, distinguished by $\isValue{d}$, as well as two cast forms: casts between function types (except identity casts) and casts from a ground type to the hole type. In both cases, the cast must be on a boxed value. % (in the prior work, ground types and values were distinguished syntactically, but it is cleaner and more faithful to the mechanization to make these distinctions judgementally).

The judgement $\isIndet{d}$ distinguishes \emph{indeterminate} forms, so named because they arise from the presence of expression holes and failed casts in a program, and so, conceptually, their ultimate value awaits programmer action (see Sec.~\ref{sec:commutativity}). The first two rules specify that hole closures are indeterminate (in the case of non-empty hole closures, when the inner expression is final). The rules defining the remaining indeterminate forms are explained simultaneously with the corresponding transition rules below.\todo{relationship to weak head normal forms}

% !TEX root = hazelnut-dynamics.tex
\begin{comment}
\begin{figure}[t]

\begin{comment}
\vsepRule

\judgbox{\isevalctx{\evalctx}}{$\evalctx$ is an evaluation context}
\begin{mathpar}
\inferrule[ECDot]{ }{
  \isevalctx{\evalhole}
}

%% \inferrule[ECLam]{
%%   \isevalctx{\evalctx}
%% }{
%%   \isevalctx{\halam{x}{\htau}{\evalctx}}
%% }

\inferrule[ECAp1]{
  \isevalctx{\evalctx}
}{
  \isevalctx{\hap{\evalctx}{\dexp}}
}

\inferrule[ECAp2]{
  \maybePremise{\isFinal{\dexp}}\\
  \isevalctx{\evalctx}
}{
  \isevalctx{\hap{\dexp}{\evalctx}}
}

\inferrule[ECNEHole]{
  \isevalctx{\evalctx}
}{
  \isevalctx{\dhole{\evalctx}{\mvar}{\subst}{}}
}

\inferrule[ECCast]{
  \isevalctx{\evalctx}
}{
  \isevalctx{\dcasttwo{\evalctx}{\htau_1}{\htau_2}}
}

\inferrule[ECFailedCast]{
  \isevalctx{\evalctx}
}{
  \isevalctx{\dcastfail{\evalctx}{\htau_1}{\htau_2}}
}
\end{mathpar}
% \end{comment}
\vsepRule


\caption{Evaluation Contexts}
\label{fig:eval-contexts}
\end{figure}
\end{comment}

%% \vsepRule

\begin{figure}
\judgbox{\reducesE{}{\dexp}{\dexp'}}{$\dexp$ takes an instruction transition to $\dexp'$}
\begin{mathpar}
\inferrule[ITBeta]{
  \maybePremise{\isFinal{\dexp_2}}
}{
  \reducesE{}{\hap{(\halam{x}{\htau}{\dexp_1})}{\dexp_2}}{[\dexp_2/x]\dexp_1}
}

\inferrule[ITApCast]{
  \maybePremise{\isFinal{\dexp_1}}\\
  \maybePremise{\isFinal{\dexp_2}}\\
  \tarr{\htau_1}{\htau_2} \neq \tarr{\htau_1'}{\htau_2'}
}{
  \reducesE{}
    {\hap{\dcasttwo{\dexp_1}{\tarr{\htau_1}{\htau_2}}{\tarr{\htau_1'}{\htau_2'}}}{\dexp_2}}
    {\dcasttwo{(\hap{\dexp_1}{\dcasttwo{\dexp_2}{\htau_1'}{\htau_1}})}{\htau_2}{\htau_2'}}
}

\inferrule[ITCastId]{
  \maybePremise{\isFinal{\dexp}}
}{
  \reducesE{}{\dcasttwo{\dexp}{\htau}{\htau}}{\dexp}
}

\inferrule[ITCastSucceed]{
  \maybePremise{\isFinal{\dexp}}\\
  \isGround{\htau}
}{
  \reducesE{}{\dcastthree{\dexp}{\htau}{\tehole}{\htau}}{\dexp}
}

\inferrule[ITCastFail]{
  \maybePremise{\isFinal{\dexp}}\\
  \htau_1\neq\htau_2\\\\
  \isGround{\htau_1}\\
  \isGround{\htau_2}
}{
  \reducesE{}
    {\dcastthree{\dexp}{\htau_1}{\tehole}{\htau_2}}
    {\dcastfail{\dexp}{\htau_1}{\htau_2}}
}

\inferrule[ITGround]{
  \maybePremise{\isFinal{\dexp}}\\
  \groundmatch{\htau}{\htau'}
}{
  \reducesE{}
    {\dcasttwo{\dexp}{\htau}{\tehole}}
    {\dcastthree{\dexp}{\htau}{\htau'}{\tehole}}
}

\inferrule[ITExpand]{
  \maybePremise{\isFinal{\dexp}}\\
  \groundmatch{\htau}{\htau'}
}{
  \reducesE{}
    {\dcasttwo{\dexp}{\tehole}{\htau}}
    {\dcastthree{\dexp}{\tehole}{\htau'}{\htau}}
}

%% \inferrule[ITCast]{
%%   \isFinal{d}\\
%%   \hasType{\Delta}{\emptyset}{d}{\tau_2}\\
%%   \tconsistent{\tau_1}{\tau_2}
%% }{
%%   \reducesE{\Delta}{\dcast{\htau_1}{d}}{d}
%% }
%% 
%% \inferrule[ITEHole]{ }{
%%   \reducesE{\Delta}{\dehole{\mvar}{\subst}{\unevaled}}{\dehole{\mvar}{\subst}{\evaled}}
%% }
%% 
%% \inferrule[ITNEHole]{
%%   \isFinal{d}
%% }{
%%   \reducesE{\Delta}{\dhole{d}{\mvar}{\subst}{\unevaled}}{\dhole{d}{\mvar}{\subst}{\evaled}}
%% }
\end{mathpar}
\CaptionLabel{Instruction Transitions}{fig:instruction-transitions}
\end{figure}

\begin{figure}
$\arraycolsep=4pt\begin{array}{rllllll}
\mathsf{EvalCtx} & \evalctx & ::= &
  \evalhole ~\vert~
  \hap{\evalctx}{\dexp} ~\vert~
  \hap{\dexp}{\evalctx} ~\vert~
  \dhole{\evalctx}{\mvar}{\subst}{} ~\vert~
  \dcasttwo{\evalctx}{\htau}{\htau} ~\vert~
  \dcastfail{\evalctx}{\htau}{\htau}
\end{array}$

\vsepRule

\judgbox{\selectEvalCtx{\dexp}{\evalctx}{\dexp'}}{$\dexp$ is obtained by placing $\dexp'$ at the mark in $\evalctx$}
\begin{mathpar}
\inferrule[FHOuter]{ }{
  \selectEvalCtx{\dexp}{\evalhole}{\dexp}
}

%% \inferrule[FLam]{
%%   \selectEvalCtx{d}{\evalctx}{d'}
%% }{
%%   \selectEvalCtx{\halam{x}{\htau}{d}}{\halam{x}{\htau}{\evalctx}}{d'}
%% }

\inferrule[FHAp1]{
  \selectEvalCtx{\dexp_1}{\evalctx}{\dexp_1'}
}{
  \selectEvalCtx{\hap{\dexp_1}{\dexp_2}}{\hap{\evalctx}{\dexp_2}}{\dexp_1'}
}

\inferrule[FHAp2]{
  \maybePremise{\isFinal{\dexp_1}}\\
  \selectEvalCtx{\dexp_2}{\evalctx}{\dexp_2'}
}{
  \selectEvalCtx{\hap{\dexp_1}{\dexp_2}}{\hap{\dexp_1}{\evalctx}}{\dexp_2'}
}

%% \inferrule[FHEHole]{ }{
%%   \selectEvalCtx{\dehole{\mvar}{\subst}{}}{\evalhole}{\dehole{\mvar}{\subst}{}}
%% }
%% 
%% \inferrule[FHNEHoleEvaled]{ }{
%%   \selectEvalCtx{\dhole{d}{\mvar}{\subst}{\evaled}}{\evalhole}{\dhole{d}{\mvar}{\subst}{\evaled}}
%% }

\inferrule[FHNEHoleInside]{
  \selectEvalCtx{\dexp}{\evalctx}{\dexp'}
}{
  \selectEvalCtx{\dhole{\dexp}{\mvar}{\subst}{}}{\dhole{\evalctx}{\mvar}{\subst}{}}{\dexp'}
}

%% \inferrule[FHNEHoleFinal]{
%%   \isFinal{d}
%% }{
%%   \selectEvalCtx{\dhole{d}{\mvar}{\subst}{\unevaled}}{\evalhole}{\dhole{d}{\mvar}{\subst}{\unevaled}}
%% }

\inferrule[FHCastInside]{
  \selectEvalCtx{\dexp}{\evalctx}{\dexp'}
}{
  \selectEvalCtx{\dcasttwo{\dexp}{\htau_1}{\htau_2}}
                {\dcasttwo{\evalctx}{\htau_1}{\htau_2}}
                {\dexp'}
}

\inferrule[FHFailedCast]{
  \selectEvalCtx{\dexp}{\evalctx}{\dexp'}
}{
  \selectEvalCtx{\dcastfail{\dexp}{\htau_1}{\htau_2}}
                {\dcastfail{\evalctx}{\htau_1}{\htau_2}}
                {\dexp'}
}

%% \inferrule[FHCastFinal]{
%%   \isFinal{d}
%% }{
%%   \selectEvalCtx{\dcast{\htau}{d}}{\evalhole}{\dcast{\htau}{d}}
%% }
\end{mathpar}

\vsepRule

\judgbox{\stepsToD{}{\dexp}{\dexp'}}{$\dexp$ steps to $\dexp'$}
\vspace{-10px}
\begin{mathpar}
\inferrule[Step]{
  \selectEvalCtx{d}{\evalctx}{\dexp_0}\\
  \reducesE{}{\dexp_0}{\dexp_0'}\\
  \selectEvalCtx{\dexp'}{\evalctx}{\dexp_0'}
}{
  \stepsToD{}{\dexp}{\dexp'}
}
\end{mathpar}
\CaptionLabel{Evaluation Contexts and Steps}{fig:step}
\end{figure}

The transition rules are defined in Fig.~\ref{fig:instruction-transitions}-\ref{fig:step}. Top-level transitions, i.e. \emph{steps}, $\stepsToD{}{d}{d'}$, are governed by a single rule, Rule {Step} in Fig.~\ref{fig:step}, which (1) decomposes $d$ into an evaluation context, $\evalctx$, and a selected sub-term, $d_0$; (2) takes an \emph{instruction transition}, $\reducesE{}{d_0}{d_0'}$, as specified in Fig.~\ref{fig:instruction-transitions}; and (3) places $d_0'$ in the selected position, which is marked\footnote{In the literature, the form $\evalhole$ in the grammar of evaluation contexts is referred to as the \emph{hole}, but this hole is a technical device entirely orthogonal to the holes of this paper, so we use the term ``mark'' instead.} in the evaluation context by $\evalhole$, to obtain $d'$. This approach was originally developed in the reduction semantics of \citet{DBLP:journals/tcs/FelleisenH92} and is the predominant style of operational semantics in the literature on gradual typing. Because we distinguish final forms judgementally, rather than syntactically, we use a judgemental formulation of this approach called a \emph{contextual dynamics} by \citet{pfpl}. It would be straightforward to derive an equivalent structural operational semantics \cite{DBLP:journals/jlp/Plotkin04a} by using search rules instead of evaluation contexts (\citet{pfpl} compares the two approaches).\todo{Put the search rules in the appendix?}{}



%% \begin{figure}[t]
\judgbox{\stepsToD{\Delta}{\dexp_1}{\dexp_2}}{$\dexp_1$
steps to $\dexp_2$}
\begin{mathpar}
\inferrule[STEHoleEvaled]
{ }
{\stepsToD{\Delta}{\dehole{\mvar}{\subst}{\unevaled}}{\dehole{\mvar}{\subst}{\evaled}}}

\inferrule[STNEHoleStep]
{\stepsToD{\Delta}{\dexp_1}{\dexp_2} }
{\stepsToD{\Delta}{\dhole{\dexp_1}{\mvar}{\subst}{\unevaled}}{\dhole{\dexp_2}{\mvar}{\subst}{\unevaled}}}

\inferrule[STNEHoleEvaled]
{\isFinal{\dexp}}
{\stepsToD{\Delta}{\dhole{\dexp}{\mvar}{\subst}{\unevaled}}{\dhole{\dexp}{\mvar}{\subst}{\evaled}}}

\inferrule[STCast]
{
\isValue{\dexp}\\
\hasType{\Delta}{\emptyset}{\dexp}{\htau_2} \\
\tconsistent{\tau_1}{\tau_2}}
{\stepsToD{\Delta}{\dcast{\htau_1}{\dexp}}{\dexp}}

\inferrule[STApStep1]
{\stepsToD{\Delta}{\dexp_1}{\dexp_1'}}
{\stepsToD{\Delta}{\dap{\dexp_1}{\dexp_2}}{\dap{\dexp_1'}{\dexp_2}}}

\inferrule[STApStep2]
{ \isFinal{\dexp_1} \\ \stepsToD{\Delta}{\dexp_2}{\dexp_2'}}
{\stepsToD{\Delta}{\dap{\dexp_1}{\dexp_2}}{\dap{\dexp_1}{\dexp_2'}}}

\inferrule[STApBeta]
{ \isFinal{\dexp_2} }
{\stepsToD{\Delta}{\dapP{\dlam{x}{\htau}{\dexp_1}}{\dexp_2}}{ [\dexp_2/x]\dexp_1 }}
\end{mathpar}
\caption{Structural Dynamics}
\label{fig:stepsTo}
\end{figure}


\subsubsection{Application and Substitution} Rule {ITBeta} in Fig.~\ref{fig:instruction-transitions} specifies the standard beta reduction transition. The bracketed premises of the form $\maybePremise{\isFinal{\dexp}}$ in Fig.~\ref{fig:instruction-transitions}-\ref{fig:step} can be included to specify an eager, left-to-right evaluation strategy, or excluded to leave the evaluation order unspecified. We do the latter in our metatheory, both for the sake of generality and for reasons we return to in Sec.~\ref{sec:commutativity}.

Substitution, $[d/x]d'$, operates in the standard capture-avoiding manner \cite{pfpl}. The only cases of special interest are when substitution reaches a hole closure:
\[
\begin{array}{rcl}
  [d/x]\dehole{u}{\sigma}{} & = & \dehole{u}{[d/x]\sigma}{} \\%
  \substitute{d}{x}{\dhole{d'}{u}{\sigma}{}} & = & \dhole{[d/x]d'}{u}{[d/x]\sigma}{}
\end{array}
\]
In both cases, we write $[d/x]\sigma$ to perform substitution on each expression in the hole environment, $\sigma$. For example, $\stepsToD{}{\hap{(\halam{x}{b}{\halam{y}{b}{\dehole{u}{[x/x, y/y]}{}}})}{c}}{\halam{y}{b}{\dehole{u}{[c/x, y/y]}{}}}$. Hole closures can be duplicated like any other term by substitution, and the environments of different closures for the same hole name can differ, e.g. because a function with a hole closure in its body is applied multiple times.\todo{show example here or refer back to sec 2?}. Hole closures can also appear within the environments of other hole closures (this gives rise to the closure paths described in Sec.~\ref{sec:paths}).% If we uniquely number each hole closure in an expression, then each unique hole closure can be identified by a path consisting of a hole name, an instance number and a variable.

The ITBeta rule is not the only rule we need to handle function application, because lambdas are not the only final forms of arrow type. There are two other classes of possibilities. 

The expression in function position might be a cast between arrow types, in which case we apply the arrow cast conversion rule, Rule {ITApCast}, to rewrite the application form, obtaining an equivalent application where the expression under the function cast, $d_1$, is exposed. We know from inverting the typing rules that $d_1$ is of type $\tarr{\htau_1}{\htau_2}$ and that $d_2$ is of type $\htau_1'$ where $\tconsistent{\htau_1}{\htau_1'}$, so to maintain type safety, we must place a cast on $d_2$ from  $\htau_1'$ to $\htau_1$. The result of this application will be of type $\htau_2$, but the original cast promised that the result would be of consistent type $\htau_2'$, so we also need a cast on the result from $\htau_2$ to $\htau_2'$.

The last possibility is that the expression in function position is indeterminate and arrow cast conversion is not applicable, e.g. $\hap{(\dehole{u}{\sigma}{})}{c}$. In this case, no further progress can be made, so the function application as a whole is also deemed indeterminate by Rule {IAp} in Fig.~\ref{fig:isFinal}. Note that we are careful to maintain the following property, which establishes that expressions that step are not final, and that final expressions do not step.\todo{prove this as stated from Ian's disjointness proofs}\todo{do we need a typing premise?}
\begin{thm}[Finality] There does not exist $d$ such that both $\isFinal{d}$ and $\stepsToD{}{d}{d'}$ for some $d'$.
\end{thm}

\subsubsection{Casts}
The remaining instruction transition rules handle casts directly. Rule {ITCastId} drops identity casts.  As discussed in Sec.~\ref{sec:cast-insertion}, the structure of a term cast to hole type is statically obscure, so we need to wait until it is actually used at some other type, via a cast away from hole type, to be able to determine whether there is a problem. Rules {ITCastSucceed} and {ITCastFail} handle this situation when the two types involved are ground types (Fig.~\ref{fig:isGround}). If the two ground types are equal, then the cast succeeds and the cast forms can be dropped. If they are not equal, then the cast fails and the failed cast form, $\dcastfail{\dexp}{\htau_1}{\htau_2}$, arises. Rule {TAFailedCast} specifies as expected from Rule {ITCastFail} that a failed cast is well-typed exactly when $d$ has ground type $\tau_1$ and $\tau_2$ is a ground type disequal to $\tau_1$. Rule {IFailedCast} specifies that a failed cast operates as an indeterminate form once $d$ is final. This allows evaluation to proceed after cast failure, as with holes.

The two rules just described operate at ground type. The two remaining instruction transition rules, Rule {ITGround} and {ITExpand}, operate as technical devices that allow us to restrict our interest to ground types exclusively by inserting intermediate casts from (resp. to) non-ground types to (resp. from) consistent ground types. In the core calculus, the only non-ground types are the arrow types, so the grounding judgement, $\groundmatch{\tau_1}{\htau_2}$, defined in Fig.~\ref{fig:groundmatch}, produces the ground arrow type, $\tarr{\tehole}{\tehole}$. More generally, this judgement is governed by the following invariant:
\begin{lem}[Grounding] 
  If $\groundmatch{\htau}{\htau'}$
  then $\isGround{\htau'}$
  and $\tconsistent{\htau}{\htau'}$
  and $\htau\neq\htau'$.
\end{lem}

\subsubsection{Type Safety} afdasfdas dfasdfs
\begin{thm}[Preservation]
  If $\hasType{\Delta}{\emptyset}{\dexp}{\htau}$ and
  $\stepsToD{\Delta}{\dexp}{\dexp'}$ then
  $\hasType{\Delta}{\emptyset}{\dexp'}{\htau}$.
\end{thm}

\begin{thm}[Progress]
  If $\hasType{\Delta}{\emptyset}{\dexp}{\htau}$ then either
  (i) $\stepsToD{}{\dexp}{\dexp'}$ or
  (ii) $\isIndet{\dexp}$ or
  (iii) $\isBoxedValue{\dexp}$.
\end{thm}

\begin{thm}[Canonical Boxed Forms]
  If $\hasType{\hDelta}{\emptyset}{\dexp}{\htau}$ and $\isBoxedValue{\dexp}$
  then
  \begin{enumerate}[label=(\alph*)]
    \item If $\htau=b$ then $\dexp=c$.
    \item If $\htau=\tarr{\htau_1}{\htau_2}$ then either
      \begin{enumerate}
        \item[i.]
          $\dexp=\halam{x}{\htau_1}{\dexp'}$
          where $\hasType{\hDelta}{x : \htau_1}{\dexp'}{\htau_2}$, or
        \item[ii.]
          $\dexp=\dcasttwo{\dexp'}{\tarr{\htau_1'}{\htau_2'}}{\tarr{\htau_1}{\htau_2}}$
          where $\tarr{\htau_1'}{\htau_2'}\neq\tarr{\htau_1}{\htau_2}$
          and $\hasType{\hDelta}{\emptyset}{\dexp'}{\tarr{\htau_1'}{\htau_2'}}$.
      \end{enumerate}
    \item If $\htau=\tehole$
          then $\dexp=\dcasttwo{\dexp'}{\htau'}{\tehole}$
          where $\isGround{\htau'}$
          and $\hasType{\hDelta}{\emptyset}{\dexp'}{\htau'}$.
  \end{enumerate}
\end{thm}

\begin{thm}[Canonical Indeterminate Forms]
  If $\hasType{\hDelta}{\emptyset}{\dexp}{\htau}$
  and $\isIndet{\dexp}$
  then either
  \begin{enumerate}[label=(\alph*)]
    \item
      $\dexp=\dehole{u}{\subst}{}$
      and $\Dbinding{u}{\Gamma'}{\htau}\in\hDelta$, or
    \item
      $\dexp=\dhole{\dexp'}{u}{\subst}{}$
      and $\isFinal{\dexp'}$
      and $\hasType{\hDelta}{\emptyset}{\dexp'}{\htau'}$
      and $\Dbinding{u}{\Gamma'}{\htau}\in\hDelta$, or
    \item
      $\dexp=\dap{\dexp_1}{\dexp_2}$
      and $\hasType{\hDelta}{\emptyset}{\dexp_1}{\tarr{\htau_2}{\htau}}$
      and $\hasType{\hDelta}{\emptyset}{\dexp_2}{\htau_2}$
      and $\isIndet{\dexp_1}$
      and $\isFinal{\dexp_2}$
      and $\dexp_1\neq\dcasttwo{\dexp_1}{\tarr{\htau_3}{\htau_4}}
                                        {\tarr{\htau_3'}{\htau_4'}}$, or
    %% \item
    %%   \begin{enumerate}
    %%     \item blah
    %%     \item blah
    %%     \item blah
    %%   \end{enumerate}
    \item
      $\htau=b$
      and $\dexp=\dcasttwo{\dexp'}{\tehole}{b}$
      and $\isIndet{\dexp'}$
      and $\dexp'\neq\dcasttwo{\dexp''}{\htau'}{\tehole}$, or
    \item
      $\htau=b$
      and $\dexp=\dcastfail{\dexp'}{\htau'}{b}$
      and $\isGround{\htau'}$
      and $\htau'\neq{b}$
      and $\hasType{\hDelta}{\emptyset}{\dexp'}{\htau'}$, or
    \item
      $\htau=\tarr{\htau_{11}}{\htau_{12}}$
      and $\dexp=\dcasttwo{\dexp'}{\tarr{\htau_1}{\htau_2}}
                                  {\tarr{\htau_{11}}{\htau_{12}}}$
      and $\isIndet{\dexp'}$
      and $\tarr{\htau_1}{\htau_2}\neq\tarr{\htau_{11}}{\htau_{12}}$, or
    \item
      $\htau=\tarr{\tehole}{\tehole}$
      %% $\htau=\tarr{\htau_{11}}{\htau_{12}}$
      %% and $\htau_{11}=\tehole$
      %% and $\htau_{12}=\tehole$
      and $\dexp=\dcastthree{\dexp'}{\tehole}{\tehole}{\tehole}$
      and $\isIndet{\dexp'}$
      and $\dexp'\neq\dcasttwo{\dexp''}{\htau'}{\tehole}$, or
    \item
      $\htau=\tarr{\tehole}{\tehole}$
      %% $\htau=\tarr{\htau_{11}}{\htau_{12}}$
      %% and $\htau_{11}=\tehole$
      %% and $\htau_{12}=\tehole$
      and $\dexp=\dcastfail{\dexp'}{\htau'}{\tarr{\tehole}{\tehole}}$
      %% and $\dexp=\dcastfail{\dexp'}{\htau'}{\tarr{\htau_{11}}{\htau_{12}}}$
      and $\htau'\neq\htau$
      and $\isGround{\htau'}$
      and $\isIndet{\dexp'}$
      and $\hasType{\hDelta}{\emptyset}{\dexp'}{\htau'}$, or
    \item
      $\htau=\tehole$
      and $\dexp=\dcasttwo{\dexp'}{\htau'}{\tehole}$
      and $\isGround{\htau'}$
      and $\isIndet{\dexp'}$.
  \end{enumerate}
\end{thm}

\subsubsection{Complete Terms}\label{sec:complete-terms}

\autoref{fig:complete} defines \emph{complete} types, expressions, and
dynamic expressions to be those terms without type or expression holes.
\begin{theorem}[Canonical Value Forms]
  If $\hasType{\hDelta}{\emptyset}{\dexp}{\htau}$ and $\isValue{\dexp}$
  then $\htau\neq\tehole$ and
  \begin{enumerate}[label=(\alph*)]
    \item If $\htau=b$ then $\dexp=c$.
    \item If $\htau=\tarr{\htau_1}{\htau_2}$
          then $\dexp=\halam{x}{\htau_1}{\dexp'}$
          where $\hasType{\hDelta}{x : \htau_1}{\dexp'}{\htau_2}$.
  \end{enumerate}
\end{theorem}

\begin{theorem}[Complete Progress]
  If $\hasType{\hDelta}{\emptyset}{\dexp}{\htau}$
  and $\isComplete{\dexp}$
  then either $\stepsToD{}{\dexp}{\dexp'}$
  or $\isValue{\dexp}$.
\end{theorem}

\begin{theorem}[Complete Preservation]
  If $\hasType{\hDelta}{\emptyset}{\dexp}{\htau}$
  and $\isComplete{\dexp}$
  and $\stepsToD{}{\dexp}{\dexp'}$
  then $\hasType{\hDelta}{\emptyset}{\dexp'}{\htau}$
  and $\isComplete{\dexp'}$.
\end{theorem}

\begin{theorem}[Complete Expansion] ~
  \begin{enumerate}
    \item
      If $\isComplete{\hexp}$
      and $\expandSyn{\hGamma}{\hexp}{\htau}{\dexp}{\Delta}$
      then $\isComplete{\dexp}$.
    \item
      If $\isComplete{\hexp}$
      and $\expandAna{\hGamma}{\hexp}{\htau}{\dexp}{\htau'}{\Delta}$
      then $\isComplete{\dexp}$.
  \end{enumerate}
\end{theorem}




\cy{definition of complete types and terms? do we care about completeness
  of $\hexp$s or $\dexp$s?}


%% \begin{figure}[!ht]
%%   \begin{definition}
%%     $\hasType{\Delta}{\hGamma}{\sigma}{\hGamma'}$ iff for each $\dexp/x \in \sigma$, we have $x : \htau \in \hGamma'$ and $\hasType{\Delta}{\hGamma}{\dexp}{\htau}$.
%%   \end{definition}
%%   \caption{substitution type assignment}
%%   \label{fig:subassign}
%% \end{figure}


%% \begin{figure}[!ht]
%%   \caption{substitution type assignment}
%% \end{figure}


\subsection{Agda Mechanization}
\label{sec:agda-mechanization}



%%   \halam{x}{\htau}{\evalctx} ~\vert~

{\color{gray}\blindtext}

%%%%%%%%%%%%%%%%%%%%%%%%%%%%%%%%%%%%%%%%%%%%%%%%%%%%%%%%%%%%%%%%%%%%%%%%%%%

\section{Metatheory: Correctness of Edit-and-Resume}
\label{sec:resumption}

\begin{figure}[t]
\judgbox
  {[\dexp_1 / x] \dexp_2 = \dexp_3}
  {$\dexp_3$ is the result of substituting $\dexp_1$ for $x$ in $\dexp_2$}
\[
\begin{array}{lcll}
[\dexp_1 / x] c
&=&
c\\
{[\dexp_1 / x] x}
&=&
\dexp_1\\
{[\dexp_1 / x] y}
&=&
y & (y \neq x)\\
{[\dexp_1 / x] \dlam{y}{\htau}{\dexp_2}}
&=&
{\dlam{y}{\htau}{[\dexp_1 / x] \dexp_2}}
& (y \neq x)
\\
{[\dexp_1 / x] \dap{d_2}{d_3}}
&=&
{\dapP{[\dexp_1 / x] d_2}{[\dexp_1 / x] d_3}}
\\
% {[\dexp_1 / x] \dinj{i}{e_2}}
% &=&
% {\dinj{i}{[\dexp_1 / x] e_2}}
% \\
% {[\dexp_1 / x] \dcase{e}{x}{e_x}{y}{e_y}}
% &=&
% {\dcase{[\dexp_1 / x]e}{x}{[\dexp_1 / x]e_x}{y}{[\dexp_1 / x]e_y}}
% \\
{[\dexp_1 / x]} \dehole{\mvar}{\subst}{}
&=&
\dehole{\mvar}{[\dexp_1 / x] \subst}{}
\\
{[\dexp_1 / x] \dhole{\dexp_2}{\mvar}{\subst}{}}
&=&
\dhole{ [ \dexp_1 / x ] \dexp_2 }{\mvar}{[\dexp_1 / x] \subst}{}
\\
{[\dexp_1 / x] \dcast{\htau}{\dexp}}
&=&
\dcast{\htau}{[\dexp_1 / x] \dexp}
\end{array}
\]
\caption{Substitution}
\end{figure}

\begin{figure}[t]
\judgbox{\multiStepsTo{\dexp}{\dexp'}}{$\dexp$ multi-steps to $\dexp'$}
\begin{mathpar}
\inferrule[MultiStepRefl]{~}{
  \multiStepsTo{\dexp}{\dexp}
}

\inferrule[MultiStepSteps]{
  \stepsToD{}{\dexp}{\dexp'}\\
  \multiStepsTo{\dexp'}{\dexp''}
}{
  \multiStepsTo{\dexp}{\dexp''}
}
\end{mathpar}
\CaptionLabel{Multi-Step Dynamics}{fig:multi-step}
\end{figure}


{\color{gray}\blindtext}

\rkc{some of these statements may need additional typing premises}

\begin{theorem}[Instantiation]
  If $\hasType{\hDelta}{\hGamma}{\dexp}{\tau}$
  and $\Dbinding{u}{\hGamma'}{\htau'}\in\hDelta$
  and $\hasType{\hDelta}{\hGamma'}{\dexp'}{\htau'}$
  then $\hasType{\hDelta}{\hGamma}{\instantiate{\dexp'}{u}{\dexp}}{\tau}$.
\end{theorem}

\autoref{fig:multi-step} defines the multi-step relation as the reflexive,
transitive closure of the single-step relation.

\begin{lemma}[Finality]
  If $\isFinal{\dexp}$ and $\multiStepsTo{\dexp}{\dexp'}$
  then $\dexp=\dexp'$.
\end{lemma}

\begin{theorem}[Commutativity]
  If $\multiStepsTo{\dexp_0}{\dexp_1}$
  then $\multiStepsTo{\instantiate{\dexp}{u}{\dexp_0}}
                     {\instantiate{\dexp}{u}{\dexp_1}}$.
\end{theorem}

\begin{theorem}[Confluence]
  If $\multiStepsTo{\dexp}{\dexp_1}$
  and $\multiStepsTo{\dexp}{\dexp_2}$
  then there exists $\dexp'$ such that
  $\multiStepsTo{\dexp_1}{\dexp'}$ and
  $\multiStepsTo{\dexp_2}{\dexp'}$.
\end{theorem}

\begin{corollary}[Final Confluence]
  If $\multiStepsTo{\dexp}{\dexp_1}$
  and $\isFinal{\dexp_1}$
  and $\multiStepsTo{\dexp}{\dexp_2}$
  then $\multiStepsTo{\dexp_2}{\dexp_1}$.
  \begin{proof}
    By Confluence and Finality.
  \end{proof}
\end{corollary}

\begin{theorem}[Resumption]
  If $\multiStepsTo{\dexp_1}{\dexp_2}$
  and $\isFinal{\dexp_2}$
  and $\multiStepsTo{\instantiate{\dexp_3}{u}{\dexp_1}}{\dexp_4}$
  and $\isFinal{\dexp_4}$
  then $\multiStepsTo{\instantiate{\dexp_3}{u}{\dexp_2}}{\dexp_4}$.
  \begin{proof}
    By Commutativity,
    $\multiStepsTo{\instantiate{\dexp_3}{u}{\dexp_1}}
                  {\instantiate{\dexp_3}{u}{\dexp_2}}$.
    By Final Confluence, we can conclude.
  \end{proof}
\end{theorem}

Talk about weak vs strong conversion vs sharing -- see \url{http://gallium.inria.fr/~maranget/papers/jwk60.pdf} and cited work in the intro. See also \url{https://www.lri.fr/~blsk/Docs/Balabonski-FullLaziness-POPL12.pdf}?? Original Church-Rosser paper -- \url{http://www.ams.org/journals/tran/1936-039-03/S0002-9947-1936-1501858-0/S0002-9947-1936-1501858-0.pdf}.


\begin{comment}

\begin{theorem}[Maximum Informativity]
If the expansion produces $t1$, and there exists another possible type choice
$t2$, then $t1 \sim t2$ and $t1 JOIN t2 = t1$
\end{theorem}\footnote{idea is that special casing the holes in EANEHole gives you ``the
most descriptive hole types'' for some sense of what that means -- they'd
all just be hole other wise. from Matt:
\begin{quote}
It sounds like we need a something akin to an abstract domain (a lattice),
where hole has the least information, and a fully-defined type (without
holes) has the most information.  You can imagine that this lattice really
expands the existing definition we have of type consistency, which is
merely the predicate that says whether two types are comparable
(“join-able”) in this lattice.  lattice join is the operation that goes
through the structure of two (consistent) types, and chooses the structure
that is more defined (i.e., non-hole, if given the choice between hole and
non-hole).

The rule choosenonhole below is the expansion of this consistency rule that
we already have (hole consistent with everything)
\end{quote}}

\begin{verbatim}
t not hole
-------------------- :: choose-non-hole
hole JOIN t  = t
\end{verbatim}
\begin{verbatim}
------------ :: hole-consistent-with-everything
hole ~ t
\end{verbatim}

\end{comment}
