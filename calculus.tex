% !TEX root = hazelnut-dynamics.tex

\clearpage
\newcommand{\calculusSec}{Hazelnut Live, Formally}
\section{\protect\calculusSec}
\label{sec:calculus}
\todo{exposition}

\begin{figure}[!ht]
$\arraycolsep=4pt\begin{array}{rllllll}
\mathsf{HTyp} & \htau & ::= &
  b ~\vert~
  \tarr{\htau}{\htau} ~\vert~
  % \tprod{\htau}{\htau} ~\vert~
  % \tsum{\htau}{\htau} ~\vert~
  \tehole\\
\mathsf{HExp} & \hexp & ::= &
  c ~\vert~
  x ~\vert~
  \halam{x}{\htau}{\hexp} ~\vert~
  \hap{\hexp}{\hexp} ~\vert~
  % \hpair{\hexp}{\hexp} ~\vert~
  % \hprj{i}{\hexp} ~\vert~
  % \hinj{i}{\hexp} ~\vert~
  % \hcase{\hexp}{x}{\hexp}{x}{\hexp} ~\vert~
  % \hadd{\hexp}{\hexp} ~\vert~
  \hehole{u} ~\vert~
  \hhole{\hexp}{u} ~\vert~
  {\hlam{x}{\hexp}} ~\vert~
  \hexp : \htau\\
\mathsf{Mark} & \markname{} & ::= &
  \evaled{} ~\vert~  \unevaled{}\\
 \mathsf{DHExp} & \dexp  & ::= &
  c ~\vert~
  x ~\vert~
  {\halam{x}{\htau}{\dexp}} ~\vert~
  \hap{\dexp}{\dexp} ~\vert~
  % \hpair{\dexp}{\dexp} ~\vert~
  % \hprj{i}{\dexp} ~\vert~
  % \hinj{i}{\dexp} ~\vert~
  % \hcase{\dexp}{x}{\dexp}{x}{\dexp} ~\vert~
  % \hadd{\dexp}{\dexp} ~\vert~
  \dehole{\mvar}{\subst}{\markname} ~\vert~
  \dhole{\dexp}{\mvar}{\subst}{\markname} ~\vert~
  \dcast{\htau}{\dexp}\\
\end{array}$
\caption{Syntax of H-types, H-expressions and dynamic H-expressions. We write $x$ to range over variables, $u$ over hole names and $\sigma$ over finite substitutions.}
\label{fig:HTyp}
\label{fig:HExp}
\end{figure}

\begin{figure}[!ht]
\judgbox{\tconsistent{\htau_1}{\htau_2}}{$\htau_1$ is consistent with $\htau_2$}
\begin{mathpar}
\inferrule[TCHole1]{ }{
  \tconsistent{\tehole}{\htau}
}

\inferrule[TCHole2]{ }{
  \tconsistent{\htau}{\tehole}
}

\inferrule[TCRefl]{ }{
  \tconsistent{\htau}{\htau}
}

\inferrule[TCArr]{
  \tconsistent{\htau_1}{\htau_1'}\\
  \tconsistent{\htau_2}{\htau_2'}
}{
  \tconsistent{\tarr{\htau_1}{\htau_2}}{\tarr{\htau_1'}{\htau_2'}}
}
%
% \inferrule{
%   \tconsistent{\htau_1}{\htau_1'}\\
%   \tconsistent{\htau_2}{\htau_2'}
% }{
%   \tconsistent{\tprod{\htau_1}{\htau_2}}{\tprod{\htau_1'}{\htau_2'}}
% }
%
% \inferrule{
%   \tconsistent{\htau_1}{\htau_1'}\\
%   \tconsistent{\htau_2}{\htau_2'}
% }{
%   \tconsistent{\tsum{\htau_1}{\htau_2}}{\tsum{\htau_1'}{\htau_2'}}
% }
\end{mathpar}


\judgbox{\tinconsistent{\htau_1}{\htau_2}}{$\htau_1$ is inconsistent with $\htau_2$}
\begin{mathpar}
    \inferrule[ICBaseArr1]{ }{
      \tinconsistent{\tb}{\tarr{\htau_1}{\htau_2}}
    }

    \inferrule[ICBaseArr2]{ }{
      \tinconsistent{\tarr{\htau_1}{\htau_2}}{\tb}
    }

    \inferrule[ICArr1]{
      \tinconsistent{\htau_1}{\htau_3}
    }{
      \tinconsistent{\tarr{\htau_1}{\htau_2}}{\tarr{\htau_3}{\htau_4}}
    }

    \inferrule[ICArr2]{
      \tinconsistent{\htau_2}{\htau_4}
    }{
      \tinconsistent{\tarr{\htau_1}{\htau_2}}{\tarr{\htau_3}{\htau_4}}
    }
\end{mathpar}

\judgbox{\arrmatch{\htau}{\tarr{\htau_1}{\htau_2}}}{$\htau$ has matched arrow type $\tarr{\htau_1}{\htau_2}$}
\begin{mathpar}
\inferrule[MAHole]{ }{
  \arrmatch{\tehole}{\tarr{\tehole}{\tehole}}
}

\inferrule[MAArr]{ }{
  \arrmatch{\tarr{\htau_1}{\htau_2}}{\tarr{\htau_1}{\htau_2}}
}
\end{mathpar}

% \judgbox{\prodmatch{\htau}{\tprod{\htau_1}{\htau_2}}}{$\htau$ has matched product type $\tprod{\htau_1}{\htau_2}$}
% \begin{mathpar}
% \inferrule{ }{
%   \prodmatch{\tehole}{\tprod{\tehole}{\tehole}}
% }

% \inferrule{ }{
%   \prodmatch{\tprod{\htau_1}{\htau_2}}{\tprod{\htau_1}{\htau_2}}
% }
% \end{mathpar}

% \judgbox{\summatch{\htau}{\tsum{\htau_1}{\htau_2}}}{$\htau$ has matched sum type $\tsum{\htau_1}{\htau_2}$}
% \begin{mathpar}
% \inferrule{ }{
%   \summatch{\tehole}{\tsum{\tehole}{\tehole}}
% }

% \inferrule{ }{
%   \summatch{\tsum{\htau_1}{\htau_2}}{\tsum{\htau_1}{\htau_2}}
% }
% \end{mathpar}
\caption{Type Consistency, Inconsistency, and Matching}
\label{fig:tconsistent}
\label{fig:arrmatch}
\end{figure}

\begin{figure}[!ht]
\judgbox{\hsyn{\hGamma}{\hexp}{\htau}}{$\htau$ synthesizes type $\htau$}
\begin{mathpar}
\inferrule[SConst]{ }{
  \hsyn{\hGamma}{c}{b}
}

\inferrule[SVar]{ }{
  \hsyn{\hGamma, x : \htau}{x}{\htau}
}

\inferrule[SLam]{
  \hsyn{\hGamma, x : \htau_1}{\hexp}{\htau_2}
}{
  \hsyn{\hGamma}{\halam{x}{\htau_1}{\hexp}}{\tarr{\htau_1}{\htau_2}}
}

\inferrule[SAp]{
  \hsyn{\hGamma}{\hexp_1}{\htau_1}\\
  \arrmatch{\htau_1}{\tarr{\htau_2}{\htau}}\\\\
  \hana{\hGamma}{\hexp_2}{\htau_2}
}{
  \hsyn{\hGamma}{\hap{\hexp_1}{\hexp_2}}{\htau}
}

\inferrule[SEHole]{ }{
  \hsyn{\hGamma}{\hehole{u}}{\tehole}
}

\inferrule[SNEHole]{
  \hsyn{\hGamma}{\hexp}{\htau}
}{
  \hsyn{\hGamma}{\hhole{\hexp}{u}}{\tehole}
}

\inferrule[SAsc]{
  \hana{\hGamma}{\hexp}{\htau}
}{
  \hsyn{\hGamma}{\hexp : \htau}{\htau}
}
\end{mathpar}

\judgbox{\hana{\hGamma}{\hexp}{\htau}}{$\htau$ analyzes against type $\htau$}
\begin{mathpar}
\inferrule[ALam]{
  \arrmatch{\htau}{\tarr{\htau_1}{\htau_2}}\\
  \hana{\hGamma, x : \htau_1}{\hexp}{\htau_2}
}{
  \hana{\hGamma}{\hlam{x}{\hexp}}{\htau}
}

\inferrule[ASubsume]{
  \hsyn{\hGamma}{\hexp}{\htau}\\
  \tconsistent{\htau}{\htau'}
}{
  \hana{\hGamma}{\hexp}{\htau'}
}
\end{mathpar}
\caption{Bidirectional Typing of H-Expressions}
\end{figure}

\todo{definition of complete types and terms? do we care about completeness
  of $\hexp$s or $\dexp$s?}

\begin{figure}[!ht]
\judgbox{\expandSyn{\hGamma}{\hexp}{\htau}{\dexp}{\Delta}}{ }
\begin{mathpar}
\inferrule[ESConst]{ }{
  \expandSyn{\hGamma}{c}{b}{c}{\emptyset}
}

\inferrule[ESVar]{
  x : \htau \in \hGamma
}{
  \expandSyn{\hGamma}{x}{\htau}{x}{\emptyset}
}

\inferrule[ESLam]{
  \expandSyn{\hGamma, x : \htau_1}{\hexp}{\htau_2}{\dexp}{\Delta}
}{
  \expandSyn{\hGamma}{\halam{x}{\htau_1}{\hexp}}{\tarr{\htau_1}{\htau_2}}{\halam{x}{\htau_1}{\dexp}}{\Delta}
}

\inferrule[ESAp1]{
  \hsyn{\hGamma}{\hexp_1}{\tehole}\\
  \expandAna{\hGamma}{\hexp_1}{\tarr{\htau_2}{\tehole}}{\dexp_1}{\htau_1}{\Delta_1}\\
  \expandAna{\hGamma}{\hexp_2}{\tehole}{\dexp_2}{\htau_2}{\Delta_2}
}{
  \expandSyn{\hGamma}{\hap{\hexp_1}{\hexp_2}}{\tehole}{\hap{(\dcast{\tarr{\htau_2}{\tehole}}{\dexp_1})}{\dexp_2}}{\Dunion{\Delta_1}{\Delta_2}}
}

\inferrule[ESAp2]{
  \expandSyn{\hGamma}{\hexp_1}{\tarr{\htau_2}{\htau}}{\dexp_1}{\Delta_1}\\
  \expandAna{\hGamma}{\hexp_2}{\htau_2}{\dexp_2}{\htau'_2}{\Delta_2}\\
  \htau_2 \neq \htau'_2
}{
  \expandSyn{\hGamma}{\hap{\hexp_1}{\hexp_2}}{\htau}{\hap{\dexp_1}{\dcast{\htau_2}{\dexp_2}}}{\Dunion{\Delta_1}{\Delta_2}}
}

\inferrule[ESAp3]{
  \expandSyn{\hGamma}{\hexp_1}{\tarr{\htau_2}{\htau}}{\dexp_1}{\Delta_1}\\
  \expandAna{\hGamma}{\hexp_2}{\htau_2}{\dexp_2}{\htau_2}{\Delta_2}
}{
  \expandSyn{\hGamma}{\hap{\hexp_1}{\hexp_2}}{\htau}{\hap{\dexp_1}{\dexp_2}}{\Dunion{\Delta_1}{\Delta_2}}
}\\
%
% \inferrule[expand-pair]{
%   \expandSyn{\hGamma}{\hexp_1}{\htau_1}{\dexp_1}{\Delta_1}\\
%   \expandSyn{\hGamma}{\hexp_2}{\htau_2}{\dexp_2}{\Delta_2}
% }{
%   \expandSyn{\hGamma}{\hpair{\hexp_1}{\hexp_2}}{\tprod{\htau_1}{\htau_2}}{\hpair{\dexp_1}{\dexp_2}}{\Dunion{\Delta_1}{\Delta_2}}
% }
%
% \inferrule[expand-prj]{
%   a
% }{
%   b
% }
%
% (inj)

%
% \inferrule[expand-plus]{ }{
%   \expandSyn{\hGamma}{\hadd{\hexp_1}{\hexp_2}}{\tnum}{\hadd{\dexp_1}{\dexp_2}}{\Dunion{\Delta_1}{\Delta_2}}
% }
%
\inferrule[ESEHole]{ }{
  \expandSyn{\hGamma}{\hehole{u}}{\tehole}{\dehole{u}{\idof{\hGamma}}{\unevaled}}{\Dbinding{u}{\hGamma}{\tehole}}
}

\inferrule[ESNEHole]{
  \expandSyn{\hGamma}{\hexp}{\htau}{\dexp}{\Delta}
}{
  \expandSyn{\hGamma}{\hhole{\hexp}{u}}{\tehole}{\dhole{\dexp}{u}{\idof{\hGamma}}{\unevaled}}{\Delta, \Dbinding{u}{\hGamma}{\tehole}}
}\\
%
\inferrule[ESAsc1]{
  \expandAna{\hGamma}{\hexp}{\htau}{\dexp}{\htau'}{\Delta}\\
  \htau \neq \htau'
}{
  \expandSyn{\hGamma}{\hexp : \htau}{\htau}{\dcast{\htau}{\dexp}}{\Delta}
}

\inferrule[ESAsc2]{
  \expandAna{\hGamma}{\hexp}{\htau}{\dexp}{\htau}{\Delta}
}{
  \expandSyn{\hGamma}{\hexp : \htau}{\htau}{\dexp}{\Delta}
}
\end{mathpar}

\judgbox{\expandAna{\hGamma}{\hexp}{\htau}{\dexp}{\htau}{\Delta}}{ }
\begin{mathpar}
\inferrule[EALam]{
  \expandAna{\hGamma, x : \htau_1}{\hexp}{\htau_2}{\dexp}{\htau'_2}{\Delta}
}{
  \expandAna{\hGamma}{\hlam{x}{\hexp}}{\tarr{\htau_1}{\htau_2}}{\halam{x}{\htau_1}{\dexp}}{\tarr{\htau_1}{\htau_2'}}{\Delta}
}

\inferrule[EALamHole]{
  \expandAna{\hGamma, x : \tehole}{\hexp}{\tehole}{\dexp}{\htau}{\Delta}
}{
  \expandAna{\hGamma}{\hlam{x}{\hexp}}{\tehole}{\halam{x}{\tehole}{\dexp}}{\tarr{\tehole}{\htau}}{\Delta}
}

\inferrule[EASubsume]{
  \hexp \neq \hehole{u}\\
  \hexp \neq \hhole{\hexp'}{u}\\\\
  \expandSyn{\hGamma}{\hexp}{\htau'}{\dexp}{\Delta}\\
  \tconsistent{\htau}{\htau'}
}{
  \expandAna{\hGamma}{\hexp}{\htau}{\dexp}{\htau'}{\Delta}
}

\inferrule[EAEHole]{ }{
  \expandAna{\hGamma}{\hehole{u}}{\htau}{\dehole{u}{\idof{\hGamma}}{\unevaled}}{\htau}{\Dbinding{u}{\hGamma}{\htau}}
}

\inferrule[EANEHole]{
  \expandSyn{\hGamma}{\hexp}{\htau'}{\dexp}{\Delta}\\
}{
  \expandAna{\hGamma}{\hhole{\hexp}{u}}{\htau}{\dhole{\dexp}{u}{\idof{\hGamma}}{\unevaled}}{\htau}{\Delta, \Dbinding{u}{\hGamma}{\htau}}
}
\end{mathpar}
\caption{Algorithmic Expansion}
\label{fig:expandSyn}
\label{fig:expandAna}
\end{figure}

\begin{figure}[!ht]
  \begin{definition}
    $\hasType{\Delta}{\hGamma}{\sigma}{\hGamma'}$ iff for each $\dexp/x \in \sigma$, we have $x : \htau \in \hGamma'$ and $\hasType{\Delta}{\hGamma}{\dexp}{\htau}$.
  \end{definition}
  \caption{substitution type assignment}
  \label{fig:subassign}
\end{figure}

\begin{figure}[!ht]
\judgbox{\hasType{\Delta}{\hGamma}{\dexp}{\htau}}{$\dexp$ is assigned type $\htau$}
\begin{mathpar}
\inferrule[TAConst]{ }{
  \hasType{\Delta}{\hGamma}{c}{b}
}

\inferrule[TAVar]{
  x : \htau \in \hGamma
}{
	\hasType{\Delta}{\hGamma}{x}{\htau}
}

\inferrule[TALam]{
  \hasType{\Delta}{\hGamma, x : \htau_1}{\dexp}{\htau_2}
}{
  \hasType{\Delta}{\hGamma}{\halam{x}{\htau_1}{\dexp}}{\tarr{\htau_1}{\htau_2}}
}

\inferrule[TAAp]{
  \hasType{\Delta}{\hGamma}{\dexp_1}{\htau_1}\\
  \arrmatch{\htau_1}{\tarr{\htau_2}{\htau}}\\\\
  \hasType{\Delta}{\hGamma}{\dexp_2}{\htau_2}
}{
  \hasType{\Delta}{\hGamma}{\hap{\dexp_1}{\dexp_2}}{\htau}
}

\inferrule[TAEHole]{
  \Dbinding{u}{\hGamma'}{\htau} \in \Delta\\
  \hasType{\Delta}{\hGamma}{\sigma}{\hGamma'}
}{
  \hasType{\Delta}{\hGamma}{\dehole{u}{\sigma}{\markname}}{\htau}
}

\inferrule[TANEHole]{
  \Dbinding{u}{\hGamma'}{\htau} \in \Delta\\\\
  \hasType{\Delta}{\hGamma}{\dexp}{\htau'}\\
  \hasType{\Delta}{\hGamma}{\sigma}{\hGamma'}
}{
  \hasType{\Delta}{\hGamma}{\dhole{\dexp}{u}{\sigma}{\markname}}{\htau}
}

\inferrule[TACast]{
  \hasType{\Delta}{\Gamma}{\dexp}{\htau'}\\
  \tconsistent{\htau}{\htau'}
}{
  \hasType{\Delta}{\hGamma}{\dcast{\htau}{\dexp}}{\htau}
}
\end{mathpar}
\caption{Type Assignment for Dynamic H-Expressions}
\label{fig:hasType}
\end{figure}

\begin{figure}[!ht]
  \begin{definition}
    $\hasType{\Delta}{\hGamma}{\sigma}{\hGamma'}$ iff $\domof{\sigma} = \domof{\Gamma'}$ and for each $\dexp/x \in \sigma$, we have $x : \htau \in \hGamma'$ and $\hasType{\Delta}{\hGamma}{\dexp}{\htau}$.
  \end{definition}
  \caption{substitution type assignment}
\end{figure}

\begin{figure}[!ht]
\judgbox{[\dexp_1 / x] \dexp_2 = \dexp_3}{Substitution}
\[
\begin{array}{lcll}
[\dexp_1 / x] c
&=&
c\\
{[\dexp_1 / x] x}
&=&
\dexp_1\\
{[\dexp_1 / x] y}
&=&
y & (y \neq x)\\
{[\dexp_1 / x] \dlam{y}{\htau}{\dexp_2}}
&=&
{\dlam{y}{\htau}{[\dexp_1 / x] \dexp_2}}
& (y \neq x)
\\
{[\dexp_1 / x] \dap{d_2}{d_3}}
&=&
{\dapP{[\dexp_1 / x] d_2}{[\dexp_1 / x] d_3}}
\\
% {[\dexp_1 / x] \dinj{i}{e_2}}
% &=&
% {\dinj{i}{[\dexp_1 / x] e_2}}
% \\
% {[\dexp_1 / x] \dcase{e}{x}{e_x}{y}{e_y}}
% &=&
% {\dcase{[\dexp_1 / x]e}{x}{[\dexp_1 / x]e_x}{y}{[\dexp_1 / x]e_y}}
% \\
{[\dexp_1 / x]} \dehole{\mvar}{\subst}{m}
&=&
\dehole{\mvar}{[\dexp_1 / x] \subst}{m}
\\
{[\dexp_1 / x] \dhole{\dexp_2}{\mvar}{\subst}{m}}
&=&
\dhole{ [ \dexp_1 / x ] \dexp_2 }{\mvar}{[\dexp_1 / x] \subst}{m}
\\
{[\dexp_1 / x] \dcast{\htau}{\dexp}}
&=&
\dcast{\htau}{[\dexp_1 / x] \dexp}
\end{array}
\]
\end{figure}

\begin{figure}[!ht]
\judgbox{\isValue{\dexp}}{$\dexp$ is a value}
\begin{mathpar}
\inferrule[VConst]{ }{
  \isValue{c}
}

\inferrule[VLam]{ }{
  \isValue{\halam{x}{\htau}{\dexp}}
}
\end{mathpar}

\judgbox{\isIndet{\dexp}}{$\dexp$ is indeterminate}
\begin{mathpar}
\inferrule[IEHole]
{ }
{\isIndet{\dehole{\mvar}{\subst}{\evaled}}}

\inferrule[INEHole]
{\isFinal{\dexp}}
{\isIndet{\dhole{\dexp}{\mvar}{\subst}{\evaled}}}

\inferrule[IAp]
{\isIndet{\dexp_1}\\
% \isFinal{\dexp_2}~\text{\todo{??}}}
 \isFinal{\dexp_2}}
{\isIndet{\dap{\dexp_1}{\dexp_2}}}

\inferrule[ICast]
{\isIndet{\dexp}}
{\isIndet{\dcast{\htau}{\dexp}}}

\end{mathpar}

\judgbox{\isFinal{\dexp}}{$\dexp$ is final (it cannot be stepped)}
\begin{mathpar}
\inferrule[FVal]
{\isValue{\dexp}}{\isFinal{\dexp}}
\and
\inferrule[FIndet]
{\isIndet{\dexp}}{\isFinal{\dexp}}
\end{mathpar}

\judgbox{\isErr{\Delta}{\dexp}}{$\dexp$ contains a cast error}
\begin{mathpar}
\inferrule[ECastError]{
  \hasType{\Delta}{\emptyset}{\dexp}{\htau_2}\\
  \tinconsistent{\htau_1}{\htau_2}
}{
  \isErr{\Delta}{\dcast{\htau_1}{\dexp}}
}

\inferrule[EConst]{
}{
  \isErr{\Delta}{\dcast{\tau}{\dconst}}
}

\inferrule[EAp1]{
  \isErr{\Delta}{\dexp_1}
}{
  \isErr{\Delta}{\dap{\dexp_1}{\dexp_2}}
}

\inferrule[EAp2]{
  \isErr{\Delta}{\dexp_2}
}{
  \isErr{\Delta}{\dap{\dexp_1}{\dexp_2}}
}

\inferrule[ENEHole]{
  \isErr{\Delta}{\dexp}
}{
  \isErr{\Delta}{\dhole{\dexp}{\mvar}{\subst}{m}}
}

\inferrule[ECastProp]{
  \isErr{\Delta}{\dexp}
}{
  \isErr{\Delta}{\dcast{\htau_1}{\dexp}}
}
\end{mathpar}
\caption{Auxiliary Judgments for the Dynamics}
\label{fig:isValue}
\label{fig:isIndet}
\label{fig:isFinal}
\label{fig:isErr}
\end{figure}

\begin{figure}
\judgbox{\stepsToD{\Delta}{\dexp_1}{\dexp_2}}{$\dexp_1$
steps to $\dexp_2$}
\begin{mathpar}
\inferrule[STEHoleEvaled]
{ }
{\stepsToD{\Delta}{\dehole{\mvar}{\subst}{\unevaled}}{\dehole{\mvar}{\subst}{\evaled}}}

\inferrule[STNEHoleStep]
{\stepsToD{\Delta}{\dexp_1}{\dexp_2} }
{\stepsToD{\Delta}{\dhole{\dexp_1}{\mvar}{\subst}{\unevaled}}{\dhole{\dexp_2}{\mvar}{\subst}{\unevaled}}}

\inferrule[STNEHoleEvaled]
{\isFinal{\dexp}}
{\stepsToD{\Delta}{\dhole{\dexp}{\mvar}{\subst}{\unevaled}}{\dhole{\dexp}{\mvar}{\subst}{\evaled}}}

\inferrule[STCast]
{
\isValue{\dexp}\\
\hasType{\Delta}{\emptyset}{\dexp}{\htau_2} \\
\tconsistent{\tau_1}{\tau_2}}
{\stepsToD{\Delta}{\dcast{\htau_1}{\dexp}}{\dexp}}

\inferrule[STApStep1]
{\stepsToD{\Delta}{\dexp_1}{\dexp_1'}}
{\stepsToD{\Delta}{\dap{\dexp_1}{\dexp_2}}{\dap{\dexp_1'}{\dexp_2}}}

\inferrule[STApStep2]
{ \isFinal{\dexp_1} \\ \stepsToD{\Delta}{\dexp_2}{\dexp_2'}}
{\stepsToD{\Delta}{\dap{\dexp_1}{\dexp_2}}{\dap{\dexp_1}{\dexp_2'}}}

\inferrule[STApBeta]
{ \isFinal{\dexp_2} }
{\stepsToD{\Delta}{\dapP{\dlam{x}{\htau}{\dexp_1}}{\dexp_2}}{ [\dexp_2/x]\dexp_1 }}
\end{mathpar}
\caption{Structural Dynamics}
\label{fig:stepsTo}
\end{figure}


\begin{figure}
$\arraycolsep=4pt\begin{array}{rllllll}
\mathsf{EvalCtx} & \evalctx & ::= &
  \evalhole ~\vert~
  \halam{x}{\htau}{\evalctx} ~\vert~
  \hap{\evalctx}{\dexp} ~\vert~
  \hap{\dexp}{\evalctx} ~\vert~
  \dhole{\evalctx}{\mvar}{\subst}{\markname} ~\vert~
  \dcast{\htau}{\evalctx}
\end{array}$

\judgbox{\isevalctx{\evalctx}}{$\evalctx$ is an evaluation context}
\begin{mathpar}
\inferrule[ECDot]{ }{
  \isevalctx{\evalhole}
}

\inferrule[ECLam]{
  \isevalctx{\evalctx}
}{
  \isevalctx{\halam{x}{\htau}{\evalctx}}
}

\inferrule[ECAp1]{
  \isFinal{\dexp}\\
  \isevalctx{\evalctx}
}{
  \isevalctx{\hap{\dexp}{\evalctx}}
}

\inferrule[ECAp2]{
  \isevalctx{\evalctx}
}{
  \isevalctx{\hap{\evalctx}{\dexp}}
}

\inferrule[ECNEHole]{
  \isevalctx{\evalctx}
}{
  \isevalctx{\dhole{\evalctx}{\mvar}{\subst}{\markname}}
}

\inferrule[ECCast]{
  \isevalctx{\evalctx}
}{
  \isevalctx{\dcast{\htau}{\evalctx}}
}
\end{mathpar}

\judgbox{\selectEvalCtx{d}{\evalctx}{d'}}{$d$ is the result of filling the
hole in $\evalctx$ with $d'$}
\begin{mathpar}
\inferrule[FDot]{ }{
  \selectEvalCtx{d}{\evalhole}{d}
}

%% \todo{d used to be c -- TODO IEV}

\inferrule[FLam]{
  \selectEvalCtx{d}{\evalctx}{d'}
}{
  \selectEvalCtx{\halam{x}{\htau}{d}}{\halam{x}{\htau}{\evalctx}}{d'}
}

\inferrule[FAp1]{
  \isFinal{d_1}\\
  \selectEvalCtx{d_2}{\evalctx}{d_2'}
}{
  \selectEvalCtx{\hap{d_1}{d_2}}{\hap{d_1}{\evalctx}}{d_2'}
}

\inferrule[FAp2]{
  \selectEvalCtx{d_1}{\evalctx}{d_1'}
}{
  \selectEvalCtx{\hap{d_1}{d_2}}{\hap{\evalctx}{d_2}}{d_1'}
}

%% I think these rules are not needed, if you make FDot work for any dhexp
%% d. The only reason they might be needed is because they're specified to
%% have check marks, not x marks, but i don't think that's important. i
%% think the more general FDot makes sense. -- TODO IEV
%%
%%
%% \inferrule[\todo{??}]{ }{
%%   \selectEvalCtx{\dehole{\mvar}{\subst}{\evaled}}{\evalhole}{\dehole{\mvar}{\subst}{\evaled}}
%% }

%% \inferrule[\todo{??}]{ }{
%%   \selectEvalCtx{\dhole{d}{\mvar}{\subst}{\evaled}}{\evalhole}{\dhole{d}{\mvar}{\subst}{\evaled}}
%% }


\inferrule[\todo{??}]{
  \selectEvalCtx{d}{\evalctx}{d'}
}{
  \selectEvalCtx{\dhole{d}{\mvar}{\subst}{\unevaled}}{\dhole{\evalctx}{\mvar}{\subst}{\unevaled}}{d'}
}

\inferrule[\todo{??}]{
  \isFinal{d}
}{
  \selectEvalCtx{\dhole{d}{\mvar}{\subst}{\unevaled}}{\evalhole}{\dhole{d}{\mvar}{\subst}{\unevaled}}
}

\inferrule[FCast]{
  \selectEvalCtx{d}{\evalctx}{d'}
}{
  \selectEvalCtx{\dcast{\htau}{d}}{\dcast{\htau}{\evalctx}}{d'}
}
\end{mathpar}


\judgbox{\reducesE{\Delta}{\dexp_1}{\dexp_2}}{$\dexp_1$
transitions to $\dexp_2$}
\begin{mathpar}
\inferrule[ITLam]{
  \isFinal{d_2}
}{
  \reducesE{\Delta}{\hap{(\halam{x}{\htau}{d})}{d_2}}{[d_2/x]d}
}

\inferrule[ITCast]{
  \isFinal{d}\\
  \hasType{\Delta}{\emptyset}{d}{\tau_2}\\
  \tconsistent{\tau_1}{\tau_2}
}{
  \reducesE{\Delta}{\dcast{\htau_1}{d}}{d}
}

\inferrule[ITEHole]{ }{
  \reducesE{\Delta}{\dehole{\mvar}{\subst}{\unevaled}}{\dehole{\mvar}{\subst}{\evaled}}
}

\inferrule[ITNEHole]{
  \isFinal{d}
}{
  \reducesE{\Delta}{\dhole{d}{\mvar}{\subst}{\unevaled}}{\dhole{d}{\mvar}{\subst}{\evaled}}
}
\end{mathpar}

\judgbox{\stepsToD{\Delta}{\dexp_1}{\dexp_2}}{$\dexp_1$ steps to $\dexp_2$}
\begin{mathpar}
\inferrule[Step]{
  \selectEvalCtx{d}{\evalctx}{d_0}\\
  \reducesE{\Delta}{d_0}{d_0'}\\
  \selectEvalCtx{d'}{\evalctx}{d_0'}
}{
  \stepsToD{\Delta}{\dexp}{\dexp'}
}
\end{mathpar}
\caption{Contextual Dynamics}
\label{fig:contextual-dynamics}
\end{figure}


\clearpage
\subsection{Metatheory}

(these theorems should be taken as fairly provisional)

\begin{theorem}[Expandability \todo{??}] ~
  \begin{enumerate}
    \item If $\hsyn{\hGamma}{\hexp}{\htau}$ then $\expandSyn{\hGamma}{\hexp}{\htau}{\dexp}{\Delta}$ for some $\dexp$ and $\Delta$.
    \item If $\hana{\hGamma}{\hexp}{\htau}$ then $\expandAna{\hGamma}{\hexp}{\htau}{\dexp}{\htau'}{\Delta}$ for some $\dexp$ and $\htau'$ and $\Delta$.
  \end{enumerate}
\end{theorem}

\begin{theorem}[Correspondence \todo{??}] ~
  \begin{enumerate}
    \item If $\expandSyn{\hGamma}{\hexp}{\htau}{\dexp}{\Delta}$ then $\hsyn{\hGamma}{\hexp}{\htau}$.
    \item If $\expandAna{\hGamma}{\hexp}{\htau}{\dexp}{\htau'}{\Delta}$ then $\hana{\hGamma}{\hexp}{\htau}$.
  \end{enumerate}
\end{theorem}

\begin{theorem}[Typed Expansion] ~
  \begin{enumerate}
    \item If $\expandSyn{\hGamma}{\hexp}{\htau}{\dexp}{\Delta}$ then $\hasType{\Delta}{\hGamma}{\dexp}{\htau}$.
    \item If $\expandAna{\hGamma}{\hexp}{\htau}{\dexp}{\htau'}{\Delta}$ then $\tconsistent{\htau}{\htau'}$ and $\hasType{\Delta}{\hGamma}{\dexp}{\htau'}$.
  \end{enumerate}
\end{theorem}

\begin{theorem}[Expansion Unicity] ~
  \begin{enumerate}
    \item If $\expandSyn{\hGamma}{\hexp}{\htau}{\dexp}{\Delta}$ and $\expandSyn{\hGamma}{\hexp}{\htau'}{\dexp'}{\Delta'}$ then $\htau=\htau'$ and $\dexp=\dexp'$ and $\Delta=\Delta'$.
    \item If $\expandAna{\hGamma}{\hexp}{\htau_1}{\dexp}{\htau_2}{\Delta}$ and $\expandAna{\hGamma}{\hexp}{\htau_1}{\dexp'}{\htau_2'}{\Delta'}$ then $\dexp=\dexp'$ and $\htau_2=\htau_2'$ and $\Delta=\Delta'$.
  \end{enumerate}
\end{theorem}

\begin{theorem}[Type Assignment Unicity]
If $\hasType{\Delta}{\hGamma}{\dexp}{\htau}$ and $\hasType{\Delta}{\hGamma}{\dexp}{\htau'}$ then $\htau=\htau'$.
\end{theorem}

\begin{theorem}[Progress]
If $\hasType{\Delta}{\emptyset}{\dexp}{\htau}$ then either
$\isValue{\dexp}$ or $\isIndet{\dexp}$ or $\isErr{\Delta}{\dexp}$ or
$\stepsToD{\Delta}{\dexp}{\dexp'}$ for some $\dexp'$.
\end{theorem}

\begin{theorem}[Preservation]
  If $\hasType{\Delta}{\emptyset}{\dexp}{\htau}$ and
  $\stepsToD{\Delta}{\dexp}{\dexp'}$ then
  $\hasType{\Delta}{\emptyset}{\dexp'}{\htau}$. \todo{or maybe up to
  consistency?}
\end{theorem}

\begin{theorem}[Complete Progress \todo{??}]
If $\hcomplete{\hexp}$ and
$\expandSyn{\emptyset}{\hexp}{\htau}{\dexp}{\Delta}$ then either
$\isValue{\dexp}$ or $\stepsToD{\Delta}{\dexp}{\dexp'}$ for some $\dexp'$.

\end{theorem}

\begin{theorem}[Complete Preservation \todo{??}]
If $\hcomplete{\hexp}$ and
$\expandSyn{\emptyset}{\hexp}{\htau}{\dexp}{\Delta}$ and
$\stepsToD{\Delta}{\dexp}{\dexp'}$ then
$\hasType{\Delta}{\emptyset}{\dexp'}{\htau}$
\end{theorem}

\begin{theorem}[Commutativity \todo{state this}]
\end{theorem}


\begin{theorem}[Maximum Informativity]
If the expansion produces $t1$, and there exists another possible type choice
$t2$, then $t1 \sim t2$ and $t1 JOIN t2 = t1$
\end{theorem}\footnote{idea is that special casing the holes in EANEHole gives you ``the
most descriptive hole types'' for some sense of what that means -- they'd
all just be hole other wise. from Matt:
\begin{quote}
It sounds like we need a something akin to an abstract domain (a lattice),
where hole has the least information, and a fully-defined type (without
holes) has the most information.  You can imagine that this lattice really
expands the existing definition we have of type consistency, which is
merely the predicate that says whether two types are comparable
(“join-able”) in this lattice.  lattice join is the operation that goes
through the structure of two (consistent) types, and chooses the structure
that is more defined (i.e., non-hole, if given the choice between hole and
non-hole).

The rule choosenonhole below is the expansion of this consistency rule that
we already have (hole consistent with everything)
\end{quote}}

\begin{verbatim}
t not hole
-------------------- :: choose-non-hole
hole JOIN t  = t
\end{verbatim}
\begin{verbatim}
------------ :: hole-consistent-with-everything
hole ~ t
\end{verbatim}

\todo{canonical forms}

\todo{indeterminate forms}
