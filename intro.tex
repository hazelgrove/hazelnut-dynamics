% !TEX root = hazelnut-dynamics.tex
\newcommand{\introSec}{Introduction}
\section{\protect\introSec} % don't like the all-caps thing that the template does, so protecting it from that
\label{sec:intro}

In a traditional programming environment, the user iteratively edits the program
text, compiles and runs the program, and views the output.
%
To make this workflow more usable and efficient, \emph{live programming
environments} ``promise to narrow the temporal and perceptive gap between
program development and code execution''~\cite{burckhardt2013s} with
fine-grained interleaving of editing and evaluation~\cite{DBLP:journals/vlc/Tanimoto90,DBLP:conf/icse/Tanimoto13}.


There have been several promising advances in building live programming
environments in many settings, such as:
%
{lab notebook environments}, like the popular
IPython/Jupyter~\cite{PER-GRA:2007}, which allow the programmer to interactively
edit and evaluate program fragments organized into a sequence of cells (an
extension of the ubiquitous read-eval-print loop (REPL));
%
spreadsheets, which provide a reactive dataflow programming model;
%
live direct manipulation programming environments like SuperGlue
\cite{McDirmid:2007}, \sns{}~\cite{sns-pldi,sns-uist}, and the tools
demonstrated by Bret Victor in his lectures \cite{victor2012inventing};
%
the TouchDevelop live UI framework \cite{burckhardt2013s};
%
live mobile application development systems like Flutter \cite{flutter};
%
live visual image processing languages~\cite{DBLP:journals/vlc/Tanimoto90};
%
and live visual and auditory dataflow languages \cite{DBLP:conf/vl/BurnettAW98},
to name several prominent examples.

Despite these advances, however, there is a fundamental limitation which hinders
the full realization of live programming: traditional programming languages
cannot evaluate \emph{incomplete programs}, \ie{}, programs that are either
syntactically ill-formed, or ill-typed.
%
Without a well-defined semantics to dictate the treatment of incomplete
programs, various editor services may be provided in ad-hoc, restricted
forms---if available at all---during the periods of time between complete
program states.
%
%% ``flickering in and out''
%
%% In other words, the ``Edit-Compile-Run'' cycle routinely fails (either in the
%% first or second steps) during the program editing process.
%
The resulting editor experience includes services ``flickering in and out,''
thus introducing ``temporal and perceptive gaps'' because programmers often
leave a program malformed or ill-typed for extended periods of time, \eg{}~as
they think about what to enter at the cursor, or as they work on a different
part of the program.
%
These times are, arguably, when rich, live editor services would be most helpful
to the user.

In view of this general problem---the lack of semantics for \emph{incomplete
programs}---\citet{popl-paper} recently developed a \emph{structure editor
calculus} called Hazelnut where every edit state consists of a well-formed and
statically meaningful (\ie{}~well-typed) incomplete expression, in particular,
one with \emph{expression holes} or \emph{type holes}.
%
That work provides a static semantics for incomplete programs but not a
\emph{dynamic} semantics, without which certain editor services---such as
\emph{stepwise debuggers} (\eg{}~\cite{XXX}) and advanced \emph{code completion
suggestions} (\eg{}~\cite{XXX})---cannot be built.
%
The topic of this paper is to develop such a dynamic semantics for functional
programs with holes.

%% \parahead{Prior Approaches: Exceptions as Expression Holes}
\parahead{Prior Approaches}

Given the existing lack of dynamic semantic foundations for incomplete programs,
what alternative mechanisms do programmers have at their disposal to run and
debug incomplete programs?

\paragraph{Exceptions as Expression Holes}
%
In lieu of proper language and tool support for expression holes,
today's programmers commonly emulate them with placeholder exceptions, such as
\verb+Debug.crash "not yet implemented"+ or
\verb+raise NotYetImplementedException+.
%
Some languages even provide syntax for these kind of ``holes''
%
(written \verb+_+ in GHC Haskell~\cite{XXX},
\verb+?+ in Agda~\cite{XXX}, and
\verb+{x}+ in Idris~\cite{XXX})
%
which, semantically, behave similarly to exceptions.

Such exceptions serve to appease the static type checker---because an
error expression is allowed to have any type required by the surrounding
context.
%
Thus, this mechanism permits the program to run, while preserving dynamic type safety, and
simply halting execution as soon as the expression appears in evaluation
position.

Although useful, an exception-based mechanism for ``program hole'' has limitations.
%
First,
the program must halt as soon as this expression is reached (ensuring type safety), and consequently, the programmer cannot benefit from seeing how
\emph{other parts} of the program, which may not depend on the missing expression,
continue to evaluate.

Second, exceptions typically carry nothing more than the
stack trace that led to the error.
%
But there are other forms of information that could be useful as the programmer
works to fill the hole---\eg{}, the dynamic environment around the failed
expression, and the subsequent computations that depend on the missing value
(which could only be deduced if evaluation continued).
%
Thus, the dynamic semantics of exceptions do not adequately address the
particular needs for expression holes.

\rkc{(mention something about manually commenting out partial expressions,
rather than something nicer like marking them inside non-empty holes?)}

%%%%%%%%%
%% Matt says: I added the remarks below; My intention is to tie the
%% last bit into the ``edit and resume'' feature, without making a
%% huge deal about it.

Finally, exceptions are often thrown with the intent that some calling
context will possibly ``handle'' them, and will thus resume ordinary
(non-exception-based) program control flow.
%
Meanwhile, it never really makes sense to handle a ``not yet
implemented'' exception, save for \emph{editing the program to further
  complete it}, which is beyond the scope and control of ordinary
exception handlers, and (ordinary) programs' execution behavior.
%
Indeed, this is one (of many) ``perceptual gaps'' for live programming
environments to close.
%
In this work, we refer to the programmer-user interaction pattern that
follows such exceptions as ``edit and resume''.

\paragraph{Gradual Typing}
%
A well-developed approach for programming with ``type holes'' is \emph{gradual
typing}~\cite{XXX,XXX,XXX,XXX}, where the \verb+Any+ type can be assigned to
expressions either when
%
(a)~a valid type assignment is unknown and cannot be inferred, or
%
(b)~there is no single valid type assignment.
%
To ensure type safety of expressions that have been statically assigned the
\verb+Any+ type, dynamic casts are attached to monitor how the expression is
used at operations that require specific types.
%
A failed dynamic cast halts execution immediately, much like other kinds of
run-time exceptions.
%
As with ad-hoc expression holes, the dynamics of existing gradually typed
languages~\cite{XXX,XXX,XXX,XXX} do not allow other portions of the program,
which do not depend on failed casts, to proceed. 

\parahead{Our Approach: Evaluating ``Around'' Indeterminate Expressions}

In contrast to using run-time exceptions, as above, our approach is to treat
expression and type holes with distinct semantic characteristics.
%
In particular, we distinguish a new set of \emph{indeterminate expressions},
which are neither fully-reduced to values, nor stuck, nor erroneous.
%
Instead, indeterminate expressions are ``paused'' during evaluation, at which
point the programmer can
%
(a) inspect the dynamic environment at the point at which evaluation left off,
%
(b) inspect how the indeterminate expression is used (abstractly) in subsequent
calculations, and
%
(c) observe the evaluation of other parts of the program that do not depend on
the value of the indeterminate expression.
%
Then, if the programmer makes an edit---to fill in an empty expression hole, to
replace a non-empty hole with a type-correct expression, or to replace a failed
cast with a type-correct expressions---the paused evaluation can resume.
%
Together, these semantic properties can enable a class of live programming
services---such as a debugging inspector and a suggestion system for filling
holes based on dynamic environments---that are not possible with run-time
exceptions.

\parahead{Contributions and Paper Outline}

\newcommand{\contribution}[2]{\paragraph{#1. #2}}

This paper makes the following contributions:

%% \begin{itemize}

%% \item
%
\contribution{1}{Dynamic semantics with indeterminate expressions}
%
We define a dynamic semantics for functional programs with holes, by
distinguishing \emph{indeterminate expressions} from those which are fully
reduced to values and those which are stuck.
%
The semantics proceeds by evaluating around indeterminate expressions, allowing
programs to continue to produce meaningful results despite dynamic cast errors
(from type holes) or missing expressions (expression holes).
%
Formally, our semantics combines features from gradual type theory~\cite{XXX}
(to handle incomplete types) and contextual modal type theory~\cite{XXX} (CMTT,
which provides a logical foundation for hole environment tracking).
%
With a formal development in Agda, we prove that our calculus enjoys standard
type safety properties, extended to account for indeterminate expressions.
%
(\autoref{sec:calculus})

%% \item
%
\contribution{2}{Resumption of suspended evaluation}
%
We describe how our semantics allows the evaluation of a program with
indeterminate expressions to \emph{resume} evaluation, as long as the only edits
to the program involve filling expression holes.
%
This ``edit-and-resume'' feature may be a useful optimization for large,
long-running programs in a development workflow that makes ``forward progress''
through filling holes.
%
We provide a proof sketch of commutativity and confluence properties that ensure
the correctness of this feature.
%
Our Adga formalization codifies this proof sketch, which assumes several
unproven lemmas that we expect to be similar to those used to prove confluence
in standard lambda-calculi~\cite{XXX}.
%
(\autoref{sec:resumption})

%% \item
%
\contribution{3}{Prototype implementation}
%
We extend the Hazelnut system of \citet{popl-paper}, which defined an edit
action calculus and static semantics for incomplete programs, with our new
dynamic semantics.
%
The resulting tool, \HazelnutLive{}, demonstrates a proof-of-concept where every
edit results in a program state that can be statically and dynamically analyzed
in order to provide meaningful feedback to the programmer.
%
\HazelnutLive{} can be used to and run the examples in this paper, modulo
several standard syntactic conveniences that our current implementation does not
provide.
%
(\autoref{sec:implementation})

%% \end{itemize}

\vspace{5pt}

Thus, our work provides an additional step towards the ultimate goal for live
programming environments to reduce the temporal and perceptive gap between
program development and code execution.
%
Our implementation, examples, and some videos are available in \suppMaterials{}.
%
Next, in \autoref{sec:examples}, we describe an overview example to introduce
the main features in \HazelnutLive{}, before describing each of the above
contributions in detail.
%
We wrap up in \autoref{sec:relatedWork} and \autoref{sec:discussion} with
discussions of related and future work.


\rkc{Cyrus/Matt: I moved all the ``raw'' material that was here into the
Appendix. Please take a look there if there's anything else that you'd like to
include in the Intro.}
